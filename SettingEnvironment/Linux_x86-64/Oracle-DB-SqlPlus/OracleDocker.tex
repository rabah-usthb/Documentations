\documentclass{article}
\usepackage[colorlinks=true, linkcolor=blue, urlcolor=blue, citecolor=blue]{hyperref}
\usepackage{xcolor}
\definecolor{greenCommand}{RGB}{107, 201, 132}
\definecolor{Sudo}{RGB}{35, 172, 140}
\definecolor{violetOption}{RGB}{150, 70, 185}
\definecolor{lightBlue}{RGB}{0,246,255}
\usepackage[a4paper, paperwidth=25.5cm, paperheight=24cm, left=2cm, right=2cm, top=2cm, bottom=2cm]{geometry}
\usepackage{tikz,tcolorbox}

\begin{document}
\section{Introduction : }
This guide aims to help users easily download and set up Oracle Database using Docker on Linux.
\section{Requirements : }

\subsection{Installing git}

\vspace{0.25cm}
\begin{tcolorbox}[title = Arch Based]
    \begin{verbatim}
sudo pacman -Sy git
    \end{verbatim}
\end{tcolorbox}

\vspace{0.5cm}
\begin{tcolorbox}[title = Debian And Ubuntu Based]
    \begin{verbatim}
sudo apt install git
    \end{verbatim}
\end{tcolorbox}

\vspace{0.25cm}
\subsection{Installing Docker}

\vspace{0.25cm}
\begin{tcolorbox}[title = Arch Based]
    \begin{verbatim}
sudo pacman -S docker
   \end{verbatim}
\end{tcolorbox}

\vspace{0.5cm}
\begin{tcolorbox}[title = Debian and Ubuntu Based]
    \begin{verbatim}
sudo apt install apt-transport-https ca-certificates curl software-properties-common
curl -fsSL https://download.docker.com/linux/ubuntu/gpg | sudo apt-key add -
sudo add-apt-repository "deb [arch=amd64] https://download.docker.com/linux/ubuntu \$(lsb\_release -cs) stable"
sudo apt update
sudo apt install docker-ce
 \end{verbatim}
\end{tcolorbox}

\vspace{0.25cm}
\subsection{Installing SqlDeveloper}
While installing sqldeveloper is optional , it allows for a better experience since sqlplus prompt terminal
has issues with arrow keys not functionning

\begin{tcolorbox}[title = Arch Based]
\textbf{\underline{Requirement:}} we need to install the yay package 

    \begin{verbatim}
git clone https://aur.archlinux.org/yay.git
cd yay/
makepkg -si
yay -S oracle-sqldeveloper
    \end{verbatim}
\end{tcolorbox}

\vspace{0.5cm}
\begin{tcolorbox}[title = Debian and Ubuntu Based]


\vspace{0.25cm}
\textbf{\underline{Requirements}}

\begin{itemize}
    \item wget (optional) : To download sqldeveloper rpm file in the terminal to download : 
        \begin{verbatim}
            sudo apt install wget
        \end{verbatim}
    \item alien : To convert the rpm into a deb file to download : 
        \begin{verbatim}
            sudo apt install alien
        \end{verbatim}
    \item java : To be able to run sqldeveloper we need a java jdk \textgreater=17
        \begin{verbatim}
            sudo apt install openjdk-21-jdk
        \end{verbatim}
\end{itemize}
\begin{verbatim}
sudo wget https://download.oracle.com/otn_software/java/sqldeveloper/sqldeveloper-24.3.0-284.2209.noarch.rpm
sudo alien -k sqldeveloper-24.3.0-284.2209.noarch.rpm
sudo dpkg -i sqldeveloper_24.3.0-284.2209_all.deb 
\end{verbatim}

\end{tcolorbox}

\vspace{0.5cm}
\begin{tcolorbox}[title = Running Sqldeveloper]
    \begin{verbatim}
sqldeveloper
    \end{verbatim}
\end{tcolorbox}

\vspace{0.25cm}
\section{Docker : }
\subsection{Start Docker :}

\vspace{0.25cm}
\begin{tcolorbox}[title = starting docker]
\begin{verbatim}
sudo systemctl start docker
sudo systemctl enable docker
\end{verbatim}
\end{tcolorbox}

\vspace{0.25cm}
\subsection{Building Oracle Container Image :}

\vspace{0.25cm}
\begin{tcolorbox}[title = Pulling oracle-xe]
Clone the docker oracle images into your local folder then navigate to the docker-images/OracleDatabase/
SingleInstance/dockerfiles/
folder use ls to print all the available versions and run the shell script and give it the version as parameter
\begin{verbatim}
git clone https://github.com/oracle/docker-images.git
cd docker-images/OracleDatabase/SingleInstance/dockerfiles/
ls
11.2.0.2  12.1.0.2  12.2.0.1  18.3.0  18.4.0  19.3.0  21.3.0  23.5.0  buildContainerImage.sh
./buildContainerImage.sh -v <version> -x
./buildContainerImage.sh -v 18.4.0 -x 
\end{verbatim}

\end{tcolorbox}

\vspace{0.5cm}
\begin{tcolorbox}[title = Note]
When running the buildContainerImage.sh make sure no docker container is running as it can lead to the script
not running 
\begin{verbatim}
    sudo docker ps 
    sudo docker stop <container_id>
\end{verbatim}
\end{tcolorbox}

\vspace{0.25cm}
\subsection{Starting Oracle Container And Executing It :}
We will create an instance of the oracle image we built by using the run command , and give it a host and container name
then we will fetch the sysdba password using the logs command after that we can either execute sql oracle in termianl
by running sqlplus inside the minimalistic docker terminal or use sqldeveloper for better experience 

\vspace{0.25cm}
\begin{tcolorbox}[title = Container Instance Creation]

\begin{verbatim}
sudo docker run -i -t -d --hostname <hostName> --name <containerName> -p 1521:1521
-v ~/Docker/shared/:/shared oracle/database:<version>-xe
sudo docker run -i -t -d --hostname ora18xe --name ora18xe -p 1521:1521
-v ~/Docker/shared/:/shared oracle/database:18.4.0-xe
\end{verbatim}

\end{tcolorbox}

\vspace{0.5cm}
\begin{tcolorbox}[title = Fetch Oracle Password]
\begin{verbatim}
sudo docker logs <containerName> -f
sudo docker logs ora18xe -f

ORACLE PASSWORD FOR SYS AND SYSTEM: 22cc7d22227cdd38

\end{verbatim}
\end{tcolorbox}
\vspace{1cm}
\begin{tcolorbox}[title= Connecting As SYSDBA and Creating PDB]
We will first connect to XE as sysdba to be able to create and mount the PDB note that all of that can be done
within sqldeveloper

\vspace{0.15cm}
\begin{verbatim}
sudo docker exec -i -t <containerName> /bin/bash
sudo docker exec -i -t ora18xe /bin/bash
\end{verbatim}
inside docker bash terminal :
\begin{verbatim}
sqlplus sys/<password>@//<hostName>:<port>/XE as sysdba
sqlplus sys/22cc7d22227cdd38@//ora18xe:1521/XE as sysdba
\end{verbatim}
inside sqlplus prompt :
\begin{verbatim}
create pluggable database <pdbName> admin user <userName> identified by <password> file_name_convert=('/opt
/oracle/oradata/XE/','/opt/oracle/oradata/XE/<pdbName>');
alter pluggable database <pdbName> open;
alter pluggable database <pdbName> save state;

create pluggable database tp_3bdd admin user rabah identified by 1234 file_name_convert=('/opt/oracle/oradata
/XE/','/opt/oracle/oradata/XE/tp_3bdd');
alter pluggable database tp_3bdd open;
alter pluggable database tp_3bdd save state;
exit;

sqlplus rabah/1234@//ora18xe:1521/tp_3bdd

\end{verbatim}
\end{tcolorbox}

\vspace{0.5cm}
\begin{tcolorbox}[title = Note]
We create the instance of the docker oracle image once then we can start and stop the instance as much as we want :
\begin{verbatim}
sudo docker start <containerName>
sudo docker stop <containerName>

sudo docker start ora18xe
sudo docker stop ora18xe
\end{verbatim}
\end{tcolorbox}
\end{document}
