\documentclass{article}
\usepackage[a4paper,left=1.5cm, right=1.5cm, top=1cm, bottom=2cm]{geometry}
\usepackage{tikz,tcolorbox}
\usepackage{amsmath}
\usepackage[table,xcdraw]{xcolor}
\usepackage{listings}
\usepackage{array,multirow} % For customizing tables
\usepackage{booktabs} % For better horizontal lines
\usepackage{makecell}
\setlength{\parindent}{0pt}

\lstset{
    language=SQL,                    % Set language to SQL
    basicstyle=\ttfamily\small,      % Font size and family for code
    keywordstyle=\color{blue}\bfseries, % Color for SQL keywords
    commentstyle=\color{gray},       % Color for comments
    stringstyle=\color{red},         % Color for strings
    numbers=left,                    % Show line numbers on the left
    numberstyle=\tiny\color{gray},   % Line number font and color
    stepnumber=1,                    % Line number step
    breaklines=true,                 % Wrap long lines
    frame=single,                    % Add a frame around code
    tabsize=2,                       % Set tab size
    showstringspaces=false           % Hide spaces in strings
}

\usepackage[colorlinks=true, linkcolor=blue, urlcolor=blue]{hyperref}

\newcommand{\ques}[1]{
  \section*{Question #1}
  \vspace{-0.5cm}
  \noindent\rule{\textwidth}{0.5pt}%
}

\newcommand{\tit}[1]{
\begin{center}
    \Large{\textbf{{#1}}}
\end{center}
}

\definecolor{commentgray}{HTML}{676160}
\definecolor{messagegreen}{HTML}{17B867}
\definecolor{myblue}{HTML}{10C2C4}

\tcbuselibrary{skins, breakable, theorems}


\newtcolorbox{prettyBox}[2]{
  enhanced,
  colback=white!90!#2,   % Background color based on the second parameter (color)
  colframe=#2!60!black,  % Frame color based on the second parameter (color)
  coltitle=white,        % Title color (white)
  fonttitle=\bfseries\Large,
  title=#1,              % Title from the first parameter
  boxrule=1mm,
  arc=0.5mm,
  drop shadow=#2!35!gray, % Drop shadow color based on the second parameter (color)
}



\begin{document}
\tit{TP N\(^{\boldsymbol{\circ}}\)\hspace{0.1cm}3}

\vspace{0.5cm}
\begin{enumerate}
    \item Lancer les deux machines virtuelles kali linux et ubuntu, et assurer la connectivité avec la commande \texttt{ping}.
    \item On va utiliser \texttt{iptables} dans ubuntu pour mettre des restrictions antérieures et extérieures du firewall pour définir des 
        conditions sur les paquets reçus et envoyés.
\begin{itemize}
    \item \texttt{sudo iptables -I INPUT -p icmp --icmp-type echo-request -j DROP} bloque les pings entrants vers Ubuntu
    \item \texttt{sudo iptables -L} affiche toutes les règles iptables actuellement configurées , que remarquez-vous?
    \item Essayez de ping ubuntu depuis kali. Que remarquez-vous? 
    \item \texttt{sudo iptables -I OUTPUT -p icmp --icmp-type echo-request -j DROP} bloque les pings sortants d'Ubuntu
    \item \texttt{sudo iptables -L} que remarquez-vous? 
    \item Essayez de ping kali depuis ubuntu. Que remarquez-vous? 
\end{itemize}
\end{enumerate}
\end{document}
