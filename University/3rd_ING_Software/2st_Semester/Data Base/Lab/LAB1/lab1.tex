\documentclass{article}
\usepackage[a4paper, paperwidth=25cm, paperheight=25.5cm, left=1.5cm, right=1.5cm, top=1cm, bottom=2cm]{geometry}
\usepackage{tikz,tcolorbox}
\usepackage{amsmath}
\usepackage[table,xcdraw]{xcolor}
\usepackage{listings}
\usepackage{array,multirow} % For customizing tables
\usepackage{booktabs} % For better horizontal lines
\usepackage{makecell}
\setlength{\parindent}{0pt}

\lstset{
    language=SQL,                    % Set language to SQL
    basicstyle=\ttfamily\small,      % Font size and family for code
    keywordstyle=\color{blue}\bfseries, % Color for SQL keywords
    commentstyle=\color{gray},       % Color for comments
    stringstyle=\color{red},         % Color for strings
    numbers=left,                    % Show line numbers on the left
    numberstyle=\tiny\color{gray},   % Line number font and color
    stepnumber=1,                    % Line number step
    breaklines=true,                 % Wrap long lines
    frame=single,                    % Add a frame around code
    tabsize=2,                       % Set tab size
    showstringspaces=false           % Hide spaces in strings
}


\newcommand{\ques}[1]{
  \section*{Question #1}
  \vspace{-0.5cm}
  \noindent\rule{\textwidth}{0.5pt}%
}

\newcommand{\tit}[1]{
\begin{center}
    \Large{\textbf{{#1}}}
\end{center}
}

\definecolor{commentgray}{HTML}{676160}
\definecolor{messagegreen}{HTML}{17B867}
\definecolor{myblue}{HTML}{10C2C4}

\tcbuselibrary{skins, breakable, theorems}


\newtcolorbox{prettyBox}[2]{
  enhanced,
  colback=white!90!#2,   % Background color based on the second parameter (color)
  colframe=#2!60!black,  % Frame color based on the second parameter (color)
  coltitle=white,        % Title color (white)
  fonttitle=\bfseries\Large,
  title=#1,              % Title from the first parameter
  boxrule=1mm,
  arc=0.5mm,
  drop shadow=#2!35!gray, % Drop shadow color based on the second parameter (color)
}



\begin{document}
\tit{TP N\(^{\boldsymbol{\circ}}\)\hspace{0.1cm}1}
\begin{enumerate}
\item Se connecter en tant que \texttt{SYSTEM}, créer un utilisateur \texttt{ING} et lui donner tous les droits,  
 puis se connecter en tant que \texttt{ING}.
\item Montrer l'utilisateur courrant avec 2 methodes.
\item Creation des tables :
    \begin{itemize}
        \item Usine(\underline{NU} , NomU , Ville)
        \item Produit(\underline{NP} , NomP , Couleur , Poids)
        \item Fournisseur(\underline{NF} , NomF , Statut , Ville , Email)
        \item Livraison (\underline{NP\(^*\) , NU\(^*\) , NF\(^*\)} , Quantite)
    \end{itemize}
 Le poids et la quantite sont strictement positive et l'email est unique et doit contenir un @

\item Afficher le nom des tables creees en utilisant les catalogues.
\item Afficher les contraintes de toutes les table en utilisant les catalogues.
\item Ajoutter l'attribut adresse a la tablr fournisseur, puis ajoutter la contrainte not null ,
    verifier avec description et les catalogue , supprimer l'attribut et verifier avec description.
\item Remplir les tables
\item Afficher tout les attribut de tout les usines.
\item Donnez les nom , num de tout les usines de sochaux.
\item Donnez les num des fournisseurs qui livre le produit num 3 a l'usine num 1.
\item Donnez les num ,nom des produits sans couleurs.
\item Donnez les nom distinct des usines par ordre croissant.
\item Donnez les num des usines dont le nom commence par C.
\item Donnez les num des produits dont le nom contient s ou S.
\item Donnez les noms des fournisseurs qui livrent le produit num 3 a l'usine num1. 
\item Donnez les noms , couleurs des prouduits livres par fournisseur num 2.
\item Donnez les num des fournisseurs qui livrent a l'usine 1 des produits de couleur rouge.
\item Donnez les noms des fournisseurs qui livrent a des usines de paris ou sochaux des produits de couleur rouge.
\item Donnez les nums des produits livres a une usine par un fournisseur de meme ville.
\item Donnez les nums des usines livres par au moins un fournisseur qui n'est pas de la meme ville.
\item Donnez les numéros des fournisseurs qui approvisionnent les usines 1 et 2 au moins.
\item Donnez les nums des usines qui utilisent des produits du fournisseur num 3 (pas forcement livre a l'usine).
\item Donnez les nums des usines qui s'approvisent du fournisseur num 3 seulement. 
\item Donnez les nums des usines qui recoit jamais des produits rouge de fournisseur parisien.
\item Donnez le nombre total de fournisseur.
\item Donnez le nombre de produits avec couleur.
\item Donnez la moyenne des poids des produits.
\item Donnez la somme des poids des produits vert.
\item Donnez la somme des poids des produits vert.
\item Donnez le plus petit poid des produits avec couleur.
\item Donnez les poids moyen de produits par couleur. 
\item Donnez les couleurs des produit dont le poids moyen \(\geq 10\). 
\item Donnez le nombre de produits livres par chaque fournisseur de paris.
\item Donnez numero des produits les plus leges.
\item Donnez le nombre de produits livres par chaque fournisseur(nom).
\item Donnez les num des usines qui utilise tout les produits du fournisseur num 3.
\item Donnez les num des produits qui sont livres a toute les usines parisiens.
\item Donnez pour chaque usine (num) nombre de produits livres (si jamais livre 0).
\item Donnez les nom des fournisseur avec le plus grand nombre de produit fournis.
\item Supprimer les produits de couleur noir et numero compris entre 1 et 3.
\item Changer la ville du fournisseur numero 3 a Lyon.
\item Connecter en tant que system et creer tablespace permanente livraison\_TBS et temporaire livraison\_TempTBS taille initial 100MO avec extension.
\item Affecter les nouvelles tablespace a l'utilisateur ING et verifie le changement en utilisant le catalogue dba\_users.
\item Connecter en tant que ING et cree les tables suivant :
\begin{itemize}
    \item TableErreurs(Adresse, Utilisateur , nomTable , nomContrainte)
    \item Fournisseur1(\underline{NF} , NomF , Statut , Ville , Email) 
\end{itemize}
Adresse est de type ROWID , l'email est unique et doit contenir un @
\item Desactiver la tablspace livraison\_TBS puis : 
    \begin{itemize}
        \item Afficher la table founisseur.
        \item Creation de la table test(a int , b char).
        \item Afficher la table fournisseur1.
    \end{itemize}
\item Activer la tablspace livraison\_TBS puis : 
    \begin{itemize}
        \item Afficher les nom , type et status des contraintes de la table fournisseur1.
        \item Inserer les elements suivant dans la table fournisseur1 :
            \begin{itemize}
 \item ( 1 ,'SaintGobain','Producteur','Paris','saint\_goubain@gmail.com') 
\item ( 2 ,'Au bon siege','Soustraitant','Limoges','au\_bon\_siegegmail.com') 
            \end{itemize}
        \item Desactiver la contrainte check de la table fournisseur1.
        \item Afficher les nom , type et status des contraintes de la table fournisseur1.
        \item Inserer dans la table fournisseur1 ( 2 ,'Au bon siege','Soustraitant','Limoges','au\_bon\_siegegmail.com').
  \end{itemize}
\newpage
\item Afficher les rowid , nf , email de fourniseur1 puis :
    \begin{itemize} 
    \item Afficher la table tableErreurs.
    \item Mettre les rowid qui ne respectent pas la contrainte check dans la table tableErreurs.
    \item Afficher la table tableErreurs.
    \item Supprimmer les elements de fournisseur1 qui ne respectent pas la contrainte check en utilisant la table tableErreurs.
    \item Activer contrainte check.
    \item Afficher fournisseur1.
    \end{itemize}
\end{enumerate}
\end{document}
