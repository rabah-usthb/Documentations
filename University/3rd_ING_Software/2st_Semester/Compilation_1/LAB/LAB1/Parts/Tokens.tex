\newpage

\section{Langage}
\subsection{Entités Lexicales Du Langage}
\begin{prettyBox}{Entités Lexicales}{mygreen}
\begin{itemize}
    \item \textbf{Identifiants} : les noms des variables , et fonctions.
    \item \textbf{Mots Clés} :
        \begin{itemize}
            \item \textbf{Type Prédéfinis} : 'Entier', 'Reel'.
            \item \textbf{Délimiteurs De Bloc} : 'Begin', 'end', 'Body', 'Declaration', 'MainProgram'.
        \end{itemize}
    \item \textbf{Constantes} :
         \begin{itemize}
             \item \textbf{Constante Entière}
             \item \textbf{Constante Chaine De Caractères}
         \end{itemize}
    \item \textbf{Séparateurs} :
        \begin{itemize}
            \item \textbf{Opération Arithmétique} : '/', '+', '-', '*'.
            \item \textbf{Parenthèses} : '(', ')'.
            \item \textbf{Accolades} : '\{' , '\}'.
            \item \textbf{Affectation} : ':='.
            \item \textbf{Deux Points} : ':'.
            \item \textbf{Virgule et Point-Virgule} : ',', ';'.
        \end{itemize}
\end{itemize}
\end{prettyBox}

\vspace{1cm}

\subsection{Commentaires}
\begin{prettyBox}{Commentaires}{mygreen}
    \begin{itemize}
        \item \textbf{Commentaire Simple} : commentaire écrit en 
            une ligne commençant par '\#\#'.
        \item \textbf{Commentaire Multi-ligne} : commentaire qui peut
            être écrit sur plusieurs lignes, commence par '\{ \texttt{--}' et finit par '\texttt{--} \}'.
    \end{itemize}
\end{prettyBox}


