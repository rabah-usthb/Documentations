\documentclass{article}
\usepackage[a4paper, paperwidth=25cm, paperheight=25.5cm, left=1.5cm, right=1.5cm, top=1cm, bottom=2cm]{geometry}
\usepackage{tikz,tcolorbox}
\usepackage{amsmath}
\usepackage[table,xcdraw]{xcolor}
\usepackage{listings}
\usepackage{array,multirow} % For customizing tables
\usepackage{booktabs} % For better horizontal lines
\usepackage{makecell}
\setlength{\parindent}{0pt}
\usepackage{algorithm}
\usepackage{algpseudocode}
\usepackage{amssymb}
\tcbuselibrary{skins, breakable, theorems}

\definecolor{myblue}{HTML}{10C2C4}
\definecolor{keyword}{HTML}{3C9ED1}
\definecolor{func}{HTML}{8D06E0}
\definecolor{op}{HTML}{E00606}
\definecolor{comment}{HTML}{6F7E7B}
\newtcolorbox{prettyBox}[2]{
  enhanced,
  colback=white!90!#2,   % Background color based on the second parameter (color)
  colframe=#2!60!black,  % Frame color based on the second parameter (color)
  coltitle=white,        % Title color (white)
  fonttitle=\bfseries\Large,
  title=#1,              % Title from the first parameter
  boxrule=1mm,
  arc=0.5mm,
  drop shadow=#2!35!gray, % Drop shadow color based on the second parameter (color)
}


\lstdefinestyle{simple}{
    basicstyle=\ttfamily,       % Use a monospaced font
    numbers=left,               % Enable line numbers on the left
    numberstyle=\tiny\color{gray}, % Make line numbers smaller and gray
    stepnumber=1,               % Number every line
    firstnumber=1,              % Start numbering from line 1
    frame=single,               % Start numbering from line 1
}

\newcommand{\affect}{\textcolor{op}{\(\gets\)}\hspace{0.1cm}}
\newcommand{\debut}{\State \textcolor{keyword}{\textbf{Begin}}
\vspace{0.5em}}
\newcommand{\fin}{
    \vspace{0.5em}
\State \textcolor{keyword}{\textbf{End}}}
\newcommand{\et}{\hspace{0.15cm}\textcolor{op}{AND}\hspace{0.15cm}}
\newcommand{\n}{\hspace{0.15cm}\textcolor{op}{NOT}\hspace{0.15cm}}
\newcommand{\com}[2]{\hspace{#1 cm} \textcolor{comment}{\texttt{ -- }{#2}}}
\newcommand{\lire}[1]{\State \textcolor{func}{Lire}(#1);}
\newcommand{\jmp}{\State \textcolor{func}{Saut\_Blanc}();}
\newcommand{\y}{\textcolor{op}{\(=\)}\hspace{0.1cm}}
\newcommand{\dn}{\textcolor{op}{\(\in\)}\hspace{0.1cm}}
\newcommand{\g}{\textcolor{op}{\(>\)}\hspace{0.1cm}}
\newcommand{\nin}{\textcolor{op}{\(\notin\)}\hspace{0.1cm}}
\newcommand{\pase}{\hspace{0.05cm}\textcolor{op}{!=}}
\newcommand{\e}{\textcolor{op}{=}}
\newcommand{\str}[1]{\hspace{0.1cm}\textcolor{green!50!black}{'#1'}\hspace{0.1cm}}
\newcommand{\ev}[1]{\textcolor{func}{Eval}({#1})}
\newcommand{\pr}[1]{\State \textcolor{func}{print}(\str{#1});}
\newcommand{\vide}{\(\hspace{0.1cm}\emptyset\hspace{0.1cm}\)}
\algrenewcommand\algorithmicif{\textcolor{keyword}{\textbf{if}}}
\algrenewcommand\algorithmicelse{\textcolor{keyword}{\textbf{else}}}
%\algrenewcommand\algorithmicprocedure{\textcolor{redPlot}{\textbf{procedure}}}

%\algrenewcommand\algorithmicfunction{\textcolor{redPlot}{\textbf{function}}}
\algrenewcommand\algorithmicend{\textcolor{keyword}{\textbf{end}}}
\algrenewcommand\algorithmicthen{\textcolor{keyword}{\textbf{then}}}

\algrenewcommand\algorithmicdo{\textcolor{keyword}{\textbf{do}}}
\algrenewcommand\algorithmicwhile{\textcolor{keyword}{\textbf{while}}}

%\algrenewcommand\algorithmicfor{\textcolor{redPlot}{\textbf{for}}}

\begin{document}
\exer{1}


\textbf{\underline{ex1.html}}
\vspace{0.1cm}

\lstinputlisting[style=htmlstyle]{Code/EX1/ex1.html}

\vspace{1.5cm}

\textbf{\underline{Output}}

\vspace{0.1cm}
\begin{center}
\setlength{\fboxrule}{2pt} % Set border thickness
\fbox{\includegraphics[height=0.4\textheight]{Code/EX1/ex1.PNG}}
\end{center}
 
\newpage

\begin{center}
    \Huge{\textbf{\underline{Exercise 2}}}
\end{center}

\vspace{0.5cm}
Give the grammar and automaton for each language
\begin{itemize}
    \item L\(_1\) = \{w \(\in\) 1.\{0,1\}\(^{*}\) / \texttt{|}w\texttt{|} \(\equiv\) 0 [3]  \}
    \item L\(_2\) = \{w \(\in\) \{a,b\}\(^{*}\) / w = a\(^{n}\) b\(^{m}\) , \(m > n \)  \}
    \item L\(_3\) = \{w \(\in\) \{a,b,c\}\(^{*}\) / w = a\(^{i}\) c b\(^{j}\) , i \(\equiv\) 0[2] , j \(\equiv\) 1[2]  \}
\end{itemize}

\vspace{1cm}

\textbf{\underline{\Large{Solution}} :}\\[0.4cm]
\textbf{\underline{L\(_1\)}}

    \vspace{0.3cm}

    \hspace{1.5cm}
\noindent
\begin{minipage}{0.4\textwidth}
    S \(\to\) 1A\\[0.1cm]
    A \(\to\) 1B \texttt{|} 0B\\[0.1cm]
    B \(\to\) 1C \texttt{|} 0C\\[0.1cm]
    C \(\to\) 1A \texttt{|} 0A \texttt{|} \(\epsilon\)
\end{minipage}%
\hspace{-1cm}
\begin{minipage}{0.5\textwidth}
    \centering
    \includegraphics[width=\textwidth]{Exercices/EX2/ex2.1.drawio.pdf}
\end{minipage}

\vspace{1.5cm}
\textbf{\underline{L\(_2\)}}

\vspace{0.3cm}
\noindent
\begin{center}
\begin{minipage}{0.4\textwidth}
    \centering
    \textbf{\underline{Solution\(_1\)}}\\[0.2cm]
    S \(\to\) aAb \texttt{|} bB\\[0.1cm]
    A \(\to\) aAb \texttt{|} bB\\[0.1cm]
    \hspace{-0.6cm}    B \(\to\) bB \texttt{|} \(\epsilon\)\\[0.1cm]
\end{minipage}%
\hspace{1cm} % Adjust this value for spacing
\begin{minipage}{0.4\textwidth}
    \centering
    \textbf{\underline{Solution\(_2\)}}\\[0.2cm]
    S \(\to\) aSb \texttt{|} Sb \texttt{|} b\\[0.1cm]
\end{minipage}
\end{center}

\vspace{0.3cm}

\begin{prettyBox}{No Automaton}{red}
This grammar is neither right-regular nor left-regular, it is a Type 2 grammar.  
Therefore, it does not have a corresponding automaton.
\end{prettyBox}

\newpage
\textbf{\underline{L\(_3\)}}

    \vspace{0.3cm}

    \hspace{1.5cm}
\noindent
\begin{minipage}{0.4\textwidth}
    S \(\to\) aaS \texttt{|} cA \\[0.1cm]
    A \(\to\) bbA \texttt{|} b 
\end{minipage}%
\hspace{-1cm}
\begin{minipage}{0.5\textwidth}
    \centering
    \includegraphics[width=\textwidth]{Exercices/EX2/ex2.3.drawio.pdf}
\end{minipage}

\vspace{2cm}

\textbf{\underline{Example with 'aacbb\#'}}
\begin{center} 
    \includegraphics[height=0.5\textheight]{Exercices/EX2/ex2.3.ex.drawio.pdf}
\end{center}

 
\newpage
\exer{3}


\textbf{\underline{ex3.html}}
\vspace{0.1cm}

\lstinputlisting[style=htmlstyle]{Code/EX3/ex3.html}

\vspace{1.5cm}

\textbf{\underline{Output}}

\vspace{0.1cm}
\begin{center}
\setlength{\fboxrule}{2pt} % Set border thickness
\fbox{\includegraphics[height=0.4\textheight]{Code/EX3/ex3.1.PNG}}
\end{center}

\begin{center}
\setlength{\fboxrule}{2pt} % Set border thickness
\fbox{\includegraphics[height=0.35\textheight]{Code/EX3/ex3.2.PNG}}
\end{center}

\vspace{1cm}

\begin{center}
\setlength{\fboxrule}{2pt} % Set border thickness
\fbox{\includegraphics[height=0.35\textheight]{Code/EX3/ex3.3.PNG}}
\end{center}

\vspace{1cm}

\begin{prettyBox}{Observation}{myblue}
    \begin{itemize}
        \item \(<\)p\(>\) est un element block.
        \item \(<\)strong\(>\) et \(<\)em\(>\) sont des elements inline.
    \end{itemize}
\end{prettyBox}
 
\newpage
\begin{center}
    \Huge{\textbf{\underline{Exercise 4}}}
\end{center}

\vspace{0.45cm}

\begin{center}
    \includegraphics[height=0.22\textheight]{Exercices/EX4/ex4.drawio.pdf}
\end{center}

\vspace{0.25cm}

\begin{prettyBox}{Explication}{myblue}
We used the \texttt{Constant} and \texttt{Variable} classes to create the 
\texttt{Monomial} class, which holds the name of the variable, the power, and the coefficient. 
The \texttt{Monomial} is a leaf element, and the \texttt{Polynomial} is the complex element 
that holds a list of \texttt{Monomial}.
\end{prettyBox}

 
\end{document}
