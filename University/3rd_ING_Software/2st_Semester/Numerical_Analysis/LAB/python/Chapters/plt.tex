\newpage
\begin{center}
    \Huge{\textbf{\underline{Chapter 3: MatPlotLib}}}
\end{center}

\setcounter{section}{0}


\section{Introduction}
\begin{prettyBox}{Introduction}{myblue}
Matplotlib is a powerful Python library for data visualization. It allows users to create both
static and interactive plots with ease. \\[0.1cm]
To use Matplotlib, we import the \texttt{pyplot} submodule, which provides a simple interface
for plotting. By convention, it is commonly aliased as \texttt{plt} to enhance readability and
simplify usage.
\end{prettyBox}

\vspace{0.5cm}


\section{Figure \& Axis}
\begin{prettyBox}{Difference}{myblue}
A \textbf{Figure} is the top-level container that holds everything, similar to a window or a canvas.  
An \textbf{Axis} is a plotting area inside a Figure where data is drawn.  

A single Figure can contain multiple Axes (subplots), allowing multiple plots within the same window.
\end{prettyBox}

\vspace{0.5cm}

\begin{center}
    \includegraphics[height = 0.4\textheight]{Chapters/PNG/figureAxis.png}
\end{center}


\section{Creating a Figure}
\begin{prettyBox}{Figure Creation}{myblue}
\begin{itemize}
    \item A figure is created using the \texttt{figure()} function from the \texttt{pyplot} submodule.
    \item By default, Matplotlib starts with an implicit figure.
    \item Each additional call to \texttt{figure()} creates a new figure, so the total number of figures is :
        \begin{center}
            \boxed{\texttt{1 + (number of calls to figure())}}
        \end{center}
\end{itemize}
\end{prettyBox}

\vspace{0.5cm}


\section{Drawing A Plot}
\begin{prettyBox}{Plot}{myblue}
\texttt{plot()} is a function from the \texttt{pyplot} submodule used to draw a graph on the axis of a figure.\\[0.05cm] 
\texttt{plot(x, y, linestyle='-', linewidth=1.5, marker=None, markersize=6.0, color=auto, mfc=color, mec=color, label=None, alpha=1)}

\begin{itemize}
    \item \textbf{x}: A NumPy array representing the x-coordinates of the data points.
    \item \textbf{y}: A NumPy array representing the y-coordinates of the data points.
    \item \textbf{linestyle}: Optional parameter. Default is \texttt{'-'}. Specifies the style of the connecting line.
    \item \textbf{linewidth}: Optional parameter. Default is \texttt{1.5}. Controls the thickness of the line.
    \item \textbf{marker}: Optional parameter. Default is \texttt{None}. Defines the shape of markers placed at data points.
    \item \textbf{markersize}: Optional parameter. Default is \texttt{6.0}. Sets the size of the markers.
    \item \textbf{color}: Optional parameter. Default is \texttt{auto}. If not set, Matplotlib assigns a color automatically. Accepts color names, hex codes, and RGB(A) tuples.
    \item \textbf{mfc} (Marker Face Color): Optional parameter. Default is the same as \texttt{color}. Defines the fill color of the marker.
    \item \textbf{mec} (Marker Edge Color): Optional parameter. Default is the same as \texttt{color}. Defines the outline color of the marker.
    \item \textbf{label}: Optional parameter. Default is \texttt{None}. Specifies the legend label for the plot. Supports raw strings and LaTeX expressions using \texttt{\$ \$}.
    \item \textbf{alpha}: Optional parameter. Default is \texttt{1}. Controls the transparency of both lines and markers (1 = fully opaque, 0 = fully transparent).
\end{itemize}
\end{prettyBox}

\newpage

\begin{center}
    \includegraphics[height=0.6\textheight]{Chapters/PDF/marker.drawio.pdf}
\end{center}

\begin{center}
    \includegraphics[height=0.35\textheight]{Chapters/PDF/linestyle.drawio.pdf}
\end{center}

\newpage

\section{Drawing a Single Point}
\begin{prettyBox}{Single Point}{myblue}
We can use either the \texttt{plot()} or \texttt{scatter()} function:
\begin{itemize}
    \item \texttt{plot(x, y, color=auto,linestyle='', marker, mfc=color, mec=color, alpha=1)}
    \item \texttt{scatter(x, y, color=auto, marker, c=color, edgecolors=color, alpha=1)}  
          Here, \texttt{c} is equivalent to \texttt{mfc}, and \texttt{edgecolors} is equivalent to \texttt{mec}.
\end{itemize}
\end{prettyBox}

\vspace{0.5cm}

\section{Graph Customization \& Display}
\begin{prettyBox}{Customization}{myblue}
\begin{itemize}
    \item \texttt{legend()}: Displays the labels of plotted elements (e.g., lines, scatter points) as a legend for the axis.
    \item \texttt{grid(visible=True)}: Toggles the grid on the axis. By default, the grid is off, but calling the function without arguments is equivalent to setting it to \texttt{True}.
    \item \texttt{title(label)}: Sets the title for the current axis.
    \item \texttt{suptitle(label)}: Sets the title for the current figure.
    \item \texttt{ylabel(label)}: Labels the y-axis.
    \item \texttt{xlabel(label)}: Labels the x-axis.
    \item \texttt{show()}: Renders and displays all created figures along with their content.
    \item \texttt{plt.savefig(fname)}: saves the figure in the given path fname , if fname doesn't have a file extension it will save it as png
\end{itemize}
\end{prettyBox}

\vspace{0.5cm}

\section{Subplot}
\begin{prettyBox}{Subplot}{myblue}
The \texttt{subplot()} function allows us to create multiple axes within the same figure by defining a grid layout.  

\texttt{subplot(nrows, ncols, index)}

\begin{itemize}
    \item \textbf{nrows}: Number of rows in the subplot grid.
    \item \textbf{ncols}: Number of columns in the subplot grid.
    \item \textbf{index}: The position of the subplot, starting from 1 (left to right, top to bottom).
\end{itemize}

Each call to \texttt{subplot()} activates a different subplot within the figure.  
\end{prettyBox}

\newpage
\textbf{\underline{Example}}\\[0.1cm]
\lstinputlisting[style=pythonstyle]{Chapters/Code/PLT/figure.py}

\newpage
\textbf{\underline{fig1}}\\[0.1cm]
\begin{center}
    \includegraphics[height=0.35\textheight]{Chapters/Code/PLT/fig1.pdf}
\end{center}


\textbf{\underline{fig 2}}\\[0.1cm]
\begin{center}
    \includegraphics[height=0.35\textheight]{Chapters/Code/PLT/fig2.pdf}
\end{center}


