\begin{center}
    \Huge{\textbf{\underline{Chapter 1: Introduction}}}
\end{center}

\vspace{0.35cm}


\section{Numerical Analysis}
\begin{prettyBox}{Definition}{mygreen}
Numerical analysis is \textbf{a field of study} focused on developing and analyzing methods for
solving mathematical problems that are difficult or impossible to solve exactly using traditional
algebraic techniques. It involves studying the \textbf{accuracy, stability, and efficiency} of numerical 
methods.
\end{prettyBox}

\vspace{0.35cm}
\section{Numerical Algorithm (Numerical Method)}
\begin{prettyBox}{Numerical Algorithm}{mygreen}
A Numerical Algorithms are practical techniques used to perform \textbf{numerical computations}. providing 
an approximate solution and is used to solve problems involving large amounts of data.
The algorithm is based on a well-defined iterative sequence, starting with
an initial solution that progressively converges toward the desired result with each iteration.

\[
\begin{cases}
    \hspace{0.1cm}(U_n) & \hspace{-0.1cm}n \geq 0 \\[0.15cm]
    \hspace{0.1cm}S_0 & \hspace{-0.3cm} \text{Initial Solution (Starting Point)}
\end{cases}
\]
\end{prettyBox}
\vspace{0.35cm}

\section{Convergence Speed (Order of Convergence)}
\begin{prettyBox}{Convergence Speed}{mygreen}
The number of iterations required to find the solution we are looking for:
\begin{itemize}
    \item Linear Order: 1 (slow)
    \item Quadratic Order: 2 (faster)
    \item \(>>\) 2 (very fast)
\end{itemize}
\end{prettyBox}

\vspace{0.35cm}

\section{Interpolation}
\begin{prettyBox}{Interpolation}{mygreen}
Estimates the value between two known points of a function allowing for a smoother representation of the function's behavior.
\end{prettyBox}

\vspace{0.35cm}

\section{Approximation}
\begin{prettyBox}{Approximation}{mygreen}
Approximates the formula of a function from a set of values, with the objective of finding a simpler function that represents the general trend of the data, even if it doesn't pass through every point exactly.
\end{prettyBox}

\vspace{0.35cm}

\section{Error}
\begin{prettyBox}{Error}{mygreen}
An error represents the difference between the actual solution and the computed result. It indicates how far we are from the true solution. There are two cases:
\begin{itemize}
    \item \textbf{Evaluation}: We know the exact solution, so we can directly calculate the error:
        \[   \boxed{E_r = |\overline{x} - x_{\text{app}}|} \]
    \item \textbf{Estimation}: We don't know the exact solution, so we only have an estimate of the error, based on the output of the algorithm:
        \[      \boxed{E_r = |\overline{x} - x_{\text{app}}| \leq \text{Algo}} \]
\end{itemize}

Where:
\begin{itemize}
    \item \(E_r\) : The error value.
    \item \(\overline{x}\) : The exact solution.
    \item \(x_{\text{app}}\) : The approximate solution.
    \item Algo : The error value found by the algorithm.
\end{itemize}
\end{prettyBox}

\vspace{0.35cm}

\section{Optimization}
\begin{prettyBox}{Optimization}{mygreen}
Optimization in numerical algorithms refers to two things:
\begin{itemize}
    \item \textbf{Error}: We aim to minimize the error in order to achieve the most accurate approximate solution.
    \item \textbf{Convergence Speed}: The higher the order of convergence, the less time the algorithm will take to converge to the solution we are looking for.
\end{itemize}
\end{prettyBox}

