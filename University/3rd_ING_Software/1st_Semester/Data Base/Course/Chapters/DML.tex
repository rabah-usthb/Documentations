\newpage 
\null 
\vspace{0.15cm}

\begin{center} 
\Huge{\textbf{\underline{Chapter 5: DML Commands}}}
\end{center}

\vspace{0.25cm}

\setcounter{section}{0}

\vspace{0.35cm}

\section{Insert}

\begin{prettyBox}{Insert}{myblue}
To insert rows into a table, we use the \textcolor{blue}{INSERT} command. We can insert one row at a time or multiple rows at once from the same or different tables using the \textcolor{blue}{ALL} keyword.
\end{prettyBox}

\subsection*{\underline{\textbf{Syntax}}}

\subsection*{\textbf{\underline{Insert Once}}}

\lstinputlisting{SQL/syntax/DML/insert.sql}

\subsection*{\textbf{\underline{Insert In Multiple Tables}}}

\lstinputlisting{SQL/syntax/DML/insertAll.sql}


\subsection*{\underline{\textbf{Example}}}

\subsection*{\underline{\textbf{Tables Definition}}}

\subsection*{\underline{\textbf{Student Table}}}
\begin{center}

 \renewcommand{\arraystretch}{1.5}
    \begin{tabular}{|c|c|c|}
        \hline 
        Column Name & Data Type & Constraints\\
        \hline
        id & number & primary key\\
        \hline
        lastname & varchar2(50) & not null\\
        \hline
        firstname & varchar2(50) & not null\\
        \hline
        id\_section & number & \makecell{foreign key section(id\_section)\\delete on cascade}\\
        \hline
        grade & number(4,2) & \makecell{default 0\\check between 0 and 20}\\
        \hline
        dob & date & not null\\
        \hline
    \end{tabular}
\end{center}

\subsection*{\underline{\textbf{Section Table}}}

\begin{center}

 \renewcommand{\arraystretch}{1.5}
    \begin{tabular}{|c|c|c|}
        \hline 
        Column Name & Data Type & Constraints\\
        \hline
        id\_section & number & primary key\\
        \hline
        name & varchar2(5) & \makecell{not null\\check in ('A','B','C','D',1,2,3,4)}\\
        \hline
    \end{tabular}
\end{center}

\subsection*{\textbf{\underline{Insert Once}}}

\lstinputlisting{SQL/examples/DML/insertOnce.sql}

\subsection*{\underline{\textbf{Tables After Insert}}}

\subsection*{\underline{\textbf{Section Table}}}

\begin{center}

 \renewcommand{\arraystretch}{1.5}
    \begin{tabular}{|c|c|}
        \hline 
        id\_section & name \\
        \hline
        1 & 'A'\\
        \hline
        2 & 'B'\\
        \hline
    \end{tabular}
\end{center}


\subsection*{\underline{\textbf{Student Table}}}
\begin{center}

 \renewcommand{\arraystretch}{1.5}
    \begin{tabular}{|c|c|c|c|c|c|}
        \hline
        id & lastname & firstname & id\_section & grade & dob\\
        \hline
        1 & 'chabane' & 'rabah' & 2 & 11.80 & 2002-03-19 \\
        \hline
        2 & adem & lyna & 1 & 13.24 & 2004-07-19\\
        \hline
        3 & chaouche & mohamed & null & 12.13 & 2004-02-20 \\
        \hline
    \end{tabular}
\end{center}


\newpage


\subsection*{\textbf{\underline{Insert In Multiple Tables}}}

\lstinputlisting{SQL/examples/DML/insertAll.sql}

\subsection*{\underline{\textbf{Tables After Insert}}}

\subsection*{\underline{\textbf{Section Table}}}

\begin{center}

 \renewcommand{\arraystretch}{1.5}
    \begin{tabular}{|c|c|}
        \hline 
        id\_section & name \\
        \hline
        1 & 'A'\\
        \hline
        2 & 'B'\\
        \hline
        3 & 'C'\\
        \hline
        4 & 'D'\\
        \hline
    \end{tabular}
\end{center}

\subsection*{\underline{\textbf{Student Table}}}
\begin{center}

 \renewcommand{\arraystretch}{1.5}
    \begin{tabular}{|c|c|c|c|c|c|}
        \hline
        id & lastname & firstname & id\_section & grade & dob\\
        \hline
        1 & 'chabane' & 'rabah' & 2 & 11.80 & 2002-03-19 \\
        \hline
        2 & adem & lyna & 1 & 13.24 & 2004-07-19\\
        \hline
        3 & chaouche & mohamed & null & 12.13 & 2004-02-20 \\
        \hline    
        4 & bakhti & sohaib & 4 & 10.51 & 2000-10-01 \\
        \hline    
        5 & ibtissame & ahlem & 3 & 14.83 & 2001-08-21 \\
        \hline 
        6 & yacine & salem & null & 9.80 & 2000-11-06 \\
        \hline
    \end{tabular}
\end{center}


\vspace{0.25cm}

\begin{prettyBox}{Note}{red}
We don't have to precise columns names when inserting , it's optional it just makes the code more readable
\end{prettyBox}

\vspace{0.25cm}
\section{Update}
\begin{prettyBox}{Update}{myblue}
To change the values of some rows in a table, we use the \textcolor{blue}{UPDATE} command, accompanied by the \textcolor{blue}{WHERE} clause to update only specific rows.
\end{prettyBox}

\subsection*{\underline{\textbf{Syntax}}}

\lstinputlisting{SQL/syntax/DML/update.sql}
\section{Delete} 

\begin{prettyBox}{Delete}{myblue}
To delete rows from` a table, we use the \textcolor{blue}{DELETE} command, accompanied by the \textcolor{blue}{WHERE} clause to delete specific rows. Although it is possible to delete all rows using \textcolor{blue}{DELETE}, it is better to use \textcolor{blue}{TRUNCATE} for that purpose due to performance considerations.
\end{prettyBox}

\subsection*{\underline{\textbf{Syntax}}}

\lstinputlisting{SQL/syntax/DML/delete.sql}
