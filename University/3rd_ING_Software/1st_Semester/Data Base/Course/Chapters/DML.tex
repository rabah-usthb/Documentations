
\section{DML Commands}

\subsection{Insert}

\begin{prettyBox}{Insert}{myblue}
To insert rows into a table, we use the \textcolor{blue}{INSERT} command. We can insert one row at a time or multiple rows at once from the same or different tables using the \textcolor{blue}{ALL} keyword.
\end{prettyBox}

\subsubsection*{\underline{\textbf{Syntax}}}

\subsubsection*{\textbf{Insert Once}}

\lstinputlisting{SQL/syntax/DML/insert.sql}

\subsubsection*{\textbf{Insert In Multiple Tables}}

\lstinputlisting{SQL/syntax/DML/insertAll.sql}

\begin{prettyBox}{Note}{red}
We don't have to precise columns names when inserting , it's optional it just makes the code more readable
\end{prettyBox}

\subsection{Update}
\begin{prettyBox}{Update}{myblue}
To change the values of some rows in a table, we use the \textcolor{blue}{UPDATE} command, accompanied by the \textcolor{blue}{WHERE} clause to update only specific rows.
\end{prettyBox}

\subsubsection*{\underline{\textbf{Syntax}}}

\lstinputlisting{SQL/syntax/DML/update.sql}
\subsection{Delete} 

\begin{prettyBox}{Delete}{myblue}
To delete rows from a table, we use the \textcolor{blue}{DELETE} command, accompanied by the \textcolor{blue}{WHERE} clause to delete specific rows. Although it is possible to delete all rows using \textcolor{blue}{DELETE}, it is better to use \textcolor{blue}{TRUNCATE} for that purpose due to performance considerations.
\end{prettyBox}

\subsubsection*{\underline{\textbf{Syntax}}}

\lstinputlisting{SQL/syntax/DML/delete.sql}
