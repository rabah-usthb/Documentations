\section{View}
\subsection{What Are Views ?}

\begin{prettyBox}{View}{myblue}
Views are virtual tables that helps restricts what user can see and to tables , it also
can also provide a boost in performance when it comes to joins there are 2 types of views
external and marterial
\end{prettyBox}

\begin{prettyBox}{Note}{red}
Any change that happens in a view will reflect on the source table
\end{prettyBox}


\subsubsection*{\underline{Syntax}}

\lstinputlisting{SQL/syntax/Views/createView.sql}


\subsection{External}

\begin{prettyBox}{External View}{myblue}
External view have same performance as normal table , they just make it easier
so we don't have to retype the select query each time we want to get the information
from a table , we can insert on an external view , for a view to be external it must include :
\begin{itemize}
    \item Primary Key of the table
    \item not have any kind of joins
    \item dont have any aggregation functions 
    \item all not null attributes
\end{itemize}
\end{prettyBox}

\subsection{Material}

\begin{prettyBox}{Material View}{myblue}
Material view have better performance , they server as read only tables , we can't insert
into them , a view is material any of the condition are met :
\begin{itemize}
    \item doesn't include primary key
    \item doesn't include all not null attributes
    \item has joins
    \item has aggregation functions
\end{itemize}
\end{prettyBox}

\subsubsection*{\underline{Example}}

