\newpage 
\null 
\vspace{0.15cm}

\begin{center} 
\Huge{\textbf{\underline{Chapter 4: Database Administration}}}
\end{center}

\vspace{0.25cm}

\setcounter{section}{0}

\vspace{1cm}

\section{Introduction}

\begin{prettyBox}{Introduction}{myblue} 
Oracle SQL is known for its robust security features and management of users' 
rights and resources. In this section, we will explore how to manage rights 
and resources.
\end{prettyBox}

\vspace{0.35cm}

\section{Users}

\subsection{Sysdba (Root)} 
\begin{prettyBox}{Sysdba User}{myblue} 
The Sysdba user is the root user. He has access to all PDBs, has access to
certain commands and tables that only he can use, and obviously, he has all
the privileges.
\end{prettyBox}

\vspace{0.25cm}
\subsection{System}

\begin{prettyBox}{System User}{myblue} 
The system user has all rights, but he lacks DBA privileges.
\end{prettyBox}

\vspace{0.25cm}
\subsection{User Creation}
\begin{prettyBox}{User Creation}{myblue}
To create a user, we first need a user with the appropriate rights to do so 
(such as the system user or DBA).
Then, we simply use the following syntax:
\end{prettyBox}

\vspace{0.15cm}
\subsection*{\underline{Syntax}}

\lstinputlisting{SQL/syntax/Admin/createUser.sql}

\vspace{0.15cm}

\subsection*{\underline{Example}}
\lstinputlisting{SQL/examples/Admin/createUser.sql}

\vspace{0.25cm}
\subsection{User Deletion} 
\begin{prettyBox}{User Deletion}{myblue}
To delete a user, we use the \textcolor{blue}{DROP} command as follows: 
\end{prettyBox}

\vspace{0.15cm}
\subsection*{\underline{Syntax}}
\lstinputlisting{SQL/syntax/Admin/dropUser.sql}

\vspace{0.15cm}
\subsection*{\underline{Example}}
\lstinputlisting{SQL/examples/Admin/dropUser.sql}


\vspace{0.15cm}
\begin{prettyBox}{Note}{red}
To delete a DBA user, you must be a DBA yourself. However,
to delete a non-DBA user, you only need the right to drop users. 
\end{prettyBox}


\vspace{0.35cm}

\section{Rights Management} 
\begin{prettyBox}{Rights Management}{myblue}
We can manage users' privileges by granting and revoking them. 
There are 3 types of privileges:

\begin{itemize} 
\item \textbf{System Privileges:} Generalized rights (create session, select any table ,...etc) 
\item \textbf{Object Privileges:} Rights to perform actions on specific objects (drop a certain table, select a certain view ,...etc) 
\item \textbf{Modifier Privileges:} Rights to grant or revoke privileges. 
\end{itemize} 

\end{prettyBox}

\subsection{Granting Rights} 
\begin{prettyBox}{Granting Rights}{myblue} 
The user must have the right to grant rights. We can: 
\begin{itemize} 
\item Give the right to execute a command (general system privilege). 
\item Grant access to a specific object (object privilege). 
\item Grant the right to grant/revoke rights to others (with admin/grant option).
\end{itemize} 
\end{prettyBox}

\vspace{0.15cm}
\subsection*{\underline{Syntax}} 
\lstinputlisting{SQL/syntax/Admin/grant.sql}

\vspace{0.15cm}
\subsection*{\underline{Example}} 

\lstinputlisting[basicstyle=\ttfamily\scriptsize]{SQL/examples/Admin/grant.sql}

\vspace{0.25cm}
\subsection{Revoking Rights} 
\begin{prettyBox}{Revoking Rights}{myblue} 
The user must have the right to revoke rights. When revoking a right, we not 
only remove the privilege itself from the user but also the right to
grant/revoke that privilege.
\end{prettyBox}

\vspace{0.15cm}
\subsection*{\underline{Syntax}}
\lstinputlisting{SQL/syntax/Admin/revoke.sql}

\vspace{0.15cm}
\subsection*{\underline{Example}}
\lstinputlisting[basicstyle=\ttfamily\footnotesize]{SQL/examples/Admin/revoke.sql}

\vspace{0.35cm}
\subsection{Roles} 
\begin{prettyBox}{Roles}{myblue}
A role represents a set of privileges, that we can affect to a user.
\end{prettyBox}

\vspace{0.25cm}

\subsubsection{Creating Roles} 
\begin{prettyBox}{Creating Roles}{myblue}
We need to first create the role, then grant privileges to it to populate
the role. We can also remove privileges from the role using 
the \textcolor{blue}{REVOKE} command.
\end{prettyBox}

\vspace{0.15cm}
\subsection*{\underline{Syntax}}
\lstinputlisting{SQL/syntax/Admin/createRole.sql}

\vspace{0.15cm}
\subsection*{\underline{Example}}
\lstinputlisting[basicstyle=\ttfamily\footnotesize]{SQL/examples/Admin/createRole.sql}

\vspace{0.25cm}
\subsubsection{Dropping Roles} 
\begin{prettyBox}{Dropping Roles}{myblue}
The user must have the right to delete a role. We use the 
\textcolor{blue}{DROP} command to do so.
\end{prettyBox}

\vspace{0.15cm}
\subsection*{\underline{Syntax}}
\lstinputlisting{SQL/syntax/Admin/dropRole.sql}

\vspace{0.15cm}
\subsection*{\underline{Example}}
\lstinputlisting{SQL/examples/Admin/dropRole.sql}

\vspace{0.15cm}
\begin{prettyBox}{Note}{red} 
If we drop a role that is still in use by other users, they will automatically
lose the privileges associated with that role.
\end{prettyBox}

\vspace{0.25cm}
\subsubsection{Granting Roles to Users}
\begin{prettyBox}{Granting Roles}{myblue} The user must have the right to 
grant roles to others. We can:
\begin{itemize} 
\item Grant the privileges of a role to users. 
\item Grant the right to grant/revoke the role to others (with admin option).
\end{itemize} 
\end{prettyBox}

\vspace{0.15cm}
\subsection*{\underline{Syntax}} 
\lstinputlisting{SQL/syntax/Admin/grantRole.sql}

\vspace{0.15cm}
\subsection*{\underline{Example}} 
\lstinputlisting{SQL/examples/Admin/grantRole.sql}


\vspace{0.25cm}
\subsubsection{Revoking Roles from Users} 
\begin{prettyBox}{Revoking Roles}{myblue}
The user needs the right to revoke roles from others. By revoking a role from
a user, we remove all the privileges of that role, as well as the right to
grant or revoke that role.
\end{prettyBox}

\vspace{0.15cm}
\subsection*{\underline{Syntax}} 
\lstinputlisting{SQL/syntax/Admin/revokeRole.sql}

\vspace{0.15cm}
\subsection*{\underline{Example}} 
\lstinputlisting{SQL/examples/Admin/revokeRole.sql}

\vspace{0.35cm}
\section{Resource Management}

\subsection{Profiles} 
\begin{prettyBox}{Profiles}{myblue} To manage the resources used by users,
we use profiles, which represent a set of limitations that we can assign
to users.
\end{prettyBox}

\vspace{0.25cm}
\subsection{Profile Creation}
\begin{prettyBox}{Profile Creation}{myblue} The user needs the right to create
profiles. We can limit many resources, such as the number of simultaneous 
sessions for a user, idle time per session, and more. 
\end{prettyBox}

\vspace{0.15cm}
\subsection*{\underline{Syntax}} 
\lstinputlisting{SQL/syntax/Admin/createProfile.sql}

\vspace{0.15cm}
\subsection*{\underline{Example}} 
\lstinputlisting[basicstyle=\ttfamily\footnotesize]{SQL/examples/Admin/createProfile.sql}


\vspace{0.25cm}
\subsection{Dropping Profiles}
\begin{prettyBox}{Dropping Profiles}{myblue} The user needs the right to 
drop profiles. We use the \textcolor{blue}{DROP} command to do so.
\end{prettyBox}

\vspace{0.15cm}
\subsection*{\underline{Syntax}}
\lstinputlisting{SQL/syntax/Admin/dropProfile.sql}

\vspace{0.15cm}
\subsection*{\underline{Example}}
\lstinputlisting{SQL/examples/Admin/dropProfile.sql}


\vspace{0.25cm}
\subsection{Assigning Profiles} 
\begin{prettyBox}{Assigning Profiles}{myblue} 
The user needs the right to alter other users. We use the 
\textcolor{blue}{ALTER} command to assign profiles.
\end{prettyBox}

\vspace{0.15cm}
\subsection*{\underline{Syntax}} 
\lstinputlisting{SQL/syntax/Admin/profile.sql}

\vspace{0.15cm}
\subsection*{\underline{Example}} 
\lstinputlisting{SQL/examples/Admin/profile.sql}



\vspace{0.25cm}
\subsection{Unassigning Profiles} 
\begin{prettyBox}{Unassigning Profiles}{myblue}
The user needs the right to alter other users. We use the
\textcolor{blue}{ALTER} command to assign the default profile to a user. 
\end{prettyBox}

\vspace{0.15cm}
\subsection*{\underline{Syntax}} 
\lstinputlisting{SQL/syntax/Admin/def.sql}

\vspace{0.15cm}
\subsection*{\underline{Example}} 
\lstinputlisting{SQL/examples/Admin/def.sql}
