\section{Data Types}


\begin{prettyBox}{Data Type}{myblue}
Data types enforce integrity constraints on columns in a database. There are many data types
available, but we will focus on the most commonly used ones.
\end{prettyBox}

\subsection{Number}

\begin{prettyBox}{Number}{myblue}
Number is a generic data type that allows for  numerical value : real and integer numbers 
and we can decide the floating point and size :
\begin{itemize}
    \item Number : stores large values of integer and decimal numbers 
    \item Number(p) : represent an integer number where p is max number of digits , p \(\in\)[1,38]
    \item Number(p,s) : p represent number of total digit p \(\in\)[1,38] , s: represent scale
number of digits after decimal point , s \(\in\)[0,p-1]
\end{itemize}

it has some sub types like Integer which is \(\Leftrightarrow\) Number(38)
\end{prettyBox}
\subsection*{\underline{Examples :}}

\begin{center}
\begin{tabular}{|c|c|c|}
    \hline
    Definition & Input & Stored As \\
    \hline
    NUMBER  & \makecell{124.56\\-99999\\44343} & \makecell{12.56\\-99999\\44343} \\
    \hline
    NUMBER(5) & \makecell{17.5\\123456\\44300} & \makecell{18\\Error\\44300} \\
    \hline
    NUMBER(3) & \makecell{99.3\\-677.9\\5432} & \makecell{99\\678\\Error}\\
    \hline
    INTEGER  & \makecell{16.89\\-234532\\13.1} & \makecell{17\\-234532\\13} \\
    \hline
    INTEGER(2)  &   \makecell{234.9\\10.4\\-20} & \makecell{Error\\10\\-20}    \\
    \hline
    INTEGER(4)  &  \makecell{1240\\932.82\\-32330} & \makecell{1240\\933\\Error}  \\
    \hline
    NUMBER(6,2)  &  \makecell{34670.56\\-9890.98\\23.232} & \makecell{Error\\-9890.98\\23.23}  \\
    \hline
    NUMBER(5,3) &  \makecell{24.1562\\99\\343.77} & \makecell{24.156\\99.000\\Error} \\
    \hline
\end{tabular}
\end{center}

\begin{prettyBox}{Note}{red}
s can be \(>\) p , i just didn't want to include that case as it can be confusing and is rarely used
\end{prettyBox}



\subsection{Date}

\begin{prettyBox}{Date}{myblue}
Date is a data type that stores both the date and the time it accept a wide range of format
and has many function  
\end{prettyBox}


\subsubsection{Format}
\begin{center}
 \renewcommand{\arraystretch}{1.5}
    \begin{tabular}{|c|c|}
        \hline
        Format & Example\\
        \hline
        YYYY-MM-DD & 2024-12-01 \\
        \hline
        DD-MON-YYYY & 30-NOV-2022\\
        \hline
        YYYY/MM/DD & 2000/04/19\\
        \hline
        HH24:MI:SS & 14:34:21\\
        \hline
        HH12:MI:SS AM/PM & 07:45:15 AM\\
        \hline
        YYYY-MM-DD HH24:MI:SS & 2021-01-30 22:50:10\\
        \hline 
        YYYY-MM-DD HH12:MI:SS AM/PM & 2014-03-19 1:21:45 PM\\
        \hline
    \end{tabular}
\end{center}


\subsubsection{Function}

\begin{center}

 \renewcommand{\arraystretch}{1.5}
    \begin{tabular}{|c|l|}
    \hline
    Fonction & \makecell{Definition}\\
    \hline
    SYSDATE & \makecell[l]{returns the current date and time of the machine running the oracle date base(server)\\ in the format YYYY-MM-DD HH24:MI:SS} \\
    \hline
    CURRENT\_DATE & \makecell[l]{returns the current date and time of the user machine connecting to the oracle date base\\ in the format YYYY-MM-DD HH24:MI:SS} \\
    \hline
    TO\_DATE(string , format) & \makecell[l]{converts a string into date in the given format}\\
    \hline
    TO\_CHAR(date , format) & \makecell[l]{converts a date into a formatted (given format) string}\\
    \hline
    ADD\_MONTHS(date , n) & \makecell[l]{returns a date which it adds/substracts n months to the given date}\\
    \hline
    MONTHS\_BETWEEN (date1 , date2) & \makecell[l]{returns an integer number that represents number of months between date1 and date2}\\
    \hline
    NEXT\_DAY(date , day\_of\_week) & \makecell[l]{returns date of the next given day string ('SUNDAY', 'MONDAY'...etc) starting to\\ search from the given date}\\
    \hline
    EXTRACT(field FROM date) & \makecell[l]{returns an integer number that represents the given field (MONTH , YEAR, DAY ,\\ HOUR , MINUTE , SECOND , WEEK ...etc) from given date} \\
    \hline
\end{tabular}
\end{center}

\begin{prettyBox}{Note}{red}
When inserting a date in a table using TO\_DATE it doesn't matter which format we use , we can use any format we
want and the same thing is valid when needing to print a date usin TO\_CHAR , because oracle stores the data object not
the format in insert
\end{prettyBox}

\subsection{Char}

\begin{prettyBox}{Char}{myblue}
Char(len) stores string of len size , if the inputed string is smaller than the definition
oracle will pad it with space char , len \(\in\)[1,2000]
\end{prettyBox}

\subsection{VARCHAR2}

\begin{prettyBox}{Varchar2}{myblue}
Varchar2(len) stores string of len size , if the inputed string is smaller than the definition
oracle will store it without any padding , in older version len \(\in\)[1,4000] but in more recent
version len \(\in\)[1,32767] 
\end{prettyBox}



\subsubsection{Function}

\begin{center}
 
 \renewcommand{\arraystretch}{1.5}
    \begin{tabular}{|c|c|}
        \hline
        Fonction & Definition\\
        \hline
        LENGTH(string) & \makecell[l]{returns intger : length of given string}\\
        \hline
        TRIM(string) & \makecell[l]{returns string : removes all leading/trailling spaces}\\
        \hline
        TRIM(char FROM string) & \makecell[l]{returns string : removes all char that are in the beginning or end of given string}\\
        \hline
        UPPER(string) & \makecell[l]{returns string : convert all characters of the given string to upper case}\\
        \hline
        LOWER(string) & \makecell[l]{returns string : convert all characters of the given string to lower case}  \\
        \hline
        CONCAT(string1,string2) & \makecell[l]{returns string : concat string1 with string2}\\
        \hline
        SUBSTR(string,i,j) & \makecell[l]{returns string : extract substring from given string from index i to index j}\\
        \hline
        REPLACE(string,sub\_string,replace\_string) & \makecell[l]{returns string : replace all occurences of sub\_string in the given string with \\replace\_string , not case sensitive}\\
        \hline
        LPAD(string,nb,char) & \makecell[l]{returns string : pads the given string to the left Length(string)-nb times with given char}\\
        \hline
        RPAD(string,nb,char) & \makecell[l]{returns string : pads the given string to the right Length(string)-nb times with given char} \\
        \hline
       INSTR(string,sub\_string) & \makecell[l]{returns integer : find the index of the first occurence of sub\_string in the given string\\if sub string doesn't exist returns 0}\\
        \hline 
   \end{tabular}
\end{center}

\subsubsection*{\underline{Example :}}

\begin{center}
    
\renewcommand{\arraystretch}{1.5}
    \begin{tabular}{|c|c|}
        \hline
        Fonction Call & Output\\
        \hline
        LENGTH('Hello World') & 11 \\
        \hline
        TRIM('   Hello  World   ') & 'Hello  World'\\
        \hline
        TRIM('!' FROM '!!!! Hello  World  !!!!!!') & ' Hello  World  ' \\
        \hline
        UPPER('Hello World') &  'HELLO WORLD'\\
        \hline
        LOWER('HeLlO WorLd') & 'hello world'\\
        \hline
        CONCAT('hello ','world!') & 'hello world!'\\
        \hline
        SUBSTR('I Love Java',8,11) & 'Java'\\
        \hline
        REPLACE('Hello world , I missed you world','world','toto') & Hello toto , I missed you toto\\
        \hline
        LPAD('hello',10,'*') & '*****hello'\\
        \hline
        RPAD('world',11,'*') & 'world******' \\
        \hline
        LPAD('toto',4,'*') & 'toto'\\
        \hline
        INSTR('I Hate Javascript','Java') & 8\\
        \hline
        INSTR('I Hate Javascript','Python') & 0\\
        \hline 
   \end{tabular}

\end{center}

\begin{prettyBox}{Note}{red}
  \textbf{\underline{Difference between CHAR and VARCHAR2}:} CHAR will take up the full specified size, even if not all space is used, and will pad the value with spaces until it reaches the full size. 
In contrast, VARCHAR2 only stores the exact amount of space needed for the data, without padding with spaces.

\vspace{0.25cm}
\underline{\textbf{Strings are 1-based}(first index is 1)}
\end{prettyBox}

