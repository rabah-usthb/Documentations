\newpage 
\null 
\vspace{0.15cm}

\begin{center} 
\Huge{\textbf{\underline{Chapter 5: Tablespace}}}
\end{center}

\vspace{0.25cm}

\setcounter{section}{0}

\vspace{1cm}

\section{Tablespace}
\begin{prettyBox}{Definition}{myblue}
A tablespace is a logical storage container that groups related objects 
and maps them to physical data files on disk. Tablespaces help manage storage, 
improve performance, and simplify administration. There are two types: permanent and temporary.
\end{prettyBox}

\vspace{0.25cm}

\subsection{Permanent Tablespace}
\begin{prettyBox}{Permanent}{myblue}
A permanent tablespace stores metadata, objects, and user data.
\begin{itemize}
    \item \textbf{SYSTEM}: The default permanent tablespace that stores 
          metadata such as table constraints, procedures, functions, triggers, etc.
    \item \textbf{USERS}: The default permanent tablespace that stores 
          user-created data like tables and views.
    \item \textbf{User-Created Tablespace}: A custom permanent tablespace 
          created by users, similar to the \textbf{USERS} tablespace.
\end{itemize}
\end{prettyBox}

\vspace{0.25cm}

\subsection{Temporary Tablespace}
\begin{prettyBox}{Temporary}{myblue}
A temporary tablespace is used to assist the server in computing 
SQL queries more efficiently when server memory is insufficient. 
For example, during the sorting of a large table with an \textcolor{blue}{ORDER BY} clause, 
the operation relies on the temporary tablespace.

\begin{itemize}
    \item \textbf{TEMP}: The default temporary tablespace.
    \item \textbf{User-Created Tablespace}: A custom temporary tablespace created by users.
\end{itemize}
\end{prettyBox}

\vspace{0.25cm}

\subsection{Tablespace Creation}
\begin{prettyBox}{Tablespace Creation}{myblue}
To create a tablespace we need to specify the type , the data file by giving the path ,
though one tablespace might use many datafile/tempfile : .dbf for permanet and .tmp for temporary ,
we give the inital size and other option like how much we extend it each time we exced and the maximum
\end{prettyBox}

\vspace{0.25cm}

\subsection*{\underline{Syntax}}

\subsection*{\underline{Permanent Tablespace}}
\lstinputlisting{SQL/syntax/Tab/createPerm.sql}


\subsection*{\underline{Temporary Tablespace}}
\lstinputlisting{SQL/syntax/Tab/createTmp.sql}

\vspace{0.15cm}

\subsection*{\underline{Syntax}}

\subsection*{\underline{Permanent Tablespace}}
\lstinputlisting{SQL/examples/Tab/perm.sql}


\subsection*{\underline{Temporary Tablespace}}
\lstinputlisting{SQL/examples/Tab/temp.sql}



\vspace{0.25cm}

\begin{prettyBox}{Note}{red}
\begin{itemize}
    \item The \textcolor{blue}{REUSE} keyword allows a tablespace to reuse an existing 
          datafile without throwing an error if the file already exists.
    \item The state of a permanent tablespace or datafile can be defined during creation. By default, 
          it is set to \textcolor{blue}{ONLINE} (allowing read/write operations). This can be explicitly 
          specified but is optional. Alternatively, the tablespace can be set to 
          \textcolor{blue}{OFFLINE}, which prevents usage until it is brought back online.
    \item Using \textcolor{blue}{AUTOEXTEND ON} without specifying values will increment the datafile 
          size by a default value, which is typically 1 MB in most systems.
    \item Specifying \textcolor{blue}{MAXSIZE UNLIMITED} or omitting the \textcolor{blue}{MAXSIZE} clause entirely 
          results in the file having no upper size limit.
\end{itemize}
\end{prettyBox}

\vspace{0.25cm}
\subsection{Tablespace Deletion}
\begin{prettyBox}{Tablespace Deletion}{myblue}
To delete a tablespace we use the \textcolor{blue}{DROP} , we can decide
if we want to delete just the tablespace contents , or to also delete its data file.
\end{prettyBox}

\vspace{0.15cm}
\subsection*{\underline{Syntax}}
\lstinputlisting{SQL/syntax/Tab/drop.sql}

\vspace{0.15cm}
\subsection*{\underline{Example}}
\lstinputlisting{SQL/examples/Tab/drop.sql}

\vspace{0.15cm}
\begin{prettyBox}{Note}{red}
If we try to drop a tablespace that is still in use by other users , it will result in an error
\begin{center}
ORA-01549: tablespace not empty, cannot drop
\end{center}
\end{prettyBox}

\vspace{0.25cm}

\subsection{Changing State}
\begin{prettyBox}{State}{myblue}
We can change the state of a permanent tablespace or datafile to online/offline using the 
\textcolor{blue}{ALTER} command
\end{prettyBox}

\vspace{0.15cm}

\subsection*{\underline{Syntax}}
\lstinputlisting[basicstyle=\ttfamily\footnotesize]{SQL/syntax/Tab/state.sql}

\vspace{0.15cm}
\subsection*{\underline{Example}}
\lstinputlisting{SQL/examples/Tab/state.sql}

\vspace{0.25cm}

\subsection{Assigning Tablespaces}
\begin{prettyBox}{Assigning Tablespaces}{myblue}
When creating the user we can assing him user-created tablespaces
\end{prettyBox}

\vspace{0.15cm}
\subsection*{\underline{Syntax}}
\lstinputlisting{SQL/syntax/Tab/tab.sql}

\vspace{0.15cm}
\subsection*{\underline{Example}}
\lstinputlisting{SQL/examples/Tab/tab.sql}


\vspace{0.25cm}

\subsection{Changing Tablespaces}
\begin{prettyBox}{Changing Tablespaces}{myblue}
    We use the \textcolor{blue}{ALTER} keyword , we have two possibilities:
\begin{itemize}
    \item \textbf{Default} (works for both permanent and temporary tablespaces): This change only updates the reference to the new tablespace for future objects.
    \begin{itemize}
        \item \textbf{Permanent Tablespace}: The user will still rely on the datafiles of the old tablespace for objects created in it. For newly created objects, the user will rely on the datafiles of the new tablespace.
        \item \textbf{Temporary Tablespace}: The user will switch to the new tablespace and use it for subsequent SQL queries.
    \end{itemize}
    \item \textbf{Move} (only for permanent tablespaces): This operation physically moves the data of a table stored in the datafiles of the previous
tablespace to the datafiles of the new one, doing that
        also requires to move the index of the table.
\end{itemize}
\end{prettyBox}

\vspace{0.15cm}
\subsection*{\underline{Syntax}}
\lstinputlisting[basicstyle=\ttfamily\footnotesize]{SQL/syntax/Tab/change.sql}

\vspace{0.15cm}
\subsection*{\underline{Example}}
\lstinputlisting[basicstyle=\ttfamily\footnotesize]{SQL/examples/Tab/change.sql}



\vspace{0.25cm}


\subsection{The Quota}
\begin{prettyBox}{Quota}{myblue}
The quota defines how much space a user can use in a permanent tablespace. 
By default, the quota is set to \textbf{0}. You can modify it using the \textcolor{blue}{ALTER} command.
\end{prettyBox}

\newpage
\subsection*{\underline{Syntax}}
\lstinputlisting{SQL/syntax/Tab/qouta.sql}

\vspace{0.15cm}
\subsection*{\underline{Example}}
\lstinputlisting{SQL/examples/Tab/qouta.sql}

\vspace{0.25cm}
\begin{prettyBox}{Note}{red}
If a user tries to write to or create objects in a permanent tablespace with insufficient quota, it will result in an error.
\end{prettyBox}

