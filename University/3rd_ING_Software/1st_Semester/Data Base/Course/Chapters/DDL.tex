\newpage 
\null 
\vspace{0.15cm}

\begin{center} 
\Huge{\textbf{\underline{Chapter 3: DDL Commands}}}
\end{center}

\vspace{0.25cm}

\setcounter{section}{0}


\section{Create Table}
\begin{prettyBox}{Table Creation}{myblue}
    To create a table in oracle sql we use the \textcolor{blue}{CREATE} command , we just have to give the table a name and define each column known as attribut
by giving each of them a name , a dataType and an optional constraint that can be added in same line of
attribut definition(in-line method) or on its own line(out-of-line method) , we will see contraint in details in the next section
\end{prettyBox}

\vspace{0.5cm}
\subsection*{\underline{Syntax :}}

\lstinputlisting{SQL/syntax/DDL/create.sql}


\vspace{0.5cm}
\section{Table Constraints}

\begin{prettyBox}{Constraints}{myblue}
Constraints are conditions set on the columns (attributes) of a table to ensure data integrity and consistency. Constraints can be defined:
\begin{itemize}
    \item During table creation, either on the same line as the attribute definition(inline) or on a separate line(out-of-line)
    \item After table creation using the ALTER TABLE command
\end{itemize}

There are two types of constraints: static and dynamic.
\end{prettyBox}

\subsection{\underline{Static Constraints}}

\vspace{0.25cm}
\begin{prettyBox}{Static}{myblue}
\begin{itemize} 
\item \textbf{Data Type} : Ensures Integrity of the column
\item \textbf{NOT NULL}: Ensures that the attribute must have a value when inserting into the table.
\item \textbf{UNIQUE}: Ensures that each value in the attribute is distinct. Unlike PRIMARY KEY, it allows null values.
\item \textbf{PRIMARY KEY}: Combines UNIQUE and NOT NULL properties to ensure each value is unique and not null.
Used to identify rows uniquely. 
\item \textbf{FOREIGN KEY}: References a primary key from another table to establish a relationship between tables , can be null.
\item \textbf{DELETE ON CASCADE}: When deleting a row from the referenced (parent) table, all rows in the child table 
that contain the matching foreign key are also deleted.
\item \textbf{CHECK}: Validates a specified condition before allowing data to be inserted or updated. 
\item \textbf{DEFAULT}: Sets a default value for the attribute if no value is provided during insertion. 
\end{itemize}
\end{prettyBox}

\vspace{0.5cm}
\subsection{\underline{Dynamic Constraints}} 

\vspace{0.25cm}
\begin{prettyBox}{Dynamic}{myblue}
\begin{itemize} 
\item \textbf{TRIGGER}: Acts like a call back function , a block of code that gets executed automatically when 
a defined event is triggered
\end{itemize}
\end{prettyBox}

\vspace{0.5cm}
\subsection*{\underline{Syntax}}

\vspace{0.25cm}

\subsection*{\underline{In-Line Method}}

\vspace{0.25cm}
\lstinputlisting{SQL/syntax/DDL/inline.sql}
\newpage

\subsection*{\underline{Out-Of-Line Method}}

\vspace{0.25cm}
\lstinputlisting{SQL/syntax/DDL/outline.sql}

\vspace{0.5cm}

\vspace{1cm}
\begin{prettyBox}{Naming Convention Of Constraints}{myblue}
  \begin{itemize}
        \item  \textbf{Primary Key} : PK\_\textless tableName\textgreater  
        \item  \textbf{Foreign Key} : FK\_\textless tableName\textgreater\_\textless referencedTableName\textgreater
        \item  \textbf{Unique} : UQ\_\textless tableName\textgreater\_\textless columnName\textgreater
        \item  \textbf{Check} : CHK\_\textless tableName\textgreater\_\textless columnName\textgreater
        \item  \textbf{Default} : DF\_\textless tableName\textgreater\_\textless columnName\textgreater
        \item  \textbf{Not Null} : NN\_\textless tableName\textgreater\_\textless columnName\textgreater
    \end{itemize}   
\end{prettyBox}

\vspace{0.35cm}
\begin{prettyBox}{Note}{red}

    \textbf{\underline{Constraint Name Must Be Unique}}

\vspace{0.15cm}
Tables inside the same PDB (pluggable data base) can't share the same constraints name

\vspace{0.25cm}
\textbf{\underline{Multiple Constraints}}

\vspace{0.15cm}
It is possible to define multiple constraints on a single attribute using the inline method. 
However, with the outline method, each constraint needs to be specified individually.

\end{prettyBox}

\vspace{0.35cm}
\section{Drop Table}
\begin{prettyBox}{Drop}{myblue}
    We can remove a table using the \textcolor{blue}{DROP} command
\end{prettyBox}

\newpage
\subsection*{\underline{Syntax}}

\lstinputlisting{SQL/syntax/DDL/drop.sql}

\vspace{0.35cm}
\section{Rename Table}

\begin{prettyBox}{Rename}{myblue}
    We can rename tables by using the \textcolor{blue}{RENAME} command 
\end{prettyBox}

\subsection*{\underline{Syntax}}

\lstinputlisting{SQL/syntax/DDL/rename.sql}

\vspace{0.35cm}
\section{Alter Table}
\begin{prettyBox}{Alter}{myblue}
    The \textcolor{blue}{ALTER} command is a versatile command that allows us to change various aspects of a table:
\begin{itemize}
    \item Columns
        \begin{itemize}
            \item \textbf{Renaming Column}: Rename the column.
            \item \textbf{Modify Column}: Change the constraint and data type.
            \item \textbf{Add Column}: Add a new column.
            \item \textbf{Remove Column}: Remove a column.
        \end{itemize}
    \item Constraints
        \begin{itemize}
            \item \textbf{Add Constraint}: Add a new constraint. 
            \item \textbf{Remove Constraint}: Remove a constraint.
            \item \textbf{Enable Constraint}: Enable an already existing constraint.
            \item \textbf{Disable Constraint}: Disable an already existing constraint without deleting it.
        \end{itemize}
\end{itemize}
\end{prettyBox}

\vspace{0.15cm}
\subsection*{\underline{Syntax}}

\subsection*{Columns Modification}

\lstinputlisting{SQL/syntax/DDL/colModification.sql}

\subsection*{Constraints}

\lstinputlisting{SQL/syntax/DDL/constraints.sql}

\vspace{0.35cm}
\section{Truncate Table}
\begin{prettyBox}{Truncate}{myblue}
    To remove all rows from a table efficiently we use the \textcolor{blue}{TRUNCATE} command
\end{prettyBox}

\vspace{0.15cm}
\subsection*{\underline{Syntax}}

\lstinputlisting{SQL/syntax/DDL/truncate.sql}

\vspace{0.35cm}
\section{Describe Table}
\begin{prettyBox}{Describe}{myblue}
If we want an overview over the definition of the table we can use the 
\textcolor{blue}{DESC} command that will retrieve name of columns , their data type
and if they are null 
\end{prettyBox}

\vspace{0.15cm}
\subsection*{\underline{Syntax}}

\lstinputlisting{SQL/syntax/DDL/desc.sql}

\vspace{0.25cm}


\begin{prettyBox}{Note}{red}
The Not Null column in the output of \textcolor{blue}{DESC} is not always precise. 
If a NOT NULL constraint is applied to a column using the \textcolor{blue}{CONSTRAINT} keyword, 
the column will not display Not Null in the output, even though the constraint prevents NULL values.
\end{prettyBox}

\newpage

\subsection*{\underline{Example :}}
let the tables definition be :\\[0.2cm]
\textbf{Rank} (\underline{\textbf{numRank}},nameRank)\\[0.1cm]
\textbf{Exam} (\underline{\textbf{numExam}},nameExam,dateExam,dateResult)\\[0.1cm]
\textbf{Participant} (\underline{\textbf{numParticipant}},firstNameParticipant,lastNameParticipant,numRank*)\\[0.1cm]
\textbf{Result} (\underline{\textbf{numParticipant*,numExam*}},grade)

\vspace{0.35cm}

\begin{prettyBox}{Note}{red}
\begin{itemize}
    \item underline means primary key.
    \item asterix(*) means foreign key.
    \item domain value of nameRank is \{'TS','ING SI','Main ING'\}.
    \item domain value of nameExam is \{'BDD','SI','GL'\}.
\end{itemize}
\end{prettyBox}

\vspace{0.35cm}
\begin{enumerate}
    \item Create the tables.
    \item Delete \textbf{dateExam} column in the \textbf{Exam} table. verify its deletion.
    \item Add \textbf{not null} constraint to the following columns :
        \begin{itemize}
            \item \textbf{nameRank} in \textbf{Rank} table.
            \item \textbf{firstNameParticipant} and \textbf{lastNameParticipant} in \textbf{Participant} table.
            \item \textbf{nameExam} in \textbf{Exam} table.
        \end{itemize}
    \item Change size of \textbf{nameRank} (add,reduce)
    \item Add back the \textbf{dateExam} column,verify it.
    \item Rename  \textbf{firstNameParticipant} and \textbf{lastNameParticipant} to \textbf{firstNameP} , \textbf{lastNameP}.
    \item Change the domain value of \textbf{nameRank} to \{'TS','ING SI','ING SYS','Main ING'\}.
    \item Change the domain value of  \textbf{nameExam} to \{'BDD','SI','GL','SYS','General Culture'\}.
    \item Add a constraint so that the \textbf{dateExam} is \(<\) to \textbf{dateResult}.
    \item Change the name of the table \textbf{Exam} to \textbf{Finals}.
    \item Disactivate the date constraint.
    \item Activate the date constraint back again.
    \item Try To Delete the \textbf{Finals} table what do you notice?
\end{enumerate}

\vspace{0.35cm}
\subsection*{\underline{1 Table Creation}}
\subsection*{1.1 Exam Table}
\lstinputlisting{SQL/examples/DDL/ex1.1.sql}

\subsection*{1.2 Rank Table}
\lstinputlisting{SQL/examples/DDL/ex1.2.sql}

\newpage
\subsection*{1.3 Participant Table}
\lstinputlisting{SQL/examples/DDL/ex1.3.sql}

\subsection*{1.4 Result Table}
\lstinputlisting{SQL/examples/DDL/ex1.4.sql}

\vspace{0.35cm}
\subsection*{\underline{2 Dropping dateExam Column}}
\lstinputlisting{SQL/examples/DDL/ex2.sql}

\vspace{0.35cm}
\subsection*{\underline{3 Not Null Constraint}}
\lstinputlisting{SQL/examples/DDL/ex3.sql}

\vspace{0.35cm}
\subsection*{\underline{4 Changing Size}}
\lstinputlisting{SQL/examples/DDL/ex4.sql}

\vspace{0.35cm}
\subsection*{\underline{5 Add dateExam Column}}
\lstinputlisting{SQL/examples/DDL/ex5.sql}

\vspace{0.35cm}
\subsection*{\underline{6 Changing Column Name}}
\lstinputlisting{SQL/examples/DDL/ex6.sql}

\vspace{0.35cm}
\subsection*{\underline{7-8 Change Domain Value}}
\subsection*{\underline{7}}
\lstinputlisting{SQL/examples/DDL/ex7.sql}
\vspace{0.15cm}

\subsection*{\underline{8}}
\lstinputlisting{SQL/examples/DDL/ex8.sql}

\vspace{0.35cm}

\subsection*{\underline{9 Date Constraint}}
\lstinputlisting{SQL/examples/DDL/ex9.sql}
\vspace{0.35cm}

\subsection*{\underline{10 Disactivate Date Constraint}}
\lstinputlisting{SQL/examples/DDL/ex10.sql}
\vspace{0.35cm}

\subsection*{\underline{11 Activate Date Constraint}}
\lstinputlisting{SQL/examples/DDL/ex11.sql}
\vspace{0.35cm}

\subsection*{\underline{12 Rename Table}}
\lstinputlisting{SQL/examples/DDL/ex12.sql}
\vspace{0.35cm}

\subsection*{\underline{13 Dropping Table}}
\lstinputlisting{SQL/examples/DDL/ex13.sql}
\vspace{0.15cm}

\begin{prettyBox}{Note}{red}
It will result into an error because finals is been referenced by other tables.
\begin{center}
    ORA-02449: unique/primary keys in table referenced by foreign keys
\end{center}   
\end{prettyBox}




