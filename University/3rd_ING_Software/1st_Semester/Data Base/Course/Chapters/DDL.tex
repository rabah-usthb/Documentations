\section{DDL Commands}
\subsection{Create Table}
\begin{tcolorbox}[title = Definition]
To create a table in oracle sql we just have to give the table a name and define each column known as attribut
by giving each of them a name , a dataType and an optional constraint that can be added in same line of
attribut definition or on its own line , we will see contraint in details in the next section
\end{tcolorbox}

\vspace{0.5cm}
\subsubsection*{\underline{Syntax :}}
\begin{tcolorbox}[title = Table Creation]

\textcolor{blue}{create table} \textless table\_name\textgreater (\\
\textless attribute$_{1}$\textgreater \quad \textless DataType$_{1}$\textgreater \quad {[Constraint$_{1}$]},\\
\textless attribute$_{2}$\textgreater \quad \textless DataType$_{2}$\textgreater \quad {[Constraint$_{2}$]},\\
....................................................................\\
\textless attribute$_{n}$\textgreater \quad \textless DataType$_{n}$\textgreater \hspace{0.15cm} {[Constraint$_{n}$]},\\
{[table contraints]}\\
);
\end{tcolorbox}

\vspace{0.5cm}
\subsubsection*{Example :} let's create student table 
\begin{lstlisting}
    create table student (
    id number,
    firstname varchar2(50),
    lastname varchar2(50),
    grade number(2,2)
    );
\end{lstlisting}

\subsection{Table Constraints}
\begin{tcolorbox}[title = Definition]
Constraints are conditions set on the columns (attributes) of a table to ensure data integrity and consistency. Constraints can be defined:
\begin{itemize}
    \item During table creation, either on the same line as the attribute definition or on a separate line
    \item After table creation using the ALTER TABLE command
\end{itemize}

There are two types of constraints: static and dynamic.
\subsubsection*{\underline{Static Constraints}} Static constraints are fixed conditions that do not change based on data input.

\begin{itemize} 
\item \textbf{NOT NULL}: Ensures that the attribute must have a value when inserting into the table.
\item \textbf{UNIQUE}: Ensures that each value in the attribute is distinct. Unlike PRIMARY KEY, it allows null values.
\item \textbf{PRIMARY KEY}: Combines UNIQUE and NOT NULL properties to ensure each value is unique and not null.
Used to identify rows uniquely. 
\item \textbf{FOREIGN KEY}: References a primary key from another table to establish a relationship between tables.
\item \textbf{DELETE ON CASCADE}: When deleting a row from the referenced (parent) table, all rows in the child table 
that contain the matching foreign key are also deleted.
\end{itemize}
\subsubsection*{\underline{Dynamic Constraints}} Dynamic constraints apply conditions that can change based on specified criteria.

\begin{itemize} 
\item \textbf{CHECK}: Validates a specified condition before allowing data to be inserted or updated. 
\item \textbf{DEFAULT}: Sets a default value for the attribute if no value is provided during insertion. 
\end{itemize}
\end{tcolorbox}
\subsubsection*{\underline{Syntax}}
\begin{tcolorbox}[title = Inline Constraint]
\textless attribute$_{i}$\textgreater \, \textless DataType$_{i}$\textgreater \, \textcolor{blue}{not null} \\
\textless attribute$_{i}$\textgreater \, \textless DataType$_{i}$\textgreater \, \textcolor{blue}{unique} \\
\textless attribute$_{i}$\textgreater \, \textless DataType$_{i}$\textgreater \, \textcolor{blue}{primary key} \\
\textless attribute$_{i}$\textgreater \, \textless DataType$_{i}$\textgreater \, \textcolor{blue}{references} \, referenced\_table(references\_attribute) \\
\textless attribute$_{i}$\textgreater \, \textless DataType$_{i}$\textgreater \, \textcolor{blue}{default} (value) \\
\end{tcolorbox}

\begin{tcolorbox}[title = Outline Constraint]
\textcolor{blue}{constraint} \, \textless contraint\_name\textgreater \, \textless attribute$_{i}$\textgreater \, \textcolor{blue}{not null} \\
\textless attribute$_{i}$\textgreater \, \textcolor{blue}{not null} \\

\textcolor{blue}{constraint} \, \textless contraint\_name\textgreater \, \textless attribute$_{i}$\textgreater \, \textcolor{blue}{unique} \\
\textless attribute$_{i}$\textgreater \, \textcolor{blue}{unique} \\

\textcolor{blue}{constraint} \, \textless contraint\_name\textgreater \, \textless attribute$_{i}$\textgreater \, \textcolor{blue}{primary key} \\
\textcolor{blue}{primary key} (attribute$_{1}$ ,..., attribute$_{n}$) \\

\textcolor{blue}{constraint} \, \textless contraint\_name\textgreater \, \textcolor{blue}{foreign key} \, \textless attribute$_{i}$\textgreater \, \textcolor{blue}{references} \, referenced\_table(references\_attribute)\\
\textcolor{blue}{foreign key} \, \textless attribute$_{i}$\textgreater \, \textcolor{blue}{references} \, referenced\_table(references\_attribute) \\

\textcolor{blue}{constraint} \, \textless contraint\_name\textgreater \, \textless attribute$_{i}$\textgreater \, \textcolor{blue}{default} (value) \\
\textless attribute$_{i}$\textgreater \, \textcolor{blue}{default} (value) \\

\end{tcolorbox}
\subsubsection*{\underline{Example :}}
let's create a new table section and recreate the student table with constraints\\\\
\textbf{Creating Section Table}\\\\
\textbf{Inline Method}
\begin{lstlisting}
    create table section (
    id_section number primary key,
    name varchar2(5) not null
    );
\end{lstlisting}
\textbf{Outline Method}
\begin{lstlisting}
    create table section (
    id_section number,
    name varchar2(5),
    name not null,
    constraint pk_sec primary key (id_section)
    );
\end{lstlisting}

\textbf{Create Student Table}\\
\textbf{Inline Method}
\begin{lstlisting}
    create table student (
    id number primary key,
    lastname varchar2(50) not null,
    firstname varchar2(50) not null,
    id_section number references section(id_section) on delete cascade,
    grade number(4,2) default 00.00 check (grade between 0 and 20),
    dob date not null check (dob<= add_months(sysdate,-18*12))
    );
\end{lstlisting}

\textbf{Outline Mehtod}
\begin{lstlisting}
    create table student (
    id number,
    constraint pk_student primary key(id),
    lastname varchar2(50),
    firstname varchar2(50),
    constraint nn_student_lastname lastname not null,
    constraint nn_student_firstname firstname not null,
    id_section number,
    constraint fr_student foreign key (id_section) references section(id_section) on delete cascade,
    grade number(4,2),
    grade default 00.00,
    constraint chk_student_grade check (grade between 0 and 20),
    dob date not null,
    constraint chk_student_dob check (dob<= add_months(sysdate,-18*12))
    );
\end{lstlisting}

\begin{tcolorbox}[title = Note]
    \textbf{Name Convention Of Constraint}
    \begin{itemize}
        \item Primary Key : PK\_\textless tableName\textgreater  
        \item Foreign Key : FK\_\textless tableName\textgreater
        \item Unique : UQ\_\textless tableName\textgreater\_\textless columnName\textgreater
        \item Check : CHK\_\textless tableName\textgreater\_\textless columnName\textgreater
        \item Default : DF\_\textless tableName\textgreater\_\textless columnName\textgreater
        \item Not Null : NN\_\textless tableName\textgreater\_\textless columnName\textgreater
    \end{itemize}
    \textbf{Constraint Name Must Be Unique}\\
    Tables inside the same PDB (pluggable data base) can't share the same constraints name\\
    \textbf{Multiple Constraints}\\
 It is possible to define multiple constraints on a single attribute using the inline method. 
However, with the outline method, each constraint needs to be specified individually.
\end{tcolorbox}
\subsection{Delete Table}
\begin{tcolorbox}[title = Definition]
    We can delete table using the drop command
\end{tcolorbox}
\subsubsection{\underline{Syntax}}
\begin{tcolorbox}[title = Table Deletion]
    \textcolor{blue}{drop table}  \textless tableName\textgreater;
\end{tcolorbox}
\subsubsection{\underline{Example}}
lets delete the section table we created 
\begin{lstlisting}
    drop table section;
\end{lstlisting}
\subsection{Rename Table}
\begin{tcolorbox}[title = Definition]
We can rename tables by using the rename command 
\end{tcolorbox}
\subsubsection*{\underline{Syntax}}
\begin{tcolorbox}[title = Renaming Table]
   rename \textless old\_tableName\textgreater to \textless new\_tableName\textgreater;
\end{tcolorbox}
\subsubsection*{\underline{Example}}
\begin{lstlisting}
    rename section to mama;
\end{lstlisting}

\subsection{Alter Table}
\begin{tcolorbox}[title = Definition]
The `ALTER` command is a versatile command that allows us to change various aspects of a table:
\begin{itemize}
    \item Columns
        \begin{itemize}
            \item \textbf{Renaming Column}: Rename the column.
            \item \textbf{Modify Column}: Change the constraint and data type.
            \item \textbf{Add Column}: Add a new column.
            \item \textbf{Remove Column}: Remove a column.
        \end{itemize}
    \item Constraints
        \begin{itemize}
            \item \textbf{Add Constraint}: Add a new constraint. 
            \item \textbf{Remove Constraint}: Remove a constraint.
            \item \textbf{Enable Constraint}: Enable an already existing constraint.
            \item \textbf{Disable Constraint}: Disable an already existing constraint without deleting it.
        \end{itemize}
\end{itemize}
\end{tcolorbox}

\subsubsection*{\underline{Syntax}}
\begin{tcolorbox}[title = Columns Modification]
\textbf{\underline{Renaming Column}}\\
\textcolor{blue}{alter table} \textless tableName\textgreater rename \textcolor{blue}{column} \textless old\_columnName\textgreater to \textless new\_columnName\textgreater;

\textbf{\underline{Modify Column}}\\
\textcolor{blue}{alter table} \textless tableName\textgreater modify (columnName [new column definition \& constraints]);

\textbf{\underline{Add Column}}\\
\textcolor{blue}{alter table} \textless tableName\textgreater \textcolor{blue}{add} (columnName [column definition \& constraints]);

\textbf{\underline{Remove Column}}\\
\textcolor{blue}{alter table} \textless tableName\textgreater \textcolor{blue}{drop column} \textless columnName\textgreater;
\end{tcolorbox}

\begin{tcolorbox}[title = Constraints]
    \textbf{\underline{Rename Constraint}}\\
    \textcolor{blue}{alter table} \textless tableName\textgreater rename \textcolor{blue}{constraint} \textless old\_constraintName\textgreater to \textless new\_constraintName\textgreater;

    \textbf{\underline{Add Constraint}}\\
    \textcolor{blue}{alter table} \textless tableName\textgreater \textcolor{blue}{add constraint} \textless constraintName\textgreater [Constraint];
    
    \textbf{\underline{Remove Constraint}}\\
    \textcolor{blue}{alter table} \textless tableName\textgreater \textcolor{blue}{drop constraint} \textless constraintName\textgreater;

    \textbf{\underline{Enable Constraint}}\\
    \textcolor{blue}{alter table} \textless tableName\textgreater enable \textcolor{blue}{constraint} \textless constraintName\textgreater;
    
    \textbf{\underline{Disable Constraint}}\\
    \textcolor{blue}{alter table} \textless tableName\textgreater disable \textcolor{blue}{constraint} \textless constraintName\textgreater;
\end{tcolorbox}
\subsubsection*{\underline{Example}}
\subsection{Truncate Table}
\begin{tcolorbox}[title = Definition]
    To remove all rows from a table efficiently we use the truncate command
\end{tcolorbox}

\subsubsection*{\underline{Syntax}}
\begin{tcolorbox}[title = Truncing Table]
\textcolor{blue}{truncate table} \textless tableName\textgreater;
\end{tcolorbox}

\subsubsection*{\underline{Example}}
lets delete all records from student table
\begin{lstlisting}
    truncate table student;
\end{lstlisting}
