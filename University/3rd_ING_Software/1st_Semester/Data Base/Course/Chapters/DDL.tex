\newpage
\section{DDL Commands}
\subsection{Create Table}
\begin{prettyBox}{Table Creation}{myblue}
To create a table in oracle sql we just have to give the table a name and define each column known as attribut
by giving each of them a name , a dataType and an optional constraint that can be added in same line of
attribut definition(in-line method) or on its own line(out-of-line method) , we will see contraint in details in the next section
\end{prettyBox}

\vspace{0.5cm}
\subsubsection*{\underline{Syntax :}}

\lstinputlisting{SQL/syntax/DDL/create.sql}


\vspace{0.5cm}
\subsubsection*{Example :} let's create student table 

\lstinputlisting{SQL/examples/DDL/createStudent.sql}

\vspace{0.5cm}
\subsection{Table Constraints}

\begin{prettyBox}{Constraints}{myblue}
Constraints are conditions set on the columns (attributes) of a table to ensure data integrity and consistency. Constraints can be defined:
\begin{itemize}
    \item During table creation, either on the same line as the attribute definition(inline) or on a separate line(out-of-line)
    \item After table creation using the ALTER TABLE command
\end{itemize}

There are two types of constraints: static and dynamic.
\end{prettyBox}

\subsubsection{\underline{Static Constraints}}

\vspace{0.25cm}
\begin{prettyBox}{Static}{myblue}
Static constraints are fixed conditions that do not change based on data input.
\begin{itemize} 
\item \textbf{Data Type} : Ensures Integrity of the column
\item \textbf{NOT NULL}: Ensures that the attribute must have a value when inserting into the table.
\item \textbf{UNIQUE}: Ensures that each value in the attribute is distinct. Unlike PRIMARY KEY, it allows null values.
\item \textbf{PRIMARY KEY}: Combines UNIQUE and NOT NULL properties to ensure each value is unique and not null.
Used to identify rows uniquely. 
\item \textbf{FOREIGN KEY}: References a primary key from another table to establish a relationship between tables , can be null.
\item \textbf{DELETE ON CASCADE}: When deleting a row from the referenced (parent) table, all rows in the child table 
that contain the matching foreign key are also deleted.
\item \textbf{CHECK}: Validates a specified condition before allowing data to be inserted or updated. 
\item \textbf{DEFAULT}: Sets a default value for the attribute if no value is provided during insertion. 
\end{itemize}
\end{prettyBox}

\vspace{0.5cm}
\subsubsection{\underline{Dynamic Constraints}} 

\vspace{0.25cm}
\begin{prettyBox}{Dynamic}{myblue}
Dynamic constraints apply conditions that can change based on specified criteria.
\begin{itemize} 
\item \textbf{TRIGGER}: Acts like a call back function , a block of code that gets executed automatically when 
a defined event is triggered
\end{itemize}
\end{prettyBox}

\vspace{0.5cm}
\subsubsection*{\underline{Syntax}}

\vspace{0.25cm}

\subsubsection*{\underline{In-Line Method}}

\vspace{0.25cm}
\lstinputlisting{SQL/syntax/DDL/inline.sql}
\newpage

\subsubsection*{\underline{Out-Of-Line Method}}

\vspace{0.25cm}
\lstinputlisting{SQL/syntax/DDL/outline.sql}

\vspace{0.5cm}
\subsubsection*{\underline{Example :}}

\vspace{0.25cm}
let's create a new table section and recreate the student table with constraints\\\\
\textbf{\underline{Creating Section Table}}\\\\
\textbf{\underline{In-Line Method}}

\vspace{0.25cm}
\lstinputlisting{SQL/examples/DDL/sectionInline.sql}

\vspace{0.5cm}
\textbf{\underline{Out-Of-Line Method}}

\vspace{0.25cm}
\lstinputlisting{SQL/examples/DDL/sectionOutline.sql}


\vspace{1cm}
\newpage
\textbf{\underline{Create Student Table}}\\\\
\textbf{\underline{In-Line Method}}

\vspace{0.25cm}
\lstinputlisting{SQL/examples/DDL/studentInline.sql}

\vspace{0.5cm}
\textbf{\underline{Out-Of-Line Mehtod}}

\vspace{0.25cm}
\lstinputlisting{SQL/examples/DDL/studentOutline.sql}

\vspace{1cm}
\begin{prettyBox}{Naming Convention Of Constraints}{myblue}
  \begin{itemize}
        \item  \textbf{Primary Key} : PK\_\textless tableName\textgreater  
        \item  \textbf{Foreign Key} : FK\_\textless tableName\textgreater
        \item  \textbf{Unique} : UQ\_\textless tableName\textgreater\_\textless columnName\textgreater
        \item  \textbf{Check} : CHK\_\textless tableName\textgreater\_\textless columnName\textgreater
        \item  \textbf{Default} : DF\_\textless tableName\textgreater\_\textless columnName\textgreater
        \item  \textbf{Not Null} : NN\_\textless tableName\textgreater\_\textless columnName\textgreater
    \end{itemize}   
\end{prettyBox}

\begin{prettyBox}{Note}{red}

    \textbf{\underline{Constraint Name Must Be Unique}}

\vspace{0.15cm}
Tables inside the same PDB (pluggable data base) can't share the same constraints name

\vspace{0.25cm}
\textbf{\underline{Multiple Constraints}}

\vspace{0.15cm}
It is possible to define multiple constraints on a single attribute using the inline method. 
However, with the outline method, each constraint needs to be specified individually.

\end{prettyBox}


\subsection{Drop Table}
\begin{prettyBox}{Drop}{myblue}
    We can remove a table using the drop command
\end{prettyBox}

\subsubsection*{\underline{Syntax}}

\lstinputlisting{SQL/syntax/DDL/drop.sql}

\subsubsection*{\underline{Example}}
lets delete the section table we created 

\lstinputlisting{SQL/examples/DDL/sectionDrop.sql}

\subsection{Rename Table}

\begin{prettyBox}{Rename}{myblue}
We can rename tables by using the rename command 
\end{prettyBox}

\subsubsection*{\underline{Syntax}}

\lstinputlisting{SQL/syntax/DDL/rename.sql}

\subsubsection*{\underline{Example}}

\lstinputlisting{SQL/examples/DDL/sectionRename.sql}

\subsection{Alter Table}
\begin{prettyBox}{Alter}{myblue}
The `ALTER` command is a versatile command that allows us to change various aspects of a table:
\begin{itemize}
    \item Columns
        \begin{itemize}
            \item \textbf{Renaming Column}: Rename the column.
            \item \textbf{Modify Column}: Change the constraint and data type.
            \item \textbf{Add Column}: Add a new column.
            \item \textbf{Remove Column}: Remove a column.
        \end{itemize}
    \item Constraints
        \begin{itemize}
            \item \textbf{Add Constraint}: Add a new constraint. 
            \item \textbf{Remove Constraint}: Remove a constraint.
            \item \textbf{Enable Constraint}: Enable an already existing constraint.
            \item \textbf{Disable Constraint}: Disable an already existing constraint without deleting it.
        \end{itemize}
\end{itemize}
\end{prettyBox}

\subsubsection*{\underline{Syntax}}

\subsubsection*{Columns Modification}

\lstinputlisting{SQL/syntax/DDL/colModification.sql}

\subsubsection*{Constraints}

\lstinputlisting{SQL/syntax/DDL/constraints.sql}

\subsubsection*{\underline{Example}}
\subsection{Truncate Table}
\begin{prettyBox}{Truncate}{myblue}
    To remove all rows from a table efficiently we use the truncate command
\end{prettyBox}

\subsubsection*{\underline{Syntax}}

\lstinputlisting{SQL/syntax/DDL/truncate.sql}


\subsubsection*{\underline{Example}}
lets delete all records from student table

\lstinputlisting{SQL/examples/DDL/studentTruncate.sql}

