\documentclass{article}
\usepackage[a4paper, paperwidth=25cm, paperheight=25.5cm, left=1.5cm, right=1.5cm, top=1cm, bottom=2cm]{geometry}
\usepackage{tikz,tcolorbox}
\usepackage{amsmath}
\usepackage[table,xcdraw]{xcolor}
\usepackage{listings}
\usepackage{array,multirow} % For customizing tables
\usepackage{booktabs} % For better horizontal lines
\usepackage{makecell}
\setlength{\parindent}{0pt}

\lstset{
    language=SQL,                    % Set language to SQL
    basicstyle=\ttfamily\small,      % Font size and family for code
    keywordstyle=\color{blue}\bfseries, % Color for SQL keywords
    commentstyle=\color{gray},       % Color for comments
    stringstyle=\color{red},         % Color for strings
    numbers=left,                    % Show line numbers on the left
    numberstyle=\tiny\color{gray},   % Line number font and color
    stepnumber=1,                    % Line number step
    breaklines=true,                 % Wrap long lines
    frame=single,                    % Add a frame around code
    tabsize=2,                       % Set tab size
    showstringspaces=false           % Hide spaces in strings
}


\newcommand{\exer}[1]{
  \section*{Exercice #1}
  \vspace{-0.5cm}
  \noindent\rule{\textwidth}{0.5pt}%
}

\newcommand{\tit}[1]{
\begin{center}
    \Large{\textbf{{#1}}}
\end{center}
}

\definecolor{commentgray}{HTML}{676160}
\definecolor{messagegreen}{HTML}{17B867}
\definecolor{myblue}{HTML}{10C2C4}

\tcbuselibrary{skins, breakable, theorems}


\newtcolorbox{prettyBox}[2]{
  enhanced,
  colback=white!90!#2,   % Background color based on the second parameter (color)
  colframe=#2!60!black,  % Frame color based on the second parameter (color)
  coltitle=white,        % Title color (white)
  fonttitle=\bfseries\Large,
  title=#1,              % Title from the first parameter
  boxrule=1mm,
  arc=0.5mm,
  drop shadow=#2!35!gray, % Drop shadow color based on the second parameter (color)
}



\begin{document}
\tit{TP N\(^{\boldsymbol{\circ}}\)\hspace{0.1cm}5-Les Vues}

\exer{1}
\section{Connectivité Des Machines Virtuelles}

\begin{center}
    \includegraphics[width=0.8\textwidth]{Question/SC/1-ping1.PNG}
\end{center}

\vspace{0.25cm}

\begin{center}
    \includegraphics[width=0.8\textwidth]{Question/SC/1-ping2.PNG}
\end{center}

\vspace{0.35cm}

\begin{prettyBox}{Connectivité}{myblue}
D'après les captures d'écran, on remarque que les adresses IP des machines sont :
\begin{itemize}
    \item \textbf{Windows : } \texttt{192.168.100.13}
    \item \textbf{Kali Linux : } \texttt{192.168.100.15}
\end{itemize}
Et le ping est réussi, donc il y a une connectivité.
\end{prettyBox}


\subsection*{Insert}
i can't put the insert queries in this pdf because they have a lot of lines , you can find all the insert sql code inside the SQL/EX1/insert/ folder


we have \(f(n) = \sum fr\) since \(f(n)\) is sum of frequency of execution (\(fr\)) we need to figure out the
\(fr\) of each instruction and sum them : 

\subsubsection*{\underline{Best Case (N is prime)}}

\vspace{0.5cm}
cmpt $\gets$ 0  \hspace{4cm} \(fr = 1\) (one affectation)

\vspace{0.15cm}
i $\gets$ 1  \hspace{4.65cm} \(fr = 1\) (one affectation)

\vspace{0.15cm}
\textcolor{purplePlot!80!black}{print}(\textcolor{blueArea!60!black}{'Input Integer N\(>\)1 : '})  \hspace{0.95cm} \(fr = 1\) (one print)

\vspace{0.15cm}
\textcolor{purplePlot!80!black}{Read}(n)  \hspace{4.25cm} \(fr = 1\) (one read)

\vspace{0.15cm}

\textbf{while} i \textcolor{redPlot}{\textbf{\textless=}} \(\sqrt{n}\)  \textbf{do} \hspace{2.45cm} \(fr = \sqrt{n}+1\) (check the while condition \(\sqrt{n}+1\) times)


\vspace{0.15cm}
cmpt $\gets$ cmpt \textcolor{redPlot}{ \textbf{+}} 1 \hspace{2.6cm} \(fr = 2\) (one affectation and one arithmetic operation (2) , repeated once because N is Prime)

\vspace{0.15cm}
i $\gets$ i \textcolor{redPlot}{ \textbf{+}} 1 \hspace{3.95cm} \(fr = 2\times\sqrt{n}\) (one affectation and one arithmetic operation (2) , inside a while that loops \(\sqrt{n}\))

\vspace{0.15cm}

\textcolor{purplePlot!80!black}{print}(\textcolor{blueArea!60!black}{'Prime Number'}) \hspace{1.95cm} \(fr = 1\) (one print)

\vspace{0.75cm}
\begin{align*}
f_5(n) &= \sum fr \\
       &= 1 + 1 + 1 + 1 + (\sqrt{n}+1) + 2 + 2\times\sqrt{n} + 1 \\
     &= 3\times\sqrt{n} + 8 \\
     &= \boxed{3\times\sqrt{n} + 8}
\end{align*}

\vspace{0.5cm}
now that we have the complexity function \(f_5(n)\) we need to find the \(T_5(n)\)  we have \(T_5(n) = f_5(n) \times \Delta t\)

\begin{align*}
T_5(n) &= f_5(n) \times \Delta t\\ 
&= (3\times\sqrt{n} + 8) \times \Delta t \\
&= \underbrace{3\Delta t}_{a_5} \sqrt{n} + \underbrace{\Delta t 8}_{b_5} \\
&= \boxed{a_5\sqrt{n}+b_5} 
\end{align*}

\subsubsection*{\underline{Worst Case (N is not prime)}}

\vspace{0.5cm}
cmpt $\gets$ 0  \hspace{4cm} \(fr = 1\) (one affectation)

\vspace{0.15cm}
i $\gets$ 1  \hspace{4.65cm} \(fr = 1\) (one affectation)

\vspace{0.15cm}
\textcolor{purplePlot!80!black}{print}(\textcolor{blueArea!60!black}{'Input Integer N\(>\)1 : '})  \hspace{0.95cm} \(fr = 1\) (one print)

\vspace{0.15cm}
\textcolor{purplePlot!80!black}{Read}(n)  \hspace{4.25cm} \(fr = 1\) (one read)

\vspace{0.15cm}

\textbf{while} i \textcolor{redPlot}{\textbf{\textless=}} \(\sqrt{n}\)  \textbf{do} \hspace{2.45cm} \(fr = \sqrt{n}+1\) (check the while condition \(\sqrt{n}+1\) times)


\vspace{0.15cm}
cmpt $\gets$ cmpt \textcolor{redPlot}{ \textbf{+}} 1 \hspace{2.6cm} \(fr = 2\times\sqrt{n}\) (one affectation and one arithmetic operation (2) , repeated at worst \(\sqrt{n}\))

\vspace{0.15cm}
i $\gets$ i \textcolor{redPlot}{ \textbf{+}} 1 \hspace{3.95cm} \(fr = 2\times\sqrt{n}\) (one affectation and one arithmetic operation (2) , inside a while that loops \(\sqrt{n}\))

\vspace{0.15cm}

\textcolor{purplePlot!80!black}{print}(\textcolor{blueArea!60!black}{'Prime Number'}) \hspace{1.95cm} \(fr = 1\) (one print)

\vspace{0.75cm}

\begin{align*}
f_6(n) &= \sum fr \\
       &= 1 + 1 + 1 + 1 + (\sqrt{n}+1) + 2\sqrt{n} + 2\sqrt{n} + 1 \\
       &= 5\sqrt{n} + 6 \\
       &= \boxed{5\sqrt{n} + 6}
\end{align*}

\vspace{0.5cm} 
now that we have the complexity function \(f_4(n)\) we need to find the \(T_6(n)\)  we have \(T_6(n) = f_6(n) \times \Delta t\)

\begin{align*}
T_4(n) &= f_4(n) \times \Delta t\\ 
       &= (5\sqrt{n} + 6) \times \Delta t \\
       &= \underbrace{5 \Delta t}_{a_6} \sqrt{n} + \underbrace{\Delta t 6}_{b_6} \\
&= \boxed{a_6\sqrt{n}+b_6} 
\end{align*}

\subsubsection*{\underline{Conclusion}}
Both \(T_5(n)\) and \(T_6(n)\) are square root complexity therefore they both $\sim$ \(O(\sqrt{n})\)




\section{Installation Office 2013}

\begin{center}
    \includegraphics[width=0.8\textwidth]{Question/SC/3-.PNG}
\end{center}

\vspace{0.35cm}

\begin{prettyBox}{L'installation}{myblue}
Installation via le lien : \href{https://www.malavida.com/en/soft/microsoft-office-2013/download}{https://www.malavida.com/en/soft/microsoft-office-2013/download} , et on aura 
besoin de 7-zip pour unzip le fichier : \href{https://www.7-zip.org/}{https://www.7-zip.org/} , apres telechargement on mount le fichier.
\end{prettyBox}


\subsubsection*{4.}
On affiche la structure des tables ALL\_TAB\_COLUMNS,USER\_TAB\_COLUMNS avec desc

\lstinputlisting[style=sqlstyle]{SQL/Partie5/descr.sql}

\begin{center}
    \includegraphics[width=\textwidth]{ScreenShot/Partie5/desc1.png}
\end{center}

\begin{center}
    \includegraphics[width=\textwidth]{ScreenShot/Partie5/desc5.png}
\end{center}

\begin{prettyBox}{Difference}{myblue}
les deux tables ont les memes attributes , la seul difference est que la table ALL\_TAB\_COLUMNS a un attribut owner en plus compare a USER\_TAB\_COLUMNS
\end{prettyBox}

\subsubsection*{5.}
On doit d'abord se connecter en tant que DBAIOT, puis afficher l'attribut \texttt{table\_name}
de la table \texttt{USER\_TABLES}, qui contient les tables appartenant à l'utilisateur actuellement connecté.

\lstinputlisting[style=sqlstyle]{SQL/Partie5/q5.sql}

\begin{center}
    \includegraphics[width=\textwidth]{ScreenShot/Partie5/q5.png}
\end{center}




\newpage
\exer{2}
\section{Connectivité Des Machines Virtuelles}

\begin{center}
    \includegraphics[width=0.8\textwidth]{Question/SC/1-ping1.PNG}
\end{center}

\vspace{0.25cm}

\begin{center}
    \includegraphics[width=0.8\textwidth]{Question/SC/1-ping2.PNG}
\end{center}

\vspace{0.35cm}

\begin{prettyBox}{Connectivité}{myblue}
D'après les captures d'écran, on remarque que les adresses IP des machines sont :
\begin{itemize}
    \item \textbf{Windows : } \texttt{192.168.100.13}
    \item \textbf{Kali Linux : } \texttt{192.168.100.15}
\end{itemize}
Et le ping est réussi, donc il y a une connectivité.
\end{prettyBox}


\subsection*{Insert}
i can't put the insert queries in this pdf because they have a lot of lines , you can find all the insert sql code inside the SQL/EX1/insert/ folder


we have \(f(n) = \sum fr\) since \(f(n)\) is sum of frequency of execution (\(fr\)) we need to figure out the
\(fr\) of each instruction and sum them : 

\subsubsection*{\underline{Best Case (N is prime)}}

\vspace{0.5cm}
cmpt $\gets$ 0  \hspace{4cm} \(fr = 1\) (one affectation)

\vspace{0.15cm}
i $\gets$ 1  \hspace{4.65cm} \(fr = 1\) (one affectation)

\vspace{0.15cm}
\textcolor{purplePlot!80!black}{print}(\textcolor{blueArea!60!black}{'Input Integer N\(>\)1 : '})  \hspace{0.95cm} \(fr = 1\) (one print)

\vspace{0.15cm}
\textcolor{purplePlot!80!black}{Read}(n)  \hspace{4.25cm} \(fr = 1\) (one read)

\vspace{0.15cm}

\textbf{while} i \textcolor{redPlot}{\textbf{\textless=}} \(\sqrt{n}\)  \textbf{do} \hspace{2.45cm} \(fr = \sqrt{n}+1\) (check the while condition \(\sqrt{n}+1\) times)


\vspace{0.15cm}
cmpt $\gets$ cmpt \textcolor{redPlot}{ \textbf{+}} 1 \hspace{2.6cm} \(fr = 2\) (one affectation and one arithmetic operation (2) , repeated once because N is Prime)

\vspace{0.15cm}
i $\gets$ i \textcolor{redPlot}{ \textbf{+}} 1 \hspace{3.95cm} \(fr = 2\times\sqrt{n}\) (one affectation and one arithmetic operation (2) , inside a while that loops \(\sqrt{n}\))

\vspace{0.15cm}

\textcolor{purplePlot!80!black}{print}(\textcolor{blueArea!60!black}{'Prime Number'}) \hspace{1.95cm} \(fr = 1\) (one print)

\vspace{0.75cm}
\begin{align*}
f_5(n) &= \sum fr \\
       &= 1 + 1 + 1 + 1 + (\sqrt{n}+1) + 2 + 2\times\sqrt{n} + 1 \\
     &= 3\times\sqrt{n} + 8 \\
     &= \boxed{3\times\sqrt{n} + 8}
\end{align*}

\vspace{0.5cm}
now that we have the complexity function \(f_5(n)\) we need to find the \(T_5(n)\)  we have \(T_5(n) = f_5(n) \times \Delta t\)

\begin{align*}
T_5(n) &= f_5(n) \times \Delta t\\ 
&= (3\times\sqrt{n} + 8) \times \Delta t \\
&= \underbrace{3\Delta t}_{a_5} \sqrt{n} + \underbrace{\Delta t 8}_{b_5} \\
&= \boxed{a_5\sqrt{n}+b_5} 
\end{align*}

\subsubsection*{\underline{Worst Case (N is not prime)}}

\vspace{0.5cm}
cmpt $\gets$ 0  \hspace{4cm} \(fr = 1\) (one affectation)

\vspace{0.15cm}
i $\gets$ 1  \hspace{4.65cm} \(fr = 1\) (one affectation)

\vspace{0.15cm}
\textcolor{purplePlot!80!black}{print}(\textcolor{blueArea!60!black}{'Input Integer N\(>\)1 : '})  \hspace{0.95cm} \(fr = 1\) (one print)

\vspace{0.15cm}
\textcolor{purplePlot!80!black}{Read}(n)  \hspace{4.25cm} \(fr = 1\) (one read)

\vspace{0.15cm}

\textbf{while} i \textcolor{redPlot}{\textbf{\textless=}} \(\sqrt{n}\)  \textbf{do} \hspace{2.45cm} \(fr = \sqrt{n}+1\) (check the while condition \(\sqrt{n}+1\) times)


\vspace{0.15cm}
cmpt $\gets$ cmpt \textcolor{redPlot}{ \textbf{+}} 1 \hspace{2.6cm} \(fr = 2\times\sqrt{n}\) (one affectation and one arithmetic operation (2) , repeated at worst \(\sqrt{n}\))

\vspace{0.15cm}
i $\gets$ i \textcolor{redPlot}{ \textbf{+}} 1 \hspace{3.95cm} \(fr = 2\times\sqrt{n}\) (one affectation and one arithmetic operation (2) , inside a while that loops \(\sqrt{n}\))

\vspace{0.15cm}

\textcolor{purplePlot!80!black}{print}(\textcolor{blueArea!60!black}{'Prime Number'}) \hspace{1.95cm} \(fr = 1\) (one print)

\vspace{0.75cm}

\begin{align*}
f_6(n) &= \sum fr \\
       &= 1 + 1 + 1 + 1 + (\sqrt{n}+1) + 2\sqrt{n} + 2\sqrt{n} + 1 \\
       &= 5\sqrt{n} + 6 \\
       &= \boxed{5\sqrt{n} + 6}
\end{align*}

\vspace{0.5cm} 
now that we have the complexity function \(f_4(n)\) we need to find the \(T_6(n)\)  we have \(T_6(n) = f_6(n) \times \Delta t\)

\begin{align*}
T_4(n) &= f_4(n) \times \Delta t\\ 
       &= (5\sqrt{n} + 6) \times \Delta t \\
       &= \underbrace{5 \Delta t}_{a_6} \sqrt{n} + \underbrace{\Delta t 6}_{b_6} \\
&= \boxed{a_6\sqrt{n}+b_6} 
\end{align*}

\subsubsection*{\underline{Conclusion}}
Both \(T_5(n)\) and \(T_6(n)\) are square root complexity therefore they both $\sim$ \(O(\sqrt{n})\)




\section{Installation Office 2013}

\begin{center}
    \includegraphics[width=0.8\textwidth]{Question/SC/3-.PNG}
\end{center}

\vspace{0.35cm}

\begin{prettyBox}{L'installation}{myblue}
Installation via le lien : \href{https://www.malavida.com/en/soft/microsoft-office-2013/download}{https://www.malavida.com/en/soft/microsoft-office-2013/download} , et on aura 
besoin de 7-zip pour unzip le fichier : \href{https://www.7-zip.org/}{https://www.7-zip.org/} , apres telechargement on mount le fichier.
\end{prettyBox}


\exer{3}

\exer{4}
\end{document}
