\subsection*{Chefs That Worked In Maximums Of Differents Restaurants}

\lstinputlisting{Questions/SQL/EX2/q3.sql}

\begin{prettyBox}{Nested Select}{myblue}
\textbf{\underline{First Inner Query}}
\begin{lstlisting}
select count(distinct s.numr) from servicetp5 s where s.numc= c.numc
\end{lstlisting}
Calculate the number of distinct restaurant a chef (s.numc=c.numc) has worked in 

\vspace{0.15cm}

\textbf{\underline{Second Inner Query}}
\begin{lstlisting}
select max(count(distinct s.numr)) from servicetp5 s group by s.numc
\end{lstlisting}
Calculate the max number of distinct restaurant chefs have worked in , \textbf{note} the group by
is curcial so we calculate number of distinct restaurant for each chef then pass it to max 

\vspace{0.15cm}

\textbf{\underline{Outer Query}}
\begin{lstlisting}
create view vmaxcheftp5 as select c.nomc , 
( number of distinct restaurants a chef has worked in ) max from cheftp5 c 
where ( number of distinct restaurants a chef has worked in ) = 
(max number of distint restaurants chefs have worked in );
\end{lstlisting}
Creating a view that fetches name of chef that worked in maximum distinct restraurants , \textbf{note} we have to redo the query
to be able to print the number because where can't acess aliases because they are executed before select
\end{prettyBox}

\vspace{0.25cm}
\begin{prettyBox}{Can We Insert On VMaxChefTP5}{myblue}
No we can't insert because the select has aggregation function count , max
\end{prettyBox}
