\documentclass[fleqn]{article}
\usepackage{tikz,tcolorbox}
\usepackage{array} % For customizing tables
\usepackage{booktabs} % For better horizontal lines
\usepackage[a4paper, paperwidth=25cm, paperheight=25.5cm, left=2cm, right=2cm, top=2cm, bottom=2cm]{geometry}
\usepackage{multicol}
\usepackage{amsmath}
\usepackage{pgfplots}

\pgfplotsset{compat=1.18}
\usepackage{makecell}
\usetikzlibrary{patterns}
\definecolor{mygreen}{HTML}{14C877}
\definecolor{orangePlot}{HTML}{EA6E12}
\definecolor{purplePlot}{HTML}{4C12EA}
\definecolor{blueArea}{HTML}{10D9EE}
\definecolor{redPlot}{HTML}{ED014A}
\definecolor{myblue}{HTML}{338AC7}
\definecolor{p}{HTML}{D813E7}
\definecolor{y}{HTML}{F5F806}
\usepackage{amssymb}
\setlength{\parindent}{0pt}
\setcellgapes{3pt}  % Adjust padding as needed
\makegapedcells
\tcbuselibrary{skins, breakable, theorems}
\usepackage{algorithm}
\usepackage{algpseudocode}
\usepackage{xcolor}
\setlength{\mathindent}{8cm}
\renewcommand{\thealgorithm}{}
\usepackage{mathtools}
\newtcolorbox{prettyBox}[2]{
  enhanced,
  colback=white!90!#2,   % Background color based on the second parameter (color)
  colframe=#2!60!black,  % Frame color based on the second parameter (color)
  coltitle=white,        % Title color (white)
  fonttitle=\bfseries\Large,
  title=#1,              % Title from the first parameter
  boxrule=1mm,
  arc=0.5mm,
  drop shadow=#2!35!gray, % Drop shadow color based on the second parameter (color)
}

\algrenewcommand\algorithmicif{\textcolor{redPlot}{\textbf{if}}}
\algrenewcommand\algorithmicelse{\textcolor{redPlot}{\textbf{else}}}
\algrenewcommand\algorithmicprocedure{\textcolor{redPlot}{\textbf{procedure}}}

\algrenewcommand\algorithmicfunction{\textcolor{redPlot}{\textbf{function}}}
\algrenewcommand\algorithmicend{\textcolor{redPlot}{\textbf{end}}}
\algrenewcommand\algorithmicthen{\textcolor{redPlot}{\textbf{then}}}

\algrenewcommand\algorithmicdo{\textcolor{redPlot}{\textbf{do}}}
\algrenewcommand\algorithmicwhile{\textcolor{redPlot}{\textbf{while}}}

\algrenewcommand\algorithmicfor{\textcolor{redPlot}{\textbf{for}}}



\begin{document}
\renewcommand{\arrayrulewidth}{0.75mm} % Set line thickness
\setlength{\tabcolsep}{12pt} % Set horizontal padding
\renewcommand{\arraystretch}{1.5} % Set vertical padding (1.0 is default)

\begin{center}
    \Huge{\textbf{\underline{Mathemathical Formulas}}}
\end{center}

\vspace{0.65cm}
\begin{enumerate}
    \item \(\sum_{i=1}^{n} 1 = n\)
    \vspace{0.15cm}
    \item \(\sum_{i=1}^{n} i = \frac{n(n+1)}{2}\)
    \vspace{0.15cm}
    \item \(\sum_{i=1}^{n} i^2 = \frac{n(n+1)(2n+1)}{6}\)
    \vspace{0.15cm}
    \item \(\sum_{i=1}^{n} i^3 =  (\sum_{i=1}^{n} i)^2  \)
    \vspace{0.15cm}
    \item \(\sum_{i=0}^{k} 2^i =  2^{k+1} - 1  \)
    \vspace{0.15cm}
    \item \(\sum_{i=0}^{k} \frac{1}{2^i} =  2\)\hspace{0.15cm} when \(k \to +\infty\)
    \vspace{0.15cm}
    \item \(\sum_{i=0}^{n} x^i =  \frac{x^{n+1}-1}{x-1}\)\hspace{0.15cm} with \(x \neq 1\)
    \vspace{0.15cm}
    \item \(\sum_{k=1}^{n} \frac{1}{k} =  1+\frac{1}{2}+\frac{1}{3}+\dots+\frac{1}{n} \sim \ln(n) \)
    \vspace{0.15cm}
    \item \(\log(n!) \sim n\log(n)\)
    \vspace{0.15cm}
    \item \(\lim\limits_{n\to +\infty} \sum_{k = 0}^n x^k  = \frac{1}{1-x}\)\hspace{0.15cm} if \(|x| < 1\) 
    \vspace{0.15cm}                                                                    
    \item \(\sum_{i = 0}^n ix^i  = \frac{1}{(1-x)^2}\)\hspace{0.15cm} if \(|x| < 1\) 
    \vspace{0.15cm}
    \item \(\log(ab) = \log(a) + \log(b)\)
    \vspace{0.15cm}
    \item \(\log(\frac{a}{b}) = \log(a) - \log(b)\) 
    \vspace{0.15cm}
    \item \(\log(a^b) = b\log(a)\)
    \vspace{0.15cm}
    \item \(\log_b(a) = \frac{\ln(a)}{\ln(b)}\) 
    \vspace{0.15cm}
    \item \(a^{\log_b(n)} = n^{\log_b(a)}\) 
    \vspace{0.15cm}
    \item \(a^n\times a^m = a^{n+m}\)
    \vspace{0.15cm}
    \item \((a^m)^n = a^{m*n}\)
    \vspace{0.15cm}
\item \(a^{(m)^{n}} \neq (a^{m})^n\)
\end{enumerate}


\newpage
\begin{center}
    \Huge{\textbf{\underline{How To Calculate Sums}}}
\end{center}

\vspace{0.65cm}
\section{Nested Sum}

\begin{prettyBox}{Calculate Nested Sum}{mygreen}
We always start from the most inner sum until we finish calculating 
\end{prettyBox}

\vspace{0.25cm}
\subsubsection*{\underline{Example :}}
\begin{align*}
    \sum_{i=1}^{n} \sum_{j=1}^{i} \sum_{k=1}^{j} 1 &= \sum_{i=1}^{n} \sum_{j=1}^i j\\[0.15cm] 
                                                   &= \sum_{i=1}^{n} \frac{i(i+1)}{2}\\[0.15cm]
                                                   &= \frac{1}{2}\sum_{i=1}^{n} i(i+1)\\[0.15cm]
                                                   &= \frac{1}{2}(\sum_{i=1}^{n} i^2 + \sum_{i=1}^{n} i)\\[0.15cm]
                                                   &= \frac{1}{2}(\frac{n(n+1)(2n+1)}{6} + \frac{n(n+1)}{2})\\[0.15cm]
                                                 &=\frac{n(n+1)(2n+4)}{12}
\end{align*}

\vspace{0.5cm}
\newpage
\section{Adjusting Boundaries To Match A Known Sum}
\begin{prettyBox}{Adjusting Boundaries}{mygreen}
We might come across a known sum where the boundaries aren't the same as the rule.
All we need to do is calculate the sum with the boundaries of the rule and then add
or subtract the term accordingly.
\end{prettyBox}

\subsubsection*{\underline{Example:}}
\begin{align*}
    \sum_{i=0}^{k-1} 2^i &= \sum_{i=0}^{k} 2^i - \sum_{i=k}^{k} 2^i \\[0.15cm]
                          &= 2^{k+1} - 1 - 2^{k}
\end{align*}

\vspace{0.25cm}
We have a sum that, by the rule, goes till the \(k\)-th term, but here it goes up to the \((k-1)\)-th term.
So, to get the sum up to \((k-1)\), we take the sum up to \(k\) and remove the last term.

\vspace{0.75cm}

\hspace{6cm}\(\text{term}(0) + \text{term}(1) + \dots + \text{term}(k-1) + \text{term}(k) = \sum_{i=0}^{k} 2^i\)
\[\sum_{i=0}^{k-1} 2^i + \text{term}(k) = \sum_{i=0}^{k} 2^i \]
\[\sum_{i=0}^{k-1} 2^i = \sum_{i=0}^{k} 2^i - \text{term}(k)\]


\newpage
\begin{center}
    \Huge{\textbf{\underline{Big Notation}}}
\end{center}


\vspace{0.5cm}
\begin{prettyBox}{Notations}{mygreen}
\begin{itemize}
    \item \textbf{Big O} : \(O\) upper bound
    \item \textbf{Big Omega}: \(\Omega\) lower bound 
    \item \textbf{Big Theta} : \(\Theta\) average
\end{itemize}
\end{prettyBox}

\vspace{0.5cm}

\begin{prettyBox}{Limit Definition}{mygreen}
    if \quad\(\lim\limits_{n \to +\infty}\left| \frac{f(n)}{g(n)}\right| =\)\hspace{0.15cm}\(k\)\quad,\quad\( k > \)0  \quad \(=> \) \quad \(f(n) \in \Theta (g(n))\)

    \vspace{0.25cm}
    if \quad\(\lim\limits_{n \to +\infty}\left| \frac{f(n)}{g(n)}\right| =\)\hspace{0.15cm}\(0\) \hspace{1.45cm} \quad \(=> \) \quad \(f(n) \in O (g(n))\)
    
    \vspace{0.25cm}
    if \quad\(\lim\limits_{n \to +\infty}\left| \frac{f(n)}{g(n)}\right| =\)\hspace{0.15cm}\(+\infty\) \hspace{1cm} \quad \(=> \) \quad \(f(n) \in \Omega (g(n))\)
\end{prettyBox}


\vspace{0.5cm}

\begin{prettyBox}{Enequalities Definition}{mygreen}
    \(f(n) = \Omega(g(n))\) \quad \( \exists c > 0\)\hspace{0.1cm},\quad \(c . g(n) \leq f(n)\) \quad ,\hspace{0.1cm}\(\forall n \geq n_0\)


\vspace{0.15cm}
\(f(n) = O(g(n))\) \quad \(\exists c > 0\)\hspace{0.1cm},\quad\(c . g(n) \geq f(n)\) \quad ,\hspace{0.1cm}\(\forall n \geq n_0\)

\vspace{0.15cm}
\(f(n) = \Theta(g(n))\) \quad \(\exists c_1 > 0\)\hspace{0.1cm},\hspace{0.1cm}\(\exists c_2 > 0\)\hspace{0.1cm},\quad \(c_1 . g(n) \leq f(n) \leq c_2. g(n)\)\quad  ,\hspace{0.1cm}\(\forall n \geq n_0\)

\end{prettyBox}

\vspace{0.5cm}
\begin{prettyBox}{Note}{red}
\begin{itemize}
    \item \(c\) is a strictly positive real number.
    \item \(n\) is a strictly positive integer.
\end{itemize}
\end{prettyBox}

\newpage
\subsubsection*{\underline{Example}}
\vspace{0.25cm}
\(\sqrt{2\log_2{n}+3+7n} \sim O(\sqrt{n})\)

\begin{align*}
\sqrt{2\log_2{n}+3+7n} &\leq \sqrt{n} \cdot c\\[0.25cm]  
\sqrt{\frac{2\log_2{n}+3+7n}{n}} &\leq c\\[0.25cm]
\sqrt{\frac{2\log_2{n}}{n}+\frac{3}{n}+7} &\leq c
\end{align*}

\[
\begin{rcases}
\frac{2\log_2{n}}{n} \leq 1 \quad &\Longrightarrow \quad n\geq 4\\[0.25cm]
\frac{3}{n} \leq 1 \quad &\Longrightarrow \quad n\geq 3 
\end{rcases}\longrightarrow \boxed{n_0 = 4}
\]

\vspace{0.15cm}

\begin{center}
    \(\sqrt{1+1+7} \leq c\)\\[0.2cm]
    \(\sqrt{9} \leq c  \longrightarrow \boxed{c = \sqrt{9}}\)
\end{center}


\vspace{1cm}
\(2^n \sim \Omega(5^{\log_e(n)})\)

\begin{align*}
2^n &\geq 5^{\log_e(n)}\cdot c\\[0.25cm]  
2^n &\geq n^{\log_e(5)}\cdot c\\[0.25cm] 
2^n &\geq n^{1.6}\cdot c\\[0.25cm]  
\frac{2^n}{n^{1.6}} &\geq  c\\[0.25cm] 
\end{align*}

\begin{center}
\(\frac{2^n}{n^{1.6}} \geq 1 \Longrightarrow n\geq 1 \longrightarrow \boxed{n_0 = 1}\hspace{0.15cm} \text{and} \hspace{0.15cm} \boxed{c = 1}\)
\end{center}

\newpage

\(n^{2^n} + 6\cdot2^n  \sim \Theta(n^{2^n})\)\\[0.15cm]

\begin{center}
 \(c_1\cdot n^{2^n} \leq n^{2^n} + 6\cdot2^n \leq c_2 \cdot n^{2^n}  \)\\[0.15cm]
\(c_1 \leq 1 + \frac{6\cdot2^n}{n^{2^n}} \leq c_2\)

   
\end{center}

\begin{center}
\( \frac{6\cdot2^n}{n^{2^n}} \leq 1 \Longrightarrow n\geq 3 \longrightarrow \boxed{n_0 = 3} \)\\[0.4cm]
\(c_1 \leq 2 \leq c_2 \longrightarrow \boxed{c_1 = 1} \hspace{0.15cm} \text{and} \hspace{0.15cm} \boxed{c_2 =2}\)
\end{center}




\vspace{1.5cm}
\begin{center}
    \Huge{\textbf{\underline{Induction}}}
\end{center}

\vspace{0.35cm}
\begin{prettyBox}{Induction's Steps}{mygreen}
\begin{enumerate}
    \item Verify for \(n =\) term
    \item Assume true for \(n\)
    \item Prove for \(n+1\)
\end{enumerate}
\end{prettyBox}

\vspace{0.35cm}
\subsubsection*{\underline{Example}}
\(a_1 = 1\)\hspace{0.5cm}\(a_2 = 1\)\\[0.25cm]
\(a_n = a_{n-1} + a_{n-2}\) \hspace{0.25cm} if \(n \geq 3\)\\[0.35cm]
Prove That : \(a_n = \frac{(\frac{1+\sqrt{5}}{2})^n - (\frac{1-\sqrt{5}}{2})^n}{\sqrt{5}}\)\\[0.5cm]

\subsubsection*{Verify For n = 1}

\vspace{0.35cm}
\[
\begin{rcases}
a_1  = 1\\[0.25cm]
a_1 = \frac{(\frac{1+\sqrt{5}}{2})^1 - (\frac{1-\sqrt{5}}{2})^1}{\sqrt{5}} = 1\hspace{0.1cm}
\end{rcases}\longrightarrow \text{True For} \hspace{0.25cm} n=1
\]

\newpage

\subsubsection*{Assume True For n And Prove For n+1}
\vspace{0.15cm}
\(a_{n+1} = a_{n} + a_{n-1}\) \hspace{0.25cm} if \(n \geq 3\)\\[0.35cm]
Prove That : \(a_{n+1} = \frac{\left(\frac{1+\sqrt{5}}{2}\right)^{n+1} - \left(\frac{1-\sqrt{5}}{2}\right)^{n+1}}{\sqrt{5}}\) 


\begin{align*}
a_{n+1} &= a_{n} + a_{n-1} \\[0.3cm]
&= \frac{\left(\frac{1+\sqrt{5}}{2}\right)^n - \left(\frac{1-\sqrt{5}}{2}\right)^n}{\sqrt{5}}+ \frac{\left(\frac{1+\sqrt{5}}{2}\right)^{n-1} - \left(\frac{1-\sqrt{5}}{2}\right)^{n-1}}{\sqrt{5}}\\[0.3cm]
&= \frac{\left(\frac{1+\sqrt{5}}{2}\right)^{n} \cdot   \left(1 + \frac{2}{1+\sqrt{5}}\right)    -  \left(\frac{1-\sqrt{5}}{2}\right)^{n} \cdot \left(1 + \frac{2}{1-\sqrt{5}}\right)}{\sqrt{5}}\\[0.3cm]
&=\frac{\left(\frac{1+\sqrt{5}}{2}\right)^{n} \cdot   \left(\frac{(3+\sqrt{5})(1-\sqrt{5})}{(1+\sqrt{5})(1-\sqrt{5})}\right)    -  \left(\frac{1-\sqrt{5}}{2}\right)^{n} \cdot \left(\frac{(3-\sqrt{5})(1+\sqrt{5})}{(1-\sqrt{5})(1+\sqrt{5})}\right)}{\sqrt{5}}\\[0.3cm] 
&=\frac{\left(\frac{1+\sqrt{5}}{2}\right)^{n} \cdot   \left(\frac{1+\sqrt{5}}{2}\right)    -  \left(\frac{1-\sqrt{5}}{2}\right)^{n} \cdot \left(\frac{1-\sqrt{5}}{2}\right)}{\sqrt{5}}\\[0.3cm]
&= \frac{\left(\frac{1+\sqrt{5}}{2}\right)^{n+1} - \left(\frac{1-\sqrt{5}}{2}\right)^{n+1}}{\sqrt{5}}\\[0.2cm]
&= a_{n+1}
\end{align*}


True For \(n+1\)

\subsubsection*{Conclusion True For n}



\vspace{1.5cm}
\begin{center}
    \Huge{\textbf{\underline{Order Of Complexity}}}
\end{center}



\vspace{0.45cm}
\begin{prettyBox}{Order}{mygreen}
\begin{center}
\(1\leq \log(n) \leq \sqrt{n} \leq n \leq n\log(n) \leq n^2 \leq n^3 \leq 2^n \leq n! \leq n^n\)
\end{center}
\end{prettyBox}



\vspace{1cm}

\begin{center}
    \Huge{\textbf{\underline{Complexity}}}
\end{center}

\vspace{0.6cm}

\begin{prettyBox}{Some Terminology}{mygreen}
\begin{itemize}
    \item \textbf{Frequency of Execution (\(F_r\))}: The number of times an instruction is executed.
    \item \textbf{Execution Time of Basic Instructions (\(\Delta t\))}: We assume that all basic instructions
(such as print, read, assignment, arithmetic operations, etc.) have the same execution time, denoted as \(\Delta t\).
    \item \textbf{Function (\(f(n)\))}: Represents the total frequency of the program's instructions in relation to the data size \(n\).
    \item \textbf{Exact Execution Time Function (\(T(n)\))}: The execution time function in relation to the data size \(n\).
    \item \textbf{Approximate Theoretical Complexity (Asymptotic)}: 
It approximates \(T(n)\) by omitting all constants and taking the term with the highest growth rate.
\end{itemize}
\end{prettyBox}

\vspace{0.1cm}
\subsubsection*{\underline{Example}}

\begin{algorithm}
\caption{Sum of First N Integers}
\begin{algorithmic}[1]
\State \textbf{\textcolor{redPlot}{Var}}
\State n, sum, i \textcolor{blue}{integer};
\vspace{0.5em}
\State  \textbf{\textcolor{redPlot}{Begin}}
\State sum $\gets$ 0; 
\State  i $\gets$ 1;
\vspace{0.5em}
\State \textcolor{purplePlot!80!black}{print}(\textcolor{blueArea!60!black}{'Input Integer N : '}) 
\State \textcolor{purplePlot!80!black}{Read}(n);
\vspace{0.5em}
\While{i \textcolor{redPlot}{\textbf{\textless=}} n}
\State sum $\gets$ sum \textcolor{redPlot}{ \textbf{+}} i; 
\State i $\gets$ i \textcolor{redPlot}{ \textbf{+}} 1; 
\EndWhile

\vspace{0.5em}
\State \textcolor{purplePlot!80!black}{print}(\textcolor{blueArea!60!black}{'Sum is ' },{sum});
\vspace{0.5em}
\State  \textbf{\textcolor{redPlot}{End}}
\end{algorithmic}
\end{algorithm}

\newpage
\subsection*{\underline{Time Complexity :}}
we have \(f(n) = \sum fr\) since \(f(n)\) is sum of frequency of execution (\(fr\)) we need to figure out the
\(fr\) of each instruction and sum them : 

\vspace{0.5cm}
sum $\gets$ 0  \hspace{4.15cm} \(fr = 1\) (one affectation)

\vspace{0.15cm}
i $\gets$ 1  \hspace{4.65cm} \(fr = 1\) (one affectation)

\vspace{0.15cm}
\textcolor{purplePlot!80!black}{print}(\textcolor{blueArea!60!black}{'Input Integer N : '})  \hspace{1.5cm} \(fr = 1\) (one print)

\vspace{0.15cm}
\textcolor{purplePlot!80!black}{Read}(n)  \hspace{4.25cm} \(fr = 1\) (one read)

\vspace{0.15cm}

\textbf{while} i \textcolor{redPlot}{\textbf{\textless=}} n  \textbf{do} \hspace{2.75cm} \(fr = n+1\) (check the while condition n+1 times)

\vspace{0.15cm}
sum $\gets$ sum \textcolor{redPlot}{ \textbf{+}} i \hspace{3cm} \(fr = 2n\) (one affectation and one arithmetic operation + (2) inside a while that loops n times (2n))

\vspace{0.15cm}
i $\gets$ i \textcolor{redPlot}{ \textbf{+}} 1 \hspace{4cm} \(fr = 2n\) (one affectation and one arithmetic operation + (2) inside a while that loops n times (2n))

\vspace{0.15cm}

\textcolor{purplePlot!80!black}{print}(\textcolor{blueArea!60!black}{'Sum is ' },{sum}) \hspace{2.25cm} \(fr = 1\) (one print)

\vspace{0.75cm}
\begin{align*}
f(n) &= \sum fr \\
     &= 1 + 1 + 1 + 1 + (n+1) + 2n + 2n + 1 \\
     &= 5n + 6 \\
     &= \boxed{5n + 6}
\end{align*}

\vspace{0.5cm}
now that we have the complexity function \(f(n)\) we need to find the \(T(n)\)  we have \(T(n) = f(n) \times \Delta t\)

\begin{align*}
T(n) &= f(n) \times \Delta t\\ 
&= (5n + 6) \times \Delta t \\
&= \underbrace{5 \Delta t}_{a} n + \underbrace{\Delta t 6}_{b} \\
&= \boxed{an+b} 
\end{align*}

\vspace{0.35cm}

In this example the exact theoritical complexity is \(T(n) = an+b\) and its approximate theoritical complexity is 
\(an+b \sim O(n)\) we notice that its time complexity is linear 




\newpage

\begin{center}
    \Huge{\textbf{\underline{Iterative Algorithm}}}
\end{center}

\vspace{0.6cm}


\begin{prettyBox}{Iterative Algorithm}{mygreen}
Iterative algorithm complexity is calculated by summing the frequency of all instructions.
\end{prettyBox}

\vspace{0.35cm}

\begin{prettyBox}{For Loop}{mygreen}
\begin{itemize}
    \item If the index is decremented or incremented by just 1:
    \begin{itemize}
        \item The number of iterations is \textbf{upper bound - lower bound + 1}
        \item Each loop can be represented as a sum and used to calculate complexity
    \end{itemize}
    
    \item If the index is decremented or incremented by a constant \( c \geq 2 \):
    \begin{itemize}
        \item The number of iterations is \( c \cdot k = n \rightarrow k = \frac{n}{c} \)
        \item Cannot be represented as a sum
    \end{itemize}
    
    \item If the index is multiplied or divided by a constant \( c \geq 2 \):
    \begin{itemize}
        \item The number of iterations is \( c^k = n \rightarrow k = \log_c(n) \)
        \item Cannot be represented as a sum
    \end{itemize}
\end{itemize}
\end{prettyBox}


\vspace{0.35cm}
\begin{prettyBox}{While Loop}{mygreen}
\begin{itemize}
        \item Can never be represented as a sum
        \item If the index is incremented or decremented by just 1, the number of iterations is the difference between the upper and lower bounds.
        \item If the index is incremented or decremented by a constant \(c \geq 2\), the number of iterations is \( \frac{n}{c} \).
        \item If the index is multiplied or divided by a constant \(c \geq 2\), the number of iterations is \( \log_c(n) \).
\end{itemize}
\end{prettyBox}

\newpage

\subsubsection*{\underline{Example}}

\begin{algorithm}[ht]
\caption{}
\begin{algorithmic}
    \For{ \( (i = n \hspace{0.15cm} ;\hspace{0.15cm} i\geq 0 \hspace{0.15cm}; \hspace{0.15cm}i \gets i/2) \)}
\State \textcolor{purplePlot!80!black}{print}(\textcolor{blueArea!60!black}{'Index i : '},i) 
\EndFor
\end{algorithmic}
\end{algorithm}

Iteration : 
\(\frac{n}{2^0},\frac{n}{2^1},\frac{n}{2^2},\frac{n}{2^3},\dots,\frac{n}{2^k}\)\\[0.25cm]
Loop stops when \(i < 0 \Rightarrow \frac{n}{2^k} < 0\)
since it's an natural division it means :
\(2^k > n \Longrightarrow k > \log_{2}(n) \Longrightarrow k= \lceil \log_{2}(n) \rceil = \boxed{\log_{2}(n) + 1 } \)\\[0.25cm]
In conslusion \(O(\log(n))\)

\vspace{0.75cm}
\begin{algorithm}[ht]
\caption{}
\begin{algorithmic}
\State \(r \gets 0;\)

\For{ \( (i = 1 \hspace{0.15cm} ;\hspace{0.15cm} i\leq n^2 \hspace{0.15cm}; \hspace{0.15cm}i \gets i+1) \)}
\For{ \( (j = 1 \hspace{0.15cm} ;\hspace{0.15cm} j\leq 2n-1 \hspace{0.15cm}; \hspace{0.15cm}j \gets j+1) \)}
\State \(r\gets r+1;\)
\EndFor
\EndFor
\end{algorithmic}
\end{algorithm}

\begin{align*}
\sum_{i=1}^{n^2} \sum_{j=1}^{2n-1} 1 &=\sum_{i=1}^{n^2} 2n-1\\
                                     &=(2n-1)\cdot n^2\\
                                     &=\boxed{2n^3 - n^2}
\end{align*}

In conslusion \(O(n^3)\)


\newpage
\begin{algorithm}[ht]
\caption{}
\begin{algorithmic}
\State \(r \gets 0;\)

\For{ \((i = 1 \hspace{0.15cm} ;\hspace{0.15cm} i\leq n \hspace{0.15cm}; \hspace{0.15cm}i \gets i+1) \)}
\For{ \((j = i+1 \hspace{0.15cm} ;\hspace{0.15cm} j\leq n \hspace{0.15cm}; \hspace{0.15cm}j \gets j+1) \)}
\For{ \((k = 1 \hspace{0.15cm} ;\hspace{0.15cm} k\leq j \hspace{0.15cm}; \hspace{0.15cm}j \gets j+1) \)}
\State \(r\gets r+1;\)
\EndFor
\EndFor
\EndFor
\end{algorithmic}
\end{algorithm}

\begin{align*}
\sum_{i=1}^n \sum_{j=i+1}^n \sum_{k=1}^{j} 1 &= \sum_{i=1}^n \sum_{j=i+1}^n j\\
                                             &= \sum_{i=1}^n (\sum_{j=1}^n j - \sum_{j=1}^{i} j)\\
                                             &= \sum_{i=1}^n (\frac{n(n+1)}{2} - \frac{i(i+1)}{2})\\
                                             &= \frac{1}{2}(\sum_{i=1}^n n(n+1) - \sum_{i=1}^n i^2 - \sum_{i=1}^n i)\\
                                             &= \frac{1}{2}(n^2(n+1) - \frac{n(n+1)(2n+1)}{6} - \frac{n(n+1)}{2})\\
                                             &= \frac{4n^3 - 3n^2 - 7n}{12}
\end{align*}

In conslusion \(O(n^3)\)

\newpage

\begin{algorithm}[ht]
\caption{}
\begin{algorithmic}
\State \(r \gets 0;i\gets 1\)

\While{ \((i\leq n) \)}
\For{ \((j = n^2 \hspace{0.15cm} ;\hspace{0.15cm} j\leq 5 \hspace{0.15cm}; \hspace{0.15cm}j \gets j-1) \)}
\State \(r\gets r+1;\)
\EndFor
\State \(i \gets i+2;\)
\EndWhile
\end{algorithmic}
\end{algorithm}

For Loop iterates : \(n^2 - 5+1 = \boxed{n^2-4}\)\\[0.35cm]
While Loop :
\(2k>n \Rightarrow k>\frac{n}{2} \Rightarrow k = \lceil \frac{n}{2} \rceil = \boxed{\frac{n}{2} + 1}\)\\[0.35cm]

In conslusion \(O((n^2-4)(\frac{n}{2} + 1)) = O(n^3)\)

\vspace{1cm}

\begin{algorithm}[ht]
\caption{}
\begin{algorithmic}
\State \(i \gets 2;j\gets 1\)

\While{ \((i\leq n) \)}
\State \(i \gets i*i;\)
\EndWhile

\While{ \((j\leq i) \)}
\State \(j \gets 4*j;\)
\EndWhile

\end{algorithmic}
\end{algorithm}

First While Iteration :
\(2,2^{2^{1}},2^{2^{2}}, 2^{2^{3}},\dots,2^{2^{k}}\)\\[0.15cm]
\(2^{2^k} > n \Rightarrow k > \log(\log(n)) \Rightarrow k = \lceil \log(\log(n)) \rceil = \boxed{\log(\log(n)) + 1}\)\\[0.35cm]

Second While Iteration :
\(1,4^1,4^2,4^3,\dots,4^k\)\\[0.15cm]
\(4^k > i \Rightarrow 4^k > n \Rightarrow k>\log(n) \Rightarrow  k = \lceil \log(n) \rceil = \boxed{\log(n) + 1} \)\\[0.35cm]

In Conclusion \(O(\log(\log(n))+1+\log(n)+1) = O(\log(n))\)





\end{document}
