\begin{prettyBox}{Some Terminology}{mygreen}
\begin{itemize}
    \item \textbf{Frequency of Execution (\(F_r\))}: The number of times an instruction is executed.
    \item \textbf{Execution Time of Basic Instructions (\(\Delta t\))}: We assume that all basic instructions
(such as print, read, assignment, arithmetic operations, etc.) have the same execution time, denoted as \(\Delta t\).
    \item \textbf{Function (\(f(n)\))}: Represents the total frequency of the program's instructions in relation to the data size \(n\).
    \item \textbf{Exact Execution Time Function (\(T(n)\))}: The execution time function in relation to the data size \(n\).
    \item \textbf{Approximate Theoretical Complexity (Asymptotic)}: 
It approximates \(T(n)\) by omitting all constants and taking the term with the highest growth rate.
\end{itemize}
\end{prettyBox}

\vspace{0.1cm}
\subsubsection*{\underline{Example}}

\begin{algorithm}
\caption{Sum of First N Integers}
\begin{algorithmic}[1]
\State \textbf{\textcolor{redPlot}{Var}}
\State n, sum, i \textcolor{blue}{integer};
\vspace{0.5em}
\State  \textbf{\textcolor{redPlot}{Begin}}
\State sum $\gets$ 0; 
\State  i $\gets$ 1;
\vspace{0.5em}
\State \textcolor{purplePlot!80!black}{print}(\textcolor{blueArea!60!black}{'Input Integer N : '}) 
\State \textcolor{purplePlot!80!black}{Read}(n);
\vspace{0.5em}
\While{i \textcolor{redPlot}{\textbf{\textless=}} n}
\State sum $\gets$ sum \textcolor{redPlot}{ \textbf{+}} i; 
\State i $\gets$ i \textcolor{redPlot}{ \textbf{+}} 1; 
\EndWhile

\vspace{0.5em}
\State \textcolor{purplePlot!80!black}{print}(\textcolor{blueArea!60!black}{'Sum is ' },{sum});
\vspace{0.5em}
\State  \textbf{\textcolor{redPlot}{End}}
\end{algorithmic}
\end{algorithm}

\newpage
\subsection*{\underline{Time Complexity :}}
we have \(f(n) = \sum fr\) since \(f(n)\) is sum of frequency of execution (\(fr\)) we need to figure out the
\(fr\) of each instruction and sum them : 

\vspace{0.5cm}
sum $\gets$ 0  \hspace{4.15cm} \(fr = 1\) (one affectation)

\vspace{0.15cm}
i $\gets$ 1  \hspace{4.65cm} \(fr = 1\) (one affectation)

\vspace{0.15cm}
\textcolor{purplePlot!80!black}{print}(\textcolor{blueArea!60!black}{'Input Integer N : '})  \hspace{1.5cm} \(fr = 1\) (one print)

\vspace{0.15cm}
\textcolor{purplePlot!80!black}{Read}(n)  \hspace{4.25cm} \(fr = 1\) (one read)

\vspace{0.15cm}

\textbf{while} i \textcolor{redPlot}{\textbf{\textless=}} n  \textbf{do} \hspace{2.75cm} \(fr = n+1\) (check the while condition n+1 times)

\vspace{0.15cm}
sum $\gets$ sum \textcolor{redPlot}{ \textbf{+}} i \hspace{3cm} \(fr = 2n\) (one affectation and one arithmetic operation + (2) inside a while that loops n times (2n))

\vspace{0.15cm}
i $\gets$ i \textcolor{redPlot}{ \textbf{+}} 1 \hspace{4cm} \(fr = 2n\) (one affectation and one arithmetic operation + (2) inside a while that loops n times (2n))

\vspace{0.15cm}

\textcolor{purplePlot!80!black}{print}(\textcolor{blueArea!60!black}{'Sum is ' },{sum}) \hspace{2.25cm} \(fr = 1\) (one print)

\vspace{0.75cm}
\begin{align*}
f(n) &= \sum fr \\
     &= 1 + 1 + 1 + 1 + (n+1) + 2n + 2n + 1 \\
     &= 5n + 6 \\
     &= \boxed{5n + 6}
\end{align*}

\vspace{0.5cm}
now that we have the complexity function \(f(n)\) we need to find the \(T(n)\)  we have \(T(n) = f(n) \times \Delta t\)

\begin{align*}
T(n) &= f(n) \times \Delta t\\ 
&= (5n + 6) \times \Delta t \\
&= \underbrace{5 \Delta t}_{a} n + \underbrace{\Delta t 6}_{b} \\
&= \boxed{an+b} 
\end{align*}

\vspace{0.35cm}

In this example the exact theoritical complexity is \(T(n) = an+b\) and its approximate theoritical complexity is 
\(an+b \sim O(n)\) we notice that its time complexity is linear 


