
\begin{prettyBox}{Recursive Algorithm}{mygreen}
A recursive algorithm is a function that calls itself and includes a base case to terminate the recursion.  
We analyze its complexity using recursive equations.
\end{prettyBox}

\vspace{0.5cm}

\begin{prettyBox}{Recursive Equations}{mygreen}
    \textbf{\underline{Divide and Conquer:}}  
    \[
    T(n) = aT\left(\frac{n}{b}\right) + f(n)
    \]  
    Where: 
    \begin{itemize}
    \item \(a\) : Number of recursive calls at once.  
    \item \(b\) : Size of subproblem.  
    \item \(f(n)\) : Cost of composition/decomposition.  
    \end{itemize}

    \vspace{0.45cm}

    \textbf{\underline{Decrement and Conquer:}}  
    \[
    T(n) = aT\left(n-b\right) + c
    \]  
    Where:  
    \begin{itemize}
    \item \(a\) : Number of recursive calls at once.  
    \item \(b\) : Size of subproblem.  
    \item \(c\) : Cost of performing an operation.  
    \end{itemize}

\end{prettyBox}

\vspace{0.5cm}


\begin{prettyBox}{By Substitution}{mygreen}
The substitution method involves repeatedly substituting into the recursive equation  
until the \(k\)-th step, where the base case is reached.
\end{prettyBox}

\newpage

\subsubsection*{\underline{Example}}
\vspace{0.1cm}
\(T(0) = 1\)\\
\(T(n) = T(n-1) + 2^n\)

\vspace{-0.1cm}
\begin{align*}
    T(n) &= T(n-1) + 2^n\\
         &= T(n-2) + 2^{n-1} + 2^n\\
         &= T(n-3) + 2^{n-2} + 2^{n-1} + 2^{n}\\
         &= T(n-k) + 2^{n-k+1} + \dots + 2^{n-2} + 2^{n-1} + 2^{n}
\end{align*}

\(n-k = 0 \Rightarrow \boxed{k = n}\)

\begin{align*}
    &=T(0) + 2^{1} + \dots + 2^{n-2} + 2^{n-1} + 2^{n}\\
    &=2^{0} + 2^{1} + \dots + 2^{n-2} + 2^{n-1} + 2^{n}\\
    &=\sum_{i=0}^{n} 2^{i}\\
    &=2^{n+1} - 1 = \boxed{O(2^n)}
\end{align*}    

\vspace{0.75cm}
\( T(1) = 1\)\\
\( T(n) = 2n+1+2T(\frac{n}{2}) \)

\begin{align*}
T(n) &= 2n+1+2T(\frac{n}{2})\\
     &= 2n+1+2[2(\frac{n}{2} + 1 + 2T(\frac{n}{2^2}))]\\
     &= 2n+2n+1+2+2^2T(\frac{n}{2^2})\\
     &= 2n+2n+1+2+2^2[2\frac{n}{2^2}+1+2T(\frac{n}{2^3})]\\
     &= 2n+2n+2n+1+2+2^2+2^3T(\frac{n}{2^3})\\
     &= 2n\cdot k +1+2+2^2+\dots+2^{k-1}+2^kT(\frac{n}{2^k})\\
     &= 2n\cdot k+ \sum_{i=0}^{k-1} 2^i+2^kT(\frac{n}{2^k})
\end{align*}


\(\frac{n}{2^k} = 1 \Rightarrow \boxed{k = \log(n)}\)

\begin{align*}
&= 2n\log(n)+ \sum_{i=0}^{k-1} 2^i+nT(1)\\
&= 2n\log(n) + 2^{k+1} - 1 - 2^{k} +n\\
&= 2n\log(n) + 2n - 1 - n+n\\
&= 2n\log(n) + 2n - 1 = \boxed{O(n\log(n))}\\
\end{align*}


\newpage
\(T(1) = 1\)\\
\(T(n) = T(\frac{n}{2}) + \log\log n\)

\begin{align*}
    T(n) &= T(\frac{n}{2}) + \log\log n\\
  &= T(\frac{n}{2^2}) +\log\log(\frac{n}{2}) +\log\log n\\
  &= T(\frac{n}{2^3})+ \log\log(\frac{n}{2^2}) +\log\log(\frac{n}{2}) +\log\log n\\
  &= T(\frac{n}{2^k})+ \log\log(\frac{n}{2^{k-1}})+\dots +\log\log(\frac{n}{2}) +\log\log n\\ 
  &= T(\frac{n}{2^k}) +\sum_{i=0}^{k-1} \log\log(\frac{n}{2^i})\\
  &= T(\frac{n}{2^k}) +\sum_{i=0}^{k-1} \log(\log(n) - \log(2^i))\\
  &= T(\frac{n}{2^k}) +\sum_{i=0}^{k-1} \log(\log(n) - i\log(2))\\
  &= T(\frac{n}{2^k}) +\sum_{i=0}^{k-1} \log(\log(n) - i)\\
\end{align*}


\(\frac{n}{2^k} = 1 \Rightarrow \boxed{k = \log(n)}\)

\begin{align*}
    &= T(1) +\sum_{i=0}^{\log(n)-1} \log(\log(n) - i)\\
    &= 1 + \log(\log(n)) + \log(\log(n) - 1)+ \dots +\log(\log(n) - \log(n) +1)\\
    &= 1 + \log(\log(n)*(\log(n)-1)*\dots*1) \\
    &= 1+\log((\log(n))!)\\
    &= 1+\log(n)\log(\log(n)) = \boxed{ O(\log(n)\log(\log(n)))}
\end{align*}

\vspace{1.5cm}


\begin{prettyBox}{Change of Variable}{mygreen}
To simplify the recursive equation, we substitute \(n\) with a value related to \(m\), transforming \(T(n)\) into \(S(m)\).  
This allows us to analyze the complexity in terms of \(m\).  
Finally, we revert the substitution to express the complexity in terms of \(n\).
\end{prettyBox}

\newpage

\subsubsection*{\underline{Example}}
\(T(1) = 1\)\\
\(T(n) = 2T(\sqrt{n}) + \log(n)\)\\[0.15cm]

let \(n = 2^m\)
\begin{center}
\(T(2^m) = 2T(2^{\frac{m}{2}}) + m\)\\[0.15cm]
\end{center}
let \(T(2^m) = S(m)\)

\begin{center}
\(S(m) = 2S(\frac{m}{2}) + m\)\\[0.15cm]
\end{center}

\vspace{0.5cm}
\begin{center}
This is the same as merge sort so complexity in terms of m is
\(O(m\log(m))\) , in terms of n is \(\boxed{O(\log(n)\log(\log(n)))}\)
\end{center}

\vspace{1cm}


\begin{prettyBox}{Upper \& Lower Bounds}{mygreen}
To analyze the complexity of a recursive equation with multiple calls, we calculate both the upper and lower bounds.  
Ultimately, the result remains consistent.  

\begin{itemize}
    \item \textbf{Lower Bound}: Replace larger terms call with smaller ones.
    \item \textbf{Upper Bound}: Replace smaller terms call with larger ones.  
\end{itemize}
\end{prettyBox}


\subsubsection*{\underline{Example}}

\begin{algorithm}
\caption{Fibonnaci}
\begin{algorithmic}[1]
\Function{Fib\_rec}{(I/n:Integer): Integer}
\State  \textbf{\textcolor{redPlot}{Begin}}
\If{(n = 0)}
\State return 0;
\ElsIf{(n = 1)}
\State return 1;
\Else 
\State return Fib\_rec(n-1)+Fib\_rec(n-2);
\EndIf
\EndFunction
\end{algorithmic}
\end{algorithm}

\(T(0) = T(1) = 0\)\\
\(T(n) = T(n-1) + T(n-2) + 1\)\\

\newpage


\subsubsection*{\underline{Lower Bound}}
\(T(n) = 2T(n-2) + 1\)\\

\begin{align*}
T(n) &= 2T(n-2) + 1\\
     &= 2[2T(n-4)+1] + 1\\
     &= 2^2T(n-4)+2+ 1\\
     &= 2^2[2T(n-6)+1]+2+1\\
     &= 2^3T(n-6)+2^2+2+1\\
     &= 2^kT(n-2k)+\sum_{i=0}^{k-1} 2^i\\
     &= 2^kT(n-2k)+ 2^{k+1} - 1 - 2^{k}\\
\end{align*}

\(n-2k = 0 \Rightarrow \boxed{k =\frac{n}{2}} \)

\begin{align*}
&= 2^{\frac{n}{2}}T(0)+ 2^{\frac{n}{2}+1} - 1 - 2^{\frac{n}{2}}\\
&= 2^{\frac{n}{2}}+ 2^{\frac{n}{2}+1} - 1 - 2^{\frac{n}{2}}\\
&= 2^{\frac{n}{2}+1} - 1  = O(2^n)\\
\end{align*}

\subsubsection*{\underline{Upper Bound}}
\(T(n) = 2T(n-1) + 1\)\\

\begin{align*}
T(n) &= 2T(n-1) + 1\\
&= 2[2T(n-2)+1] + 1\\
&= 2^2T(n-2)+2 + 1\\
&= 2^2[2T(n-3)+1]+2 + 1\\
&= 2^3T(n-3)+2^2+2 + 1\\
&= 2^3T(n-3)+2^2+2 + 1\\
&= 2^kT(n-k)+\sum_{i=0}^{k-1} 2^i\\
&= 2^kT(n-k)+ 2^{k+1} - 1 - 2^k\\
\end{align*}

\newpage

\(n-k = 0 \Rightarrow \boxed{k = n} \)

\begin{align*}
&= 2^{n}T(0)+ 2^{n+1} - 1 - 2^{n}\\
&= 2^{n}+ 2^{n+1} - 1 - 2^{n}\\
&= 2^{n+1} - 1  = O(2^n)\\
\end{align*}

\begin{center}
\boxed{Lower Bound = Upper Bound = \(O(2^n) \Rightarrow \) Complexity = \(O(2^n)\) }
\end{center}

\begin{prettyBox}{Tree}{mygreen}
    prettyBox
\end{prettyBox}

\begin{prettyBox}{Guess \& Test}{mygreen}
    prettyBox
\end{prettyBox}
