
\begin{prettyBox}{Iterative Algorithm}{mygreen}
Iterative algorithm complexity is calculated by summing the frequency of all instructions.
\end{prettyBox}

\vspace{0.35cm}

\begin{prettyBox}{For Loop}{mygreen}
\begin{itemize}
    \item If the index is decremented or incremented by just 1:
    \begin{itemize}
        \item The number of iterations is \textbf{upper bound - lower bound + 1}
        \item Each loop can be represented as a sum and used to calculate complexity
    \end{itemize}
    
    \item If the index is decremented or incremented by a constant \( c \geq 2 \):
    \begin{itemize}
        \item The number of iterations is \( c \cdot k = n \rightarrow k = \frac{n}{c} \)
        \item Cannot be represented as a sum
    \end{itemize}
    
    \item If the index is multiplied or divided by a constant \( c \geq 2 \):
    \begin{itemize}
        \item The number of iterations is \( c^k = n \rightarrow k = \log_c(n) \)
        \item Cannot be represented as a sum
    \end{itemize}
\end{itemize}
\end{prettyBox}


\vspace{0.35cm}
\begin{prettyBox}{While Loop}{mygreen}
\begin{itemize}
        \item Can never be represented as a sum
        \item If the index is incremented or decremented by just 1, the number of iterations is the difference between the upper and lower bounds.
        \item If the index is incremented or decremented by a constant \(c \geq 2\), the number of iterations is \( \frac{n}{c} \).
        \item If the index is multiplied or divided by a constant \(c \geq 2\), the number of iterations is \( \log_c(n) \).
\end{itemize}
\end{prettyBox}

\newpage

\subsubsection*{\underline{Example}}

\begin{algorithm}[ht]
\caption{}
\begin{algorithmic}
    \For{ \( (i = n \hspace{0.15cm} ;\hspace{0.15cm} i\geq 0 \hspace{0.15cm}; \hspace{0.15cm}i \gets i/2) \)}
\State \textcolor{purplePlot!80!black}{print}(\textcolor{blueArea!60!black}{'Index i : '},i) 
\EndFor
\end{algorithmic}
\end{algorithm}

Iteration : 
\(\frac{n}{2^0},\frac{n}{2^1},\frac{n}{2^2},\frac{n}{2^3},\dots,\frac{n}{2^k}\)\\[0.25cm]
Loop stops when \(i < 0 \Rightarrow \frac{n}{2^k} < 0\)
since it's an natural division it means :
\(2^k > n \Longrightarrow k > \log_{2}(n) \Longrightarrow k= \lceil \log_{2}(n) \rceil = \boxed{\log_{2}(n) + 1 } \)\\[0.25cm]
In conslusion \(O(\log(n))\)

\vspace{0.75cm}
\begin{algorithm}[ht]
\caption{}
\begin{algorithmic}
\State \(r \gets 0;\)

\For{ \( (i = 1 \hspace{0.15cm} ;\hspace{0.15cm} i\leq n^2 \hspace{0.15cm}; \hspace{0.15cm}i \gets i+1) \)}
\For{ \( (j = 1 \hspace{0.15cm} ;\hspace{0.15cm} j\leq 2n-1 \hspace{0.15cm}; \hspace{0.15cm}j \gets j+1) \)}
\State \(r\gets r+1;\)
\EndFor
\EndFor
\end{algorithmic}
\end{algorithm}

\begin{align*}
\sum_{i=1}^{n^2} \sum_{j=1}^{2n-1} 1 &=\sum_{i=1}^{n^2} 2n-1\\
                                     &=(2n-1)\cdot n^2\\
                                     &=\boxed{2n^3 - n^2}
\end{align*}

In conslusion \(O(n^3)\)


\newpage
\begin{algorithm}[ht]
\caption{}
\begin{algorithmic}
\State \(r \gets 0;\)

\For{ \((i = 1 \hspace{0.15cm} ;\hspace{0.15cm} i\leq n \hspace{0.15cm}; \hspace{0.15cm}i \gets i+1) \)}
\For{ \((j = i+1 \hspace{0.15cm} ;\hspace{0.15cm} j\leq n \hspace{0.15cm}; \hspace{0.15cm}j \gets j+1) \)}
\For{ \((k = 1 \hspace{0.15cm} ;\hspace{0.15cm} k\leq j \hspace{0.15cm}; \hspace{0.15cm}j \gets j+1) \)}
\State \(r\gets r+1;\)
\EndFor
\EndFor
\EndFor
\end{algorithmic}
\end{algorithm}

\begin{align*}
\sum_{i=1}^n \sum_{j=i+1}^n \sum_{k=1}^{j} 1 &= \sum_{i=1}^n \sum_{j=i+1}^n j\\
                                             &= \sum_{i=1}^n (\sum_{j=1}^n j - \sum_{j=1}^{i} j)\\
                                             &= \sum_{i=1}^n (\frac{n(n+1)}{2} - \frac{i(i+1)}{2})\\
                                             &= \frac{1}{2}(\sum_{i=1}^n n(n+1) - \sum_{i=1}^n i^2 - \sum_{i=1}^n i)\\
                                             &= \frac{1}{2}(n^2(n+1) - \frac{n(n+1)(2n+1)}{6} - \frac{n(n+1)}{2})\\
                                             &= \frac{4n^3 - 3n^2 - 7n}{12}
\end{align*}

In conslusion \(O(n^3)\)

\newpage

\begin{algorithm}[ht]
\caption{}
\begin{algorithmic}
\State \(r \gets 0;i\gets 1\)

\While{ \((i\leq n) \)}
\For{ \((j = n^2 \hspace{0.15cm} ;\hspace{0.15cm} j\leq 5 \hspace{0.15cm}; \hspace{0.15cm}j \gets j-1) \)}
\State \(r\gets r+1;\)
\EndFor
\State \(i \gets i+2;\)
\EndWhile
\end{algorithmic}
\end{algorithm}

For Loop iterates : \(n^2 - 5+1 = \boxed{n^2-4}\)\\[0.35cm]
While Loop :
\(2k>n \Rightarrow k>\frac{n}{2} \Rightarrow k = \lceil \frac{n}{2} \rceil = \boxed{\frac{n}{2} + 1}\)\\[0.35cm]

In conslusion \(O((n^2-4)(\frac{n}{2} + 1)) = O(n^3)\)

\vspace{1cm}

\begin{algorithm}[ht]
\caption{}
\begin{algorithmic}
\State \(i \gets 2;j\gets 1\)

\While{ \((i\leq n) \)}
\State \(i \gets i*i;\)
\EndWhile

\While{ \((j\leq i) \)}
\State \(j \gets 4*j;\)
\EndWhile

\end{algorithmic}
\end{algorithm}

First While Iteration :
\(2,2^{2^{1}},2^{2^{2}}, 2^{2^{3}},\dots,2^{2^{k}}\)\\[0.15cm]
\(2^{2^k} > n \Rightarrow k > \log(\log(n)) \Rightarrow k = \lceil \log(\log(n)) \rceil = \boxed{\log(\log(n)) + 1}\)\\[0.35cm]

Second While Iteration :
\(1,4^1,4^2,4^3,\dots,4^k\)\\[0.15cm]
\(4^k > i \Rightarrow 4^k > n \Rightarrow k>\log(n) \Rightarrow  k = \lceil \log(n) \rceil = \boxed{\log(n) + 1} \)\\[0.35cm]

In Conclusion \(O(\log(\log(n))+1+\log(n)+1) = O(\log(n))\)
