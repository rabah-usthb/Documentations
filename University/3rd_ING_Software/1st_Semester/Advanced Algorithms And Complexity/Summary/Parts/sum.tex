\section{Nested Sum}

\begin{prettyBox}{Calculate Nested Sum}{mygreen}
We always start from the most inner sum until we finish calculating 
\end{prettyBox}

\vspace{0.25cm}
\subsubsection*{\underline{Example :}}
\begin{align*}
    \sum_{i=1}^{n} \sum_{j=1}^{i} \sum_{k=1}^{j} 1 &= \sum_{i=1}^{n} \sum_{j=1}^i j\\[0.15cm] 
                                                   &= \sum_{i=1}^{n} \frac{i(i+1)}{2}\\[0.15cm]
                                                   &= \frac{1}{2}\sum_{i=1}^{n} i(i+1)\\[0.15cm]
                                                   &= \frac{1}{2}(\sum_{i=1}^{n} i^2 + \sum_{i=1}^{n} i)\\[0.15cm]
                                                   &= \frac{1}{2}(\frac{n(n+1)(2n+1)}{6} + \frac{n(n+1)}{2})\\[0.15cm]
                                                 &=\frac{n(n+1)(2n+4)}{12}
\end{align*}

\vspace{0.5cm}
\newpage
\section{Adjusting Boundaries To Match A Known Sum}
\begin{prettyBox}{Adjusting Boundaries}{mygreen}
We might come across a known sum where the boundaries aren't the same as the rule.
All we need to do is calculate the sum with the boundaries of the rule and then add
or subtract the term accordingly.
\end{prettyBox}

\subsubsection*{\underline{Example:}}
\begin{align*}
    \sum_{i=0}^{k-1} 2^i &= \sum_{i=0}^{k} 2^i - \sum_{i=k}^{k} 2^i \\[0.15cm]
                          &= 2^{k+1} - 1 - 2^{k}
\end{align*}

\vspace{0.25cm}
We have a sum that, by the rule, goes till the \(k\)-th term, but here it goes up to the \((k-1)\)-th term.
So, to get the sum up to \((k-1)\), we take the sum up to \(k\) and remove the last term.

\vspace{0.75cm}

\hspace{6cm}\(\text{term}(0) + \text{term}(1) + \dots + \text{term}(k-1) + \text{term}(k) = \sum_{i=0}^{k} 2^i\)
\[\sum_{i=0}^{k-1} 2^i + \text{term}(k) = \sum_{i=0}^{k} 2^i \]
\[\sum_{i=0}^{k-1} 2^i = \sum_{i=0}^{k} 2^i - \text{term}(k)\]
