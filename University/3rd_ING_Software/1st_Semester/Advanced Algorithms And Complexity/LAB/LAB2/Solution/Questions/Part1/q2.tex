we have \(f(n) = \sum fr\) since \(f(n)\) is sum of frequency of execution (\(fr\)) we need to figure out the
\(fr\) of each instruction and sum them : 

\subsubsection*{\underline{Best Case (N is prime)}}

\vspace{0.5cm}
cmpt $\gets$ 0  \hspace{4cm} \(fr = 1\) (one affectation)

\vspace{0.15cm}
i $\gets$ 1  \hspace{4.65cm} \(fr = 1\) (one affectation)

\vspace{0.15cm}
\textcolor{purplePlot!80!black}{print}(\textcolor{blueArea!60!black}{'Input Integer N\(>\)1 : '})  \hspace{0.95cm} \(fr = 1\) (one print)

\vspace{0.15cm}
\textcolor{purplePlot!80!black}{Read}(n)  \hspace{4.25cm} \(fr = 1\) (one read)

\vspace{0.15cm}

\textbf{while} i \textcolor{redPlot}{\textbf{\textless=}} n  \textbf{do} \hspace{2.75cm} \(fr = n+1\) (check the while condition n+1 times)


\vspace{0.15cm}
cmpt $\gets$ cmpt \textcolor{redPlot}{ \textbf{+}} 1 \hspace{2.6cm} \(fr = 4\) (one affectation and one arithmetic operation (2) , repeated twice because N is Prime)

\vspace{0.15cm}
i $\gets$ i \textcolor{redPlot}{ \textbf{+}} 1 \hspace{4cm} \(fr = 2n\) (one affectation and one arithmetic operation (2) ,  inside a while that loops n times (2n))

\vspace{0.15cm}

\textcolor{purplePlot!80!black}{print}(\textcolor{blueArea!60!black}{'Prime Number'}) \hspace{1.95cm} \(fr = 1\) (one print)

\vspace{0.75cm}
\begin{align*}
f_1(n) &= \sum fr \\
     &= 1 + 1 + 1 + 1 + (n+1) + 4 + 2n + 1 \\
     &= 3n + 10 \\
     &= \boxed{3n + 10}
\end{align*}

\vspace{0.5cm}
now that we have the complexity function \(f_1(n)\) we need to find the \(T_1(n)\)  we have \(T_1(n) = f_1(n) \times \Delta t\)

\begin{align*}
T_1(n) &= f_1(n) \times \Delta t\\ 
&= (3n + 10) \times \Delta t \\
&= \underbrace{3 \Delta t}_{a_1} n + \underbrace{\Delta t 10}_{b_1} \\
&= \boxed{a_1n+b_1} 
\end{align*}

\subsubsection*{\underline{Worst Case (N is not prime)}}

\vspace{0.5cm}
cmpt $\gets$ 0  \hspace{4cm} \(fr = 1\) (one affectation)

\vspace{0.15cm}
i $\gets$ 1  \hspace{4.65cm} \(fr = 1\) (one affectation)

\vspace{0.15cm}
\textcolor{purplePlot!80!black}{print}(\textcolor{blueArea!60!black}{'Input Integer N\(>\)1 : '})  \hspace{0.95cm} \(fr = 1\) (one print)

\vspace{0.15cm}
\textcolor{purplePlot!80!black}{Read}(n)  \hspace{4.25cm} \(fr = 1\) (one read)

\vspace{0.15cm}

\textbf{while} i \textcolor{redPlot}{\textbf{\textless=}} n  \textbf{do} \hspace{2.75cm} \(fr = n+1\) (check the while condition n+1 times)


\vspace{0.15cm}
cmpt $\gets$ cmpt \textcolor{redPlot}{ \textbf{+}} 1 \hspace{2.6cm} \(fr = n\) (one affectation and one arithmetic operation (2) ,  repeated at worst \(\frac{n}{2}\) time)

\vspace{0.15cm}
i $\gets$ i \textcolor{redPlot}{ \textbf{+}} 1 \hspace{4cm} \(fr = 2n\) (one affectation and one arithmetic operation (2) ,  inside a while that loops n times (2n))

\vspace{0.15cm}

\textcolor{purplePlot!80!black}{print}(\textcolor{blueArea!60!black}{'Not A Prime Number'}) \hspace{0.9cm} \(fr = 1\) (one print)

\vspace{0.75cm}
\begin{align*}
f_2(n) &= \sum fr \\
     &= 1 + 1 + 1 + 1 + (n+1) + n + 2n + 1 \\
     &= 4n + 6 \\
     &= \boxed{4n + 6}
\end{align*}

\vspace{0.5cm}
now that we have the complexity function \(f_2(n)\) we need to find the \(T_2(n)\)  we have \(T_2(n) = f_2(n) \times \Delta t\)

\begin{align*}
T_2(n) &= f_2(n) \times \Delta t\\ 
&= (3n + 10) \times \Delta t \\
&= \underbrace{3 \Delta t}_{a_2} n + \underbrace{\Delta t 10}_{b_2} \\
&= \boxed{a_2n+b_2} 
\end{align*}

\subsubsection*{\underline{Conclusion}}
Both \(T_1(n)\) and \(T_2(n)\) are linear complexity therefore they both $\sim$ \(O(n)\)

\vspace{1cm}


