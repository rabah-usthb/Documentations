\subsubsection*{\underline{Experimental}}

\begin{tabular}{|c|c|c|c|c|c|c|c|c|c|c|}
\hline
N & 1000003 & 2000003 & 4000037 & 8000009 & 16000057 & 32000011 & 64000031 & 128000003 & 256000001 & 512000009 \\
\hline
\makecell{T(n)\\\(10^{-3}\)} & 3.643 & 10.051 & 12.12 & 24.284 & 48.268 & 99.244 & 191.605 & 380.24 & 753.837 & 1517.66\\
\hline
\end{tabular}

\vspace{0.25cm}

\begin{tabular}{|c|c|c|}
    \hline
    N & 1024000009 & 2048000011\\
    \hline
    \makecell{T(n)\\\(10^{-3}\)}  & 3042.77 & 6038.826\\
    \hline
\end{tabular}

\vspace{0.5cm}


\subsubsection*{\underline{Theoritical}}
We first need to find  \(\Delta t\) , for that we will take one runtime value from the experimental study and solve a simple equation 

\vspace{0.15cm}

for n = 8000009 and execution time T(n) = \(24.284\times10^{-3}\) :

\vspace{0.75cm}
\begin{align*}
&f(n)\times\Delta t = T(n)\\[0.15cm]
&\Delta t = \frac{T(n)}{f(n)} \\[0.15cm]
&\Delta t = \frac{T(n)}{3n + 10}\\[0.15cm]
&\Delta t = \frac{24.284\times10^{-3}}{3\times 8000009 + 10} \\[0.15cm]
&\boxed{\Delta t = 1.01\times10^{-9}}
\end{align*}



\subsubsection*{\underline{Theoritical Best Case}}

\begin{tabular}{|c|c|c|c|c|c|c|c|c|c|c|}
\hline
N & 1000003 & 2000003 & 4000037 & 8000009 & 16000057 & 32000011 & 64000031 & 128000003 & 256000001 & 512000009 \\
\hline
\makecell{T(n)\\\(10^{-3}\)} & 3.03 & 6.06 & 12.12 & 24.24 & 48.48 & 96.96 & 193.9 & 387.84 & 775.837 & 1551.13\\
\hline
\end{tabular}

\vspace{0.25cm}

\begin{tabular}{|c|c|c|}
    \hline
    N & 1024000009 & 2048000011\\
    \hline
    \makecell{T(n)\\\(10^{-3}\)}  & 3102.72 & 6205.4\\
    \hline
\end{tabular}


