\documentclass[fleqn]{article}
\usepackage{tikz,tcolorbox}
\usepackage{array} % For customizing tables
\usepackage{booktabs} % For better horizontal lines
\usepackage[a4paper, paperwidth=25cm, paperheight=25.5cm, left=2cm, right=2cm, top=2cm, bottom=2cm]{geometry}
\usepackage{multicol}
\usepackage{amsmath}
\usepackage{pgfplots}

\usepackage[utf8]{inputenc}

\usepackage{listings}
\usepackage{xcolor}

% Define the style for C code
\lstdefinestyle{cstyle}{
    language=C,
    basicstyle=\ttfamily\small,
    keywordstyle=\color{blue}\bfseries,
    stringstyle=\color{red},
    commentstyle=\color{green!50!black}\itshape,
    numbers=left,
    numberstyle=\tiny\color{gray},
    stepnumber=1,
    breaklines=true,
    frame=tb,
    tabsize=4,
    showstringspaces=false,
    captionpos=b
}


\lstdefinestyle{pythonstyle2}{
    language=python,                    % Language set to Python
    basicstyle=\ttfamily\footnotesize,   % Change basic font size
    keywordstyle=\color{blue}\bfseries, % Different keyword style
    stringstyle=\color{red},         % Different string color
    commentstyle=\color{green!60!black}\itshape, % Adjust comment color
    numbers=left,                       % Line numbers on the left
    numberstyle=\tiny\color{gray},      % Smaller number font and color
    stepnumber=1,                       % Number each line
    frame=single,                       % Single frame around code
    tabsize=4,                          % Adjust tab size
    showstringspaces=false,             % Do not show spaces in strings
    captionpos=b,% Position of caption
    breaklines=true,
    inputencoding=utf8
}


\pgfplotsset{compat=1.18}
\usepackage{makecell}
\usetikzlibrary{patterns}
\definecolor{greenPlot}{HTML}{14C877}
\definecolor{orangePlot}{HTML}{EA6E12}
\definecolor{purplePlot}{HTML}{4C12EA}
\definecolor{blueArea}{HTML}{10D9EE}
\definecolor{redPlot}{HTML}{ED014A}
\definecolor{myblue}{HTML}{338AC7}
\definecolor{p}{HTML}{D813E7}
\definecolor{y}{HTML}{F5F806}
\usepackage{amssymb}
\setlength{\parindent}{0pt}
\setcellgapes{3pt}  % Adjust padding as needed
\makegapedcells
\tcbuselibrary{skins, breakable, theorems}
\usepackage{algorithm}
\usepackage{algpseudocode}
\usepackage{xcolor}
\setlength{\mathindent}{8cm}


\newtcolorbox{prettyBox}[2]{
  enhanced,
  colback=white!90!#2,   
  colframe=#2!60!black,  
  coltitle=white,        
  fonttitle=\bfseries\Large,
  title=#1,              
  boxrule=1mm,
  arc=0.5mm,
  drop shadow=#2!35!gray, 
}

\begin{document}
\renewcommand{\arrayrulewidth}{0.75mm} % Set line thickness
\setlength{\tabcolsep}{5.5pt} % Set horizontal padding
\renewcommand{\arraystretch}{1.5} % Set vertical padding (1.0 is default)
 \textbf{Family Name} : Chabane Chaouche

 \textbf{First Name} : Rabah

 \textbf{ID }: 222231485010

 \textbf{Soft ING 3rd}
 
 \vspace{1cm}

\section*{Part I}
\subsection*{Q1 . Sum Algorithm}
\section{Connectivité Des Machines Virtuelles}

\begin{center}
    \includegraphics[width=0.8\textwidth]{Question/SC/1-ping1.PNG}
\end{center}

\vspace{0.25cm}

\begin{center}
    \includegraphics[width=0.8\textwidth]{Question/SC/1-ping2.PNG}
\end{center}

\vspace{0.35cm}

\begin{prettyBox}{Connectivité}{myblue}
D'après les captures d'écran, on remarque que les adresses IP des machines sont :
\begin{itemize}
    \item \textbf{Windows : } \texttt{192.168.100.13}
    \item \textbf{Kali Linux : } \texttt{192.168.100.15}
\end{itemize}
Et le ping est réussi, donc il y a une connectivité.
\end{prettyBox}



\vspace{1cm}
\subsection*{Q2 . Time Complexity}

\vspace{0.25cm}
we have \(f(n) = \sum fr\) since \(f(n)\) is sum of frequency of execution (\(fr\)) we need to figure out the
\(fr\) of each instruction and sum them : 

\subsubsection*{\underline{Best Case (N is prime)}}

\vspace{0.5cm}
cmpt $\gets$ 0  \hspace{4cm} \(fr = 1\) (one affectation)

\vspace{0.15cm}
i $\gets$ 1  \hspace{4.65cm} \(fr = 1\) (one affectation)

\vspace{0.15cm}
\textcolor{purplePlot!80!black}{print}(\textcolor{blueArea!60!black}{'Input Integer N\(>\)1 : '})  \hspace{0.95cm} \(fr = 1\) (one print)

\vspace{0.15cm}
\textcolor{purplePlot!80!black}{Read}(n)  \hspace{4.25cm} \(fr = 1\) (one read)

\vspace{0.15cm}

\textbf{while} i \textcolor{redPlot}{\textbf{\textless=}} \(\sqrt{n}\)  \textbf{do} \hspace{2.45cm} \(fr = \sqrt{n}+1\) (check the while condition \(\sqrt{n}+1\) times)


\vspace{0.15cm}
cmpt $\gets$ cmpt \textcolor{redPlot}{ \textbf{+}} 1 \hspace{2.6cm} \(fr = 2\) (one affectation and one arithmetic operation (2) , repeated once because N is Prime)

\vspace{0.15cm}
i $\gets$ i \textcolor{redPlot}{ \textbf{+}} 1 \hspace{3.95cm} \(fr = 2\times\sqrt{n}\) (one affectation and one arithmetic operation (2) , inside a while that loops \(\sqrt{n}\))

\vspace{0.15cm}

\textcolor{purplePlot!80!black}{print}(\textcolor{blueArea!60!black}{'Prime Number'}) \hspace{1.95cm} \(fr = 1\) (one print)

\vspace{0.75cm}
\begin{align*}
f_5(n) &= \sum fr \\
       &= 1 + 1 + 1 + 1 + (\sqrt{n}+1) + 2 + 2\times\sqrt{n} + 1 \\
     &= 3\times\sqrt{n} + 8 \\
     &= \boxed{3\times\sqrt{n} + 8}
\end{align*}

\vspace{0.5cm}
now that we have the complexity function \(f_5(n)\) we need to find the \(T_5(n)\)  we have \(T_5(n) = f_5(n) \times \Delta t\)

\begin{align*}
T_5(n) &= f_5(n) \times \Delta t\\ 
&= (3\times\sqrt{n} + 8) \times \Delta t \\
&= \underbrace{3\Delta t}_{a_5} \sqrt{n} + \underbrace{\Delta t 8}_{b_5} \\
&= \boxed{a_5\sqrt{n}+b_5} 
\end{align*}

\subsubsection*{\underline{Worst Case (N is not prime)}}

\vspace{0.5cm}
cmpt $\gets$ 0  \hspace{4cm} \(fr = 1\) (one affectation)

\vspace{0.15cm}
i $\gets$ 1  \hspace{4.65cm} \(fr = 1\) (one affectation)

\vspace{0.15cm}
\textcolor{purplePlot!80!black}{print}(\textcolor{blueArea!60!black}{'Input Integer N\(>\)1 : '})  \hspace{0.95cm} \(fr = 1\) (one print)

\vspace{0.15cm}
\textcolor{purplePlot!80!black}{Read}(n)  \hspace{4.25cm} \(fr = 1\) (one read)

\vspace{0.15cm}

\textbf{while} i \textcolor{redPlot}{\textbf{\textless=}} \(\sqrt{n}\)  \textbf{do} \hspace{2.45cm} \(fr = \sqrt{n}+1\) (check the while condition \(\sqrt{n}+1\) times)


\vspace{0.15cm}
cmpt $\gets$ cmpt \textcolor{redPlot}{ \textbf{+}} 1 \hspace{2.6cm} \(fr = 2\times\sqrt{n}\) (one affectation and one arithmetic operation (2) , repeated at worst \(\sqrt{n}\))

\vspace{0.15cm}
i $\gets$ i \textcolor{redPlot}{ \textbf{+}} 1 \hspace{3.95cm} \(fr = 2\times\sqrt{n}\) (one affectation and one arithmetic operation (2) , inside a while that loops \(\sqrt{n}\))

\vspace{0.15cm}

\textcolor{purplePlot!80!black}{print}(\textcolor{blueArea!60!black}{'Prime Number'}) \hspace{1.95cm} \(fr = 1\) (one print)

\vspace{0.75cm}

\begin{align*}
f_6(n) &= \sum fr \\
       &= 1 + 1 + 1 + 1 + (\sqrt{n}+1) + 2\sqrt{n} + 2\sqrt{n} + 1 \\
       &= 5\sqrt{n} + 6 \\
       &= \boxed{5\sqrt{n} + 6}
\end{align*}

\vspace{0.5cm} 
now that we have the complexity function \(f_4(n)\) we need to find the \(T_6(n)\)  we have \(T_6(n) = f_6(n) \times \Delta t\)

\begin{align*}
T_4(n) &= f_4(n) \times \Delta t\\ 
       &= (5\sqrt{n} + 6) \times \Delta t \\
       &= \underbrace{5 \Delta t}_{a_6} \sqrt{n} + \underbrace{\Delta t 6}_{b_6} \\
&= \boxed{a_6\sqrt{n}+b_6} 
\end{align*}

\subsubsection*{\underline{Conclusion}}
Both \(T_5(n)\) and \(T_6(n)\) are square root complexity therefore they both $\sim$ \(O(\sqrt{n})\)





\vspace{1cm}

\subsection*{Q3 . Space Complexity}


\vspace{0.25cm}
\section{Installation Office 2013}

\begin{center}
    \includegraphics[width=0.8\textwidth]{Question/SC/3-.PNG}
\end{center}

\vspace{0.35cm}

\begin{prettyBox}{L'installation}{myblue}
Installation via le lien : \href{https://www.malavida.com/en/soft/microsoft-office-2013/download}{https://www.malavida.com/en/soft/microsoft-office-2013/download} , et on aura 
besoin de 7-zip pour unzip le fichier : \href{https://www.7-zip.org/}{https://www.7-zip.org/} , apres telechargement on mount le fichier.
\end{prettyBox}




\vspace{1cm}

\subsection*{Q4 . C Code PSUM\_1.c}

\vspace{0.25cm}
\subsubsection*{4.}
On affiche la structure des tables ALL\_TAB\_COLUMNS,USER\_TAB\_COLUMNS avec desc

\lstinputlisting[style=sqlstyle]{SQL/Partie5/descr.sql}

\begin{center}
    \includegraphics[width=\textwidth]{ScreenShot/Partie5/desc1.png}
\end{center}

\begin{center}
    \includegraphics[width=\textwidth]{ScreenShot/Partie5/desc5.png}
\end{center}

\begin{prettyBox}{Difference}{myblue}
les deux tables ont les memes attributes , la seul difference est que la table ALL\_TAB\_COLUMNS a un attribut owner en plus compare a USER\_TAB\_COLUMNS
\end{prettyBox}



\vspace{1cm}
\section*{Part II}
\subsection*{Q1 . C Code With Clock PSUM\_2.c}

\vspace{0.25cm}
\section{Connectivité Des Machines Virtuelles}

\begin{center}
    \includegraphics[width=0.8\textwidth]{Question/SC/1-ping1.PNG}
\end{center}

\vspace{0.25cm}

\begin{center}
    \includegraphics[width=0.8\textwidth]{Question/SC/1-ping2.PNG}
\end{center}

\vspace{0.35cm}

\begin{prettyBox}{Connectivité}{myblue}
D'après les captures d'écran, on remarque que les adresses IP des machines sont :
\begin{itemize}
    \item \textbf{Windows : } \texttt{192.168.100.13}
    \item \textbf{Kali Linux : } \texttt{192.168.100.15}
\end{itemize}
Et le ping est réussi, donc il y a une connectivité.
\end{prettyBox}




\vspace{1cm}
\subsection*{Q2 . Tables}

\vspace{0.25cm}
we have \(f(n) = \sum fr\) since \(f(n)\) is sum of frequency of execution (\(fr\)) we need to figure out the
\(fr\) of each instruction and sum them : 

\subsubsection*{\underline{Best Case (N is prime)}}

\vspace{0.5cm}
cmpt $\gets$ 0  \hspace{4cm} \(fr = 1\) (one affectation)

\vspace{0.15cm}
i $\gets$ 1  \hspace{4.65cm} \(fr = 1\) (one affectation)

\vspace{0.15cm}
\textcolor{purplePlot!80!black}{print}(\textcolor{blueArea!60!black}{'Input Integer N\(>\)1 : '})  \hspace{0.95cm} \(fr = 1\) (one print)

\vspace{0.15cm}
\textcolor{purplePlot!80!black}{Read}(n)  \hspace{4.25cm} \(fr = 1\) (one read)

\vspace{0.15cm}

\textbf{while} i \textcolor{redPlot}{\textbf{\textless=}} \(\sqrt{n}\)  \textbf{do} \hspace{2.45cm} \(fr = \sqrt{n}+1\) (check the while condition \(\sqrt{n}+1\) times)


\vspace{0.15cm}
cmpt $\gets$ cmpt \textcolor{redPlot}{ \textbf{+}} 1 \hspace{2.6cm} \(fr = 2\) (one affectation and one arithmetic operation (2) , repeated once because N is Prime)

\vspace{0.15cm}
i $\gets$ i \textcolor{redPlot}{ \textbf{+}} 1 \hspace{3.95cm} \(fr = 2\times\sqrt{n}\) (one affectation and one arithmetic operation (2) , inside a while that loops \(\sqrt{n}\))

\vspace{0.15cm}

\textcolor{purplePlot!80!black}{print}(\textcolor{blueArea!60!black}{'Prime Number'}) \hspace{1.95cm} \(fr = 1\) (one print)

\vspace{0.75cm}
\begin{align*}
f_5(n) &= \sum fr \\
       &= 1 + 1 + 1 + 1 + (\sqrt{n}+1) + 2 + 2\times\sqrt{n} + 1 \\
     &= 3\times\sqrt{n} + 8 \\
     &= \boxed{3\times\sqrt{n} + 8}
\end{align*}

\vspace{0.5cm}
now that we have the complexity function \(f_5(n)\) we need to find the \(T_5(n)\)  we have \(T_5(n) = f_5(n) \times \Delta t\)

\begin{align*}
T_5(n) &= f_5(n) \times \Delta t\\ 
&= (3\times\sqrt{n} + 8) \times \Delta t \\
&= \underbrace{3\Delta t}_{a_5} \sqrt{n} + \underbrace{\Delta t 8}_{b_5} \\
&= \boxed{a_5\sqrt{n}+b_5} 
\end{align*}

\subsubsection*{\underline{Worst Case (N is not prime)}}

\vspace{0.5cm}
cmpt $\gets$ 0  \hspace{4cm} \(fr = 1\) (one affectation)

\vspace{0.15cm}
i $\gets$ 1  \hspace{4.65cm} \(fr = 1\) (one affectation)

\vspace{0.15cm}
\textcolor{purplePlot!80!black}{print}(\textcolor{blueArea!60!black}{'Input Integer N\(>\)1 : '})  \hspace{0.95cm} \(fr = 1\) (one print)

\vspace{0.15cm}
\textcolor{purplePlot!80!black}{Read}(n)  \hspace{4.25cm} \(fr = 1\) (one read)

\vspace{0.15cm}

\textbf{while} i \textcolor{redPlot}{\textbf{\textless=}} \(\sqrt{n}\)  \textbf{do} \hspace{2.45cm} \(fr = \sqrt{n}+1\) (check the while condition \(\sqrt{n}+1\) times)


\vspace{0.15cm}
cmpt $\gets$ cmpt \textcolor{redPlot}{ \textbf{+}} 1 \hspace{2.6cm} \(fr = 2\times\sqrt{n}\) (one affectation and one arithmetic operation (2) , repeated at worst \(\sqrt{n}\))

\vspace{0.15cm}
i $\gets$ i \textcolor{redPlot}{ \textbf{+}} 1 \hspace{3.95cm} \(fr = 2\times\sqrt{n}\) (one affectation and one arithmetic operation (2) , inside a while that loops \(\sqrt{n}\))

\vspace{0.15cm}

\textcolor{purplePlot!80!black}{print}(\textcolor{blueArea!60!black}{'Prime Number'}) \hspace{1.95cm} \(fr = 1\) (one print)

\vspace{0.75cm}

\begin{align*}
f_6(n) &= \sum fr \\
       &= 1 + 1 + 1 + 1 + (\sqrt{n}+1) + 2\sqrt{n} + 2\sqrt{n} + 1 \\
       &= 5\sqrt{n} + 6 \\
       &= \boxed{5\sqrt{n} + 6}
\end{align*}

\vspace{0.5cm} 
now that we have the complexity function \(f_4(n)\) we need to find the \(T_6(n)\)  we have \(T_6(n) = f_6(n) \times \Delta t\)

\begin{align*}
T_4(n) &= f_4(n) \times \Delta t\\ 
       &= (5\sqrt{n} + 6) \times \Delta t \\
       &= \underbrace{5 \Delta t}_{a_6} \sqrt{n} + \underbrace{\Delta t 6}_{b_6} \\
&= \boxed{a_6\sqrt{n}+b_6} 
\end{align*}

\subsubsection*{\underline{Conclusion}}
Both \(T_5(n)\) and \(T_6(n)\) are square root complexity therefore they both $\sim$ \(O(\sqrt{n})\)





\newpage
\subsection*{Q3 . Plots}
\section{Installation Office 2013}

\begin{center}
    \includegraphics[width=0.8\textwidth]{Question/SC/3-.PNG}
\end{center}

\vspace{0.35cm}

\begin{prettyBox}{L'installation}{myblue}
Installation via le lien : \href{https://www.malavida.com/en/soft/microsoft-office-2013/download}{https://www.malavida.com/en/soft/microsoft-office-2013/download} , et on aura 
besoin de 7-zip pour unzip le fichier : \href{https://www.7-zip.org/}{https://www.7-zip.org/} , apres telechargement on mount le fichier.
\end{prettyBox}




\end{document}
