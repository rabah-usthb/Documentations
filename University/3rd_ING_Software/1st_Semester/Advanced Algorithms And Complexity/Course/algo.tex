\documentclass[fleqn]{article}
\usepackage{tikz,tcolorbox}
\usepackage{array} % For customizing tables
\usepackage{booktabs} % For better horizontal lines
\usepackage[a4paper, paperwidth=25cm, paperheight=25.5cm, left=2cm, right=2cm, top=2cm, bottom=2cm]{geometry}
\usepackage{multicol}
\usepackage{amsmath}
\usepackage{pgfplots}

\pgfplotsset{compat=1.18}
\usepackage{makecell}
\usetikzlibrary{patterns}
\definecolor{greenPlot}{HTML}{14C877}
\definecolor{orangePlot}{HTML}{EA6E12}
\definecolor{purplePlot}{HTML}{4C12EA}
\definecolor{blueArea}{HTML}{10D9EE}
\definecolor{redPlot}{HTML}{ED014A}
\definecolor{myblue}{HTML}{338AC7}
\definecolor{p}{HTML}{D813E7}
\definecolor{y}{HTML}{F5F806}
\usepackage{amssymb}
\setlength{\parindent}{0pt}
\setcellgapes{3pt}  % Adjust padding as needed
\makegapedcells
\tcbuselibrary{skins, breakable, theorems}
\usepackage{algorithm}
\usepackage{algpseudocode}
\usepackage{xcolor}
\setlength{\mathindent}{8cm}
\renewcommand{\thealgorithm}{}

\begin{document}
\renewcommand{\arrayrulewidth}{0.75mm} % Set line thickness
\setlength{\tabcolsep}{12pt} % Set horizontal padding
\renewcommand{\arraystretch}{1.5} % Set vertical padding (1.0 is default)

\begin{tcolorbox}[enhanced, colback=white!90!greenPlot, colframe=greenPlot!60!black, coltitle=white, fonttitle=\bfseries\Large, title=Some Terminology, boxrule=1mm, arc=0.5mm, drop shadow=greenPlot!35!gray]
\begin{itemize}
    \item \textbf{Theoritical Complexity}: It's theoritical complexity calculated through \(f(n)\) and \(T(n)\) by supposing that each basic instruction has same execution time \(\Delta t\)
    \item \textbf{Experimental Complexity}: It's real life complexity calculated by executing a program in a machine and using a clock to get the exact time
\end{itemize}
\end{tcolorbox}


\begin{center}
    \Huge{\textbf{\underline{Chapter 1: Introduction}}}
\end{center}

\setcounter{section}{0}

\vspace{0.35cm}

\section{Steps Of An Attack}

\vspace{0.25cm}
\begin{center}
    \includegraphics[width=0.9\textwidth]{Chapters/Diagram/Introduction/attack.drawio.pdf}
\end{center}

\vspace{0.35cm}
\section{Reasons for Poor Security}
\begin{prettyBox}{Reasons}{myblue}
\begin{itemize}
    \item \textbf{Insufficient Budget}: Approximately \(\frac{1}{4}\) of issues arise due to inadequate funding for cybersecurity initiatives and personnel.
    \item \textbf{Unqualified Personnel}: A lack of skilled and properly trained cybersecurity professionals.
    \item \textbf{Poor Administration}: Inefficient management and lack of synchronization in security policies and practices.
\end{itemize}
\end{prettyBox}

\vspace{0.35cm}

\section{Impacts of a Cyberattack}
\begin{prettyBox}{Impacts}{myblue}
\begin{itemize}
    \item \textbf{Data Breach}: Unauthorized access to sensitive client or organizational data. This may include\\ encrypting data for ransom (ransomware), sharing confidential information, or selling it on the dark web.  
    \item \textbf{Denial of Service (DoS)}: Disrupting or halting the services of an organization, making them inaccessible to users.   
    \item \textbf{Financial Loss}: Hacking into bank accounts, demanding ransom (ransomware attacks), or causing service interruptions that result in revenue loss.  
    \item \textbf{Damage to Reputation}: Eroding client trust or tarnishing someone's reputation by exposing compromised or sensitive data.  
    \item \textbf{Loss of Clients}: Organizations may lose clients due to the exposure of sensitive information, compromised systems, server outages, and interruptions in services.  
\end{itemize}
\end{prettyBox}

\newpage

\section{Information System}
\begin{prettyBox}{Definition}{myblue}
A set of active applications, services, and other components that allow for the management of informations.
Vulnerabilities can affect all components of the information system (IS).  
\end{prettyBox}

\section{Responses to Cyberattacks}
\begin{prettyBox}{Responses}{myblue}
\begin{itemize}
    \item \textbf{Reduce the Impact}: Taking measures to minimize the damage caused by the cyberattack, such as isolating affected systems, restoring backups, or limiting access.  
    \item \textbf{Accept the Risk}: The least effective response, where no action is taken to counter the attack. This could include surrendering to attackers’ demands, such as paying a ransom.  
    \item \textbf{Refuse or Resolve the Risk}: Actively countering the attack by refusing to comply with attackers and taking corrective actions to fix vulnerabilities or breaches.  
    \item \textbf{Transfer the Responsibility}: Shifting the burden of dealing with the cyberattack to a third party, such as an insurance provider or a managed cybersecurity service.  
\end{itemize}
\end{prettyBox}

\vspace{0.5cm}

\section{Steps For Protection (Deming's Wheel)}
\begin{prettyBox}{PDCA}{myblue}
    The PDCA (Plan-Do-Check-Act) cycle is a continuous improvement process widely used in
cybersecurity to ensure effective protection and adapt to evolving threats. Below are the four key steps:

    \begin{itemize}
        \item \textbf{\textcolor{green}{P}lan}: Identify goals, assess risks, and develop strategies to strengthen cybersecurity.
        \item \textbf{\textcolor{green}{D}o}: Implement the cybersecurity measures, like deploying firewalls and training staff.
        \item \textbf{\textcolor{green}{C}heck}: Monitor and evaluate the effectiveness of security measures through audits and testing.
        \item \textbf{\textcolor{green}{A}ct}: Address weaknesses and refine security measures to adapt to new threats.
    \end{itemize}
\end{prettyBox}

\begin{center}
    \includegraphics[width=0.5\textwidth]{Chapters/Diagram/Introduction/pdca.drawio.pdf}
\end{center}



\section{Complexity Analysis Basics}

\begin{tcolorbox}[enhanced, colback=white!90!greenPlot, colframe=greenPlot!60!black, coltitle=white, fonttitle=\bfseries\Large, title=Some Terminology, boxrule=1mm, arc=0.5mm, drop shadow=greenPlot!35!gray]
\begin{itemize}
    \item \textbf{Theoritical Complexity}: It's theoritical complexity calculated through \(f(n)\) and \(T(n)\) by supposing that each basic instruction has same execution time \(\Delta t\)
    \item \textbf{Experimental Complexity}: It's real life complexity calculated by executing a program in a machine and using a clock to get the exact time
\end{itemize}
\end{tcolorbox}

\vspace{0.25cm}

\begin{center}
   \begin{tikzpicture}
           \begin{axis}[
        legend style={at={(1.5,0.75)}, anchor=north east}, % Position the legend   
        height = 9.5cm,
        width = 10cm,
        axis lines = middle,           % Ensures axes cross at (0, 0)
        xlabel=$n$, 
        ylabel=$time$, 
        xmin = 0, xmax = 12,           % Set x-axis range
        ymin = 0, ymax = 30,           % Set y-axis range
        ytick={0,5,...,30},             % Y-tick values
        xtick={0,2,...,12},
    xlabel style={align=center, xshift = 16pt,yshift=-10pt}
    ]
 \addplot[thick ,domain = 0:12,samples=200, blueArea] {x};
 \addlegendentry{x}
 \addplot[thick ,domain = 0:12,samples=200, orangePlot ,restrict y to domain=0:30] {x^2};
 
 \addlegendentry{$x^2$}
 \addplot[thick , domain = 0:80,samples=500, blue, restrict y to domain=0:32] {x^3};

 \addlegendentry{$x^3$}
 \addplot[thick ,domain = 0:12,samples=200, redPlot] {sqrt(x)};
 
 \addlegendentry{$\sqrt x$}
 \addplot[thick ,domain = 0.1:12,samples=200, purplePlot] {log10(x)};

 \addlegendentry{$\log_{10}(x)$}
 \addplot[thick ,domain = 0.1:12,samples=200, greenPlot] {log10(x)*x};

 \addlegendentry{x $\log_{10}(x)$}
 \addplot[thick ,domain = 0.01:80,samples=500, restrict y to domain=0:32,p] {2^x};
 \addlegendentry{$2^x$}
 \addplot[thick ,domain = 0.01:80,samples=500, restrict y to domain=0:36,color =y] {factorial(x)};
 \addlegendentry{$factorial(x)$}

\end{axis}
    \end{tikzpicture}
\end{center}

\subsection*{\underline{Example :}}
suppose we have \(f(n) = 2 ^ n \quad, \quad \Delta t = 10^{-6} s \)
\begin{align*}
    T(n) &= f(n) \times \Delta t\\
    T(60) &= 2^{60} \times 10^{-6}\\
    &= \boxed{36559 years}
\end{align*}

\begin{tcolorbox}[enhanced, colback=white!90!greenPlot, colframe=greenPlot!60!black, coltitle=white, fonttitle=\bfseries\Large, title=Landau Notation, boxrule=1mm, arc=0.5mm, drop shadow=greenPlot!35!gray]
\begin{itemize}
    \item \textbf{Big O} : \(O\) upper bound
    \item \textbf{Big Omega}: \(\Omega\) lower bound 
    \item \textbf{Big Theta} : \(\Theta\) average
\end{itemize}
\end{tcolorbox}
% 3 graphs

\vspace{0.5cm}

\begin{tcolorbox}[enhanced, colback=white!90!greenPlot, colframe=greenPlot!60!black, coltitle=white, fonttitle=\bfseries\Large, title=Landau Notation Mathematical Definition, boxrule=1mm, arc=0.5mm, drop shadow=greenPlot!35!gray]
    \(f(n) = \Omega(g(n))\) \quad \( \exists c > 0\)\hspace{0.1cm},\quad \(c . g(n) \leq f(n)\) \quad ,\hspace{0.1cm}\(\forall n \geq n_0\)


\vspace{0.15cm}
\(f(n) = O(g(n))\) \quad \(\exists c > 0\)\hspace{0.1cm},\quad\(c . g(n) \geq f(n)\) \quad ,\hspace{0.1cm}\(\forall n \geq n_0\)

\vspace{0.15cm}
\(f(n) = \Theta(g(n))\) \quad \(\exists c_1 > 0\)\hspace{0.1cm},\hspace{0.1cm}\(\exists c_2 > 0\)\hspace{0.1cm},\quad \(c_1 . g(n) \leq f(n) \leq c_2. g(n)\)\quad  ,\hspace{0.1cm}\(\forall n \geq n_0\)

\end{tcolorbox}
\newpage
\subsection*{\underline{Example :}}
\(f(n) = 2n + 5 => O(n)\)
 
\vspace{0.5cm} 
\hspace{9cm}\(f(n) \leq c.g(n)\)

\vspace{0.15cm}
\hspace{9cm}\(2n+5 \leq c.n\)

\vspace{0.15cm}
\hspace{9cm}\(2 + \frac{5}{n} \leq c\) 

\vspace{0.15cm}
\hspace{7.75cm}\(\frac{5}{n} \leq 1\)\quad \quad \(\forall n \geq 5\) \quad \(\boxed{n_0 = 5}\)

\vspace{0.15cm}
\hspace{7.75cm}\(2 + \frac {5}{5} \leq c\)\quad \(3 \leq c\)\hspace{0.5cm} \(\boxed{c = 3}\)

\vspace{1cm}

\(f(n) = 2n + 5 => \Omega (n)\)

\vspace{0.5cm} 
\hspace{9cm}\(f(n) \geq c.g(n)\)

\vspace{0.15cm}
\hspace{9cm}\(2n+5 \geq c.n\)

\vspace{0.15cm}
\hspace{9cm}\(2 + \frac{5}{n} \geq c\) 

\vspace{0.15cm}
\hspace{7.75cm}\(\frac{5}{n} \leq 1\)\quad \quad \(\forall n \geq 5\) \quad \(\boxed{n_0 = 5}\)

\vspace{0.15cm}
\hspace{7.75cm}\(2 + \frac {5}{5} \geq c\)\quad \(3 \geq c\)\hspace{0.5cm} \(\boxed{c = 2}\)

\vspace{1cm}

\(f(n) = 2n + 5 => \Theta (n)\)

\vspace{0.5cm} 
\hspace{8.25cm}\( c_1.g(n) \leq f(n) \leq c_2.g(n)\)

\vspace{0.15cm}
\hspace{8.5cm}\( c_1.n \leq 2n+5 \leq c_2.n\)

\vspace{0.15cm}
\hspace{8.85cm}\( c_1 \leq 2 + \frac{5}{n} \leq c_2\) 

\vspace{0.15cm}
\hspace{7.75cm}\(\frac{5}{n} \leq 1\)\quad \quad \(\forall n \geq 5\) \quad \(\boxed{n_0 = 5}\)

\vspace{0.15cm}
\hspace{7.75cm}\(2 + \frac {5}{5} \geq c_1\)\quad \(3 \geq c_1\)\hspace{0.35cm} \(\boxed{c_1 = 2}\)

\vspace{0.15cm}
\hspace{7.75cm}\(2 + \frac {5}{5} \leq c_2\)\quad \(3 \leq c_2\)\hspace{0.35cm} \(\boxed{c_2 = 3}\)

\vspace{0.75cm}
\begin{tcolorbox}[enhanced, colback=white!90!greenPlot, colframe=greenPlot!60!black, coltitle=white, fonttitle=\bfseries\Large, title=Note, boxrule=1mm, arc=0.5mm, drop shadow=greenPlot!35!gray]
\begin{itemize}
    \item  \textbf{n is integer} : it represents the size of the inputed data
    \item  \textbf{c is float} : it represents a coefficient
    \item  \textbf{Why we didn't took \(c_1 = 3\) even though \(c_1 \leq 3\) when n = 5} : because the inequality \(2 + \frac {5}{n} \geq c_1\)
must be verified \(\forall n \geq n_0 = 5\) if we take n = 6 , \(2+\frac{5}{6} \approx 2.8\) and \(c_1 = 3\) isn't \(\leq\) 2.8
\end{itemize}
\end{tcolorbox}

\vspace{0.75cm}
\begin{tcolorbox}[enhanced, colback=white!90!greenPlot, colframe=greenPlot!60!black, coltitle=white, fonttitle=\bfseries\Large, title=Laudau Notation Limit Definition, boxrule=1mm, arc=0.5mm, drop shadow=greenPlot!35!gray]
    if \quad\(\lim\limits_{n \to +\infty}\left| \frac{f(n)}{g(n)}\right| =\)\hspace{0.15cm}\(k\)\quad,\quad\( k > \)0  \quad \(=> \) \quad \(f(n) \in \Theta (g(n))\)

    \vspace{0.25cm}
    if \quad\(\lim\limits_{n \to +\infty}\left| \frac{f(n)}{g(n)}\right| =\)\hspace{0.15cm}\(0\) \hspace{1.45cm} \quad \(=> \) \quad \(f(n) \in O (g(n))\)
    
    \vspace{0.25cm}
    if \quad\(\lim\limits_{n \to +\infty}\left| \frac{f(n)}{g(n)}\right| =\)\hspace{0.15cm}\(+\infty\) \hspace{1cm} \quad \(=> \) \quad \(f(n) \in \Omega (g(n))\)
\end{tcolorbox}


\subsection*{\underline{Exercice :}} show that
\(f(n) = 5n^2 - 6n => \Theta(n^2)\)
 
\vspace{0.5cm} 
\hspace{8.25cm}\( c_1.g(n) \leq f(n) \leq c_2.g(n)\)

\vspace{0.15cm}
\hspace{8.5cm}\( c_1.n^2 \leq 5n^2-6n \leq c_2.n^2\)

\vspace{0.15cm}
\hspace{8.85cm}\( c_1 \leq 5 - \frac{6}{n} \leq c_2\) 

\vspace{0.15cm}
\hspace{7.75cm}\(\frac{6}{n} \leq 1\)\quad \quad \(\forall n \geq 6\) \quad \(\boxed{n_0 = 6}\)

\vspace{0.15cm}
\hspace{7.75cm}\(5 - \frac {6}{6} \geq c_1\)\quad \(4 \geq c_1\)\hspace{0.35cm} \(\boxed{c_1 = 4}\)

\vspace{0.15cm}
\hspace{7.75cm}\(5 - \frac {6}{6} \leq c_2\)\quad \(4 \leq c_2\)\hspace{0.35cm} \(\boxed{c_2 = 5}\)

\vspace{1cm}
\(f(n) = 6n^3 \neq \Theta(n^2)\)
 
\vspace{0.5cm} 
\hspace{8.25cm}\( c_1.g(n) \leq f(n) \leq c_2.g(n)\)

\vspace{0.15cm}
\hspace{8.5cm}\( c_1.n^2 \leq 6n^3 \leq c_2.n^2\)

\vspace{0.15cm}
\hspace{9cm}\(c_1 \leq 6n \leq c_2\) \hspace{3cm} \(c_1 \nexists , c_2 \nexists\)

\vspace{1cm}
\(f(n) = n^2 => O(10^{-5}n^3)\)
 
\vspace{0.5cm} 
\hspace{9cm}\(f(n) \leq c.g(n)\)

\vspace{0.15cm}
\hspace{9cm}\(n^2 \leq c.10^{-5}n^3\)

\vspace{0.15cm}
\hspace{9.25cm}\(\frac{10^{5}}{n} \leq c\) 

\vspace{0.15cm}
\hspace{7.75cm}\(\frac{10^{5}}{n} \leq 1\)\quad \quad \(\forall n \geq 10^{5}\) \quad \(\boxed{n_0 = 10^{5}}\)

\vspace{0.15cm}
\hspace{8.25cm}\(\frac {10^{5}}{10^{5}} \leq c\)\quad \(1 \leq c\)\hspace{0.5cm} \(\boxed{c = 1}\)

\newpage
\(f(n) = 10n^3 + 3n^2 + 5n + 1  => O(n^3)\)
 
\vspace{0.5cm} 
\hspace{9cm}\(f(n) \leq c.g(n)\)

\vspace{0.15cm}
\hspace{8cm}\(10n^3 + 3n^2 + 5n + 1 \leq c.n^3\)

\vspace{0.15cm}
\hspace{8.25cm}\( 10 + \frac{3}{n} + \frac{5}{n^2} + \frac{1}{n^3} \leq c\) 

\vspace{0.15cm}
\hspace{8.25cm}\(\frac{3}{n} \leq 1\)\quad \(\forall n \geq 3\)  

\vspace{0.15cm}
\hspace{7.75cm}\(\frac{5}{n^2} \leq 1\)\quad \(n^2 \leq 5\)\quad \(n \leq \sqrt{5}\)\quad \(\forall n \geq 3\)

\vspace{0.15cm}
\hspace{7.75cm}\(\frac{1}{n^3} \leq 1\)\quad \(\forall n \geq 1\) 

\vspace{0.15cm}
\hspace{8cm}\(\forall n \geq 1 \cap \forall n \geq 3\)\quad\(=>\)\quad\(\forall n \geq 3\)\quad \(\boxed{n_0 = 3}\)

\vspace{0.15cm}
\hspace{8.25cm}\(10 + \frac{3}{3} + \frac{5}{3^2} + \frac{1}{3^3} \approx 11.59\)\hspace{0.5cm} \(\boxed{c = 13}\)




\end{document}
