
\newpage
\null
\begin{center}
    \Huge{\textbf{\underline{Chapter 1.2: Informed}}}
\end{center}

\setcounter{section}{0}

\vspace{0.35cm}

\section{Heuristic Function \(h(n)\)}
\begin{prettyBox}{Definition}{myblue}
The Heuristic Function takes a state as input and returns a heuristic value, which indicates how close the state is to the solution.
\end{prettyBox}

\vspace{0.35cm}

\begin{prettyBox}{Better Heuristic \(h(n)\)}{myblue}
A single problem can have many heuristic functions. These heuristics can be compared on two main points:
\begin{itemize}
    \item \textbf{Cost:} We aim for a low cost since each node we traverse will involve applying the heuristic function.
    \item \textbf{Effectiveness:} How efficient the heuristic is at guiding the search. If the ratio is equal to 1, it is perfect.
    \[
    \text{Effectiveness} = \frac{\text{Nb}_{\text{Seen Nodes}}}{|\text{Path}|}
    \]
\end{itemize}
\end{prettyBox}

\vspace{0.35cm}

\section{Types of Heuristic Functions}
\begin{prettyBox}{Types}{myblue}
    Heuristic functions can be divided into the following types:
    \begin{itemize}
        \item \textbf{Static (Dependent on the Domain):} A static rule derived from the goal state is applied to a given state to compute its heuristic value.
        \item \textbf{Dynamic (Independent of the Domain):} Uses a relaxed problem, typically simplifying the problem constraints to guide the search.
    \end{itemize}
\end{prettyBox}

\newpage
\section{Searching Algorithms}

\subsection{Best-First Search}

\begin{algorithm}[ht]
\caption{Best-First Search}
\begin{algorithmic}
\State Open \(\gets\) \{\{S, \text{nil}, h(S)\}\}; 
\State Closed \(\gets\) \{\}; 
\vspace{0.1cm}
\While{ Open \(\neq\) nil) }
    \State N \(\gets\) Head(Open); 
    
    \vspace{0.07cm}
    \If{GoalTest(N.current)} 
        \State return reconstructPath(N, Closed); 
    \Else
        \State Closed \(\gets\) Closed \(\bigcup\) \{N\}; 
        \For{each new state in MoveGen(\(N.\text{current}\))}
            \If{new \(\notin\) Open \(\bigcup\) Closed}
                \State Open \(\gets\) append(\{new, N, h(new)\}); 
            \EndIf
        \EndFor

        \State sort\(_{h}\)(Open); 
    \EndIf
    \vspace{0.07cm}
\EndWhile

\vspace{0.1cm}
\State return nil; 
\end{algorithmic}
\end{algorithm}


\subsection{Hill Climbing}

\begin{algorithm}[ht]
\caption{Best-First Search}
\begin{algorithmic}
\State next \(\gets\) \{\{S, \text{nil}, h\}\}; 
\State value \(\gets\) Next.h;
\State b \(\gets\) true;
\State path \(\gets\) \{next\};
\vspace{0.1cm}
\While{ b }
\For {each new in MoveGen(next) }
\If{new better than next}
\State next \(\gets\) new;
\State path \(\gets\) append(new);
\EndIf
\EndFor
    
\If{value = next.h}
\State b \(\gets\) false;
\Else
\State value \(\gets\) next.h;
\EndIf
\EndWhile

\vspace{0.1cm}
\State return path; 
\end{algorithmic}
\end{algorithm}


\begin{tabular}{|c|c|c|c|c|c|}
\hline
\textbf{Algorithm} & \textbf{Space Complexity} & \textbf{Time Complexity} & \textbf{Completeness} & \textbf{Optimality} & \textbf{Scale}\\
\hline
\textbf{Best-First Search} & Dependent Of \(h(n)\) & Dependent Of \(h(n)\) & Yes & No & Global Search\\
\hline
\textbf{Hill Climbing} & \( O(1) \) & linear & No & No & Local Search\\
\hline
\end{tabular}

