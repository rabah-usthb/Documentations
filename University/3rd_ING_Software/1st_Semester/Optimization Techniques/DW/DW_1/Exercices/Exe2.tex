
\section*{Exercise 2: Optimal Production}

A food company has 1000 kilos of African coffee, 2000 kilos of Brazilian coffee, and 500 kilos of Colombian coffee. The company produces two types of coffee:

\begin{itemize}
    \item \textbf{Type 1}: A blend of three parts African coffee for one part of each of the two others (Brazilian and Colombian), sold at 10 UM per kilo.
    \item \textbf{Type 2}: A blend of three parts Brazilian coffee for one part Colombian coffee, sold at 8 UM per kilo.
\end{itemize}

The problem is to determine how much coffee of each type the company should produce in order to maximize its profit. Model this problem.

\vspace{1.5cm}
\textbf{\underline{\Large Solution:}}

\vspace{0.5cm}
\begin{minipage}[t]{0.45\textwidth}
    \textbf{\underline{Variables Definition:}} \\[1em] % Adjust vertical space here

    Let \(x_1\) be Type 1 coffee. \\
   
    Let \(x_2\) be Type 2 coffee.
\end{minipage}%
\hfill % This adds space between the two minipages
\begin{minipage}[t]{0.45\textwidth}
    \textbf{\underline{Constraint:}} \\[1em]
    \[
    \left\{
        \begin{array}{l}
            \forall x_1, x_2 \geq 0 \quad \text{(Non-negative number of type of coffee)}\\\\
            \frac{3}{5} x_1  \leq 1000 \quad \text{(Total number of African coffee)}\\\\
            \frac{1}{5}x_1 + \frac{3}{4}x_2 \leq 2000 \quad \text{(Total number of Brazilian coffee)}\\\\
            \frac{1}{5}x_1 + \frac{1}{4}x_2 \leq 500 \quad \text{(Total number of Colombian coffee)}\\\\
        \end{array}
    \right.
    \]
\end{minipage}
\vspace{0.5cm}
\begin{tcolorbox}[title = Objectif Function]
\[
f(x_1,x_2) = 10x_1 + 8x_2 
\]
\begin{center}
The goal is to maximize The Company income by maximizing \(f(x_1,x_2)\), while adhering to the limit of each coffe type constraints (Af,Col,Br).
\end{center}
\end{tcolorbox}
