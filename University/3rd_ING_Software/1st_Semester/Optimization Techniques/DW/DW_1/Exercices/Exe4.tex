\section*{Exercise 4: Stock Optimization}

A company must decide how many tons of two raw materials (M1 \& M2) to purchase in order to manufacture a product. 
Each ton of M1 costs 200 UM, and each ton of M2 costs 300 UM. For each finished product, the company needs 3 tons of M1 
and 2 tons of M2. The company has a budget of 30 000 UM and must produce at least 50 tons of the product.\\
Formulate this problem as a linear program to optimize expenses while respecting production constraints.

\vspace{1.5cm}
\textbf{\underline{\Large Solution:}}

\vspace{0.5cm}
\begin{minipage}[t]{0.45\textwidth}
    \textbf{\underline{Variables Definition:}} \\[1em] % Adjust vertical space here

    Let \(x_1\) be number of tons of M1. \\
   
    Let \(x_2\) be number of tons of M2.
\end{minipage}%
\hfill % This adds space between the two minipages
\begin{minipage}[t]{0.45\textwidth}
    \textbf{\underline{Constraint:}} \\[1em]
    \[
    \left\{
        \begin{array}{l}
            \forall x_1, x_2 \geq 0 \quad \text{(Non-negative number of tons of materials)}\\
            200x_1 + 300x_2  \leq 30 000 \quad \text{(Budget Contraint)}\\
            x_1 \geq 150 \quad \text{(At least 50 products from M1)}\\
            x_2 \geq 100 \quad \text{(At least 50 products from M2)}
        \end{array}
    \right.
    \]
\end{minipage}
\vspace{0.5cm}
\begin{tcolorbox}[title = Objectif Function]
\[
f(x_1,x_2) = 200x_1 + 300x_2 
\]
\begin{center}
The goal is to minimize The Company cost of production by minimizing \(f(x_1,x_2)\), while adhering to the limit of the budget and the min number of products.
\end{center}
\end{tcolorbox}

