\newpage
\null
\vspace{0.15cm}

\begin{center}
    \Huge{\textbf{\underline{Chapter 3: Communication Between Processes}}}
\end{center}

\setcounter{section}{0}

\vspace{0.35cm}


\section{Introduction}

\begin{prettyBox}{Introduction}{myblue}
When processes communicate with each other, it is called \textbf{Inter-Process Communication (IPC)}. IPC allows processes to share information and work together. There are two main scenarios: 
\begin{itemize}
    \item \textbf{Same Process}: A single program is divided into threads or split into child processes that communicate internally.
    \item \textbf{Different Processes}: Two separate programs running at the same time exchange information.
\end{itemize}
\end{prettyBox}

\vspace{0.35cm}

\section{Same Process}

\subsection{Child Processes}
\begin{prettyBox}{Child Processes}{myblue}
The parent process is the first instance of the program, which is split into child
processes using the primitive \textbf{fork()} that inherit all variables from parent and start executing after the fork. These processes communicate with
each other and with their parent through pipes.
\end{prettyBox}

\vspace{0.35cm}

\subsubsection{Fork}

\begin{prettyBox}{fork()}{myblue}
To use \textbf{fork()}, include the \texttt{<unistd.h>} header  
,the \textbf{fork()} function splits the current process into two: a parent process and a child process. It returns a process ID (\textbf{pid}) with the following results:
\begin{itemize}
    \item \textbf{fork() = -1} : Error, fork failed.
    \item \textbf{fork() = 0} : PID of the child process.
    \item \textbf{fork() \(>\) 0} : PID of the parent process.
\end{itemize}
\end{prettyBox}

\newpage
\null

\lstinputlisting[style=cstyle]{Chapters/C/Fork/ex1.c}


\vspace{0.5cm}

\begin{prettyBox}{exit()}{red}
    Terminates the current process with the given status. Use:
    \begin{itemize}
        \item \texttt{0} for a successful termination.
        \item \texttt{-1} for indicating an error.
    \end{itemize}
\end{prettyBox}

\vspace{0.35cm}

\begin{prettyBox}{When Forking Fails}{red}
    \begin{itemize}
        \item Exceeded the maximum number of child processes.
        \item Not enough memory or resources to fork.
    \end{itemize}
\end{prettyBox}

\vspace{0.35cm}

\subsubsection{Wait}
\begin{prettyBox}{wait(NULL)}{myblue}
To use \textbf{wait(NULL)}, include the \texttt{<sys/wait.h>} header used by parent processes , wait for its child processes to terminates
\end{prettyBox}

\newpage
\null
\lstinputlisting[style=cstyle]{Chapters/C/Fork/ex2.c}

\vspace{0.35cm}
\subsubsection{Pipe}
\begin{prettyBox}{pipe()}{myblue}
To use \texttt{pipe()} we need to include the header \texttt{<unistd.h>}\\
    The \texttt{pipe()} function creates a pipe and takes an array of two integers as input:
    \begin{itemize}
        \item \texttt{pipe(descriptor)} creates the pipe and stores the file descriptors in the \texttt{descriptor} array , if pipe creation fails, \texttt{-1} is returned.
    \end{itemize}
    The two elements of the \texttt{descriptor} array represent the ends of the pipe:
    \begin{itemize}
        \item \texttt{read(descriptor[0], \&var, sizeof(var))} reads from the pipe.
        \item \texttt{write(descriptor[1], \&var, sizeof(var))} writes to the pipe.
    \end{itemize}
    To close a pipe end, use the \texttt{close(descriptor[i])} function.

\end{prettyBox}


\newpage
\null

\begin{prettyBox}{Reminder}{red}
\textbf{Frequent Mistakes With Pipes}
\begin{itemize}
    \item \textbf{Closing All Read Descriptors Then Trying To Write On Pipe}: 
    Kernel doesn't find where to write the data, causing a SIGPIPE signal or an EPIPE error.
    \item \textbf{Not Closing All Write Descriptors Then Trying To Read From That Pipe}: 
    Kernel waits for the pipe to be written to, as it doesn't see an EOF (End Of File) signal.
\end{itemize}

\textbf{Solution}
\begin{itemize}
    \item \textbf{Leave at least one read end of the pipe open before writing to it}, 
    so the kernel knows where to write the data.
    \item \textbf{Always close all write ends of a pipe before reading from it}, 
    so the kernel knows that the pipe has reached its EOF.
\end{itemize}

\end{prettyBox}


\vspace{0.35cm}
\begin{prettyBox}{Note}{red}
    You can loop through \texttt{read()} using \texttt{read() > 0}, as it returns the number of bytes read. 
    As data is read, the cursor moves until \textbf{EOF} (End of File) is reached.
\end{prettyBox}

\vspace{1.25cm}

\subsubsection*{\underline{Example}}

\begin{itemize}
    \item \textbf{child\_1}: Accepts input of n characters. Write only alphabetic characters to the pipe, converts lowercase letters to uppercase. Stops when the user inputs '0'.
    \item \textbf{child\_2}: Prints the characters written to the pipe by \textbf{child\_1}.
    \item \textbf{parent}: Creates the child processes and waits for them to finish.
\end{itemize}

\newpage

\subsubsection*{\underline{Code Overview}}
\lstinputlisting[style=cstyle,basicstyle=\footnotesize\ttfamily]{Chapters/C/Fork/ex3.1.c}

\newpage

\subsubsection*{\underline{child\_1 function :}}
\lstinputlisting[style=cstyle,basicstyle=\footnotesize\ttfamily]{Chapters/C/Fork/ex3.2.c}

\vspace{0.5cm}

\subsubsection*{\underline{child\_1 function :}}
\lstinputlisting[style=cstyle,basicstyle=\footnotesize\ttfamily]{Chapters/C/Fork/ex3.3.c}

\vspace{1cm}

\begin{itemize}
    \item \textbf{parent}: Write 5 integers.
    \item \textbf{child}: Read the integers of parent and write their double.
    \item \textbf{parent}: Read Double integers of child.
\end{itemize}

\newpage
\null
\subsubsection*{\underline{Code Overview}}
\lstinputlisting[style=cstyle,basicstyle=\footnotesize\ttfamily]{Chapters/C/Fork/ex4.1.c}
\newpage
\subsubsection*{\underline{Parent's Functions}}
\lstinputlisting[style=cstyle,firstline=28,lastline=47,basicstyle=\footnotesize\ttfamily]{Chapters/C/Fork/ex4.c}
\subsubsection*{\underline{Child's Functions}}
\lstinputlisting[style=cstyle,firstline=12,lastline=26,basicstyle=\footnotesize\ttfamily]{Chapters/C/Fork/ex4.c}
\subsection{Threads}
\begin{prettyBox}{Thread}{myblue}
    prettyBox
\end{prettyBox}
