\newpage
\null
\vspace{0.15cm}

\begin{center}
    \Huge{\textbf{\underline{Chapter 4: Synchronization}}}
\end{center}

\vspace{0.25cm}

\setcounter{section}{0}

\section{Introduction}
\begin{prettyBox}{Introduction}{myblue}
In multi-threading and multi-processing, threads and processes often share resources.
If these resources are modified without proper synchronization, problems can arise. 
When a machine has only one CPU core, true parallelism doesn't occur; instead, the core
switches between tasks very quickly. This rapid switching, known as context switching, 
can lead to race conditions where values are overwritten unexpectedly,
causing incorrect results.
\end{prettyBox}

\vspace{0.35cm}

\section{Critical Section}
\begin{prettyBox}{Critical Section}{myblue}
A critical section is a part of a program or code that accesses shared resources
(like variables, memory, or hardware) and needs to be executed by only one thread or 
process at a time to prevent data corruption or inconsistency. Multiple threads or 
processes trying to access the same critical section simultaneously can lead 
to race conditions, where the outcome depends on the unpredictable order in which the threads run.
\end{prettyBox}

\vspace{0.35cm}

\section{Bernstein Conditions}
\begin{prettyBox}{Bernstein Conditions}{myblue}
The Bernstein Conditions are a set of conditions that verify if threads or processes
can truly execute at the same time without needing to switch between them or lock resources.

\begin{itemize}
    \item \textbf{No Read-Write Conflict:} No shared resources should be written to while they are being read by another thread or process.
    \item \textbf{No Write-Write Conflict:} No simultaneous writes to the same resource should occur to avoid data corruption.
    \item \textbf{No Read-Read Conflict:} Multiple threads or processes can read the same resource without issue, as reading does not modify the data.
\end{itemize}
\end{prettyBox}


\vspace{0.35cm}
\section{Tools To Synchronize}

\subsection{Semaphore}


\begin{prettyBox}{Semaphore}{myblue}
A semaphore is a synchronization primitive that maintains a positive integer value. This value can only be modified through its atomic operations, which are:

\begin{itemize}
    \item \textbf{P(semaphore):} If the semaphore value is 0, the process waits until it becomes greater than 0, then decrements the semaphore.
    \item \textbf{V(semaphore):} Increments the value of the semaphore by 1, signaling that a resource is available.
\end{itemize}

There are two main types of semaphores:
\begin{itemize}
    \item \textbf{Binary Semaphore:} The value is restricted to 0 or 1. It is used for locking critical sections to ensure that only one thread can access the resource at a time.
    \item \textbf{Mutex:} The value of the semaphore is unbounded and shared across multiple threads or processes. It not only enforces mutual exclusion, but also help with ressource management.
\end{itemize}
\end{prettyBox}

\vspace{0.35cm}
\begin{prettyBox}{When To Use Semaphore}{myblue}
Semaphores are commonly used in the following situations:

\begin{itemize}
    \item \textbf{Mutual Exclusion:} Ensuring that critical sections are accessed by only one thread or process at a time, preventing race conditions.
    \item \textbf{Signaling Between Processes:} Semaphores can be used to signal one process from another. For example, a producer can signal a consumer when new data is available.
    \item \textbf{Resource Management:} Managing access to a limited number of resources, such as database connections, printers, or buffers. A semaphore can keep track of how many resources are available and prevent overuse.
    \item \textbf{Synchronizing Threads:} In multi-threaded programs, semaphores can be used to coordinate the execution of threads. For example, you might use a semaphore to ensure that certain threads wait until others have completed their work.
\end{itemize}
\end{prettyBox}

\subsection{Monitor}
\begin{prettyBox}{Monitor}{myblue}
    prettyBox
\end{prettyBox}

\begin{prettyBox}{When To Use Semaphore}{myblue}
    prettyBox
\end{prettyBox}


