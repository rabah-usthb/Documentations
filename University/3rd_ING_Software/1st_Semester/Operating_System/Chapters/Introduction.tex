\section{Introduction}
\subsection{Operation System}
\subsubsection{What's An Operating System ?}
An operating system (OS) is a software program designed to manage and optimize the use of a machine's resources. 
It provides an easy to use \textbf{VM} virtual Machine to interact with the hardware and efficiently execute tasks. 
A typical machine consists of three main components:
\begin{itemize}
    \item \textbf{Memory:} Includes random access memory (RAM), registers, and storage devices such as 
        hard drives and solid-state drives (SSD).
    \item \textbf{CPU:} The central processing unit (CPU) consists of the Arithmetic Logic Unit (ALU) and 
        the Control Unit (CU), which process and execute instructions.
    \item \textbf{Peripherals:} External devices such as keyboards, mice, displays, printers, and storage media.
\end{itemize}
\subsubsection{Types Of Operating Systems}
\subsubsection{Core Functions Of An Operating System}
\subsubsection{History And Evolution Of Operating System}
