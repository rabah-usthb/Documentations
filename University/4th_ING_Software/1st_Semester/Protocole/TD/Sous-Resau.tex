\documentclass{article}
\usepackage[a4paper, left=1.5cm, right=1.5cm, top=1cm, bottom=2cm]{geometry}

\newcounter{commentCount}
\newcounter{filePrg}
\newcounter{inputPrg}

\usepackage[dvipsnames]{xcolor}
\usepackage{minted}

\usepackage[many]{tcolorbox}
\tcbuselibrary{listings}
\tcbuselibrary{minted}

\usepackage{ifthen}
\usepackage{fontawesome}

\usepackage{tabularx}
\newcolumntype{\CeX}{>{\centering\let\newline\\\arraybackslash}X}%
\newcommand{\TwoSymbolsAndText}[3]{%
  \begin{tabularx}{\textwidth}{c\CeX c}%
    #1 & #2 & #3
  \end{tabularx}%
}

\newtcblisting[use counter=inputPrg, number format=\arabic]{codeInput}[4]{
  listing engine=minted,
  minted language=#1,
  minted options={autogobble,breaklines,  firstnumber={#4}},
  listing only,
  size=title,
  arc=1.5mm,
  breakable,
  enhanced jigsaw,
  colframe=myblue,
  coltitle=White,
  boxrule=0.5mm,
  colback=white,
  coltext=Black,
  title=\TwoSymbolsAndText{\faCode}{%
  \textbf{Sql Program \thetcbcounter}\ifthenelse{\equal{#2}{}}{}{\textbf{: }#2}%
  }{\faCode},
  label=inputPrg:#3
}


\usepackage{boldline}
\usepackage{tikz,tcolorbox}
\usepackage{amsmath}
\usepackage[table,xcdraw]{xcolor}
\usepackage{listings}
\usepackage{array,multirow} % For customizing tables
\usepackage{booktabs} % For better horizontal lines
\usepackage{makecell}
\setlength{\parindent}{0pt}
\usepackage{siunitx}
\usepackage{tkz-tab}
\usepackage{amssymb}
\usepackage{amsmath}
\usepackage{enumitem}

\usepackage{caption}
\usepackage{float}


\newcommand{\exer}[1]{
  \section*{Exercice #1}
  \vspace{-0.5cm}
  \noindent\rule{\textwidth}{0.5pt}%
}

\newcommand{\tit}[1]{
\begin{center}
    \Large{\textbf{{#1}}}
\end{center}
}

\definecolor{commentgray}{HTML}{676160}
\definecolor{messagegreen}{HTML}{17B867}
\definecolor{myblue}{HTML}{10C2C4}

\tcbuselibrary{skins, breakable, theorems}


\newtcolorbox{prettyBox}[2]{
  enhanced,
  colback=white!90!#2,   % Background color based on the second parameter (color)
  colframe=#2!60!black,  % Frame color based on the second parameter (color)
  coltitle=white,        % Title color (white)
  fonttitle=\bfseries\Large,
  title=#1,              % Title from the first parameter
  boxrule=1mm,
  arc=0.5mm,
  drop shadow=#2!35!gray, % Drop shadow color based on the second parameter (color)
}


\lstdefinestyle{cmd}{
 basicstyle=\ttfamily,
 backgroundcolor=\color{lightgray!20},
 frame=single
}

\usepackage{minted}

\begin{document}


\tit{TD N\(^{\boldsymbol{\circ}}\)\hspace{0.1cm}1}

\vspace{0.15cm}

On a un reseau \verb|172.19.160.0/22| on veut le diviser en sous-reseaux :

\begin{figure}[H] 
  \centering 
  \includegraphics[width=0.5\textwidth]{r.png} 
\end{figure}

\vspace{-0.75cm}

\begin{figure}[H] 
  \centering 
  \includegraphics[width=0.25\textwidth]{tab.png} 
\end{figure}


Pour chaque sous-reseau \textbf{LAN} donner : taille idf machine ,taille idf sous-reseau, idf sous-reseau en binaire et l'adresse de ce sous-reseau

\begin{figure}[H] 
  \centering 
  \includegraphics[width=0.45\textwidth]{tab1.png} 
\end{figure}

\begin{prettyBox}{Remarque}{red}
    Il est important d'ordonner les sous-reseau de maniere decroissante par rapport a leur taille (nombre de PCs).
\end{prettyBox}

\vspace{0.35cm}

Comment Remplire Chaque Colonne ?
\begin{itemize}
    \item Taille IDF machine = $m$ \quad tell que :
        \[ 2^m - 2 \geq \text{Taille s-reseau}  \]
    \item Taille IDF s-reseau = $r$ \quad tell que :
        \[r = 32 - m\]

    \item Taille IDF s-reseau = $t$ \quad tell que :
        \[ t = r - r_{\text{initial}}\]
    \item adresse\_s-reseau 
\[
\begin{cases}
    \text{adresse}_\text{initial}/r & \text{Si premier \textbf{LAN}} \\
    \text{adresse}_\text{precedent} + 2^{m_\text{precedent}}/r & \text{Sinon} 
\end{cases}
\]

\item IDF s-reseau en binaire : $b$ = adresse\_s-reseau $[r_{\text{initial}}+1,r_{\text{initial}}+t]$ (on compte de gauche a droite)

\end{itemize}
\newpage

\section*{LAN\_E : }

\begin{itemize}
    \item On a $2^9 - 2 = 510 \geq 500 \Rightarrow m_0 = 9$
    \item $r_0 = 32 - m_0 = 32 - 9 = 23$
    \item $t_0 = r_0 - r_{\text{initial}} = 23 - 22 = 1$
    \item adresse\_s-reseau  = \verb|172.19.160.0/23|
    \item On sait que chaque partie de l'adresse occupe 8 bits et on s'interesse au 23eme bits \[[r_{\text{initial}}+1,r_{\text{initial}}+t_0] = [23,23] = 23\]
    Donc on va voir que la troisieme partie \verb|160|, on doit convertir \verb|160| en binaire :\\

\[
\begin{alignedat}{3}
160 &/ 2 &= 80  &\quad \text{rest} &= 0 \\
 80 &/ 2 &= 40  &\quad \text{rest} &= 0 \\
 40 &/ 2 &= 20  &\quad \text{rest} &= 0 \\
 20 &/ 2 &= 10  &\quad \text{rest} &= 0 \\
 10 &/ 2 &= 5   &\quad \text{rest} &= 0 \\
  5 &/ 2 &= 2   &\quad \text{rest} &= 1 \\
  2 &/ 2 &= 1   &\quad \text{rest} &= 0 \\
  1 &/ 2 &= 0   &\quad \text{rest} &= 1
\end{alignedat}
\]
Donc \verb|160| en binaire = $1010 00\underbrace{0}_{\text{23eme bits}}0 \Rightarrow b_0 = 0$

\end{itemize}

\vspace{0.25cm}

\section*{LAN\_A : }

\begin{itemize}
    \item On a $2^8 - 2 = 254 \geq 200 \Rightarrow m_1 = 8$
    \item $r_1 = 32 - m_1 = 32 - 8 = 24$
    \item $t_1 = r_1 - r_{\text{initial}} = 24 - 22 = 2$
    \item adresse\_s-reseau  = \verb|172.19.160.0/23| + $2^{m_0}$ = \verb|172.19.160.0/23| + $2^{9}$ = \verb|172.19.160.0/23| + 512

\[
\begin{array}{r}
    172.19.160.0/23 \\[0.1cm]
    +  \hspace{1.75cm}           512\\ \hline
  =172.19.162.0/24
\end{array}
\]
le max de chaque segment est de 255 quant il ya un \verb|overflow| on devise par 256 le qoution est ajoute au segment suivant
et le segment courrant recoit le rest : $521 = 256\times2+0$
    \item On s'interesse au bits 23-24 \[[r_{\text{initial}}+1,r_{\text{initial}}+t_1] = [23,24] \]
    Donc on va voir que la troisieme partie \verb|162|, on doit convertir \verb|162| en binaire = $1010 00\underbrace{10}_{\text{23-24eme bits}} \Rightarrow b_1 = 10$


\end{itemize}

\newpage

\section*{LAN\_C : }

\begin{itemize}
    \item On a $2^6 - 2 =62 \geq 60 \Rightarrow m_2 = 6$
    \item $r_2 = 32 - m_2 = 32 - 6 = 26$
    \item $t_2 = r_2 - r_{\text{initial}} = 26 - 22 = 4$
    \item adresse\_s-reseau  = \verb|172.19.162.0/24| + $2^{m_1}$ = \verb|172.19.162.0/24| + $2^{8}$ = \verb|172.19.162.0/24| + 256

\[
\begin{array}{r}
    172.19.162.0/24 \\[0.1cm]
    +  \hspace{1.75cm}           256\\ \hline
  =172.19.163.0/26
\end{array}
\]
$256 = 256\times1+0$
    \item On s'interesse au bits 23-26 \[[r_{\text{initial}}+1,r_{\text{initial}}+t_2] = [23,26] \]
        Donc on va voir que la 3eme et 4eme partie \verb|163.0|, on doit convertir \verb|163.0| en binaire = $1010 00 \hspace{-0.4cm}\underbrace{11 . 00}_{\text{23-26eme bits}} \hspace{-0.4cm} \Rightarrow b_2 = 1100$


\end{itemize}

\vspace{0.25cm}

\section*{LAN\_B : }

\begin{itemize}
    \item On a $2^6 - 2 =62 \geq 50 \Rightarrow m_3 = 6$
    \item $r_3 = 32 - m_3 = 32 - 6 = 26$
    \item $t_3 = r_3 - r_{\text{initial}} = 26 - 22 = 4$
    \item adresse\_s-reseau  = \verb|172.19.163.0/26| + $2^{m_2}$ = \verb|172.19.163.0/26| + $2^{6}$ = \verb|172.19.163.0/26| + 64

\[
\begin{array}{r}
    172.19.163.0/26 \\[0.1cm]
    +  \hspace{1.65cm}           64\\ \hline
  =172.19.163.64/26
\end{array}
\]
    \item On s'interesse au bits 23-26 \[[r_{\text{initial}}+1,r_{\text{initial}}+t_3] = [23,26] \]
        Donc on va voir que la 3eme et 4eme partie \verb|163.64|, on doit convertir \verb|163.64| en binaire = $1010 00 \hspace{-0.4cm}\underbrace{11 . 01}_{\text{23-26eme bits}} \hspace{-0.4cm}\\ \Rightarrow b_3 = 1101$


\end{itemize}


\vspace{0.25cm}

\section*{LAN\_D : }

\begin{itemize}
    \item On a $2^5 - 2 =30 \geq 25 \Rightarrow m_4 = 5$
    \item $r_4 = 32 - m_4 = 32 - 5 = 27$
    \item $t_4 = r_4 - r_{\text{initial}} = 27- 22 = 5$
    \item adresse\_s-reseau  = \verb|172.19.163.64/26| + $2^{m_3}$ = \verb|172.19.163.64/26| + $2^{6}$ = \verb|172.19.163.64/26| + 64

\[
\begin{array}{r}
    172.19.163.64/26 \\[0.1cm]
    +  \hspace{2.25cm}           64\\ \hline
  =172.19.163.128/27
\end{array}
\]
    \item On s'interesse au bits 23-27 \[[r_{\text{initial}}+1,r_{\text{initial}}+t_4] = [23,27] \]
        Donc on va voir que la 3eme et 4eme partie \verb|163.128|, on doit convertir \verb|163.128| en binaire = $1010 00 \hspace{-0.4cm}\underbrace{11 . 100}_{\text{23-27eme bits}} \hspace{-0.4cm}\\ \Rightarrow b_4 = 11100$


\end{itemize}

\newpage

\section*{R1 - R2 : }

\begin{itemize}
    \item On a $2^2 - 2 = 2 \geq 2 \Rightarrow m_5 = 2$
    \item $r_5 = 32 - m_5 = 32 - 2 = 30$
    \item $t_5 = r_5 - r_{\text{initial}} = 30- 22 = 8$
    \item adresse\_s-reseau  = \verb|172.19.163.128/27| + $2^{m_4}$ = \verb|172.19.163.128/27| + $2^{5}$ = \verb|172.19.163.128/27| + 32

\[
\begin{array}{r}
    172.19.163.128/27 \\[0.1cm]
    +  \hspace{2.25cm}           32\\ \hline
  =172.19.163.160/30
\end{array}
\]
    \item On s'interesse au bits 23-30 \[[r_{\text{initial}}+1,r_{\text{initial}}+t_5] = [23,30] \]
        Donc on va voir que la 3eme et 4eme partie \verb|163.160|, on doit convertir \verb|163.160| en binaire = $1010 00 \hspace{-0.15cm}\underbrace{11 . 1010 00}_{\text{23-30eme bits}} \hspace{-0.4cm}\\ \Rightarrow b_5 = 11101000$


\end{itemize}

\vspace{0.25cm}

\section*{R1 - R3 : }

\begin{itemize}
    \item On a $2^2 - 2 = 2 \geq 2 \Rightarrow m_6 = 2$
    \item $r_6 = 32 - m_6 = 32 - 2 = 30$
    \item $t_6 = r_6 - r_{\text{initial}} = 30- 22 = 8$
    \item adresse\_s-reseau  = \verb|172.19.163.160/30| + $2^{m_5}$ = \verb|172.19.163.160/30| + $2^{2}$ = \verb|172.19.163.160/30| + 4

\[
\begin{array}{r}
    172.19.163.160/30 \\[0.1cm]
    +  \hspace{2.25cm}           4\\ \hline
  =172.19.163.164/30
\end{array}
\]
    \item On s'interesse au bits 23-30 \[[r_{\text{initial}}+1,r_{\text{initial}}+t_6] = [23,30] \]
        Donc on va voir que la 3eme et 4eme partie \verb|163.164|, on doit convertir \verb|163.164| en binaire = $1010 00 \hspace{-0.15cm}\underbrace{11 . 1010 01}_{\text{23-30eme bits}} \hspace{-0.4cm}\\ \Rightarrow b_6 = 11101001$


\end{itemize}


\begin{figure}[H] 
  \centering 
  \includegraphics[width=0.45\textwidth]{tab2.png} 
\end{figure}

\end{document}
