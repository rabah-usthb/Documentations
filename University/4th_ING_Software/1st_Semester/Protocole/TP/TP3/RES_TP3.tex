\documentclass{article}
\usepackage[a4paper, left=1.5cm, right=1.5cm, top=1cm, bottom=2cm]{geometry}


\newcounter{commentCount}
\newcounter{filePrg}
\newcounter{inputPrg}

\usepackage[dvipsnames]{xcolor}
\usepackage{minted}

\usepackage[many]{tcolorbox}
\tcbuselibrary{listings}
\tcbuselibrary{minted}

\usepackage{ifthen}
\usepackage{fontawesome}

\usepackage{tabularx}
\newcolumntype{\CeX}{>{\centering\let\newline\\\arraybackslash}X}%
\newcommand{\TwoSymbolsAndText}[3]{%
  \begin{tabularx}{\textwidth}{c\CeX c}%
    #1 & #2 & #3
  \end{tabularx}%
}

\newtcblisting[use counter=inputPrg, number format=\arabic]{codeInput}[4]{
  listing engine=minted,
  minted language=#1,
  minted options={autogobble,breaklines,  firstnumber={#4}},
  listing only,
  size=title,
  arc=1.5mm,
  breakable,
  enhanced jigsaw,
  colframe=myblue,
  coltitle=White,
  boxrule=0.5mm,
  colback=white,
  coltext=Black,
  title=\TwoSymbolsAndText{\faCode}{%
  \textbf{Sql Program \thetcbcounter}\ifthenelse{\equal{#2}{}}{}{\textbf{: }#2}%
  }{\faCode},
  label=inputPrg:#3
}


\usepackage{boldline}
\usepackage{tikz,tcolorbox}
\usepackage{amsmath}
\usepackage[table,xcdraw]{xcolor}
\usepackage{listings}
\usepackage{array,multirow} % For customizing tables
\usepackage{booktabs} % For better horizontal lines
\usepackage{makecell}
\setlength{\parindent}{0pt}
\usepackage{siunitx}
\usepackage{tkz-tab}
\usepackage{amssymb}
\usepackage{amsmath}
\usepackage{enumitem}

\usepackage{caption}
\usepackage{float}


\newcommand{\exer}[1]{
  \section*{Exercice #1}
  \vspace{-0.5cm}
  \noindent\rule{\textwidth}{0.5pt}%
}

\newcommand{\tit}[1]{
\begin{center}
    \Large{\textbf{{#1}}}
\end{center}
}

\definecolor{commentgray}{HTML}{676160}
\definecolor{messagegreen}{HTML}{17B867}
\definecolor{myblue}{HTML}{10C2C4}

\tcbuselibrary{skins, breakable, theorems}


\newtcolorbox{prettyBox}[2]{
  enhanced,
  colback=white!90!#2,   % Background color based on the second parameter (color)
  colframe=#2!60!black,  % Frame color based on the second parameter (color)
  coltitle=white,        % Title color (white)
  fonttitle=\bfseries\Large,
  title=#1,              % Title from the first parameter
  boxrule=1mm,
  arc=0.5mm,
  drop shadow=#2!35!gray, % Drop shadow color based on the second parameter (color)
}



\lstdefinestyle{cmd}{
 basicstyle=\ttfamily,
 backgroundcolor=\color{lightgray!20},
 frame=single
}

\usepackage{minted}

\begin{document}


\tit{TP N\(^{\boldsymbol{\circ}}\)\hspace{0.1cm}3}

\begin{enumerate}

\item Realiser la topologie suivante et configurer uniquement les adresses et gateway:

\begin{figure}[H] 
  \centering 
  \includegraphics[width=0.6\textwidth]{state1.png} 
\end{figure}

configuration Des PCs : 



\begin{figure}[H] 
  \centering 
  \includegraphics[width=0.55\textwidth]{confPC0.png} 
\end{figure}

\vspace{-0.1cm}


\begin{figure}[H] 
  \centering 
  \includegraphics[width=0.55\textwidth]{confPC1.png} 
\end{figure}


\begin{figure}[H] 
  \centering 
  \includegraphics[width=0.55\textwidth]{confPC2.png} 
\end{figure}


\begin{figure}[H] 
  \centering 
  \includegraphics[width=0.55\textwidth]{confPC3.png} 
\end{figure}

\newpage
Configuration des routeurs :

\begin{figure}[H] 
  \centering 
  \includegraphics[width=0.6\textwidth]{route01.png} 
\end{figure}

\begin{figure}[H] 
  \centering 
  \includegraphics[width=0.6\textwidth]{route02.png} 
\end{figure}


\begin{figure}[H] 
  \centering 
  \includegraphics[width=0.6\textwidth]{route11.png} 
\end{figure}

\begin{figure}[H] 
  \centering 
  \includegraphics[width=0.6\textwidth]{route12.png} 
\end{figure}

\vspace{0.25cm}

\begin{prettyBox}{Remarque}{red}
    On a mis comme mask \verb|255.255.255.252| dans le reseau \verb|200.100.12.0/30|
    car la partie machine prend 30 bits : 
    \[\underbrace{11111111}_{8\text{bits} = 255} \hspace{0.15cm}. \hspace{0.15cm} \underbrace{11111111}_{8\text{bits} = 255} \hspace{0.15cm} . \hspace{0.15cm} \underbrace{11111111}_{8\text{bits} = 255} \hspace{0.15cm} . \hspace{0.15cm} \underbrace{11111100}_{6\text{bits} = 252} \]
\end{prettyBox}

\newpage

\item Comment Configure \textbf{OSPF}? 

    \vspace{0.15cm}

    \begin{prettyBox}{OSPF}{myblue}
        On a deux methodes :
        \begin{itemize}
            \item \textbf{Configuration Mode Routeur : }
                \begin{itemize}
                    \item On doit etre au niveau 3 pour activer le protocole \textbf{OSPF} avec la commande : \verb|router ospf <id_processus>|
                    \item Apres avoir executer la commande on passe au niveau 4.
                    \item Pour ajouter une route dynamic on utilise la commande : \\\verb|network <reseau_connu> <mask_inverse> area <id_region>|
                    \item Le masque inverse c'est prendre chaque partie du mask normal et lui soustraire 255 :
                        \vspace{-0.2cm}
                        \[ 255.255.255.0 \Rightarrow (255-255).(255-255).(255-255).(255-0) \Rightarrow 0.0.0.255\]
                \end{itemize}
            \item \textbf{Configuration Mode Inteface : }

                \begin{itemize}
                    \item On doit etre au niveau 3 pour acceder a une interface avec la commande : \\\verb|interface <nom_interface>|
                    \item Apres avoir executer la commande on passe au niveau 4.
                    \item Pour activer le protocole \textbf{OSPF} pour l'interface on utilise :\\\verb|ip ospf <id_processus> area <id_region>|
                \end{itemize}
            \end{itemize}
    
        \end{prettyBox}

        \vspace{1.25cm}

        \begin{prettyBox}{Remaque}{red}
            \begin{itemize}
                \item On peut faire une configuration \textbf{multi-regional} ou chaque region a une collection de routeurs pour \textbf{faciliter le managment}
                \item Dans notre cas simple on va utilise une configuration \textbf{mono-regional} on va juste utilise \\region id = 0 pour dire qu'on a \textbf{une seule
                    region globale}.
                \item L'id processus permet d'executer plusieur instances de l'\textbf{OSPF} sur le meme routeur.
            \end{itemize}
        \end{prettyBox}

    \newpage
\item Configurer l'\textbf{OSPF} sur le routeur 0 avec la premiere methode et afficher la table de routage :  

\begin{figure}[H] 
  \centering 
  \includegraphics[width=0.6\textwidth]{confroute0.png} 
\end{figure}

\begin{figure}[H] 
  \centering 
  \includegraphics[width=0.6\textwidth]{table0.png} 
\end{figure}

\item Configurer l'\textbf{OSPF} sur le routeur 1 avec la deuxieme methode et afficher la table de routage :  

\begin{figure}[H] 
  \centering 
  \includegraphics[width=0.6\textwidth]{confroute1.png} 
\end{figure}

\newpage
\item C'est quoi la distance administrative et le cout de l'\textbf{OSPF}? 

\vspace{0.25cm}

\begin{prettyBox}{Distance Administrative \& Cout}{myblue}
    \begin{itemize}
        \item Distance Administrative = 110.
        \item Le cout est calculer avec la bandwidth (capacité de transfert de données) des cables avec la formule suivant :

            \[
            \sum_{i=1}^{m} \text{round}\left(\frac{\text{Reference Bandwidth}}{\text{interface}_{i} \text{  Bandwidth}}\right)
            \]
            \[
            \text{round}(n) = 
            \begin{cases}
                \text{if  } n<1 &: 1\\
                \text{if  } n \in \mathbb{N} &: n\\
                \text{if  } n \in \mathbb{R} &: \text{Integer Part} + 1
            \end{cases}
            \]
        \item $m$ est le nombre d'interfaces emetteur pour atteindre la switch du reseau recepteur.
        \item Par defaut la reference bandwidth est de 100Mbps donc $100\times 10^{6}$ bps = $10^{8}$bps.
    \end{itemize}
\end{prettyBox}

\vspace{1.25cm}

\item changer le cout des interface g 0/0 des deux routeurs , afficher la table de routage expliquer le nouveau cout :

    \begin{prettyBox}{Changer Le Cout}{myblue}
        Pour changer le cout on doit etre au niveau 4 (conf interface) et on utilise la commande : \verb|ip ospf cost <nouveau_cout>| 
    \end{prettyBox}


\begin{figure}[H] 
  \centering 
  \includegraphics[width=0.6\textwidth]{changeCost.png} 
\end{figure}

\begin{figure}[H] 
  \centering 
  \includegraphics[width=0.6\textwidth]{changeCost2.png} 
\end{figure}

\newpage

\begin{prettyBox}{Nouveau Cout}{myblue}
    \begin{itemize}
        \item cout de la route vers \verb|192.168.10.0| est : 
            \begin{align*}
                \text{cout}_{192.168.10.0} &= \text{cout}(\text{g}0/1_{\text{router 1}}) + \text{cout}(\text{g}0/0_{\text{router 0}})\\[0.1cm]
                                           &= \text{round}\left( \frac{10^{8}\text{bps}}{1\text{Gbps}} \right) + 5 \\[0.1cm] 
                                           &= \text{round}\left( \frac{10^{8}\text{bps}}{10^{9}\text{bps}} \right) + 5\\[0.1cm]
                                           &= \text{round}(0.1) + 5\\
                                           &= 1 + 5\\
                                           &=\boxed{6}
            \end{align*}
        \item cout de la route vers \verb|192.168.20.0| est : 
            \begin{align*}
                \text{cout}_{192.168.20.0} &= \text{cout}(\text{g}0/1_{\text{router 0}}) + \text{cout}(\text{g}0/0_{\text{router 1}})\\[0.1cm]
                                           &= \text{round}\left( \frac{10^{8}\text{bps}}{1\text{Gbps}} \right) + 8 \\[0.1cm] 
                                           &= \text{round}\left( \frac{10^{8}\text{bps}}{10^{9}\text{bps}} \right) + 8\\[0.1cm]
                                           &= \text{round}(0.1) + 8\\
                                           &= 1 + 8\\
                                           &=\boxed{9}
            \end{align*}

    \end{itemize}


\begin{figure}[H] 
  \centering 
  \includegraphics[width=0.6\textwidth]{cal.png} 
\end{figure}
\end{prettyBox}

\newpage 

\item Faite la topologie suivante et configurer les adresses et les gateway :

\begin{figure}[H] 
  \centering 
  \includegraphics[width=0.6\textwidth]{state2.png} 
\end{figure}


\begin{figure}[H] 
  \centering 
  \includegraphics[width=0.6\textwidth]{2pc0.png} 
\end{figure}


\begin{figure}[H] 
  \centering 
  \includegraphics[width=0.6\textwidth]{2pc1.png} 
\end{figure}


\begin{figure}[H] 
  \centering 
  \includegraphics[width=0.6\textwidth]{2pc2.png} 
\end{figure}



\begin{figure}[H] 
  \centering 
  \includegraphics[width=0.6\textwidth]{2r0.png} 
\end{figure}

\begin{figure}[H] 
  \centering 
  \includegraphics[width=0.6\textwidth]{2r1.png} 
\end{figure}


\begin{figure}[H] 
  \centering 
  \includegraphics[width=0.6\textwidth]{2r2.png} 
\end{figure}

\newpage

\item configurer L'\textbf{OSPF} sur chaque routeur avec la deuxieme methode :
\begin{figure}[H] 
  \centering 
  \includegraphics[width=0.6\textwidth]{2cr0.png} 
\end{figure}

\begin{figure}[H] 
  \centering 
  \includegraphics[width=0.6\textwidth]{2cr1.png} 
\end{figure}


\begin{figure}[H] 
  \centering 
  \includegraphics[width=0.6\textwidth]{2cr2.png} 
\end{figure}

\newpage

\item Afficher la table de routage de tout les routeurs : 
\begin{figure}[H] 
  \centering 
  \includegraphics[width=0.6\textwidth]{2t0.png} 
\end{figure}

\begin{figure}[H] 
  \centering 
  \includegraphics[width=0.6\textwidth]{2t1.png} 
\end{figure}


\begin{figure}[H] 
  \centering 
  \includegraphics[width=0.6\textwidth]{2t2.png} 
\end{figure}

\vspace{0.15cm}

\item Faite un ping en simulation depuis le PC0 au PC1:

\begin{figure}[H] 
  \centering 
  \includegraphics[width=0.6\textwidth]{ping1.png} 
\end{figure}

\begin{prettyBox}{Remarque}{red}
    On remarque que le routeur prefere passe directement par le routeur 2 (cout \textbf{OSPF} = 2) au lieu de passe par le routeur 1 
puis le routeur 2 (cout \textbf{OSPF} = 3).
\end{prettyBox}

\newpage

\item Desactiver l'interface \verb|g0/2| du routeur 2 pour simuler une panne puis affiche la table de routage du routeur 1: 


\begin{figure}[H] 
  \centering 
  \includegraphics[width=0.6\textwidth]{sh2.png} 
\end{figure}

\begin{figure}[H] 
  \centering 
  \includegraphics[width=0.6\textwidth]{rupdate.png} 
\end{figure}

\begin{prettyBox}{Remarque}{red}
    Puisque l'interface reliant le routeur 2 au 0 n'est plus functionelle , \textbf{OSPF} a dynamiquement mis a jour la route vers 
    \verb|192.168.30.0| du routeur 0 pour qu'il passe par le routeur 1 et le cout deviendra 3.
\end{prettyBox}

\vspace{1.25cm}

\item Refaire le ping en simulation :

\begin{figure}[H] 
  \centering 
  \includegraphics[width=0.6\textwidth]{ping2.png} 
\end{figure}

\newpage

\item Ajouter un routeur  et configurer son adresse :

\begin{figure}[H] 
  \centering 
  \includegraphics[width=0.6\textwidth]{state3.png} 
\end{figure}


\begin{figure}[H] 
  \centering 
  \includegraphics[width=0.6\textwidth]{r3.png} 
\end{figure}

\vspace{0.5cm}
\item Le routeur 3 qu'on a ajouter est un routeur provider donc il ya des possibilite et combinaison infini d'adresse ip venant depuis
    ce routeur comment alors on va configurer cette route ?

\begin{prettyBox}{0.0.0.0}{myblue}
    On va utilise un routage static et on dit que l'adresse et mask du reseau inconnu est \verb|0.0.0.0| signifiant n'import quelle reseau qui ne figurent pas 
    dans la table de routage du routeur, elle est appelle la \textbf{route par defaut} . 
\end{prettyBox}

\vspace{0.5cm}
\item Configure la route static dans le routeur 1 :

\begin{figure}[H] 
  \centering 
  \includegraphics[width=0.6\textwidth]{def.png} 
\end{figure}

\newpage

\item Est-ce qu'on va configurer la \textbf{route par defaut} sur tout les autre routeurs?Ou peut on propager cette route automatiquement au routeurs?

\vspace{0.15cm}
    \begin{prettyBox}{Propagation}{myblue}
        \begin{itemize}
            \item \textbf{OSPF} nous permet de propager une \textbf{route par defaut} definit sur un routeur au autre routeurs voisin.
            \item On doit etre au niveau 4 \textbf{OSPF} avec la commande :\\ \verb|router ospf <id_processus>|
            \item On utilise cetter commande pour propager la \textbf{route par defaut} : \\\verb|default-information originiate|
        \end{itemize}
    \end{prettyBox}

\vspace{0.15cm}
\begin{figure}[H] 
  \centering 
  \includegraphics[width=0.6\textwidth]{pr.png} 
\end{figure}

\vspace{0.5cm}

\item Afficher la table de routage des deux autre routeur pour s'assure que la \textbf{route par defaut} c'est propage :

\begin{figure}[H] 
  \centering 
  \includegraphics[width=0.6\textwidth]{3t0.png} 
\end{figure}

\begin{figure}[H] 
  \centering 
  \includegraphics[width=0.6\textwidth]{3t2.png} 
\end{figure}



\end{enumerate}
\end{document}

