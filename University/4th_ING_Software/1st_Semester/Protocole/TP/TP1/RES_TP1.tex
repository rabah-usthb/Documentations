\documentclass{article}
\usepackage[a4paper, left=1.5cm, right=1.5cm, top=1cm, bottom=2cm]{geometry}


\newcounter{commentCount}
\newcounter{filePrg}
\newcounter{inputPrg}

\usepackage[dvipsnames]{xcolor}
\usepackage{minted}

\usepackage[many]{tcolorbox}
\tcbuselibrary{listings}
\tcbuselibrary{minted}

\usepackage{ifthen}
\usepackage{fontawesome}

\usepackage{tabularx}
\newcolumntype{\CeX}{>{\centering\let\newline\\\arraybackslash}X}%
\newcommand{\TwoSymbolsAndText}[3]{%
  \begin{tabularx}{\textwidth}{c\CeX c}%
    #1 & #2 & #3
  \end{tabularx}%
}

\newtcblisting[use counter=inputPrg, number format=\arabic]{codeInput}[4]{
  listing engine=minted,
  minted language=#1,
  minted options={autogobble,breaklines,  firstnumber={#4}},
  listing only,
  size=title,
  arc=1.5mm,
  breakable,
  enhanced jigsaw,
  colframe=myblue,
  coltitle=White,
  boxrule=0.5mm,
  colback=white,
  coltext=Black,
  title=\TwoSymbolsAndText{\faCode}{%
  \textbf{Sql Program \thetcbcounter}\ifthenelse{\equal{#2}{}}{}{\textbf{: }#2}%
  }{\faCode},
  label=inputPrg:#3
}


\usepackage{boldline}
\usepackage{tikz,tcolorbox}
\usepackage{amsmath}
\usepackage[table,xcdraw]{xcolor}
\usepackage{listings}
\usepackage{array,multirow} % For customizing tables
\usepackage{booktabs} % For better horizontal lines
\usepackage{makecell}
\setlength{\parindent}{0pt}
\usepackage{siunitx}
\usepackage{tkz-tab}
\usepackage{amssymb}
\usepackage{amsmath}
\usepackage{enumitem}

\usepackage{caption}
\usepackage{float}


\newcommand{\exer}[1]{
  \section*{Exercice #1}
  \vspace{-0.5cm}
  \noindent\rule{\textwidth}{0.5pt}%
}

\newcommand{\tit}[1]{
\begin{center}
    \Large{\textbf{{#1}}}
\end{center}
}

\definecolor{commentgray}{HTML}{676160}
\definecolor{messagegreen}{HTML}{17B867}
\definecolor{myblue}{HTML}{10C2C4}

\tcbuselibrary{skins, breakable, theorems}


\newtcolorbox{prettyBox}[2]{
  enhanced,
  colback=white!90!#2,   % Background color based on the second parameter (color)
  colframe=#2!60!black,  % Frame color based on the second parameter (color)
  coltitle=white,        % Title color (white)
  fonttitle=\bfseries\Large,
  title=#1,              % Title from the first parameter
  boxrule=1mm,
  arc=0.5mm,
  drop shadow=#2!35!gray, % Drop shadow color based on the second parameter (color)
}



\lstdefinestyle{cmd}{
 basicstyle=\ttfamily,
 backgroundcolor=\color{lightgray!20},
 frame=single
}

\usepackage{minted}

\begin{document}


\tit{TP N\(^{\boldsymbol{\circ}}\)\hspace{0.1cm}1}

\begin{enumerate}

\item selectioner des PC (appareille terminaux) et en mettre 2 comme suit :

\begin{figure}[H] 
  \centering 
  \includegraphics[width=0.5\textwidth]{terminalDevices.png} 
\end{figure}

\item mettre un \textbf{routeur 2911} et une \textbf{switch 2950-24} depuis appreille reseau : 


\begin{figure}[H] 
  \centering 
  \includegraphics[width=0.5\textwidth]{networkDevices.png} 
\end{figure}


\begin{figure}[H] 
  \centering 
  \includegraphics[width=0.5\textwidth]{routers.png} 
\end{figure}

\begin{figure}[H] 
  \centering 
  \includegraphics[width=0.5\textwidth]{switches.png} 
\end{figure}

\item pour connecter des appareille de differente couches (pc a switch, switch a routeur...etc)\\
on utilise un \textbf{cable droit} :


\begin{figure}[H] 
  \centering 
  \includegraphics[width=0.5\textwidth]{cables.png} 
\end{figure}

\begin{figure}[H] 
  \centering 
  \includegraphics[width=0.5\textwidth]{droit.png} 
\end{figure}

\item apres avoir place et branche tous on obtient :

\begin{figure}[H] 
  \centering 
  \includegraphics[width=0.75\textwidth]{state1.png} 
\end{figure}

\newpage

\item pour mettre des label on utilise l'outil text on va mettre un label
    sur notre reseau local '192.168.10.0/24':

\begin{figure}[H] 
  \centering 
  \includegraphics[width=0.5\textwidth]{textTool.png} 
\end{figure}

\begin{figure}[H] 
  \centering 
  \includegraphics[width=0.5\textwidth]{textTool2.png} 
\end{figure}

\item Comment configurer notre routeur?

\begin{prettyBox}{Configuration}{myblue}
    \begin{itemize}
        \item Le routeur n'a ni calvier ni ecran donc pour pouvoir 
      le configurer on le connecte a un pc grace a un cable bleau ciel appele 
  \textbf{cable console}.
        \item On met le cable dans le port \textbf{console} du routeur et dans le port
            \textbf{serie RS 232} du pc.
        \item Si le pc n'a pas de port serie on utilise un \textbf{adapteur} qu'on met dans le port 
            \textbf{USB}.
        \item On utilisera une application pour dialoguer avec le routeur depuis le PC.
    \end{itemize}
\end{prettyBox}


\begin{figure}[H] 
  \centering 
  \includegraphics[width=0.5\textwidth]{console.png} 
\end{figure}

\begin{figure}[H] 
  \centering 
  \includegraphics[width=0.3\textwidth]{portConsole.png} 
\end{figure}

\begin{figure}[H] 
  \centering 
  \includegraphics[width=0.3\textwidth]{portrs232.png} 
\end{figure}

\newpage
\item Comment acceder a l'application?

\begin{prettyBox}{Application}{myblue}
    \begin{itemize}
        \item click droit sur le pc connecte au routeur avec le cable console.
        \item selectioner la tab desktop.
        \item choisir l'application \textbf{Terminal}.
    \end{itemize}
\end{prettyBox}


\begin{figure}[H] 
  \centering 
  \includegraphics[width=0.6\textwidth]{clickPc.png} 
\end{figure}

\begin{figure}[H] 
  \centering 
  \includegraphics[width=0.6\textwidth]{deskTerminal.png} 
\end{figure}

\item lancer l'application terminal avec un click droit :

\begin{figure}[H] 
  \centering 
  \includegraphics[width=0.6\textwidth]{clickTerminal.png} 
\end{figure}

\newpage
\item Appuier sur OK :


\begin{figure}[H] 
  \centering 
  \includegraphics[width=0.7\textwidth]{no.png} 
\end{figure}

\begin{prettyBox}{Remarque}{red}
    on repond avec no pour la question \textbf{'would you like to enter the initial configuration dialog ?'} pour
    garder les configurations initiaux et avoir un demarage par defaut sans configuration personalise. 
\end{prettyBox}

\item Comment utiliser l'application terminal pour configurer le routeur?

\begin{prettyBox}{Terminal}{myblue}
    \begin{itemize}
        \item Actuellement on est au \textbf{niveau 1 (mode utilisateur)} c'est pour ca on a 
            un sign superieur :\\ 
            {\centering \verb|Router>|}
        \item Pour Configurer le routeur on doit passer par le \textbf{niveau 2 mode administrateur} on aura un
            sign dièse : \verb|Router#|
        \item Pour passer au niveau 2 on utilise : \verb|enable| ou \verb|en|.
        \item Pour Configurer les parametres d'un routeur on doit au moins etre au \textbf{niveau 3} on aura :
            \verb|Router(config)#|
        \item Pour passer au niveau 3 on utilise : \verb|conf terminal| ou \verb|conf t|.
    \end{itemize} 
\end{prettyBox}



\begin{figure}[H] 
  \centering 
  \includegraphics[width=0.6\textwidth]{admin.png} 
\end{figure}

\begin{figure}[H] 
  \centering 
  \includegraphics[width=0.6\textwidth]{configt.png} 
\end{figure}

\item C'est quoi la premiere chose a configure ?

\begin{prettyBox}{Hostname}{myblue}
    \begin{itemize}
        \item Hostname permet d'identifier le router en lui donnant un nom.
        \item Facilite le managment des routeurs pour les admins.
        \item On doit etre au niveau 3 pour le configurer.
        \item La commande : \verb|hostname <nom>|
    \end{itemize}
\end{prettyBox}


\begin{figure}[H] 
  \centering 
  \includegraphics[width=0.6\textwidth]{hostname.png} 
\end{figure}

\newpage

\item Qu'avez vous remarquer de bizzar qu'on est passe au mode administrateur?

\begin{prettyBox}{Mot De Passe}{myblue}
    \begin{itemize}
    \item Le routeur a donne acces directement sans demander de mot de passe.
    \item Pour Configurer un mot de passe on doit etre au niveau 3.
    \item La commande : \verb|enable password <mot de passe>|
    \end{itemize}
\end{prettyBox}



\begin{figure}[H] 
  \centering 
  \includegraphics[width=0.6\textwidth]{password.png} 
\end{figure}



\begin{figure}[H] 
  \centering 
  \includegraphics[width=0.6\textwidth]{enterpass.png} 
\end{figure}

\begin{prettyBox}{EXIT}{red}
    \begin{itemize}
        \item \verb|exit| permet de descendre d'un niveau.
    \end{itemize}
\end{prettyBox}

\vspace{0.5cm}

\item Pourquoi quand on a tapper le mot de passe rien ne s'affiche meme pas des etoiles asterix (*)?

\begin{prettyBox}{Taille}{myblue}
    Pour des raisons de securite afin de ne pas revele le nombre de character (taille) du mot de passe.
\end{prettyBox}

\vspace{0.5cm}
\item Comment changer les addresses IP d'une Interface ou configurer le routage dynamic \textbf{(RIP)} :

\begin{prettyBox}{Niveau 4}{myblue}
    On doit passer au niveau 4 on a plusieur palier (different type de niveau 4 chaqu'un configure quelque chose).
\end{prettyBox}

\vspace{0.5cm}

\item Configurer l'interface du routeur connecte avec la switch:

\begin{prettyBox}{Interface}{myblue}
    \begin{itemize}
        \item Hover sur le cable entre routeur et switch pour voir les interfaces.
        \item Pour passer au niveau 4 on utilise : \verb|interface <nom_interface>| ou \verb|int <nom_interface>|
        \item Pour Configurer l'addresse on utilise : \verb|ip add <ip> <mask>| ou \verb|ip address <ip> <mask>|
        \item Pour allumer l'interface : \verb|no shutdown| ou \verb|no sh|
    \end{itemize}
\end{prettyBox}


\begin{figure}[H] 
  \centering 
  \includegraphics[width=0.6\textwidth]{nomInterface.png} 
\end{figure}

\vspace{0.25cm}


\begin{prettyBox}{Remarque}{red}
    \begin{itemize}
        \item meme si on pratique (pas simulation) on a besoin de cable console et d'un pc pour configurer
    un routeur dans packet tracer pour similifier les choses et eviter de r'ajouter un cable pour chaque routeur
    on utilise le CLI du routeur (on y accede avec click droit sur routeur).
       \item Donc on supprime le cable console avec le del tool. 
    \end{itemize}
\end{prettyBox}


\begin{figure}[H] 
  \centering 
  \includegraphics[width=0.5\textwidth]{delTool.png} 
\end{figure}


\begin{figure}[H] 
  \centering 
  \includegraphics[width=0.6\textwidth]{clickRouter.png} 
\end{figure}


\begin{figure}[H] 
  \centering 
  \includegraphics[width=0.6\textwidth]{cli.png} 
\end{figure}

\newpage


\begin{figure}[H] 
  \centering 
  \includegraphics[width=0.6\textwidth]{interfaceRouter1.png} 
\end{figure}


\begin{figure}[H] 
  \centering 
  \includegraphics[width=0.6\textwidth]{confRoutSwitch.png} 
\end{figure}

\item Comment verifier l'adressage du routeur?

\begin{prettyBox}{Hover/Config}{myblue}
    \begin{itemize}
        \item Hover sur le routeur pour afficher la configuration.
        \item Click droit sur le routeur et selectioner la tab config ensuite selectioner
            l'interface.
        \item On peut aussi configurer le routeur depuis la tab config en ustilisant l'interface graphique
            cette derniere executra les commandes equivalentes qui sont visible dans une fenetre en bas.
    \end{itemize}
\end{prettyBox}


\begin{figure}[H] 
  \centering 
  \includegraphics[width=0.6\textwidth]{verify1.png} 
\end{figure}


\begin{figure}[H] 
  \centering 
  \includegraphics[width=0.6\textwidth]{verify2.png} 
\end{figure}

\newpage

\item Comment configurer les addresses des PC?

\begin{prettyBox}{IP Configuration/Config}{myblue}
    \begin{itemize}
        \item Click droit sur PC selectionner la tab desktop et choisir l'application 
        \textbf{IP Configuration}.
        \item Click droit sur PC selectionner la tab config et choisir l'interface.
    \end{itemize}
\end{prettyBox}


\begin{figure}[H] 
  \centering 
  \includegraphics[width=0.6\textwidth]{deskIPCONF.png} 
\end{figure}


\begin{figure}[H] 
  \centering 
  \includegraphics[width=0.6\textwidth]{configPCMeth1.png} 
\end{figure}


\begin{figure}[H] 
  \centering 
  \includegraphics[width=0.6\textwidth]{configPCMeth2.png} 
\end{figure}

\newpage

\item Configurer l'addresse des PC avec la methode \textbf{IP Configuration} :


\begin{figure}[H] 
  \centering 
  \includegraphics[width=0.6\textwidth]{configPC1.png} 
\end{figure}


\begin{figure}[H] 
  \centering 
  \includegraphics[width=0.6\textwidth]{configPC2.png} 
\end{figure}

\vspace{1cm}

\item Verifier avec le hover sur PC:

\begin{figure}[h] 
  \centering 
  \includegraphics[width=0.6\textwidth]{verifypc1.png} 
\end{figure}


\begin{figure}[h] 
  \centering 
  \includegraphics[width=0.6\textwidth]{verifypc2.png} 
\end{figure}

\newpage

\item Comment verifier la configuration des PC avec \textbf{C}omman\textbf{d} \textbf{p}rompt (CMD)?

\begin{prettyBox}{CMD}{myblue}
    \begin{itemize}
        \item Click droit sur le PC choisir la tab desktop selectionner l'application \textbf{Command Prompt}.
        \item utilise la commande : \verb|ipconfig| pour affichier l'addresse \textbf{ipv6 ,ipv4 et la passrelle}.
        \item utilise la commande : \verb|ipconfig /all| pour aussi afficher la \textbf{mac addresse} (adresse physique).
    \end{itemize}
\end{prettyBox}


\begin{figure}[H] 
  \centering 
  \includegraphics[width=0.6\textwidth]{deskCMD.png} 
\end{figure}



\begin{figure}[H] 
  \centering 
  \includegraphics[width=0.6\textwidth]{testConfPC1.png} 
\end{figure}

\newpage
\item Teste la connection entre le PC1 et les autre appareilles et lui meme. Le routeur avec les autre appareilles et lui meme.

\begin{prettyBox}{ping}{myblue}
    \begin{itemize}
        \item Pour faire un ping depuis un PC on utilise l'application CMD.
        \item Pour faire un ping depuis le routeur on utilise le soit le CLI du routeur ou
            terminal depuis un PC connecter avec cable console on doit etre au niveau 1 ou 2(utilisateur ou admin).
        \item la commande : \verb|ping <ip address>|
    \end{itemize}
\end{prettyBox}

\begin{figure}[h] 
  \centering 
  \includegraphics[width=0.6\textwidth]{testPingPC1.png} 
\end{figure}


\begin{figure}[h] 
  \centering 
  \includegraphics[width=0.6\textwidth]{testPingRouter.png} 
\end{figure}

\newpage
\item Copier les PC et la switch et coller puis brancher entre la switch et le routeur avec un cable droit:

\begin{figure}[h] 
  \centering 
  \includegraphics[width=0.6\textwidth]{copy1.png} 
\end{figure}


\begin{figure}[h] 
  \centering 
  \includegraphics[width=0.6\textwidth]{state2.png} 
\end{figure}

\item On a deux reseaus que faire ?

\begin{prettyBox}{Addressage}{myblue}
    \begin{itemize}
        \item On a besoin d'une addresse pour ce deuxieme reseau : \textbf{192.168.20.0/24}.
        \item Configurer les addresses IP des PC et l'interface routeur relier avec la deuxieme switch. 
    \end{itemize}
\end{prettyBox}



\begin{figure}[H] 
  \centering 
  \includegraphics[width=0.6\textwidth]{confPC3.png} 
\end{figure}


\begin{figure}[H] 
  \centering 
  \includegraphics[width=0.6\textwidth]{confPC4.png} 
\end{figure}



\begin{figure}[H] 
  \centering 
  \includegraphics[width=0.6\textwidth]{IntrefaceCheck.png} 
\end{figure}


\begin{figure}[H] 
  \centering 
  \includegraphics[width=0.6\textwidth]{confRoutInt2.png} 
\end{figure}

\vspace{0.5cm}

\item Faire un ping graphique entre le PC du deuxieme reseau et ses voisin (routeur , autre PC) puis avec le PC
    de l'autre reseau :


\begin{figure}[H] 
  \centering 
  \includegraphics[width=0.6\textwidth]{pingTool.png} 
\end{figure}


\begin{figure}[H] 
  \centering 
  \includegraphics[width=0.6\textwidth]{goodPing.png} 
\end{figure}



\begin{figure}[H] 
  \centering 
  \includegraphics[width=0.6\textwidth]{failedPing.png} 
\end{figure}

\newpage
\item Pourqoui le ping entre des pc de different reseau echou?

\begin{prettyBox}{Gateway}{myblue}
    \begin{itemize}
        \item Les reseau ne se connaissent pas entre eux donc on doit mettre une \textbf{passrelle(gateway)} pour chaque PC.
        \item On met \textbf{192.168.10.1} comme gateway pour les PCs du reseau \textbf{192.168.10.0/24}
        \item On met \textbf{192.168.20.1} comme gateway pour les PCs du reseau \textbf{192.168.20.0/24}
    \end{itemize}
\end{prettyBox}


\begin{figure}[H] 
  \centering 
  \includegraphics[width=0.6\textwidth]{gatePC1.png} 
\end{figure}


\begin{figure}[H] 
  \centering 
  \includegraphics[width=0.6\textwidth]{gatePC2.png} 
\end{figure}


\begin{figure}[H] 
  \centering 
  \includegraphics[width=0.6\textwidth]{gatePC3.png} 
\end{figure}


\begin{figure}[H] 
  \centering 
  \includegraphics[width=0.6\textwidth]{gatePC4.png} 
\end{figure}

\newpage
\item Faire un ping graphique entre PC de reseaus different :


\begin{figure}[H] 
  \centering 
  \includegraphics[width=0.6\textwidth]{goodPing2e.png} 
\end{figure}

\vspace{0.25cm}

\item Copier PCs , routeur , switch et coller :

\begin{figure}[H] 
  \centering 
  \includegraphics[width=0.6\textwidth]{state3.png} 
\end{figure}

\item Configurer les addresse des PCs avec leur gatway et les addresses routeur du troisieme reseau \textbf{192.168.30.0/24}:


\begin{figure}[H] 
  \centering 
  \includegraphics[width=0.6\textwidth]{confiPC5.png} 
\end{figure}


\begin{figure}[H] 
  \centering 
  \includegraphics[width=0.6\textwidth]{confiPC6.png} 
\end{figure}


\begin{figure}[H] 
  \centering 
  \includegraphics[width=0.6\textwidth]{confRout2.png} 
\end{figure}

\item relie les routeur avec un cable \textbf{croise} (appreille de la meme couche):


\begin{figure}[H] 
  \centering 
  \includegraphics[width=0.6\textwidth]{crossCable.png} 
\end{figure}


\begin{figure}[H] 
  \centering 
  \includegraphics[width=0.6\textwidth]{state4.png} 
\end{figure}

\item Configuration des addresses du reseau entre routeur \textbf{192.168.0.0/24} :

\begin{figure}[H] 
  \centering 
  \includegraphics[width=0.6\textwidth]{confRouteRoute1.png} 
\end{figure}


\begin{figure}[H] 
  \centering 
  \includegraphics[width=0.6\textwidth]{confRoutRout2.png} 
\end{figure}

\newpage

\item Faire un ping entre PCs du troisieme reseau et avec son routeur et les PCs d'autre reseau.

\begin{figure}[H] 
  \centering 
  \includegraphics[width=0.6\textwidth]{p1.png} 
\end{figure}

\item Pourquoi echec de ping entre les PCs du troisieme reseau avec les autre? et avec le premier routeur? 

\begin{prettyBox}{Les Routes}{myblue}
    \begin{itemize}
        \item Le premier routeur ne connait pas le reseau \textbf{192.168.30.0/24}.
        \item Le deuxieme routeur no connait pas les reseau \textbf{192.168.10.0/24} et \textbf{192.168.20.0/24}.
        \item pour consulte les routes connus d'un routeur on doit afficher la \textbf{table de routage}.
        \item la commande pour afficher la table routage :
            \begin{itemize}
                \item \verb|show ip route| (pour niveau 2)
                \item \verb|do show ip route| (pour niveau 3)
            \end{itemize}
        \item pour ajouter on route static on doit etre au niveau 3 et on utilise : \verb|ip route <ip_reseau_iconnu> <mask_reseau_iconnu> <ip_voisin_pour_acceder>|
    \end{itemize}
\end{prettyBox}

\vspace{0.35cm}

\item configurer une route static pour le premier routeur :
\begin{figure}[H] 
  \centering 
  \includegraphics[width=0.6\textwidth]{tableRoute1.png} 
\end{figure}


\begin{figure}[H] 
  \centering 
  \includegraphics[width=0.6\textwidth]{staticRoute1.png} 
\end{figure}

\newpage
\item faire une simulation ping entre PC du troisieme reseau avec un pc d'un autre reseau:


\begin{figure}[H] 
  \centering 
  \includegraphics[width=0.6\textwidth]{routeerror.png} 
\end{figure}

\item Pourquoi echec? 

\begin{prettyBox}{Deuxieme Routeur}{myblue}
    \begin{itemize}
        \item Meme si le premier routeur connait le reseau \textbf{192.168.30.0/24} grace au routage static.
        \item Le deuxieme routeur ne connait pas les autre reseau donc on doit lui ajoutter deux route static.
    \end{itemize}
\end{prettyBox}

\vspace{0.25cm}

\item Configurer les routes statics pour le deuxieme routeur :

\begin{figure}[H] 
  \centering 
  \includegraphics[width=0.6\textwidth]{table2.png} 
\end{figure}


\begin{figure}[H] 
  \centering 
  \includegraphics[width=0.6\textwidth]{confroute22.png} 
\end{figure}

\item Refaire le ping :

\begin{figure}[H] 
  \centering 
  \includegraphics[width=0.6\textwidth]{goodpingend.png} 
\end{figure}





\end{enumerate}



\end{document}
