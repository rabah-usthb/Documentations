\documentclass{article}
\usepackage[a4paper, left=1.5cm, right=1.5cm, top=1cm, bottom=2cm]{geometry}


\newcounter{commentCount}
\newcounter{filePrg}
\newcounter{inputPrg}

\usepackage[dvipsnames]{xcolor}
\usepackage{minted}

\usepackage[many]{tcolorbox}
\tcbuselibrary{listings}
\tcbuselibrary{minted}

\usepackage{ifthen}
\usepackage{fontawesome}

\usepackage{tabularx}
\newcolumntype{\CeX}{>{\centering\let\newline\\\arraybackslash}X}%
\newcommand{\TwoSymbolsAndText}[3]{%
  \begin{tabularx}{\textwidth}{c\CeX c}%
    #1 & #2 & #3
  \end{tabularx}%
}

\newtcblisting[use counter=inputPrg, number format=\arabic]{codeInput}[4]{
  listing engine=minted,
  minted language=#1,
  minted options={autogobble,breaklines,  firstnumber={#4}},
  listing only,
  size=title,
  arc=1.5mm,
  breakable,
  enhanced jigsaw,
  colframe=myblue,
  coltitle=White,
  boxrule=0.5mm,
  colback=white,
  coltext=Black,
  title=\TwoSymbolsAndText{\faCode}{%
  \textbf{Sql Program \thetcbcounter}\ifthenelse{\equal{#2}{}}{}{\textbf{: }#2}%
  }{\faCode},
  label=inputPrg:#3
}


\usepackage{boldline}
\usepackage{tikz,tcolorbox}
\usepackage{amsmath}
\usepackage[table,xcdraw]{xcolor}
\usepackage{listings}
\usepackage{array,multirow} % For customizing tables
\usepackage{booktabs} % For better horizontal lines
\usepackage{makecell}
\setlength{\parindent}{0pt}
\usepackage{siunitx}
\usepackage{tkz-tab}
\usepackage{amssymb}
\usepackage{amsmath}
\usepackage{enumitem}

\usepackage{caption}
\usepackage{float}


\newcommand{\exer}[1]{
  \section*{Exercice #1}
  \vspace{-0.5cm}
  \noindent\rule{\textwidth}{0.5pt}%
}

\newcommand{\tit}[1]{
\begin{center}
    \Large{\textbf{{#1}}}
\end{center}
}

\definecolor{commentgray}{HTML}{676160}
\definecolor{messagegreen}{HTML}{17B867}
\definecolor{myblue}{HTML}{10C2C4}

\tcbuselibrary{skins, breakable, theorems}


\newtcolorbox{prettyBox}[2]{
  enhanced,
  colback=white!90!#2,   % Background color based on the second parameter (color)
  colframe=#2!60!black,  % Frame color based on the second parameter (color)
  coltitle=white,        % Title color (white)
  fonttitle=\bfseries\Large,
  title=#1,              % Title from the first parameter
  boxrule=1mm,
  arc=0.5mm,
  drop shadow=#2!35!gray, % Drop shadow color based on the second parameter (color)
}



\lstdefinestyle{cmd}{
 basicstyle=\ttfamily,
 backgroundcolor=\color{lightgray!20},
 frame=single
}

\usepackage{minted}

\begin{document}


\tit{TP N\(^{\boldsymbol{\circ}}\)\hspace{0.1cm}2}

\begin{enumerate}

\item Realiser la topologie suivante avec routage static :

\begin{figure}[H] 
  \centering 
  \includegraphics[width=0.6\textwidth]{state1.png} 
\end{figure}

\begin{prettyBox}{Route Static}{myblue}
    On a deux route static pour chaque routeur :
    \begin{itemize}
        \item Router 0 :
            \begin{itemize}
                \item \verb|ip route 192.168.30.0 255.255.255.0 192.168.0.2|
                \item \verb|ip route 192.168.40.0 255.255.255.0 192.168.0.2|
            \end{itemize}
         
        \item Router 1 :
            \begin{itemize}
                \item \verb|ip route 192.168.10.0 255.255.255.0 192.168.0.1|
                \item \verb|ip route 192.168.20.0 255.255.255.0 192.168.0.1|
            \end{itemize}
    \end{itemize}
\end{prettyBox}

\vspace{0.5cm}

\item Activer et configurer \textbf{RIP} sur le routeur 0 :

\begin{prettyBox}{RIP}{myblue}
    \begin{itemize}
        \item Activer le protocole RIP on doit etre au niveau 3 et utilise : \verb|router rip|
        \item \verb|router rip| nous fait passse au niveau 4 aussi \verb|Router(config-router)#|.
        \item Ajouter un reseau voisin au routage dynamic on doit etre au niveau 4 et utilise: \verb|network <reseau_connu>|
        \item Enlever un reseau voisin au routage dynamic on doit etre au niveau 4 et utilise: \verb|no network <reseau_connu>|
    \end{itemize}
\end{prettyBox}


\begin{figure}[H] 
  \centering 
  \includegraphics[width=0.7\textwidth]{configRoute0.png} 
\end{figure}


\newpage

\item Qu'est ce qui manque a notre configuration \textbf{RIP}?

\begin{prettyBox}{Routeur 1}{myblue}
 On n'a ni activer ni configurer le protocole \textbf{RIP} sur l'autre routeur.
\end{prettyBox}

\vspace{0.5cm}

\item Activer et configurer \textbf{RIP} sur le routeur 1 :

\begin{figure}[H] 
  \centering 
  \includegraphics[width=0.7\textwidth]{configRoute1.png} 
\end{figure}

\vspace{0.5cm}

\item Pourquoi on ne voit aucune route R (\textbf{RIP}) dans la table de routage
de deux routeurs malgres qu'on a activer et configurer \textbf{RIP}?

\begin{prettyBox}{Distance Administrative}{myblue}
\begin{figure}[H] 
  \centering 
  \includegraphics[width=0.7\textwidth]{ex.png} 
\end{figure}
le routeur a prefere utilise le routage static car il a une distance administrative plus petite \\(static) $1<120$ (\textbf{RIP})
\end{prettyBox}

\newpage

\item Enlever tout les routages static :

\begin{prettyBox}{Enlever Routage Static}{myblue}
    \verb|no ip route <ip_reseau_iconnu> <mask_reseau_iconnu> <ip_voisin_pour_acceder> |
\end{prettyBox}


\begin{figure}[H] 
  \centering 
  \includegraphics[width=0.7\textwidth]{removeStaticRoute0.png} 
\end{figure}


\begin{figure}[H] 
  \centering 
  \includegraphics[width=0.7\textwidth]{removeStaticRoute1.png} 
\end{figure}


\begin{prettyBox}{Remarque}{red}
    On remarque que la table de routage c'est met a jour automatiquement apres avoir 
    enelever le routage static on a un routage \textbf{RIP} dynamic avec une distance
    administrative de 120 et le hop cost de 1 (passe par un routeur pour atteindre la distination).
\end{prettyBox}

\newpage

\item Changer la topologie pour qu'elle devient comme suit :

\begin{figure}[H] 
  \centering 
  \includegraphics[width=0.7\textwidth]{state2.png} 
\end{figure}



\begin{figure}[H] 
  \centering 
  \includegraphics[width=0.7\textwidth]{state3.png} 
\end{figure}

\item Configurer tout les routeurs comme il se doit :

\begin{figure}[H] 
  \centering 
  \includegraphics[width=0.7\textwidth]{route2.1.png} 
\end{figure}


\begin{figure}[H] 
  \centering 
  \includegraphics[width=0.7\textwidth]{route2.2.png} 
\end{figure}


\begin{figure}[H] 
  \centering 
  \includegraphics[width=0.7\textwidth]{route2.3.png} 
\end{figure}


\begin{figure}[H] 
  \centering 
  \includegraphics[width=0.7\textwidth]{route2.4.png} 
\end{figure}


\begin{figure}[H] 
  \centering 
  \includegraphics[width=0.7\textwidth]{route0.1.png} 
\end{figure}


\begin{figure}[H] 
  \centering 
  \includegraphics[width=0.7\textwidth]{route0.2.png} 
\end{figure}



\begin{figure}[H] 
  \centering 
  \includegraphics[width=0.7\textwidth]{route1.1.png} 
\end{figure}


\begin{figure}[H] 
  \centering 
  \includegraphics[width=0.7\textwidth]{route1.2.png} 
\end{figure}

\item Afficher les tables de routage apres mise-a-jour :


\begin{figure}[H] 
  \centering 
  \includegraphics[width=0.7\textwidth]{table0.png} 
\end{figure}


\begin{figure}[H] 
  \centering 
  \includegraphics[width=0.7\textwidth]{table1.png} 
\end{figure}



\begin{figure}[H] 
  \centering 
  \includegraphics[width=0.7\textwidth]{table2.png} 
\end{figure}

\item faite un ping simulation du PC2 au PC1:

\begin{figure}[H] 
  \centering 
  \includegraphics[width=0.7\textwidth]{goodping.png} 
\end{figure}

\begin{prettyBox}{Remarque}{red}
    On remarque que le roteur 1 prend le chemin avec le plus hop cost il est alle au routeur 2 directment
    sans avoir a passer au routeur 0 puis routeur 2.
\end{prettyBox}


\vspace{0.5cm}
\item Desactiver l'interface G 0/1 du routeur 1 :
\begin{figure}[H] 
  \centering 
  \includegraphics[width=0.7\textwidth]{disactivate.png} 
\end{figure}

\newpage
\item Afficher la table de routage du routeur 1 apres avoir desactiver l'interface :
\begin{figure}[H] 
  \centering 
  \includegraphics[width=0.7\textwidth]{update.png} 
\end{figure}

\begin{prettyBox}{Remarque}{red}
    Pour acceder au reseau 192.168.20.0/24 il passe par le routeur 0 parceque l'interface relie avec
    le routeur 2 a ete desactiver :

\begin{figure}[H] 
  \centering 
  \includegraphics[width=0.7\textwidth]{comp.png} 
\end{figure}



\end{prettyBox}

\vspace{0.5cm}

\item refaire un ping simulation du PC2 au PC1:

\begin{figure}[H] 
  \centering 
  \includegraphics[width=0.7\textwidth]{lastping.png} 
\end{figure}



\end{enumerate}


\end{document}

