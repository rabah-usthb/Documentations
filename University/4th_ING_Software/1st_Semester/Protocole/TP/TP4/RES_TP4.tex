\documentclass{article}
\usepackage[a4paper, left=1.5cm, right=1.5cm, top=1cm, bottom=2cm]{geometry}

\newcounter{commentCount}
\newcounter{filePrg}
\newcounter{inputPrg}

\usepackage[dvipsnames]{xcolor}
\usepackage{minted}

\usepackage[many]{tcolorbox}
\tcbuselibrary{listings}
\tcbuselibrary{minted}

\usepackage{ifthen}
\usepackage{fontawesome}

\usepackage{tabularx}
\newcolumntype{\CeX}{>{\centering\let\newline\\\arraybackslash}X}%
\newcommand{\TwoSymbolsAndText}[3]{%
  \begin{tabularx}{\textwidth}{c\CeX c}%
    #1 & #2 & #3
  \end{tabularx}%
}

\newtcblisting[use counter=inputPrg, number format=\arabic]{codeInput}[4]{
  listing engine=minted,
  minted language=#1,
  minted options={autogobble,breaklines,  firstnumber={#4}},
  listing only,
  size=title,
  arc=1.5mm,
  breakable,
  enhanced jigsaw,
  colframe=myblue,
  coltitle=White,
  boxrule=0.5mm,
  colback=white,
  coltext=Black,
  title=\TwoSymbolsAndText{\faCode}{%
  \textbf{Sql Program \thetcbcounter}\ifthenelse{\equal{#2}{}}{}{\textbf{: }#2}%
  }{\faCode},
  label=inputPrg:#3
}


\usepackage{boldline}
\usepackage{tikz,tcolorbox}
\usepackage{amsmath}
\usepackage[table,xcdraw]{xcolor}
\usepackage{listings}
\usepackage{array,multirow} % For customizing tables
\usepackage{booktabs} % For better horizontal lines
\usepackage{makecell}
\setlength{\parindent}{0pt}
\usepackage{siunitx}
\usepackage{tkz-tab}
\usepackage{amssymb}
\usepackage{amsmath}
\usepackage{enumitem}

\usepackage{caption}
\usepackage{float}


\newcommand{\exer}[1]{
  \section*{Exercice #1}
  \vspace{-0.5cm}
  \noindent\rule{\textwidth}{0.5pt}%
}

\newcommand{\tit}[1]{
\begin{center}
    \Large{\textbf{{#1}}}
\end{center}
}

\definecolor{commentgray}{HTML}{676160}
\definecolor{messagegreen}{HTML}{17B867}
\definecolor{myblue}{HTML}{10C2C4}

\tcbuselibrary{skins, breakable, theorems}


\newtcolorbox{prettyBox}[2]{
  enhanced,
  colback=white!90!#2,   % Background color based on the second parameter (color)
  colframe=#2!60!black,  % Frame color based on the second parameter (color)
  coltitle=white,        % Title color (white)
  fonttitle=\bfseries\Large,
  title=#1,              % Title from the first parameter
  boxrule=1mm,
  arc=0.5mm,
  drop shadow=#2!35!gray, % Drop shadow color based on the second parameter (color)
}


\lstdefinestyle{cmd}{
 basicstyle=\ttfamily,
 backgroundcolor=\color{lightgray!20},
 frame=single
}

\usepackage{minted}

\begin{document}

\tit{TP N\(^{\boldsymbol{\circ}}\)\hspace{0.1cm}4}

\begin{enumerate}

\item Realiser la topologie suivante et configurer uniquement les adresses et gateway des PCs:

\begin{figure}[H] 
  \centering 
  \includegraphics[width=0.5\textwidth]{state1.png} 
\end{figure}

configuration Des PCs : 

\begin{figure}[H] 
  \centering 
  \includegraphics[width=0.55\textwidth]{pc0.png} 
\end{figure}

\vspace{-0.25cm}


\begin{figure}[H] 
  \centering 
  \includegraphics[width=0.55\textwidth]{pc1.png} 
\end{figure}

\vspace{-0.25cm}

\begin{figure}[H] 
  \centering 
  \includegraphics[width=0.55\textwidth]{pc2.png} 
\end{figure}

\vspace{-0.25cm}

\begin{figure}[H] 
  \centering 
  \includegraphics[width=0.55\textwidth]{pc3.png} 
\end{figure}


\item Comment configurer le vlan sur les switch?

    \begin{prettyBox}{Vlan}{myblue}
        \begin{itemize}
            \item Pour creer un vlan on doit etre au niveau 3 et utilise la commande : \verb|vlan <id_vlan>|
            \item Apres l'execution de la commande on passe au niveau 4 \verb|(config-vlan)|
            \item On peut attribue un nom au vlan au niveau 4 avec la commande : \verb|name <nom_vlan>|
            \item Pour attribue des vlan au ports on fait :
                \begin{itemize}
                    \item on doit passer au niveau 4 configuration d'interface on peut soit selectioner\\ une seul interface avec
                        \verb|int <nom_int>| example \verb|int f0/1| \\
                        ou plusieur interface avec \verb|int range <range_int_1> , ... , <range_int_n>| \\example \verb|int range f0/5 ,f0/10-15|
                    \item On doit configurer le type des ports avec \verb|switchport mode <mode>|
                    \item On a deux type de mode :
                        \begin{itemize}
                            \item \textbf{access} : le mode par defaut utilise si le port est relie\\ a un appareille terminal ( pc , laptop , imprimente ...etc)
                            \item \textbf{trunk} : le port est relie a un appareille reseau ( switch , hub , routeur...etc)
                        \end{itemize}
                    \item Affecter un vlan au port : \verb|switchport <mode> vlan <id_vlan>|
                \end{itemize}
            \item On peut afficher les nom , id , port associe au vlan avec la commande : \verb|show vlan|
            \item On peut supprimer un vlan avec la commande : \verb|no vlan <id_vlan>|
            \item On peut supprimer tout les vlan avec la commande : \verb|delete vlan.dot|
        \end{itemize}
    \end{prettyBox}

    \vspace{1cm}

\item Creer les vlan 10 student , vlan 20 teacher sur les deux switchs:

\begin{figure}[H] 
  \centering 
  \includegraphics[width=0.6\textwidth]{vlan0.png} 
\end{figure}

\begin{figure}[H] 
  \centering 
  \includegraphics[width=0.6\textwidth]{vlan1.png} 
\end{figure}

\newpage
\item Attribue les vlans au ports des deux switchs:

\begin{figure}[H] 
  \centering 
  \includegraphics[width=0.6\textwidth]{port0.png} 
\end{figure}


\begin{figure}[H] 
  \centering 
  \includegraphics[width=0.6\textwidth]{port1.png} 
\end{figure}

\vspace{0.15cm}

\item Qu'est ce qui manque a la configuration :

\begin{prettyBox}{Trunk}{myblue}
    On doit mettre les ports entre switch au mode \textbf{trunk}
\end{prettyBox}


\begin{figure}[H] 
  \centering 
  \includegraphics[width=0.6\textwidth]{tr0.png} 
\end{figure}


\begin{figure}[H] 
  \centering 
  \includegraphics[width=0.6\textwidth]{tr1.png} 
\end{figure}

\newpage

\item Faire un ping entre PC du meme vlan et apres de vlan different :

\begin{figure}[H] 
  \centering 
  \includegraphics[width=0.6\textwidth]{ping.png} 
\end{figure}

\vspace{0.35cm}

\item Pourquoi le ping entre vlan different a echoue ? 

\begin{prettyBox}{Reseau Iconnu}{myblue}
    Parceque se sont des reseau different ils ne se connaissent pas entre eux donc on a besoin d'un routage.
\end{prettyBox}

\vspace{1cm}

\item Ajouter un routeur :
\begin{figure}[H] 
  \centering 
  \includegraphics[width=0.6\textwidth]{state2.png} 
\end{figure}

\newpage

\item Comment configurer le routeur?

    \begin{prettyBox}{Configuration}{myblue}
        \begin{itemize}
            \item Selection partition de l'interface : \verb|int <nom_interface>.<id_vlan>|
            \item Activer le protocole \textbf{821.1Q} qui permet de changer le vlan de la trame du vlan emetteur au vlan recepteur avec 
                la commande \verb|encapsulation dot1Q <id_vlan>|
            \item Attribution d'une adresse IP au routeur a cette partition de l'interface : \verb|ip add <ip> <mask>|
            \item Ne pas oublier de mettre en mode \textbf{trunk} l'interface du switch reliant au router.
        \end{itemize}
    \end{prettyBox}

    \vspace{1cm}

\item Configurer le router et tester le ping :

\begin{figure}[H] 
  \centering 
  \includegraphics[width=0.6\textwidth]{tr3.png} 
\end{figure}


\begin{figure}[H] 
  \centering 
  \includegraphics[width=0.6\textwidth]{confroute0.png} 
\end{figure}


\begin{figure}[H] 
  \centering 
  \includegraphics[width=0.6\textwidth]{pingGood.png} 
\end{figure}

\newpage

\item Expliquer comment la trame change si on ping du PC0 au PC1:

\begin{figure}[H] 
  \centering 
  \includegraphics[width=0.6\textwidth]{0pdu.png} 
\end{figure}

\begin{figure}[H] 
  \centering 
  \includegraphics[width=0.6\textwidth]{1pdu.png} 
\end{figure}


\begin{figure}[H] 
  \centering 
  \includegraphics[width=0.6\textwidth]{2pdu.png} 
\end{figure}

\vspace{0.15cm}

\begin{prettyBox}{Expliquation}{myblue}
    \begin{itemize}
        \item Premierment une requete \textbf{ARP} depuis le pc0 au routeur (default gateway)
            le routeur repondra avec son adresse mac.
        \item Apres ca le pc va envoye encore la trame a la switch mais cette fois la switch va mettre les information vlan , l'id du vlan
            est stocke sur les 12 dernier bit \textbf{TCI} :
            \[0\underbrace{00A}_{\text{12 bits du vlan id}} \Longrightarrow (00A)_{10} = 10 \Longrightarrow \text{vlan de l'emetteur pc0}  \]
        \item Apres ca le routeur va mettre a jour la partie \textbf{TCI} avec le protocole \textbf{862.1 Q} :
            \[0\underbrace{014}_{\text{12 bits du vlan id}} \Longrightarrow (014)_{10} = 20 \Longrightarrow \text{vlan du recepteur pc1}  \]
    \end{itemize}

\end{prettyBox}

\newpage

\item Faite la topologie suivante et les adresses et gateway du nouveau pc et laptop :

\begin{figure}[H] 
  \centering 
  \includegraphics[width=0.6\textwidth]{state3.png} 
\end{figure}


\begin{figure}[H] 
  \centering 
  \includegraphics[width=0.6\textwidth]{pc4.png} 
\end{figure}


\begin{figure}[H] 
  \centering 
  \includegraphics[width=0.6\textwidth]{lap0.png} 
\end{figure}

\newpage

\item C'est quoi le \textbf{VTP} et comment le configurer ?

\begin{prettyBox}{VTP}{myblue}
    \begin{itemize}
        \item \textbf{VTP} c'est un protocole qui permet la propagation des definition des vlan depuis une switch server
            au autre switch .
        \item Il ne propagent pas l'attribution des vlan au ports just la definition des vlan.
        \item On doit difinir une switch server qui va creer les vlan et les propager au autre.
        \item Pour definir le mode vtp d'une switch on utilise : \verb|vtp mode <mode>|
        \item on a 3 modes :
            \begin{itemize}
                \item server : mode par defaut , definit les vlan, le mot de passe , le domain et propage les vlans.
                \item client : copie les vlan du server a la \textbf{RAM} n'a pas le droit de faire un changement.
                \item transparent: ne fait que transmettre les vlan au autre switch sans utilise la configuration du server et n'a pas le droit de faire un changement .
            \end{itemize}
        \item Pour mettre un nom au domain du server on doit executer la commande suivant au switch server: \verb|vtp domain <nom_domain>|
        \item Une switch peut apparetenir a un seul domain a la fois celui de son server.
        \item Pour des raisons de securite on ajoute un mot de passe au domain dans le switch server avec la commande: \verb|vtp password <mot_de_passe>|
        \item Tout les switch du domain a besoin du mot de passe pour pouvoir communiquer.
        \item Pour voir le mot de passe dans la switch server : \verb|show vtp password|
        \item Pour voir l'etat du \textbf{VTP} : \verb|show vtp status|
    \end{itemize}
\end{prettyBox}

\vspace{1cm}

\item Configurer switch 0 comme serveur:

\begin{figure}[H] 
  \centering 
  \includegraphics[width=0.6\textwidth]{s0.png} 
\end{figure}


\begin{figure}[H] 
  \centering 
  \includegraphics[width=0.6\textwidth]{s01.png} 
\end{figure}

\newpage
\item Configurer le \textbf{VTP} sur les autre switchs:

\begin{figure}[H] 
  \centering 
  \includegraphics[width=0.6\textwidth]{s1.png} 
\end{figure}


\begin{figure}[H] 
  \centering 
  \includegraphics[width=0.6\textwidth]{s2.png} 
\end{figure}

\begin{figure}[H] 
  \centering 
  \includegraphics[width=0.6\textwidth]{s3.png} 
\end{figure}


\begin{figure}[H] 
  \centering 
  \includegraphics[width=0.6\textwidth]{s4.png} 
\end{figure}

\begin{figure}[H] 
  \centering 
  \includegraphics[width=0.6\textwidth]{s5.png} 
\end{figure}

\vspace{0.5cm}

\item Verifier la propagation des vlans sur l'un des switch qu'on ajouter comme la switch 3 :

\begin{figure}[H] 
  \centering 
  \includegraphics[width=0.6\textwidth]{verify.png} 
\end{figure}

\newpage

\item Configurer les ports switchs et fait un ping :


\begin{figure}[H] 
  \centering 
  \includegraphics[width=0.6\textwidth]{es1.png} 
\end{figure}


\begin{figure}[H] 
  \centering 
  \includegraphics[width=0.6\textwidth]{es2.png} 
\end{figure}

\begin{figure}[H] 
  \centering 
  \includegraphics[width=0.6\textwidth]{es3.png} 
\end{figure}


\begin{figure}[H] 
  \centering 
  \includegraphics[width=0.6\textwidth]{es4.png} 
\end{figure}

\begin{figure}[H] 
  \centering 
  \includegraphics[width=0.6\textwidth]{es5.png} 
\end{figure}


\begin{figure}[H] 
  \centering 
  \includegraphics[width=0.6\textwidth]{lastPing.png} 
\end{figure}

\end{enumerate}

\end{document}

