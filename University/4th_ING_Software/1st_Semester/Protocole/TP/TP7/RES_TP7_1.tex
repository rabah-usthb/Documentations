\documentclass{article}
\usepackage[a4paper, left=1.5cm, right=1.5cm, top=1cm, bottom=2cm]{geometry}

\newcounter{commentCount}
\newcounter{filePrg}
\newcounter{inputPrg}

\usepackage[dvipsnames]{xcolor}
\usepackage{minted}

\usepackage[many]{tcolorbox}
\tcbuselibrary{listings}
\tcbuselibrary{minted}

\usepackage{ifthen}
\usepackage{fontawesome}

\usepackage{tabularx}
\newcolumntype{\CeX}{>{\centering\let\newline\\\arraybackslash}X}%
\newcommand{\TwoSymbolsAndText}[3]{%
  \begin{tabularx}{\textwidth}{c\CeX c}%
    #1 & #2 & #3
  \end{tabularx}%
}

\newtcblisting[use counter=inputPrg, number format=\arabic]{codeInput}[4]{
  listing engine=minted,
  minted language=#1,
  minted options={autogobble,breaklines,  firstnumber={#4}},
  listing only,
  size=title,
  arc=1.5mm,
  breakable,
  enhanced jigsaw,
  colframe=myblue,
  coltitle=White,
  boxrule=0.5mm,
  colback=white,
  coltext=Black,
  title=\TwoSymbolsAndText{\faCode}{%
  \textbf{Sql Program \thetcbcounter}\ifthenelse{\equal{#2}{}}{}{\textbf{: }#2}%
  }{\faCode},
  label=inputPrg:#3
}


\usepackage{boldline}
\usepackage{tikz,tcolorbox}
\usepackage{amsmath}
\usepackage[table,xcdraw]{xcolor}
\usepackage{listings}
\usepackage{array,multirow} % For customizing tables
\usepackage{booktabs} % For better horizontal lines
\usepackage{makecell}
\setlength{\parindent}{0pt}
\usepackage{siunitx}
\usepackage{tkz-tab}
\usepackage{amssymb}
\usepackage{amsmath}
\usepackage{enumitem}

\usepackage{caption}
\usepackage{float}


\newcommand{\exer}[1]{
  \section*{Exercice #1}
  \vspace{-0.5cm}
  \noindent\rule{\textwidth}{0.5pt}%
}

\newcommand{\tit}[1]{
\begin{center}
    \Large{\textbf{{#1}}}
\end{center}
}

\definecolor{commentgray}{HTML}{676160}
\definecolor{messagegreen}{HTML}{17B867}
\definecolor{myblue}{HTML}{10C2C4}

\tcbuselibrary{skins, breakable, theorems}


\newtcolorbox{prettyBox}[2]{
  enhanced,
  colback=white!90!#2,   % Background color based on the second parameter (color)
  colframe=#2!60!black,  % Frame color based on the second parameter (color)
  coltitle=white,        % Title color (white)
  fonttitle=\bfseries\Large,
  title=#1,              % Title from the first parameter
  boxrule=1mm,
  arc=0.5mm,
  drop shadow=#2!35!gray, % Drop shadow color based on the second parameter (color)
}


\lstdefinestyle{cmd}{
 basicstyle=\ttfamily,
 backgroundcolor=\color{lightgray!20},
 frame=single
}

\usepackage{minted}

\begin{document}


\tit{TP N\(^{\boldsymbol{\circ}}\)\hspace{0.1cm}7 Partie N\(^{\boldsymbol{\circ}}\)\hspace{0.1cm}1 }

\vspace{-0.25cm}
\begin{enumerate}
    \item c'est quoi un chemin redondant ?

        \begin{prettyBox}{Chemin Redondnat}{myblue}
            Un chemin redodant est un chemin en plus qui mene vers la meme destination :
            \begin{figure}[H] 
  \centering 
  \includegraphics[width=0.4\textwidth]{path1.png} 
\end{figure}
\vspace{-0.35cm}
\begin{figure}[H] 
  \centering 
  \includegraphics[width=0.4\textwidth]{path2.png} 
\end{figure}


        \end{prettyBox}

\vspace{0.15cm}

\item Quelle sont les avantages des chemins redondants?

\begin{prettyBox}{Avantages}{myblue}
    \begin{itemize}
        \item \textbf{Fiabilite} : Maintient la connectivité en cas de panne.
        \item \textbf{Rapidite} : plusieurs chemins offerent une faible latence.
        \item \textbf{Équilibrage De Charge} :  Permet de distribuer le trafic sur plusieurs chemins simultanément.
    \end{itemize}
\end{prettyBox}

\vspace{0.15cm}

\item C'est quoi le probleme des chemins redondants ?

\begin{prettyBox}{Probleme}{myblue}
    Le problème est que lorsqu'un switch reçoit un message, il le diffuse en broadcast sur tous les appareils connectés.
    Le switch qui reçoit ce broadcast fait de même, ce qui provoque une \textbf{inondation de trames} ou un 
    \textbf{broadcast storm}.
\end{prettyBox}

\vspace{0.15cm}
\item C'est quoi le protocole \textbf{STP} ?
\begin{prettyBox}{STP}{myblue}
    Protocole \textbf{STP} fix the issue of \textbf{brpadcast storm} by blocking port that leads to \textbf{loops} :
    \begin{itemize}
\itemsep0em 
        \item \textbf{STP} doit definir le switch \textbf{root bridge}.
        \item Les switchs communiquent entre eux et envoient des message \textbf{BPDU} qui contiennet \textbf{Bridge ID}.
        \item \textbf{STP} prend la switch avec la plus petit \textbf{Bridge ID}.
        \item \textbf{Bridge ID} = Priorite + \textbf{Vlan\_ID} et \textbf{MAC}.
        \item Par defaut la priorite = 32768 $\in$  0 to 61440 (multiples de 4096).
        \item Si switches ont la meme priorite + \textbf{Vlan\_ID} alors \textbf{STP} prendra switch aves la plus petit \textbf{MAC}.
        \item Les ports du \textbf{root bridge} sont dit \textbf{designated ports}.
        \item \textbf{Designated Ports} sont des ports qui s'eloigne du \textbf{Bridge Root}.
        \item \textbf{STP} doit definir \textbf{Root Ports} qui sont des ports avec le chemins le plus rapids vers \textbf{Bridge Root}.
        \item Par plus rapide on veut dire avec le cout le plus petit.
        \item \textbf{STP} prend un switch \textbf{non Bridge Root} avec la plus petite \textbf{Bridge ID} et met les port restent commes \textbf{Designated Ports}.
        \item Les ports qui reste sont dit \textbf{Blocked Ports}.
        \item \textbf{STP} se met a jour en cas de panne.
    \end{itemize}
\end{prettyBox}


\begin{figure}[H] 
  \centering 
  \includegraphics[width=0.5\textwidth]{tab.png} 
\end{figure}

\item Refaite la topologie suivante :


\begin{figure}[H] 
  \centering 
  \includegraphics[width=0.6\textwidth]{state1_1.png} 
\end{figure}

\begin{prettyBox}{Remarque}{red}
    \begin{itemize}
        \item On doit choisir un model switch qui support \textbf{STP} comme \verb|2960-24 TT|
        \item Les interfaces oranges sont les interfaces bloques par \textbf{STP}.
    \end{itemize}
\end{prettyBox}

    \newpage
\item Donnez les adresses \textbf{MAC} de tout les switchs :


\begin{figure}[H] 
  \centering 
  \includegraphics[width=0.6\textwidth]{switch0_1.png} 
\end{figure}

\begin{figure}[H] 
  \centering 
  \includegraphics[width=0.55\textwidth]{switch1_1.png} 
\end{figure}


\begin{figure}[H] 
  \centering 
  \includegraphics[width=0.6\textwidth]{switch2_1.png} 
\end{figure}

\begin{prettyBox}{MAC}{myblue}
    \begin{itemize}
        \item \textbf{Switch 0} : \verb |0002.4A45.C0E9|
        \item \textbf{switch 1} : \verb |0001.C926.4D0E|
        \item \textbf{switch 2} : \verb |00E0.F984.8E44|
    \end{itemize}
\end{prettyBox}

\vspace{0.15cm}

\item Quelle est la switch \textbf{Root Bridge} ?

\begin{prettyBox}{Root Bridge}{myblue}
    c'est la switch 1 car tout les switchs on la meme priorite par defaut et 
    switch 1 a la plus petite \textbf{MAC} adresse.
\end{prettyBox}

\vspace{0.15cm}

\item Comment voir l'etat du \textbf{STP} dans un switch ?
\begin{prettyBox}{Etat}{myblue}
    On doit etre au niveau 2 admin er utilise la commande \verb|show spanning-tree|  et si niveau
    3 au plus on utilise \verb|do show spanning-tree|\\cette derniere
    donne l'adresse \textbf{MAC} du \textbf{Root Bridge} et du switch courrant ainsi que le type de port
    de chaque interface \textbf{Blocked , Designated, Root }
\end{prettyBox}

\newpage

\item Afficher l'etat \textbf{STP} de tout les switchs

\begin{figure}[H] 
  \centering 
  \includegraphics[width=0.6\textwidth]{show0_1.png} 
\end{figure}

\begin{figure}[H] 
  \centering 
  \includegraphics[width=0.6\textwidth]{show1_1.png} 
\end{figure}


\begin{figure}[H] 
  \centering 
  \includegraphics[width=0.6\textwidth]{show3_1.png} 
\end{figure}


\item Expliquer le resultat 

\begin{prettyBox}{Expliquation}{myblue}
    \begin{itemize}
        \item \textbf{switch 0} :
            \begin{itemize}
                \item Tout les ports relie au machine du \textbf{LAN} (\verb|f0/1-3|) sont \textbf{Designated}.
                \item \verb|f0/5| est plus rapid le cout est 19 et celui du \verb|f0/4| est $19\times 2$ donc
                    \verb|f0/5| est port \textbf{Root}.
                \item le switch 0 a une plus petite \textbf{Bridge ID} donc le port restant (\verb|f0/4|) est \textbf{Designated}.
            \end{itemize}
        \item \textbf{switch 1} : puisqu'il est le \textbf{Root Bridge} tout ces port sont \textbf{Designated}.
        \item \textbf{switch 2} :
            \begin{itemize}
                \item \verb|f0/2| est plus rapid le cout est 19 et celui du \verb|f0/1| est $19\times 2$ donc
                    \verb|f0/2| est port \textbf{Root}.
                \item Le port restant \verb|f0/1| est \textbf{Blocked}
            \end{itemize}
    \end{itemize}
\end{prettyBox}

\newpage
\item Faire une panne


\begin{figure}[H] 
  \centering 
  \includegraphics[width=0.6\textwidth]{sh_1.png} 
\end{figure}

\begin{figure}[H] 
  \centering 
  \includegraphics[width=0.4\textwidth]{state2_1.png} 
\end{figure}



\item Re-afficher l'etat \textbf{STP} de tout les switchs

\begin{figure}[H] 
  \centering 
  \includegraphics[width=0.6\textwidth]{show4_1.png} 
\end{figure}

\begin{figure}[H] 
  \centering 
  \includegraphics[width=0.6\textwidth]{show5_1.png} 
\end{figure}


\begin{figure}[H] 
  \centering 
  \includegraphics[width=0.6\textwidth]{show6_1.png} 
\end{figure}


\item Expliquer le resultat 
\begin{prettyBox}{Expliquation}{myblue}
    \begin{itemize}
        \item \textbf{switch 0} :
            \begin{itemize}
                \item Tout les ports relie au machine du \textbf{LAN} (\verb|f0/1-3|) sont \textbf{Designated}.
                \item \verb|f0/4| est plus rapid le cout est 19 donc c'est un port \textbf{Root}.
            \end{itemize}
        \item \textbf{switch 1} : puisqu'il est le \textbf{Root Bridge} tout ces port sont \textbf{Designated}.
        \item \textbf{switch 2} :
            \begin{itemize}
                \item \verb|f0/2| est plus rapid le cout est 19 et celui du \verb|f0/1| est $19\times 2$ donc
                    \verb|f0/2| est port \textbf{Root}.
                \item Le port restant \verb|f0/1| est \textbf{Designated}
            \end{itemize}
    \end{itemize}
\end{prettyBox}




\end{enumerate}
\end{document}
