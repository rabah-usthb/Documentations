\documentclass{article}
\usepackage[a4paper, left=1.5cm, right=1.5cm, top=1cm, bottom=2cm]{geometry}

\newcounter{commentCount}
\newcounter{filePrg}
\newcounter{inputPrg}

\usepackage[dvipsnames]{xcolor}
\usepackage{minted}

\usepackage[many]{tcolorbox}
\tcbuselibrary{listings}
\tcbuselibrary{minted}

\usepackage{ifthen}
\usepackage{fontawesome}

\usepackage{tabularx}
\newcolumntype{\CeX}{>{\centering\let\newline\\\arraybackslash}X}%
\newcommand{\TwoSymbolsAndText}[3]{%
  \begin{tabularx}{\textwidth}{c\CeX c}%
    #1 & #2 & #3
  \end{tabularx}%
}

\newtcblisting[use counter=inputPrg, number format=\arabic]{codeInput}[4]{
  listing engine=minted,
  minted language=#1,
  minted options={autogobble,breaklines,  firstnumber={#4}},
  listing only,
  size=title,
  arc=1.5mm,
  breakable,
  enhanced jigsaw,
  colframe=myblue,
  coltitle=White,
  boxrule=0.5mm,
  colback=white,
  coltext=Black,
  title=\TwoSymbolsAndText{\faCode}{%
  \textbf{Sql Program \thetcbcounter}\ifthenelse{\equal{#2}{}}{}{\textbf{: }#2}%
  }{\faCode},
  label=inputPrg:#3
}


\usepackage{boldline}
\usepackage{tikz,tcolorbox}
\usepackage{amsmath}
\usepackage[table,xcdraw]{xcolor}
\usepackage{listings}
\usepackage{array,multirow} % For customizing tables
\usepackage{booktabs} % For better horizontal lines
\usepackage{makecell}
\setlength{\parindent}{0pt}
\usepackage{siunitx}
\usepackage{tkz-tab}
\usepackage{amssymb}
\usepackage{amsmath}
\usepackage{enumitem}

\usepackage{caption}
\usepackage{float}


\newcommand{\exer}[1]{
  \section*{Exercice #1}
  \vspace{-0.5cm}
  \noindent\rule{\textwidth}{0.5pt}%
}

\newcommand{\tit}[1]{
\begin{center}
    \Large{\textbf{{#1}}}
\end{center}
}

\definecolor{commentgray}{HTML}{676160}
\definecolor{messagegreen}{HTML}{17B867}
\definecolor{myblue}{HTML}{10C2C4}

\tcbuselibrary{skins, breakable, theorems}


\newtcolorbox{prettyBox}[2]{
  enhanced,
  colback=white!90!#2,   % Background color based on the second parameter (color)
  colframe=#2!60!black,  % Frame color based on the second parameter (color)
  coltitle=white,        % Title color (white)
  fonttitle=\bfseries\Large,
  title=#1,              % Title from the first parameter
  boxrule=1mm,
  arc=0.5mm,
  drop shadow=#2!35!gray, % Drop shadow color based on the second parameter (color)
}


\lstdefinestyle{cmd}{
 basicstyle=\ttfamily,
 backgroundcolor=\color{lightgray!20},
 frame=single
}

\usepackage{minted}

\begin{document}


\tit{TP N\(^{\boldsymbol{\circ}}\)\hspace{0.1cm}7 Partie N\(^{\boldsymbol{\circ}}\)\hspace{0.1cm}2 }

\vspace{-0.25cm}
\begin{enumerate}

    \item Faite la topologie suivante :
\begin{figure}[H] 
  \centering 
  \includegraphics[width=0.4\textwidth]{state1_2.png} 
\end{figure}

\begin{prettyBox}{Remarque}{red}
    \begin{itemize}
        \item On doit utilise la model \verb|2960-24 TT| qui supporte \textbf{STP}.
        \item On utilise les ports \verb|f0/1-2| et \verb|f0/18| , \verb|f0/20|. (je mentionne ceci just
            pour simplifie le suivi de ce tp vous n'etes pas oblige d'utilise les meme interfaces)
    \end{itemize}
\end{prettyBox}

\item Afficher l'etat \textbf{STP} de tout les switchs

\begin{figure}[H] 
  \centering 
  \includegraphics[width=0.6\textwidth]{show1_2.png} 
\end{figure}

\begin{figure}[H] 
  \centering 
  \includegraphics[width=0.6\textwidth]{show2_2.png} 
\end{figure}


\item Expliquer le resultat 

\begin{prettyBox}{Expliquation}{myblue}
    \begin{itemize}
        \item \textbf{switch 0} : puisqu'il est le \textbf{Root Bridge} (plus petite \textbf{MAC}) tout ces port sont \textbf{Designated}.
        \item \textbf{switch 1} : Puisque tout les ports ont le meme cout 19 \textbf{STP} prendra comme \textbf{Root Port} celui avec l'index interface la plus petite donc  le port \verb|f0/1| est \textbf{Root Port} et les autre deviennent \textbf{Blocked Port}
    \end{itemize}
\end{prettyBox}

\vspace{0.25cm}

\item C'est quoi \textbf{EtherChannel} ?

    \begin{prettyBox}{EtherChannel}{myblue}
    C'est un protocole d'agrégation de ports de switch qui permet de regrouper plusieurs interfaces 
    (même si pas contigües) en un \textbf{canal de ports} qui est une interface virtuelle.
    
    \[\text{bandwidth}_\text{canal} = \sum_{i=1}^{n} \text{bandwidth}_i\]
    
    Un switch \verb|2960-24TT| supporte au max 6 interfaces dans un canal.\\
    Un switch peut avoir plusieurs canaux.
\end{prettyBox}

\newpage

\item Comment configurer \textbf{EtherChannel} ?

\begin{prettyBox}{Config}{myblue}
    On selectionne plusiseur interfaces avec : \verb|int range <intervalle_1> , ... , <intervalle_n>| \\
    Puis on passe au niveau 4 et on utilise : \verb|channel-group <id_group> mode on|\\
    On peut selectionner un cannal avec : \verb|int port-channel <id_group>|
\end{prettyBox}

\begin{figure}[H] 
  \centering 
  \includegraphics[width=0.6\textwidth]{ether0_2.png} 
\end{figure}

\begin{figure}[H] 
  \centering 
  \includegraphics[width=0.6\textwidth]{ether1_2.png} 
\end{figure}

\begin{prettyBox}{Remarque}{red}
    On les a mit en \verb|trunk mode| car les switchs sont relie entre-eux.
\end{prettyBox}

\vspace{0.25cm}

\item Comment voir l'etat du \textbf{EtherChannel} dans une switch :
\begin{prettyBox}{Etat}{myblue}
    \verb|show interfaces port-channel <id_group>| donne des information sur un seul cannal par mis ces information 
    on a la bandwidth du cannal dans notre exampel ca va etre 200Mbps.\\
    \verb|show etherchannel summary| donne un apprecu sur tout les cannale leur id et les interdaces concernen
\end{prettyBox}

\begin{figure}[H] 
  \centering 
  \includegraphics[width=0.6\textwidth]{show3_2.png} 
\end{figure}

\begin{figure}[H] 
  \centering 
  \includegraphics[width=0.6\textwidth]{show4_2.png} 
\end{figure}

\begin{figure}[H] 
  \centering 
  \includegraphics[width=0.6\textwidth]{show5_2.png} 
\end{figure}

\begin{figure}[H] 
  \centering 
  \includegraphics[width=0.6\textwidth]{show6_2.png} 
\end{figure}

\item Re-Afficher l'etat \textbf{STP} de tout les switchs

\begin{figure}[H] 
  \centering 
  \includegraphics[width=0.6\textwidth]{show7_2.png} 
\end{figure}

\begin{figure}[H] 
  \centering 
  \includegraphics[width=0.6\textwidth]{show8_2.png} 
\end{figure}


\item Expliquer le resultat 

\begin{prettyBox}{Expliquation}{myblue}
    \begin{itemize}
        \item \textbf{switch 0} : puisqu'il est le \textbf{Root Bridge} (plus petite \textbf{MAC}) tout ces port sont \textbf{Designated}.
        \item \textbf{switch 1} : puisque le cannale a une bandwidth de 200 Mbps donc il est le plus rapid il est \textbf{Root Port} et les autre sont \textbf{Blocked Port}. 
            
    \end{itemize}
\end{prettyBox}



\end{enumerate}
\end{document}
