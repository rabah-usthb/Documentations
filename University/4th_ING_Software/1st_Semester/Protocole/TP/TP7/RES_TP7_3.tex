\documentclass{article}
\usepackage[a4paper, left=1.5cm, right=1.5cm, top=1cm, bottom=2cm]{geometry}

\newcounter{commentCount}
\newcounter{filePrg}
\newcounter{inputPrg}

\usepackage[dvipsnames]{xcolor}
\usepackage{minted}

\usepackage[many]{tcolorbox}
\tcbuselibrary{listings}
\tcbuselibrary{minted}

\usepackage{ifthen}
\usepackage{fontawesome}

\usepackage{tabularx}
\newcolumntype{\CeX}{>{\centering\let\newline\\\arraybackslash}X}%
\newcommand{\TwoSymbolsAndText}[3]{%
  \begin{tabularx}{\textwidth}{c\CeX c}%
    #1 & #2 & #3
  \end{tabularx}%
}

\newtcblisting[use counter=inputPrg, number format=\arabic]{codeInput}[4]{
  listing engine=minted,
  minted language=#1,
  minted options={autogobble,breaklines,  firstnumber={#4}},
  listing only,
  size=title,
  arc=1.5mm,
  breakable,
  enhanced jigsaw,
  colframe=myblue,
  coltitle=White,
  boxrule=0.5mm,
  colback=white,
  coltext=Black,
  title=\TwoSymbolsAndText{\faCode}{%
  \textbf{Sql Program \thetcbcounter}\ifthenelse{\equal{#2}{}}{}{\textbf{: }#2}%
  }{\faCode},
  label=inputPrg:#3
}


\usepackage{boldline}
\usepackage{tikz,tcolorbox}
\usepackage{amsmath}
\usepackage[table,xcdraw]{xcolor}
\usepackage{listings}
\usepackage{array,multirow} % For customizing tables
\usepackage{booktabs} % For better horizontal lines
\usepackage{makecell}
\setlength{\parindent}{0pt}
\usepackage{siunitx}
\usepackage{tkz-tab}
\usepackage{amssymb}
\usepackage{amsmath}
\usepackage{enumitem}

\usepackage{caption}
\usepackage{float}


\newcommand{\exer}[1]{
  \section*{Exercice #1}
  \vspace{-0.5cm}
  \noindent\rule{\textwidth}{0.5pt}%
}

\newcommand{\tit}[1]{
\begin{center}
    \Large{\textbf{{#1}}}
\end{center}
}

\definecolor{commentgray}{HTML}{676160}
\definecolor{messagegreen}{HTML}{17B867}
\definecolor{myblue}{HTML}{10C2C4}

\tcbuselibrary{skins, breakable, theorems}


\newtcolorbox{prettyBox}[2]{
  enhanced,
  colback=white!90!#2,   % Background color based on the second parameter (color)
  colframe=#2!60!black,  % Frame color based on the second parameter (color)
  coltitle=white,        % Title color (white)
  fonttitle=\bfseries\Large,
  title=#1,              % Title from the first parameter
  boxrule=1mm,
  arc=0.5mm,
  drop shadow=#2!35!gray, % Drop shadow color based on the second parameter (color)
}


\lstdefinestyle{cmd}{
 basicstyle=\ttfamily,
 backgroundcolor=\color{lightgray!20},
 frame=single
}

\usepackage{minted}

\begin{document}


\tit{TP N\(^{\boldsymbol{\circ}}\)\hspace{0.1cm}7 Partie N\(^{\boldsymbol{\circ}}\)\hspace{0.1cm}3 }

\vspace{-0.25cm}
\begin{enumerate}

\item C'est quoi le protocole \textbf{HSRP} ? 

\begin{prettyBox}{HSRP}{myblue}
    C'est un protocole de la famille \textbf{First Hop Redundancy Protocol (FHRP)} 
    conçu pour garantir la tolérance aux pannes au niveau de la première passerelle (default gateway).\\
    Il regroupe plusieurs routeurs sous une seule adresse \textbf{IPv4} virtuelle et leur attribue une adresse 
    \textbf{MAC} virtuelle. 
    Cependant, un seul routeur est actif dans le groupe en cas de panne ou de changement de priorité 
    un autre routeur devient automatiquement actif.
\end{prettyBox}

\vspace{0.15cm}

\item Comment est-ce que le protocole \textbf{HSRP} choisit le routeur actif? 

\begin{prettyBox}{Actif}{myblue}
    \begin{itemize}
        \item Par defaut tout les priorite des routeurs est 100 cette derniere \(\in [0,255]\)
        \item Dans la premiere election du routeur actif si plusieur routeurs on activer \textbf{HSRP} :
            \begin{itemize}
                \item Il prend celui avec la plus grande priorite 
                \item Si ils ont la meme priorite il prendra celle avec la plus grand adresse \textbf{IPv4}
            \end{itemize}
        \item Chaque 3 second le routeur actif envoie un message hello au routeurs veille si
            apres 10 seconds (3$\times$3+1) et il ne repond pas le protocole remplacera le routeur actif.
        \item Le protocole \textbf{HSRP} remplace en cas si on change la priorite des routeurs.
    \end{itemize}
\end{prettyBox}

\vspace{0.35cm}

\item Faite la topologie suivante :

\begin{figure}[H] 
  \centering 
  \includegraphics[width=0.6\textwidth]{state1.png} 
\end{figure}



\item Combien de \textbf{LAN} dans la topologie?

\begin{prettyBox}{LAN}{myblue}
    nombre de reseau \textbf{LAN} est 4 :
    \begin{itemize}
        \item \verb|192.168.1.0/24|
        \item \verb|192.168.13.0/30|
        \item \verb|192.168.23.0/30|
        \item \verb|192.168.2.0/24|
    \end{itemize}
\end{prettyBox}

\newpage

\item Faite l'adressage des machines terminaux et routeurs :

\begin{figure}[H] 
  \centering 
  \includegraphics[width=0.6\textwidth]{pc0Conf.png} 
\end{figure}


\begin{figure}[H] 
  \centering 
  \includegraphics[width=0.6\textwidth]{serverConf.png} 
\end{figure}


\begin{figure}[H] 
  \centering 
  \includegraphics[width=0.6\textwidth]{router0Conf.png} 
\end{figure}


\begin{figure}[H] 
  \centering 
  \includegraphics[width=0.6\textwidth]{router1Conf.png} 
\end{figure}


\begin{figure}[H] 
  \centering 
  \includegraphics[width=0.6\textwidth]{router2Conf.png} 
\end{figure}

\newpage

\item Configurer \textbf{RIP} dans les routeurs :

\begin{figure}[H] 
  \centering 
  \includegraphics[width=0.6\textwidth]{router0RIP.png} 
\end{figure}


\begin{figure}[H] 
  \centering 
  \includegraphics[width=0.6\textwidth]{router1RIP.png} 
\end{figure}


\begin{figure}[H] 
  \centering 
  \includegraphics[width=0.6\textwidth]{router2RIP.png} 
\end{figure}

\vspace{0.15cm}

\item Comment configurer \textbf{HSRP} ?

\begin{prettyBox}{Config}{myblue}
    On doit etre au niveau 4 configuration d'interface et utliser les commandes suivantes :
    \begin{itemize}
        \item Activer version 2 du protocle \textbf{HSRP} : \verb|standby version 2|
        \item Creer un group avec un ip virtuelle : \verb|standby <id_group> ip <adresse_virtuelle>|
    \end{itemize}
\end{prettyBox}


\vspace{0.15cm}

\item Configurer \textbf{HRSP} dans les routeurs 0 et 1 :

\begin{figure}[H] 
  \centering 
  \includegraphics[width=0.6\textwidth]{router0HRSP.png} 
\end{figure}


\begin{figure}[H] 
  \centering 
  \includegraphics[width=0.6\textwidth]{router1HSRP.png} 
\end{figure}

\begin{figure}[H] 
  \centering 
  \includegraphics[width=0.6\textwidth]{state2.png} 
\end{figure}


\vspace{0.15cm}


\item Configurer la gateway du pc0 avec l'adresse virtuelle \textbf{HSRP} :

\begin{figure}[H] 
  \centering 
  \includegraphics[width=0.6\textwidth]{gatewayPC0.png} 
\end{figure}

\newpage

\item Afficher la table de routage des routeurs :

\vspace{0.15cm}

\begin{figure}[H] 
  \centering 
  \includegraphics[width=0.6\textwidth]{router0RIPTable.png} 
\end{figure}


\begin{figure}[H] 
  \centering 
  \includegraphics[width=0.6\textwidth]{router1RIPTable.png} 
\end{figure}


\begin{figure}[H] 
  \centering 
  \includegraphics[width=0.6\textwidth]{router2RIPTable.png} 
\end{figure}

\item Tester la connectivite avec des pings :

\begin{figure}[H] 
  \centering 
  \includegraphics[width=0.6\textwidth]{ping.png} 
\end{figure}

\vspace{0.15cm}

\item c'est quoi la commande \verb|tracert| ?

\begin{prettyBox}{Tracert}{myblue}
    \verb|tracert| est une commande similaire à \verb|ping|, mais en plus de tester la connectivité,
    elle affiche la route par laquelle la machine emprunte pour atteindre la destination en donnant les adresses \verb|IPv4| 
    de toutes les machines intermédiaires.
\end{prettyBox}

\newpage

\item Faite un \verb|tracert| entre pc0 et le serveur :


\begin{figure}[H] 
  \centering 
  \includegraphics[width=0.6\textwidth]{tracePC0_1.png} 
\end{figure}

\begin{prettyBox}{Remarque}{red}
    On remarque que le ping passe par le routeur 0 \verb|192.168.1.1| donc
    c'est lui le routeur actif du groupe \verb|1|.
\end{prettyBox}

\vspace{0.15cm}

\item Pourquoi est le routeur 0 actif ? malgre qu'ils sont la meme priorite (100)
et son adresse est plus petite que celle du routeur 1 ?


\begin{prettyBox}{Raison}{myblue}
    Routeur 0 est actif parceque c'est le premier qu'on activer puis routeur 1 si on avait
    activer les routeurs 0 et 1 en meme temps routeur 1 aurait ete actif.
\end{prettyBox}

\vspace{0.15cm}

\item Simuler une panne dans le routeur 0 en desactivant l'interface \verb|g0/1| : 

\begin{figure}[H] 
  \centering 
  \includegraphics[width=0.6\textwidth]{router0FAIL.png} 
\end{figure}


\begin{figure}[H] 
  \centering 
  \includegraphics[width=0.6\textwidth]{sh.png} 
\end{figure}

\begin{figure}[H] 
  \centering 
  \includegraphics[width=0.6\textwidth]{router1ActiveFAIL.png} 
\end{figure}

\newpage


\item Refaite un \verb|tracert| entre pc0 et le serveur :


\begin{figure}[H] 
  \centering 
  \includegraphics[width=0.6\textwidth]{tracePC0_2.png} 
\end{figure}

\begin{prettyBox}{Remarque}{red}
    On remarque que le ping passe par le routeur 1 \verb|192.168.1.2| donc
    c'est lui le routeur actif du groupe \verb|1|, car le routeur 0 est en panne
    et le routeur 1 ne recevait plus de message hello.
\end{prettyBox}

\vspace{0.25cm}

\item comment verifie l'etat du \textbf{HSRP} dans un routeur? 

\begin{prettyBox}{Etat}{myblue}

    \begin{itemize}
        \item niveau 2 admin : \verb|show standby brief|
        \item niveau 3 et 4 : \verb|do show standby brief|
    \end{itemize}
    La commande nous donne l'interface , id du group , priority , preempt ,l'etat du routeur, le routeur actif, les routeurs
    veille (standby) et l'adresse \textbf{IPv4} virtuelle.
\end{prettyBox}


\begin{figure}[H] 
  \centering 
  \includegraphics[width=0.6\textwidth]{brief1.png} 
\end{figure}


\begin{figure}[H] 
  \centering 
  \includegraphics[width=0.6\textwidth]{brief2.png} 
\end{figure}

\vspace{0.15cm}

\item Reactiver l'interface \verb|g0/1| du routeur 0 et affiche l'etat \textbf{HSRP} des routeur 0 et 1 :


\begin{figure}[H] 
  \centering 
  \includegraphics[width=0.6\textwidth]{brief3.png} 
\end{figure}


\begin{figure}[H] 
  \centering 
  \includegraphics[width=0.6\textwidth]{brief4.png} 
\end{figure}


\begin{prettyBox}{Remarque}{red}
    On remarque que lorsque l'interface du routeur 0 était en panne, le routeur 0 ne connaissait pas le routeur actif
    et le routeur 1 ne connaissait pas le routeur en veille (\verb|unknown|).
    Après avoir réactivé l'interface, les routeurs 0 et 1 ont de nouveau pu communiquer et se reconnaître entre eux.
\end{prettyBox}

\newpage

\item Comment modifier la priorite d'un routeur ?

\begin{prettyBox}{Priorite}{myblue}
    on doit etre au niveau 4 configuration d'interface et utilise la commande suivante :\\
    \verb|standby <id_groupe> priority <priority>| priority $\in [0,255]$ par defaut = 10.
\end{prettyBox}

\vspace{0.15cm}

\item Changer la priorite du routeur 0 a 120 et afficher l'etat du \textbf{HSRP}
\begin{figure}[H] 
  \centering 
  \includegraphics[width=0.6\textwidth]{priority1.png} 
\end{figure}

\vspace{0.15cm}

\item Pourquoi routeur 0 est toujours en veille malgre qu'il a une priorite plus grande? 

\begin{prettyBox}{Preempt}{myblue}
    Parceque \verb|preempt| est desactiver par defaut, quand on l'actif dans un routeur si il est en veille
    et sa priorite est plus grande que le routeur actif \textbf{HSRP} le remplacera.\\[0.15cm]
    Pour l'activer on doit etre au niveau 4 configuration d'interface et utilise la commande suivante :
    \verb|standby <id_groupe> preempt|
\end{prettyBox}

\vspace{0.15cm}

\item Activer la preemption dans le routeur 0 et afficher l'etat \textbf{HSRP} des routeurs  :

\begin{figure}[H] 
  \centering 
  \includegraphics[width=0.6\textwidth]{brief5.png} 
\end{figure}


\begin{figure}[H] 
  \centering 
  \includegraphics[width=0.6\textwidth]{brief6.png} 
\end{figure}

\begin{prettyBox}{Remarque}{red}
    On remarque apres avoir activer la preemeption le routeur 0 est devenu actif et le routeur 1 est devenu
    en veille
\end{prettyBox}

\newpage
\item Faite un ping en simulation du PC0 au serveur et avancer le message jusqu'au switch 0 et afficher la tramme
    outbound de la switch :

\begin{figure}[H] 
  \centering 
  \includegraphics[width=0.6\textwidth]{switch.png} 
\end{figure}

\item Consulter la table \textbf{MAC} du switch 0 :
\begin{figure}[H] 
  \centering 
  \includegraphics[width=0.6\textwidth]{mac.png} 
\end{figure}

\vspace{0.15cm}

\item Que represent l'adresse \textbf{MAC} \verb|0000.0C9F.F001| ?

\begin{prettyBox}{MAC Virtuelle}{myblue}
    Elle represente l'adresse \textbf{MAC} virtuelle genere par le protocole \textbf{HSRP}
\end{prettyBox}


\vspace{0.15cm}
\item Verifie l'adresse \textbf{MAC} du \textbf{HSRP} :

    \begin{prettyBox}{Verification}{myblue}
        On afficher en detaille l'etat du \textbf{HSRP} avec la commande : 
        \verb|show standby|
\end{prettyBox}

\begin{figure}[H] 
  \centering 
  \includegraphics[width=0.6\textwidth]{show.png} 
\end{figure}



\end{enumerate}

\end{document}

