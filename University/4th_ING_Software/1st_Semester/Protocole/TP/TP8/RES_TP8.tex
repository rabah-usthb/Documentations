\documentclass{article}
\usepackage[a4paper, left=1.5cm, right=1.5cm, top=1cm, bottom=2cm]{geometry}

\newcounter{commentCount}
\newcounter{filePrg}
\newcounter{inputPrg}

\usepackage[dvipsnames]{xcolor}
\usepackage{minted}

\usepackage[many]{tcolorbox}
\tcbuselibrary{listings}
\tcbuselibrary{minted}

\usepackage{ifthen}
\usepackage{fontawesome}

\usepackage{tabularx}
\newcolumntype{\CeX}{>{\centering\let\newline\\\arraybackslash}X}%
\newcommand{\TwoSymbolsAndText}[3]{%
  \begin{tabularx}{\textwidth}{c\CeX c}%
    #1 & #2 & #3
  \end{tabularx}%
}

\newtcblisting[use counter=inputPrg, number format=\arabic]{codeInput}[4]{
  listing engine=minted,
  minted language=#1,
  minted options={autogobble,breaklines,  firstnumber={#4}},
  listing only,
  size=title,
  arc=1.5mm,
  breakable,
  enhanced jigsaw,
  colframe=myblue,
  coltitle=White,
  boxrule=0.5mm,
  colback=white,
  coltext=Black,
  title=\TwoSymbolsAndText{\faCode}{%
  \textbf{Sql Program \thetcbcounter}\ifthenelse{\equal{#2}{}}{}{\textbf{: }#2}%
  }{\faCode},
  label=inputPrg:#3
}


\usepackage{boldline}
\usepackage{tikz,tcolorbox}
\usepackage{amsmath}
\usepackage[table,xcdraw]{xcolor}
\usepackage{listings}
\usepackage{array,multirow} % For customizing tables
\usepackage{booktabs} % For better horizontal lines
\usepackage{makecell}
\setlength{\parindent}{0pt}
\usepackage{siunitx}
\usepackage{tkz-tab}
\usepackage{amssymb}
\usepackage{amsmath}
\usepackage{enumitem}

\usepackage{caption}
\usepackage{float}


\newcommand{\exer}[1]{
  \section*{Exercice #1}
  \vspace{-0.5cm}
  \noindent\rule{\textwidth}{0.5pt}%
}

\newcommand{\tit}[1]{
\begin{center}
    \Large{\textbf{{#1}}}
\end{center}
}

\definecolor{commentgray}{HTML}{676160}
\definecolor{messagegreen}{HTML}{17B867}
\definecolor{myblue}{HTML}{10C2C4}

\tcbuselibrary{skins, breakable, theorems}


\newtcolorbox{prettyBox}[2]{
  enhanced,
  colback=white!90!#2,   % Background color based on the second parameter (color)
  colframe=#2!60!black,  % Frame color based on the second parameter (color)
  coltitle=white,        % Title color (white)
  fonttitle=\bfseries\Large,
  title=#1,              % Title from the first parameter
  boxrule=1mm,
  arc=0.5mm,
  drop shadow=#2!35!gray, % Drop shadow color based on the second parameter (color)
}


\lstdefinestyle{cmd}{
 basicstyle=\ttfamily,
 backgroundcolor=\color{lightgray!20},
 frame=single
}

\usepackage{minted}

\begin{document}


\tit{TP N\(^{\boldsymbol{\circ}}\)\hspace{0.1cm}8}

\vspace{-0.25cm}
\begin{enumerate}

\item C'est quoi un service ?
\begin{prettyBox}{Service}{myblue}
    un service est une \textbf{fonctionnalité fournie} par un appareil ou un système en réseau, tandis qu'un\\ protocole est \textbf{un ensemble de règles qui
    définissent comment la communication} entre \\les appareils ou les systèmes doit s'effectuer. Un protocole peut être utilisé par plusieurs
    services pour permettre la communication entre eux.\\[0.15cm]
    Par example un serveur web utilise le protocole \textbf{HTTP}
\end{prettyBox}

\vspace{0.25cm}

\item C'est quoi le service \textbf{DHCP} ?
\begin{prettyBox}{DHCP}{myblue}
    C'est un service qui utilise le protocole \textbf{UDP} pour attribution automatique des adresses , gateway et serveur DNS des appareille clients via
    un serveur \textbf{DHCP} (qui peut etre un routeur , switch , serveur , pc)
\begin{figure}[H] 
  \centering 
  \includegraphics[width=0.6\textwidth]{dhcp.png} 
\end{figure}


\end{prettyBox}

\vspace{0.25cm}

\item Faite la topologie suivante :

\begin{figure}[H] 
  \centering 
  \includegraphics[width=0.6\textwidth]{state1.png} 
\end{figure}

\vspace{0.25cm}
\item Configurer l'adresse au routeur \verb|g0/0| :
\begin{figure}[H] 
  \centering 
  \includegraphics[width=0.6\textwidth]{router0ADD.png} 
\end{figure}

\begin{figure}[H] 
  \centering 
  \includegraphics[width=0.6\textwidth]{state2.png} 
\end{figure}


\item Configurer un serveur \textbf{DHCP} dans le routeur :

\begin{prettyBox}{Config DHCP}{myblue}
    On peut configurer l'attribution d'adresse dynamique de plusieur reseaux dans le meme serveur
    chaque reseau est appele \textbf{pool}. pour configure on doit :
    \begin{itemize}
        \item Etre au niveau 3 et creer une nouvelle \textbf{pool} avec la commande : \verb|ip dhcp pool <nom_service>|
        \item On passe au niveau 4 configuration de dhcp
        \item On definit le reseau avec la commande : \verb|network <adresse_reseau>|
        \item On definit la gateway avec : \verb|default-router <adresse_gateway>|
        \item On definit le serveur DNS : \verb|dns-server <adresse_dns>|
        \item On peut exclure des adresses mais on doit etre au niveau 3 et utilise la commande : \verb|ip dhcp excluded-address <adresse_1> <adresse_2>| pour exclure une page
            ou on peut aussi exclure une seul adresse \verb|ip dhcp excluded-adress <adresse>|
    \end{itemize}
\end{prettyBox}

\vspace{0.15cm}

\begin{prettyBox}{Exclusion}{red}
    On exclue des adresses pour les appareilles aves des adresses static comme les serveur web, \\imprimente...etc et pour adresse de gateway.
\end{prettyBox}

\begin{figure}[H] 
  \centering 
  \includegraphics[width=0.6\textwidth]{router0DIH.png} 
\end{figure}

\vspace{0.25cm}
\item Activer \textbf{DHCP} dans les PCs :
\begin{prettyBox}{Activer DHCP}{myblue}
    On va dans \textbf{Configuration IP} et selectionner \textbf{DHCP} au lieu de \textbf{static}.
\end{prettyBox}

\begin{figure}[H] 
  \centering 
  \includegraphics[width=0.6\textwidth]{pc0.png} 
\end{figure}

\begin{figure}[H] 
  \centering 
  \includegraphics[width=0.6\textwidth]{pc1.png} 
\end{figure}
\item Faite la topologie suivante :

\begin{figure}[H] 
  \centering 
  \includegraphics[width=0.6\textwidth]{state3.png} 
\end{figure}


\item Configurer l'adresse au routeur \verb|g0/1| :
\begin{figure}[H] 
  \centering 
  \includegraphics[width=0.6\textwidth]{router0ADD2.png} 
\end{figure}

\begin{figure}[H] 
  \centering 
  \includegraphics[width=0.6\textwidth]{state4.png} 
\end{figure}

\newpage
\item Avant de configurer \textbf{DHCP} verifie l'adresse des PCs du reseau \verb|192.168.2.0| :
\begin{figure}[H] 
  \centering 
  \includegraphics[width=0.6\textwidth]{apipa.png} 
\end{figure}

\begin{figure}[H] 
  \centering 
  \includegraphics[width=0.6\textwidth]{apipa2.png} 
\end{figure}

\begin{prettyBox}{Remarque}{red}
    On remarque que le \textbf{DHCP} a echoue car on a pas encore configurer la pool pour le reseau \verb|192.168.2.0| et qu'un autre service
    \textbf{APIPA} fournit par le system d'exploitation windows a donne une adresse. 
\end{prettyBox}

\item Configurer un autre pool dans le meme routeur :
\begin{figure}[H] 
  \centering 
  \includegraphics[width=0.6\textwidth]{router0DIH2.png} 
\end{figure}



\item Re-activer \textbf{DHCP} dans les PCs :
\begin{figure}[H] 
  \centering 
  \includegraphics[width=0.6\textwidth]{pc3.png} 
\end{figure}

\begin{figure}[H] 
  \centering 
  \includegraphics[width=0.6\textwidth]{pc4.png} 
\end{figure}


\end{enumerate}
\end{document}
