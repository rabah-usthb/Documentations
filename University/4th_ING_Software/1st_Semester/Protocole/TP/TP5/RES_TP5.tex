\documentclass{article}
\usepackage[a4paper, left=1.5cm, right=1.5cm, top=1cm, bottom=2cm]{geometry}

\newcounter{commentCount}
\newcounter{filePrg}
\newcounter{inputPrg}

\usepackage[dvipsnames]{xcolor}
\usepackage{minted}

\usepackage[many]{tcolorbox}
\tcbuselibrary{listings}
\tcbuselibrary{minted}

\usepackage{ifthen}
\usepackage{fontawesome}

\usepackage{tabularx}
\newcolumntype{\CeX}{>{\centering\let\newline\\\arraybackslash}X}%
\newcommand{\TwoSymbolsAndText}[3]{%
  \begin{tabularx}{\textwidth}{c\CeX c}%
    #1 & #2 & #3
  \end{tabularx}%
}

\newtcblisting[use counter=inputPrg, number format=\arabic]{codeInput}[4]{
  listing engine=minted,
  minted language=#1,
  minted options={autogobble,breaklines,  firstnumber={#4}},
  listing only,
  size=title,
  arc=1.5mm,
  breakable,
  enhanced jigsaw,
  colframe=myblue,
  coltitle=White,
  boxrule=0.5mm,
  colback=white,
  coltext=Black,
  title=\TwoSymbolsAndText{\faCode}{%
  \textbf{Sql Program \thetcbcounter}\ifthenelse{\equal{#2}{}}{}{\textbf{: }#2}%
  }{\faCode},
  label=inputPrg:#3
}


\usepackage{boldline}
\usepackage{tikz,tcolorbox}
\usepackage{amsmath}
\usepackage[table,xcdraw]{xcolor}
\usepackage{listings}
\usepackage{array,multirow} % For customizing tables
\usepackage{booktabs} % For better horizontal lines
\usepackage{makecell}
\setlength{\parindent}{0pt}
\usepackage{siunitx}
\usepackage{tkz-tab}
\usepackage{amssymb}
\usepackage{amsmath}
\usepackage{enumitem}

\usepackage{caption}
\usepackage{float}


\newcommand{\exer}[1]{
  \section*{Exercice #1}
  \vspace{-0.5cm}
  \noindent\rule{\textwidth}{0.5pt}%
}

\newcommand{\tit}[1]{
\begin{center}
    \Large{\textbf{{#1}}}
\end{center}
}

\definecolor{commentgray}{HTML}{676160}
\definecolor{messagegreen}{HTML}{17B867}
\definecolor{myblue}{HTML}{10C2C4}

\tcbuselibrary{skins, breakable, theorems}


\newtcolorbox{prettyBox}[2]{
  enhanced,
  colback=white!90!#2,   % Background color based on the second parameter (color)
  colframe=#2!60!black,  % Frame color based on the second parameter (color)
  coltitle=white,        % Title color (white)
  fonttitle=\bfseries\Large,
  title=#1,              % Title from the first parameter
  boxrule=1mm,
  arc=0.5mm,
  drop shadow=#2!35!gray, % Drop shadow color based on the second parameter (color)
}


\lstdefinestyle{cmd}{
 basicstyle=\ttfamily,
 backgroundcolor=\color{lightgray!20},
 frame=single
}

\usepackage{minted}

\begin{document}

\tit{TP N\(^{\boldsymbol{\circ}}\)\hspace{0.1cm}5}

\begin{enumerate}

\item C'est Quoi un cable \textbf{Serie} et un cable \textbf{Fibre}:

\begin{prettyBox}{Cable}{myblue}
    \begin{itemize}
        \item \textbf{Serie} : un cable qui simule les connections \textbf{WAN} entre routeurs.    
        \item \textbf{Fibre} : un cable qui utilise des fils de verre pour transmettre des données sous forme de lumiere.
    \end{itemize}
\end{prettyBox}

\begin{figure}[H] 
  \centering 
  \includegraphics[width=0.5\textwidth]{serial.png} 
\end{figure}

\begin{figure}[H] 
  \centering 
  \includegraphics[width=0.5\textwidth]{fibre.png} 
\end{figure}




\vspace{0.25cm}


\item Comment ajouter les ports Serie et fibre dans un routeur?:

\begin{prettyBox}{Ajouter Ports}{myblue}
    \begin{itemize}
        \item Click sur le routeur
        \item Eteint le routeur 
        \item Ajouter le module \verb|HWIC-2T| pour ajouter le port \textbf{Serie}
        \item Ajouter le module \verb|HWIC-1G-SFP|
        \item Ajouter le module \verb|GLC-LH-SMD| sur \verb|HWIC-1G-SFP| pour ajouter le port \textbf{Fibre}
        \item Allumer le routeur.
    \end{itemize}
\end{prettyBox}


\begin{figure}[H] 
  \centering 
  \includegraphics[width=0.5\textwidth]{router.png} 
\end{figure}

\begin{figure}[H] 
  \centering 
  \includegraphics[width=0.5\textwidth]{turnOff.png} 
\end{figure}

\begin{figure}[H] 
  \centering 
  \includegraphics[width=0.5\textwidth]{addSerial.png} 
\end{figure}

\begin{figure}[H] 
  \centering 
  \includegraphics[width=0.5\textwidth]{fibre1.png} 
\end{figure}

\begin{figure}[H] 
  \centering 
  \includegraphics[width=0.5\textwidth]{fibre2.png} 
\end{figure}

\begin{figure}[H] 
  \centering 
  \includegraphics[width=0.5\textwidth]{turnOn.png} 
\end{figure}

\begin{figure}[H] 
  \centering 
  \includegraphics[width=0.5\textwidth]{PortRouter.png} 
\end{figure}

\vspace{0.25cm}

\item Comment ajouter Des ellipse et ajuster sa taille ?

\begin{figure}[H] 
  \centering 
  \includegraphics[width=0.5\textwidth]{draw.png} 
\end{figure}

\begin{figure}[H] 
  \centering 
  \includegraphics[width=0.5\textwidth]{resize.png} 
\end{figure}


\vspace{0.45cm}

\item Faite la topologie suivants:

\begin{figure}[H] 
  \centering 
  \includegraphics[width=0.8\textwidth]{state.png} 
\end{figure}

\newpage

\item Faite l'adressage:

\begin{figure}[H] 
  \centering 
  \includegraphics[width=0.8\textwidth]{addRouter0.png} 
\end{figure}


\begin{figure}[H] 
  \centering 
  \includegraphics[width=0.8\textwidth]{addRouter1.png} 
\end{figure}


\begin{figure}[H] 
  \centering 
  \includegraphics[width=0.8\textwidth]{addRouter2.png} 
\end{figure}


\begin{figure}[H] 
  \centering 
  \includegraphics[width=0.8\textwidth]{addRouter3.png} 
\end{figure}


\begin{figure}[H] 
  \centering 
  \includegraphics[width=0.8\textwidth]{addRouter4.png} 
\end{figure}


\begin{figure}[H] 
  \centering 
  \includegraphics[width=0.8\textwidth]{addRouter5.png} 
\end{figure}


\begin{figure}[H] 
  \centering 
  \includegraphics[width=0.8\textwidth]{addInternet.png} 
\end{figure}

\begin{figure}[H] 
  \centering 
  \includegraphics[width=0.8\textwidth]{addPC0.png} 
\end{figure}


\begin{figure}[H] 
  \centering 
  \includegraphics[width=0.8\textwidth]{addLap0.png} 
\end{figure}

\begin{figure}[H] 
  \centering 
  \includegraphics[width=0.8\textwidth]{addServ0.png} 
\end{figure}


\begin{figure}[H] 
  \centering 
  \includegraphics[width=0.8\textwidth]{addServ1.png} 
\end{figure}

\vspace{0.45cm}

\item Configurez l'\textbf{OSPF} (en mode interface) au niveau des sites A et B:

\begin{figure}[H] 
  \centering 
  \includegraphics[width=0.8\textwidth]{router1OSPF.png} 
\end{figure}


\begin{figure}[H] 
  \centering 
  \includegraphics[width=0.8\textwidth]{router2OSPF.png} 
\end{figure}


\begin{figure}[H] 
  \centering 
  \includegraphics[width=0.8\textwidth]{router3OSPF.png} 
\end{figure}


\begin{figure}[H] 
  \centering 
  \includegraphics[width=0.8\textwidth]{router4OSPF.png} 
\end{figure}


\begin{figure}[H] 
  \centering 
  \includegraphics[width=0.8\textwidth]{router5OSPF.png} 
\end{figure}

\newpage

\item Faire un ping entre machines du meme Site (Entre Machine Site A puis Entre Machine Site B) :

\begin{figure}[H] 
  \centering 
  \includegraphics[width=0.8\textwidth]{pingSiteA.png} 
\end{figure}

\begin{figure}[H] 
  \centering 
  \includegraphics[width=0.8\textwidth]{pingSiteB.png} 
\end{figure}

\vspace{0.45cm}

\item Configurez la route par défaut sur le routeur Router0 vers internet:


\begin{figure}[H] 
  \centering 
  \includegraphics[width=0.8\textwidth]{defaultRouter0.png} 
\end{figure}

\newpage


\item C'est Quoi un systeme autonome?:

\begin{prettyBox}{Système Autonome}{myblue}
    Un système autonome est un système réseau qui dispose de \textbf{edge routers} qui configurent \textbf{BGP} pour permettre
    la communication avec des systèmes autonomes externes.
\end{prettyBox}

\vspace{0.35cm}

\item C'est Quoi le protocole \textbf{BGP}?
\begin{prettyBox}{BGP}{myblue}
    \text{BGP} est un protocole de routage dynamique qui permet la communication entre les systemes autonomes , ce dernier est configurer sur
    les edges routers qui vont partager leur reseaux  \textbf{LAN} \\a ces voisins (edges routers d'autre systemes autonomes).
\end{prettyBox}

\item Comment Configurer \textbf{BGP}?

\begin{prettyBox}{Configurer BGP}{myblue}
    \begin{itemize}
        \item On doit etre au niveau 3 pour activer le \textbf{BGP} avec l'ID du system autonome du routeur avec la commande : \verb|router bgp <Current_SA_ID>|
        \item Apres la commande precedente on passe au niveau 4 (\verb|config-router|)
        \item Pour creer une session avec un voisin on utilise la commande suivant :\\\verb|neighbor <@Neighbor> remote as <Neighbor_SA_ID>|
        \item Pour propager les reseaux \textbf{LAN} au voisin on utilise la commande suivant : \\\verb|network <@LAN> mask <mask> |
    \end{itemize}
\end{prettyBox}

\item Configurez le \textbf{BGP} sur les edge routers (router0 , router2, router4)

\begin{figure}[H] 
  \centering 
  \includegraphics[width=0.8\textwidth]{BGProuter0.png} 
\end{figure}


\begin{figure}[H] 
  \centering 
  \includegraphics[width=0.8\textwidth]{BGProuter2.png} 
\end{figure}


\begin{figure}[H] 
  \centering 
  \includegraphics[width=0.8\textwidth]{BGProuter4.png} 
\end{figure}


\item Configurez la route par défaut sur le routeur Router2, Router4 et la propagation \textbf{OSPF}:


\begin{figure}[H] 
  \centering 
  \includegraphics[width=0.8\textwidth]{defaultRouter2.png} 
\end{figure}

\begin{figure}[H] 
  \centering 
  \includegraphics[width=0.8\textwidth]{defaultRouter4.png} 
\end{figure}

\item Affichez les tables de routage des différents routeurs :


\begin{figure}[H] 
  \centering 
  \includegraphics[width=0.8\textwidth]{tabRouter0.png} 
\end{figure}


\begin{figure}[H] 
  \centering 
  \includegraphics[width=0.8\textwidth]{tabRouter1.png} 
\end{figure}


\begin{figure}[H] 
  \centering 
  \includegraphics[width=0.8\textwidth]{tabRouter2.png} 
\end{figure}


\begin{figure}[H] 
  \centering 
  \includegraphics[width=0.8\textwidth]{tabRouter3.png} 
\end{figure}



\begin{figure}[H] 
  \centering 
  \includegraphics[width=0.8\textwidth]{tabRouter4.png} 
\end{figure}


\begin{figure}[H] 
  \centering 
  \includegraphics[width=0.8\textwidth]{tabRouter5.png} 
\end{figure}


\begin{prettyBox}{Remarque}{red}
    On remarque que dans la table de routage des edges routeurs ont des routes \verb|B|(\textbf{BGP}) pour acceder au \textbf{LAN} des autre systems autonomes.
\end{prettyBox}

\vspace{0.5cm}

\item Vérifiez la communication entre les différents équipements :


\begin{figure}[H] 
  \centering 
  \includegraphics[width=0.8\textwidth]{goodPing.png} 
\end{figure}



\end{enumerate}


\end{document}
