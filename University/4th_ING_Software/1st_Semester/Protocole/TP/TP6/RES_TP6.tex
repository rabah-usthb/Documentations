\documentclass{article}
\usepackage[a4paper, left=1.5cm, right=1.5cm, top=1cm, bottom=2cm]{geometry}

\newcounter{commentCount}
\newcounter{filePrg}
\newcounter{inputPrg}

\usepackage[dvipsnames]{xcolor}
\usepackage{minted}

\usepackage[many]{tcolorbox}
\tcbuselibrary{listings}
\tcbuselibrary{minted}

\usepackage{ifthen}
\usepackage{fontawesome}

\usepackage{tabularx}
\newcolumntype{\CeX}{>{\centering\let\newline\\\arraybackslash}X}%
\newcommand{\TwoSymbolsAndText}[3]{%
  \begin{tabularx}{\textwidth}{c\CeX c}%
    #1 & #2 & #3
  \end{tabularx}%
}

\newtcblisting[use counter=inputPrg, number format=\arabic]{codeInput}[4]{
  listing engine=minted,
  minted language=#1,
  minted options={autogobble,breaklines,  firstnumber={#4}},
  listing only,
  size=title,
  arc=1.5mm,
  breakable,
  enhanced jigsaw,
  colframe=myblue,
  coltitle=White,
  boxrule=0.5mm,
  colback=white,
  coltext=Black,
  title=\TwoSymbolsAndText{\faCode}{%
  \textbf{Sql Program \thetcbcounter}\ifthenelse{\equal{#2}{}}{}{\textbf{: }#2}%
  }{\faCode},
  label=inputPrg:#3
}


\usepackage{boldline}
\usepackage{tikz,tcolorbox}
\usepackage{amsmath}
\usepackage[table,xcdraw]{xcolor}
\usepackage{listings}
\usepackage{array,multirow} % For customizing tables
\usepackage{booktabs} % For better horizontal lines
\usepackage{makecell}
\setlength{\parindent}{0pt}
\usepackage{siunitx}
\usepackage{tkz-tab}
\usepackage{amssymb}
\usepackage{amsmath}
\usepackage{enumitem}

\usepackage{caption}
\usepackage{float}


\newcommand{\exer}[1]{
  \section*{Exercice #1}
  \vspace{-0.5cm}
  \noindent\rule{\textwidth}{0.5pt}%
}

\newcommand{\tit}[1]{
\begin{center}
    \Large{\textbf{{#1}}}
\end{center}
}

\definecolor{commentgray}{HTML}{676160}
\definecolor{messagegreen}{HTML}{17B867}
\definecolor{myblue}{HTML}{10C2C4}

\tcbuselibrary{skins, breakable, theorems}


\newtcolorbox{prettyBox}[2]{
  enhanced,
  colback=white!90!#2,   % Background color based on the second parameter (color)
  colframe=#2!60!black,  % Frame color based on the second parameter (color)
  coltitle=white,        % Title color (white)
  fonttitle=\bfseries\Large,
  title=#1,              % Title from the first parameter
  boxrule=1mm,
  arc=0.5mm,
  drop shadow=#2!35!gray, % Drop shadow color based on the second parameter (color)
}


\lstdefinestyle{cmd}{
 basicstyle=\ttfamily,
 backgroundcolor=\color{lightgray!20},
 frame=single
}

\usepackage{minted}

\begin{document}


\tit{TP N\(^{\boldsymbol{\circ}}\)\hspace{0.1cm}6}

\vspace{-0.25cm}
\begin{enumerate}

\item C'est quoi le protocole \textbf{IPV6} ? 


\begin{prettyBox}{IPv6}{myblue}
    \textbf{IPv6} est un protocole conçu pour remplacer \textbf{IPv4}, avec une taille d'adresse 4 fois plus grande.
    
    \textbf{IPv6} utilise des adresses de 128 bits, tandis que \textbf{IPv4} utilise des adresses de 32 bits.
    
    Les 64 premiers bits sont appelés le \textbf{préfixe} (réseau),
    et les 64 autres identifient l'\textbf{interface} (machine).\\
    \[\underbrace{2001:0\text{db}8:\text{cafe}:0001}_{\text{\textbf{Partie Prefixe}}}:\underbrace{0000:0000:0000:0001}_{\large{\text{\textbf{Partie Interface}}}} \underbrace{/64}_{\text{\textbf{Taille Prefixe}}}\]
\end{prettyBox}


\item Comment simplifie une adresse \textbf{IPV6} ? 


\begin{prettyBox}{Simplification IPv6}{myblue}
    On a deux simplifications :
    \begin{itemize}
\setlength{\itemsep}{0pt}
        \item \texttt{000x} devient \texttt{x}, \texttt{00xx} devient \texttt{xx}, \texttt{0xxx} devient \texttt{xxx}
        \item Une séquence de segments \texttt{0000} devient \texttt{::} (on ne peut le faire qu'une seule fois, sinon l'adresse devient ambiguë)
    \end{itemize}
    Exemple : \texttt{000ff:0000:0000:0000:0001:0000:0000:0001}
    \begin{itemize}
\setlength{\itemsep}{0pt}
        \item \texttt{ff:0:0:0:1:0:0:1} est correct
        \item \texttt{ff::1:0:0:1} est correct
        \item \texttt{ff::1::1} est faux
    \end{itemize}
\end{prettyBox}


\item Enummerez et expliquez les type d'adresses \textbf{IPV6}

\begin{prettyBox}{Type IPV6}{myblue}
    \begin{itemize}
        \item \textbf{L}ink \textbf{L}ocal \textbf{A}dresse (\textbf{LLA}) :  L'adresse de lien local est une adresse attribuée dès que l'interface réseau
            de la machine est activée. Elle commence toujours par le préfixe \verb|fe80::/10|. Elle peut être générée automatiquement à partir de l'adresse \textbf{MAC} 
            ou de manière aléatoire ; elle peut également être configurée manuellement. Ces adresses doivent être uniques au sein
            d'un même domaine de diffusion (même switch). Elles servent à identifier la machine et ses voisins directs, mais elles ne sont pas 
            routables : elles ne peuvent pas franchir un routeur, que ce soit vers un autre réseau local ou vers Internet.
        \item \textbf{U}nique \textbf{L}ocal \textbf{A}dresse (\textbf{ULA}) :  L'adresse locale unique est l'équivalent de l'adresse \textbf{IPv4} privée.\\ 
            Elle n'est pas visible de l'extérieur et est indépendante du fournisseur d'accès Internet. Elle commence par le préfixe \verb|fc00::/7| jusqu'a \verb|fdff::/7|.
            Ces adresses sont routables uniquement au sein d'un même systeme autonome, mais elles ne sont pas routables sur l'Internet public.
        \item \textbf{G}lobal \textbf{U}nicast \textbf{A}dresse (\textbf{GUA}) : L'adresse globale monodiffusion est une adresse unique à l'échelle
        mondiale, équivalente à l'adresse \textbf{IPv4} publique. Elle est routable aussi bien au sein d'un système autonome que sur l'Internet public.
        Elle commence par les bits de poids fort \verb|001|, ce qui signifie que le premier chiffre hexadécimal est soit un 2 (\verb|0010|) ou un 3 (\verb|0011|).
        \\donc le prefix est \verb|2xxx::/3| ou \verb|3xxx::/3|.
        \item \textbf{Multicast Adresse} :  L'adresse de multidiffusion (multicast) désigne un groupe d'interfaces réseau appartenant généralement à
            des machines différents. Elle est utilisée lorsqu'un expéditeur souhaite envoyer un message à un groupe spécifique uniquement. En \textbf{IPv6},
            le broadcast est supprimé et remplacé par le multicast pour plus d'efficacité. Elles commencent par le préfixe \verb|ffxx::/16| les deux premier 
            ff sont fixe les deux dernier chiffre hexa represent le porte :
            \begin{itemize}
\setlength{\itemsep}{0pt}
                \item \verb|ff01| : local host.
                \item \verb|ff02| : lien local (meme switch).
                \item \verb|ff05| : systeme autonome.
                \item \verb|ff0e| : global internet.
            \end{itemize}
\end{itemize}
\end{prettyBox}

\newpage

\item Donnez le pourcentage d'adresse \textbf{GUA} 

\begin{prettyBox}{GUA \%}{myblue}
    On sait que n'importe quelle adresse \textbf{IPV6} a 128 bits , et que les adresses \textbf{GUA} commence par 001 bit dans il ya
    125 bits variable (128 bits total - 3 bits fix de \textbf{GUA})

    \begin{align*}
        100&\% \xrightarrow{\hspace{3cm}} 2^{128} \text{bits} \\
        X&\% \xrightarrow{\hspace{3cm}} 2^{125} \text{bits}
    \end{align*}

    \[
        X\% = \dfrac{2^{125} \text{bits} \times 100\%}{2^{128} \text{bits}} = \boxed{12.5\%}
           \]
\end{prettyBox}

\vspace{0.15cm}

\item Faite la topologie reseau suivante : 

\begin{figure}[H] 
  \centering 
  \includegraphics[width=0.6\textwidth]{state1.png} 
\end{figure}

\vspace{0.15cm}

\item Ajoutter les adresses \textbf{IPV6} au routeur :

\begin{prettyBox}{Adresse Routeur IPV6}{myblue}
    Pour ajoutter une adresse \textbf{IPV6} a un routeur on doit
    etre au niveau 4 configuration d'interface et utilise la commande est la suivante :
    \verb|ipv6 add <adresse>|
\end{prettyBox}


\begin{figure}[H] 
  \centering 
  \includegraphics[width=0.6\textwidth]{routeipv6.png} 
\end{figure}



\vspace{0.15cm}

\item Afficher la table de routage \textbf{IPV6}

    \begin{prettyBox}{Table De Routage}{myblue}
        si on est au niveau 2 admin : \verb|show ipv6 route|\\
        si niveau 3 ou 4 : \verb|do show ipv6 route|
    \end{prettyBox}

\begin{figure}[H] 
  \centering 
  \includegraphics[width=0.6\textwidth]{showroute.png} 
\end{figure}

\newpage

\item Configuration des adresses et gateway \textbf{IPV6} des pc:  

\begin{figure}[H] 
  \centering 
  \includegraphics[width=0.6\textwidth]{pc0.png} 
\end{figure}

\begin{figure}[H] 
  \centering 
  \includegraphics[width=0.6\textwidth]{pc1.png} 
\end{figure}

\begin{figure}[H] 
  \centering 
  \includegraphics[width=0.6\textwidth]{pc2.png} 
\end{figure}

\begin{figure}[H] 
  \centering 
  \includegraphics[width=0.6\textwidth]{pc3.png} 
\end{figure}

\newpage

\item Faite un ping entre les pc du meme \textbf{LAN} puis entre different \textbf{LAN} :

\begin{figure}[H] 
  \centering 
  \includegraphics[width=0.6\textwidth]{badPing.png} 
\end{figure}

\vspace{0.15cm}

\item Que remarquez-vous ?

\begin{prettyBox}{Remarque}{red}
    On remarque que le ping entre pc du meme \textbf{LAN} fonctionne mais un ping
    entre different \textbf{LAN} echoue le packet n'a pas pu etre route. 
\end{prettyBox}

\vspace{0.15cm}

\item Pourquoi le ping entre pc de different \textbf{LAN} a echoue ?

\begin{prettyBox}{Raison}{myblue}
    Parceque on a pas activer le routage mono-diffusion des packets \textbf{IPV6} dans
    le router : on doit etre un niveau 3 et utilise la commande suivant \verb|ipv6 unicast-routing|
\end{prettyBox}

\begin{figure}[H] 
  \centering 
  \includegraphics[width=0.6\textwidth]{unicast.png} 
\end{figure}


\vspace{0.15cm}

\item Refaire un ping entre pc de different \textbf{LAN} :

\begin{figure}[H] 
  \centering 
  \includegraphics[width=0.6\textwidth]{goodPing.png} 
\end{figure}

\begin{prettyBox}{Remarque}{red}
    Apres avoir activer le routage mono-diffusion le ping entre pc de different \textbf{LAN} fonctionne.
\end{prettyBox}

\newpage

\item Faite le topologie suivante :


\begin{figure}[H] 
  \centering 
  \includegraphics[width=0.6\textwidth]{state2.png} 
\end{figure}

\vspace{0.15cm}

\item Faite l'adressage \textbf{IPV6} des routeurs :


\begin{figure}[H] 
  \centering 
  \includegraphics[width=0.6\textwidth]{router0Add.png} 
\end{figure}


\begin{figure}[H] 
  \centering 
  \includegraphics[width=0.6\textwidth]{router1Add.png} 
\end{figure}

\vspace{0.15cm}

\item Ajouter les routes static \textbf{IPV6} :

\begin{prettyBox}{Static IPV6}{myblue}
    on doit etre au niveau 3 et utiliser la commande suivante :\\ \verb|ipv6 route <adresse_reseau\taille prefixe> <adresse_routeur_voisin> |
\end{prettyBox}


\begin{figure}[H] 
  \centering 
  \includegraphics[width=0.6\textwidth]{router0S0.png} 
\end{figure}


\begin{figure}[H] 
  \centering 
  \includegraphics[width=0.6\textwidth]{router1S.png} 
\end{figure}


\newpage
\item Pourquoi ya pas de route static dans le routeur 0?

\begin{prettyBox}{Raison}{myblue}
    Parceque le routeur 0 a toujours l'adresse \verb|2001:B::1/64| dans l'interface \verb|g 0/1| car
    on \textbf{IPV6} une interface peut avoir plusieur adresses donc on doit la supprimer avec la commande :
    \verb|no ipv6 add <adresse>|
\end{prettyBox}


\begin{figure}[H] 
  \centering 
  \includegraphics[width=0.6\textwidth]{router0twoadd.png} 
\end{figure}

\vspace{0.15cm}

\item Enlever l'adresse \verb|2001:B::/64| du routeur 0 et reafficher la table de routage :

\begin{figure}[H] 
\centering 
\includegraphics[width=0.6\textwidth]{router0S1.png} 
\end{figure}

\vspace{0.15cm}

\item Activer le routage mono-diffusion dans le routeur 1 :

\begin{figure}[H] 
  \centering 
  \includegraphics[width=0.6\textwidth]{unicast2.png} 
\end{figure}

\vspace{0.15cm}

\item Faite un ping entre pc de different \textbf{LAN} :

\begin{figure}[H] 
  \centering 
  \includegraphics[width=0.6\textwidth]{goodPing2.png} 
\end{figure}

\newpage

\item Enlevez les routes static et reafficher la table de routage:

\begin{prettyBox}{No Static IPV6}{myblue}
    on doit etre au niveau 3 et utiliser la commande suivante :\\ \verb|no ipv6 route <adresse_reseau\taille prefixe> <adresse_routeur_voisin> |
\end{prettyBox}

\begin{figure}[H] 
  \centering 
  \includegraphics[width=0.6\textwidth]{noRoute0S.png} 
\end{figure}


\begin{figure}[H] 
  \centering 
  \includegraphics[width=0.6\textwidth]{noRoute1S.png} 
\end{figure}

\vspace{0.15cm}

\item Activer \textbf{RIPIng} dans les routeurs :


\begin{prettyBox}{RIPng}{myblue}
    \textbf{RIPng} est un protocole de routage dynamique pour \textbf{IPv6}, similaire à \textbf{RIPv2} de \textbf{IPV4}. 
    Dans cette version, il n'est plus nécessaire d'utiliser une commande globale \texttt{network} pour indiquer quels réseaux partager. 

    En activant le protocole directement sur l'interface, le routeur identifie automatiquement le préfixe à annoncer et utilise les adresses \textbf{LLA} (Link-Local Address) de ses voisins pour l'échange des routes.\\[0.1cm] 
    Pour l'activer, on doit passer en mode configuration d'interface pour chaque interface concernée : 
    \verb|ipv6 rip <nom_processus> enable|
\end{prettyBox}

\begin{figure}[H] 
  \centering 
  \includegraphics[width=0.6\textwidth]{router0R.png} 
\end{figure}


\begin{figure}[H] 
  \centering 
  \includegraphics[width=0.6\textwidth]{router1R.png} 
\end{figure}

\newpage
\item Faite un ping entre pc de different \textbf{LAN} :

\begin{figure}[H] 
  \centering 
  \includegraphics[width=0.6\textwidth]{goodPing3.png} 
\end{figure}

\vspace{0.15cm}

\item Activer l'\textbf{OSPFv3} dans les routeur



\begin{prettyBox}{OSPFv3}{myblue}
    \textbf{OSPFv3} est un protocole de routage dynamique pour \textbf{IPv6}, similaire à \textbf{OSPFv2} pour \textbf{IPv4}. 
    Dans cette version, il n'est plus nécessaire d'utiliser une commande globale \texttt{network} pour indiquer quels réseaux partager. 

    En activant le protocole directement sur l'interface, le routeur identifie automatiquement le préfixe à annoncer et utilise les adresses \textbf{LLA} (Link-Local Address) de ses voisins pour l'échange des routes.\\[0.15cm]

    \textbf{Étape 1 : Configuration du processus (Mode Global)} \\
    On active le processus pour définir l'identifiant 32 bits obligatoire : \\
    \verb|ipv6 router ospf <id_processus>| \\
    \verb|router-id x.x.x.x| où x $\in [0,255]$ (sauf \texttt{0.0.0.0}).\\[0.1cm]

    \textbf{Étape 2 : Configuration des interfaces (Mode Interface)} \\
    On entre dans chaque interface pour l'activer : \\
    \verb|ipv6 ospf <id_processus> area <id_zone>|
\end{prettyBox}

\begin{figure}[H] 
  \centering 
  \includegraphics[width=0.6\textwidth]{router0O.png} 
\end{figure}


\begin{figure}[H] 
  \centering 
  \includegraphics[width=0.6\textwidth]{router1O.png} 
\end{figure}

\newpage

\item Faite un ping entre pc de different \textbf{LAN} :

\begin{figure}[H] 
  \centering 
  \includegraphics[width=0.6\textwidth]{goodPing4.png} 
\end{figure}

\end{enumerate}

\end{document}
