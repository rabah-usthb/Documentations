\documentclass{article}
\usepackage[a4paper, left=1.5cm, right=1.5cm, top=1cm, bottom=2cm]{geometry}


\newcounter{commentCount}
\newcounter{filePrg}
\newcounter{inputPrg}

\usepackage[dvipsnames]{xcolor}
\usepackage{minted}

\usepackage[many]{tcolorbox}
\tcbuselibrary{listings}
\tcbuselibrary{minted}

\usepackage{ifthen}
\usepackage{fontawesome}

\usepackage{tabularx}
\newcolumntype{\CeX}{>{\centering\let\newline\\\arraybackslash}X}%
\newcommand{\TwoSymbolsAndText}[3]{%
  \begin{tabularx}{\textwidth}{c\CeX c}%
    #1 & #2 & #3
  \end{tabularx}%
}

\newtcblisting[use counter=inputPrg, number format=\arabic]{codeInput}[4]{
  listing engine=minted,
  minted language=#1,
  minted options={autogobble,breaklines,  firstnumber={#4}},
  listing only,
  size=title,
  arc=1.5mm,
  breakable,
  enhanced jigsaw,
  colframe=myblue,
  coltitle=White,
  boxrule=0.5mm,
  colback=white,
  coltext=Black,
  title=\TwoSymbolsAndText{\faCode}{%
  \textbf{Sql Program \thetcbcounter}\ifthenelse{\equal{#2}{}}{}{\textbf{: }#2}%
  }{\faCode},
  label=inputPrg:#3
}


\usepackage{boldline}
\usepackage{tikz,tcolorbox}
\usepackage{amsmath}
\usepackage[table,xcdraw]{xcolor}
\usepackage{listings}
\usepackage{array,multirow} % For customizing tables
\usepackage{booktabs} % For better horizontal lines
\usepackage{makecell}
\setlength{\parindent}{0pt}
\usepackage{siunitx}
\usepackage{tkz-tab}
\usepackage{amssymb}
\usepackage{amsmath}
\usepackage{enumitem}

\usepackage{caption}
\usepackage{float}


\newcommand{\exer}[1]{
  \section*{Exercice #1}
  \vspace{-0.5cm}
  \noindent\rule{\textwidth}{0.5pt}%
}

\newcommand{\tit}[1]{
\begin{center}
    \Large{\textbf{{#1}}}
\end{center}
}

\definecolor{commentgray}{HTML}{676160}
\definecolor{messagegreen}{HTML}{17B867}
\definecolor{myblue}{HTML}{10C2C4}

\tcbuselibrary{skins, breakable, theorems}


\newtcolorbox{prettyBox}[2]{
  enhanced,
  colback=white!90!#2,   % Background color based on the second parameter (color)
  colframe=#2!60!black,  % Frame color based on the second parameter (color)
  coltitle=white,        % Title color (white)
  fonttitle=\bfseries\Large,
  title=#1,              % Title from the first parameter
  boxrule=1mm,
  arc=0.5mm,
  drop shadow=#2!35!gray, % Drop shadow color based on the second parameter (color)
}



\lstdefinestyle{cmd}{
 basicstyle=\ttfamily,
 backgroundcolor=\color{lightgray!20},
 frame=single
}

\usepackage{minted}

\begin{document}


\tit{TP N\(^{\boldsymbol{\circ}}\)\hspace{0.1cm}1}


\exer{1}

\vspace{0.5cm}

\begin{enumerate}
    \item Création de tous les types nécessaires pour la création du type TPersonne : 


\begin{codeInput}{sql}{Type Nom}{code01}{41}
-- creation du type tnom
create or replace type tnom_tp1 as object (
    nomDeFamille varchar2(50),
    prenom varchar2(50)
);
/
\end{codeInput}

\vspace{0.1cm}

\begin{codeInput}{sql}{Type Adresse}{code01}{1}
-- creation du type tadresse
create or replace type tadresse_tp1 as object (
    rue varchar2(50),
    numero integer,
    ville varchar2(50),
    pays varchar2(50),
    codePostal integer
);
/
\end{codeInput}

\vspace{0.1cm}

\begin{codeInput}{sql}{Type Telephone}{code01}{1}
-- creation du type ttelephone on a utilise varray puisqu'on connait le max = 6
create or replace type ttelephone_tp1 as varray(6) of integer;
/
\end{codeInput}

\vspace{0.1cm}

\begin{codeInput}{sql}{Type Personne}{code01}{1}
-- creation du type tpersonne en utilisant les type compose qu'on a definie
create or replace type tpersonne_tp1 as object (
    NSS integer,
    nom tnom,
    date_naiss date, 
    adresse tadresse_tp1,
    telephone ttelephone_tp1
)not final; -- on mit not final pour permettre a d'autre objets d'herite de tpersonne
/
\end{codeInput}

\begin{figure}[H] 
  \centering 
  \includegraphics[width=0.5\textwidth]{personneEX1.png} 
\end{figure}

\vspace{0.15cm}

\item Définition la méthode permettant de connaitre l’âge d’une personne:

\begin{codeInput}{sql}{Method Age}{code01}{1}
-- ajoutter la signature de la methode a tpersonne
alter type tpersonne_tp1 add member function age_tp1 return number;

-- implementation des methodes
create or replace type body tpersonne_tp1
as

-- implementaton de la methode age avec pl/sql
member function age_tp1 return number is
age_val number;

begin
select round((sysdate-self.date_naiss)/365) into age_val from dual;
return age_val;
end ;

end;
/
\end{codeInput}


\vspace{0.15cm}

\begin{figure}[H] 
  \centering 
  \includegraphics[width=0.65\textwidth]{methodAge.png} 
\end{figure}

\newpage
\item Création la table Personne: 

\begin{codeInput}{sql}{Creation De La Table Personne}{code01}{1}
-- creation de la table personne de type tpersonne
create table personne_tp1 of tpersonne_tp1;
\end{codeInput}

\vspace{0.15cm}

\item Insertion sur la table personne:

\vspace{0.15cm}

\begin{codeInput}{sql}{Insertion Sur La Table Personne}{code01}{1}
-- faut utilise les constructeurs pour des attributs composes 
insert into personne_tp1 values ( 
    123456789,
    tnom_tp1('ADIMI','Mohamed'),
    TO_DATE('01/01/1980', 'DD/MM/YYYY'),
    tadresse_tp1('la Gare',22,'Alger','Algérie',16000),
    ttelephone_tp1(023112345,0554321921)
 );
\end{codeInput}


\begin{figure}[H] 
  \centering 
  \includegraphics[width=0.65\textwidth]{insert1EX1.png} 
\end{figure}

\vspace{0.5cm}

\item Définir les deux types TEtudiant et TEnseignant :


\begin{codeInput}{sql}{Creation Du Type Enfant TEtudiant}{code01}{1}
-- Creation Du Type Tdiplome
create or replace type tdiplome_tp1 as object (
    nom varchar2(50),
    annee integer
);
/

-- Creation Du Type T_Set_diplome table de type TDiplome
-- On a choisit une nested table parcequ'on a pas de limite ,max = n
create or replace type t_set_diplome_tp1 as table of tdiplome_tp1;
/

-- Creation Du Type TEtudiant Qui Herite Du Type Tperosnne
create or replace type tetudiant_tp1 under tpersonne_tp1(
    numEtudiant integer,
    Departement varchar2(50),
    diplome t_set_diplome_tp1
);
/
\end{codeInput}



\begin{figure}[H] 
  \centering 
  \includegraphics[width=0.65\textwidth]{tetudiant.png} 
\end{figure}


\begin{codeInput}{sql}{Creation Du Type Enfant TEnseignant}{code01}{1}
-- Creation Du Type TCompte 
create or replace type tcompte_tp1 as object (
  numero integer,
  banque varchar2(50)
);
/

-- Creation Du Type TEnseignant
create or replace type tenseignant_tp1 under tpersonne_tp1(
 numEnseignant integer,
 compte tcompte_tp1
);
/
\end{codeInput}

\begin{figure}[H] 
  \centering 
  \includegraphics[width=0.65\textwidth]{tteacher.png} 
\end{figure}


\newpage

\item Insertions :
\begin{codeInput}{sql}{Insertion D'un TEtudiant}{code01}{1}
-- Insertion D'un Etudiant    
insert into personne_tp1 values ( 
    tetudiant_tp1( -- Utilization Du Constructeur TEtudiant    
    123123123,
    tnom_tp1('MERABETI','Adam'),
    TO_DATE('01/05/1985', 'DD/MM/YYYY'),
    tadresse_tp1('Didouche Mourad',null,'Alger','Algérie',null),
    ttelephone_tp1(),
    999,
    'informatique',
    t_set_diplome_tp1()
	)
 );
\end{codeInput}

\begin{figure}[H] 
  \centering 
  \includegraphics[width=0.65\textwidth]{insertEtudiantEX1.png} 
\end{figure}

\begin{codeInput}{sql}{Insertion D'un TEnseignant}{code01}{1}
 -- Insertion D'un Enseignant    
insert into personne_tp1 values ( 
    tenseignant_tp1( -- Utilization Du Constructeur TEnseignant    
    666999666,
    tnom_tp1('LAMARI','Meriem'),
    TO_DATE('04/06/1975', 'DD/MM/YYYY'),
    tadresse_tp1('Boulevard Colonel Amirouche',99,'Alger','Algérie',null),
    ttelephone_tp1(),
    777,
    tcompte_tp1(310123456789,'BDL')
	)
 );

insert into personne_tp1 values ( 
    tenseignant_tp1(
    556978566,
    tnom_tp1('SALEMI','Ahmed'),
    TO_DATE('04/06/1965', 'DD/MM/YYYY'),
    tadresse_tp1('Ismail Yafsah',99,'Alger','Algérie',null),
    ttelephone_tp1(),
    787,
    tcompte_tp1(330123489756,'BEA')
	)
 );
\end{codeInput}


\begin{figure}[H] 
  \centering 
  \includegraphics[width=0.65\textwidth]{insertTeacherEX1.png} 
\end{figure}

\item Ajouter les numéros de téléphone 022342222 et 066543333 à l'étudiant Adam MERABETI :


\begin{codeInput}{sql}{Mise-A-Jour Etudiant}{code01}{1}

-- afficher les numeros comme object
select telephone from personne_tp1 where nom = tnom_tp1('MERABETI','Adam');


update personne_tp1 set telephone = ttelephone_tp1(022342222,066543333) -- affecter les numeros
where nom = tnom_tp1('MERABETI','Adam'); -- pour cibler adam MERABETI

-- afficher les numero comme table avec la function TABLE et column_value pour acceder au valeur
select column_value tel from personne_tp1,table(telephone) where nom = tnom_tp1('MERABETI','Adam');
\end{codeInput}



\begin{figure}[H] 
  \centering 
  \includegraphics[width=0.65\textwidth]{update1EX1.png} 
\end{figure}


\begin{figure}[H] 
  \centering 
  \includegraphics[width=0.65\textwidth]{tabletel.png} 
\end{figure}

\newpage

\item Ajouter les mêmes numéros de téléphone de l'étudiant Adam MERABETI à Meriem LAMARI:


\begin{codeInput}{sql}{Mise-A-Jour Enseignant}{code01}{1}
-- afficher les numeros comme object
select telephone from personne_tp1 where nom = tnom_tp1('LAMARI','Meriem');

-- On imbrique un select pour trouver l'objet ttelephone de adam et la mettre dans telphone de mariem
update personne_tp1 set telephone = (select telephone from personne_tp1 where nom = tnom_tp1('MERABETI','Adam') )
where nom = tnom_tp1('LAMARI','Meriem'); -- pour cibler Meriem LAMARI 
\end{codeInput}


\begin{figure}[H] 
  \centering 
  \includegraphics[width=0.65\textwidth]{update2EX1.png} 
\end{figure}


\item Lister les étudiants (nom et numéro): 

\begin{codeInput}{sql}{Mise-A-Jour Enseignant}{code01}{1}
select p.nom.nomdefamille, treat(value(p) as tetudiant_tp1).numetudiant num 
-- down cast (du general a la class special) pour acceder 
-- des attributs d'objet specifique si ligne pas une instance la cellule rest vide
from personne_tp1 p 
where value(p) is of (tetudiant_tp1); -- cibler que les instances de tetudiant;
\end{codeInput}


\begin{figure}[H] 
  \centering 
  \includegraphics[width=0.65\textwidth]{printEX1.png} 
\end{figure}

\newpage
\item Trouver les couples de noms de personnes partageant au moins un numéro de téléphone:


\begin{codeInput}{sql}{Mise-A-Jour Enseignant}{code01}{1}
select distinct  -- distinct pour eviter les doublons 
p1.nom.nomDeFamille,p1.nom.prenom , -- nom , prenom de p1
p2.nom.nomDeFamille ,p2.nom.prenom  -- nom , prenom de p2
from personne_tp1 p1,personne_tp1 p2, -- table p1 , p2
table(p2.telephone) tel2, table(p1.telephone) tel1 -- table telephone de tel1 , tel2 
where p1.nss !=p2.nss  -- eviter la meme instance
and tel1.column_value = tel2.column_value; -- verifier que tel1 = tel2
\end{codeInput}


\begin{figure}[H] 
  \centering 
  \includegraphics[width=0.65\textwidth]{same.png} 
\end{figure}

\item Creation des types tcours et tevaluation :


\begin{codeInput}{sql}{Type Tcours \& TEvaluation}{code01}{1}
-- creation du type tcours
create or replace type tcours_tp1 as object ( 
    NumCours integer,
    Libelle varchar2(50),
    Credit integer
 );
/

-- creation du type tevaluation
create or replace type tevaluation_tp1 as object (
    dateEval date,
    note number
);
/
\end{codeInput}


\begin{figure}[H] 
  \centering 
  \includegraphics[width=0.6\textwidth]{type2EX1.png} 
\end{figure}

\item Relation entre cours et enseignants :
\begin{codeInput}{sql}{Relation Entre Cours Avec Enseignants}{code01}{1}
-- un cour a plusieurs ensignants donc on met une table ref d'enseignants 
-- un enseignant est responsbale d'un seul cour donc on met une ref de cour

-- creation du type tableau de reference(pointeur) de tenseignant
create or replace type t_set_ref_enseignant_tp1 as table of ref tenseignant_tp1; 
/

-- ajouter tableau ref tenseignant a tcours
alter type tcours_tp1 add attribute assure_par t_set_ref_enseignant_tp1 cascade;

-- ajouter une ref tcours a tenseignant 
alter type tenseignant_tp1 add attribute assure ref tcours_tp1 cascade ;
\end{codeInput}

\begin{figure}[H] 
  \centering 
  \includegraphics[width=0.6\textwidth]{cour-teach .png} 
\end{figure}




\item Relation entre cours et etudiant :
\begin{codeInput}{sql}{Relation Entre Cours Avec Etudiants}{code01}{1}
-- un cour a plusieurs etudiants donc on met une table ref d'etudiants 
-- un etudiant est inscrit a plusieur cour on met une table ref de cour

-- creation du type tableau de reference(pointeur) de tetudiant
create or replace type t_set_ref_etudiant_tp1 as table of ref tetudiant_tp1; 
/

-- creation du type tableau de reference(pointeur) de tcour
create or replace type t_set_ref_cours_tp1 as table of ref tcours_tp1; 
/

-- ajouter tableau ref tetudiant a tcours
alter type tcours_tp1 add attribute est_inscrit t_set_ref_etudiant_tp1 cascade;

-- ajouter tableau ref tcours a tetudiant 
alter type tetudiant_tp1 add attribute inscrit_a t_set_ref_cours_tp1 cascade ;
\end{codeInput}

\begin{figure}[H] 
  \centering 
  \includegraphics[width=0.6\textwidth]{cour-student.png} 
\end{figure}



\begin{codeInput}{sql}{Relation Entre Cours,Etudiant Avec Evaluation}{code01}{1}
-- une evaluation represente une note d'un cour d'un etudiant donc on met ref
-- d'un cour et d'un etudiant
-- un cour a plusieurs evaluation d'etudiants donc on met une table ref d'evaluation
-- un etudiant a plusieurs evaluation de different cours donc on met
-- une table ref d'evaluation

-- creation du type tableau de reference(pointeur) de tcour
create or replace type t_set_ref_evaluation_tp1 as table ref of evaluation_tp1;
/ 

-- ajouter ref tcour a tevaluation
alter type tevaluation_tp1 add attribute cour ref tcours_tp1 cascade;

-- ajouter ref tetudiant a tevaluation
alter type tevaluation_tp1 add attribute etudiant ref tetudiant_tp1 cascade;

-- ajouter tableau ref tevaluation a tetudiant 
alter type tetudiant_tp1 add attribute evaluations t_set_ref_evaluation_tp1 cascade;

-- ajouter tableau ref tevaluation a tcours 
alter type tcours_tp1 add attribute evaluations t_set_ref_evaluation_tp1 cascade;
\end{codeInput}

\begin{figure}[H] 
  \centering 
  \includegraphics[width=0.6\textwidth]{evaluation.png} 
\end{figure}




\item Relation reflexif de cour :

\begin{codeInput}{sql}{Relation Reflexif De Cours}{code01}{1}
-- chaque cour a des pre_requis et des acquis les deux representent des
-- tableau de cour

-- le type t_set_ref_cours_tp1 a ete cree precedemment

-- ajouter tableau ref tcours a tcours (pre_requis)
alter type tcours_tp1 add attribute pre_requis t_set_ref_cours_tp1;

-- ajouter tableau ref tcours a tcours (acquis)
alter type tcours_tp1 add attribute a_pre_requis t_set_ref_cours_tp1;
\end{codeInput}

\begin{figure}[H] 
  \centering 
  \includegraphics[width=0.6\textwidth]{cour-cour.png} 
\end{figure}

\newpage 
\begin{prettyBox}{Remaque}{red}
    On peut creer un type de table de reference mais pas un type qui fait reference a un autre example :
\begin{figure}[H] 
  \centering 
  \includegraphics[width=0.6\textwidth]{note.png} 
\end{figure}
\end{prettyBox}


\item Creation De La Table cours : 

\begin{codeInput}{sql}{Creation De La Table Cours}{code01}{1}
-- error on ne peut pas creer la table directmenent car le type contient
-- des table imprique on a besoin pour chaque table imbrique de creer la table
-- physiquement et pointer vers elle
create table cours_tp1 of tcours_tp1;
/

-- error on peut pas utiliser desc si le type a une depandence circulaire
desc tcours_tp1;

-- on utilise alors le dictionnaire 
select attr_name, attr_type_name from user_type_attrs 
where type_name = 'TCOURS_TP1' -- cibler le type
order by attr_no; -- ordonner par numero de creation de l'attribut

create table cours_tp1 of tcours_tp1 (primary key (numcours))
nested table assure_par store as cours_assure_par, 
-- store as creer la table cours_assure_par et assure_par va pointer vers cette table
nested table est_inscrit store as cours_est_inscrit,
nested table evaluations store as cours_evaluations,
nested table pre_requis store as cours_pre_requis,
nested table a_pre_requis store as cours_a_pre_requis;
\end{codeInput}


\begin{figure}[H] 
  \centering 
  \includegraphics[width=0.57\textwidth]{specifyNested.png} 
\end{figure}

\vspace{-0.7cm}


\begin{figure}[H] 
  \centering 
  \includegraphics[width=0.55\textwidth]{descCant.png} 
\end{figure}

\vspace{-0.5cm}
\begin{figure}[H] 
  \centering 
  \includegraphics[width=0.56\textwidth]{cours.png} 
\end{figure}


\newpage


\end{enumerate}

\begin{codeInput}{sql}{Creation De La Table Evaluation}{code01}{1}
-- error le type a une depandence circulaire
desc tevaluation_tp1

-- dictionnaire pour voir les nom des attributs
select attr_name, attr_type_name from user_type_attrs
where type_name = 'TEVALUATION_TP1'
order by attr_no;

create table evaluation_tp1 of tevaluation_tp1 (
foreign key (cour) references cours_tp1, -- cour va reference a une ligne de la table cours
foreign key (etudiant) references personne_tp1
); 
\end{codeInput}


\begin{figure}[H] 
  \centering 
  \includegraphics[width=0.58\textwidth]{desCant2.png} 
\end{figure}

\begin{figure}[H] 
  \centering 
  \includegraphics[width=0.6\textwidth]{evaluationt.png} 
\end{figure}



\end{document}
