\documentclass{article}
\usepackage[a4paper, left=1.5cm, right=1.5cm, top=1cm, bottom=2cm]{geometry}


\newcounter{commentCount}
\newcounter{filePrg}
\newcounter{inputPrg}

\usepackage[dvipsnames]{xcolor}
\usepackage{minted}

\usepackage[many]{tcolorbox}
\tcbuselibrary{listings}
\tcbuselibrary{minted}

\usepackage{ifthen}
\usepackage{fontawesome}

\usepackage{tabularx}
\newcolumntype{\CeX}{>{\centering\let\newline\\\arraybackslash}X}%
\newcommand{\TwoSymbolsAndText}[3]{%
  \begin{tabularx}{\textwidth}{c\CeX c}%
    #1 & #2 & #3
  \end{tabularx}%
}

\newtcblisting[use counter=inputPrg, number format=\arabic]{codeInput}[4]{
  listing engine=minted,
  minted language=#1,
  minted options={autogobble,breaklines,  firstnumber={#4}},
  listing only,
  size=title,
  arc=1.5mm,
  breakable,
  enhanced jigsaw,
  colframe=myblue,
  coltitle=White,
  boxrule=0.5mm,
  colback=white,
  coltext=Black,
  title=\TwoSymbolsAndText{\faCode}{%
  \textbf{Sql Program \thetcbcounter}\ifthenelse{\equal{#2}{}}{}{\textbf{: }#2}%
  }{\faCode},
  label=inputPrg:#3
}


\usepackage{boldline}
\usepackage{tikz,tcolorbox}
\usepackage{amsmath}
\usepackage[table,xcdraw]{xcolor}
\usepackage{listings}
\usepackage{array,multirow} % For customizing tables
\usepackage{booktabs} % For better horizontal lines
\usepackage{makecell}
\setlength{\parindent}{0pt}
\usepackage{siunitx}
\usepackage{tkz-tab}
\usepackage{amssymb}
\usepackage{amsmath}
\usepackage{enumitem}

\usepackage{caption}
\usepackage{float}


\newcommand{\exer}[1]{
  \section*{Exercice #1}
  \vspace{-0.5cm}
  \noindent\rule{\textwidth}{0.5pt}%
}

\newcommand{\tit}[1]{
\begin{center}
    \Large{\textbf{{#1}}}
\end{center}
}

\definecolor{commentgray}{HTML}{676160}
\definecolor{messagegreen}{HTML}{17B867}
\definecolor{myblue}{HTML}{10C2C4}

\tcbuselibrary{skins, breakable, theorems}


\newtcolorbox{prettyBox}[2]{
  enhanced,
  colback=white!90!#2,   % Background color based on the second parameter (color)
  colframe=#2!60!black,  % Frame color based on the second parameter (color)
  coltitle=white,        % Title color (white)
  fonttitle=\bfseries\Large,
  title=#1,              % Title from the first parameter
  boxrule=1mm,
  arc=0.5mm,
  drop shadow=#2!35!gray, % Drop shadow color based on the second parameter (color)
}



\lstdefinestyle{cmd}{
 basicstyle=\ttfamily,
 backgroundcolor=\color{lightgray!20},
 frame=single
}

\usepackage{minted}

\begin{document}

\tit{Executer Plusieur Instance D'ORACLE}

\vspace{0.5cm}

\begin{prettyBox}{Creation D'une Nouvelle Instance}{myblue}
    On va executer la commande \verb|dbca| pour lancer l'application java \textbf{Database Configuration Assistant}
     qui va nous permettre de creer une nouvelle instance d'oracle :
     \begin{enumerate}
         \item Choisir \textbf{Create a database} et cliquer sur \textbf{next}.
         \item Choisir \textbf{Typical configuration}.
         \item Donner un identifiant unique a l'instance \textbf{Global database name}.
         \item Donne un mot de passe administratif \textbf{Administrative password}.
         \item Donne un nom a la \textbf{Pluggable Database} (represente une base de donne de l'instance).
         \item Cliquer sur \textbf{next} puis \textbf{finish}.
         \item Attendre le telechargement puis cliquer sur \textbf{close}.
     \end{enumerate}
\end{prettyBox}

\vspace{0.25cm}
\begin{figure}[H] 
  \centering 
  \includegraphics[width=0.8\textwidth]{dbca.png} 
\end{figure}

\vspace{0.15cm}
\begin{figure}[H] 
  \centering 
  \includegraphics[width=0.8\textwidth]{name.png} 
\end{figure}

\begin{figure}[H] 
  \centering 
  \includegraphics[width=0.8\textwidth]{finish.png} 
\end{figure}

\vspace{0.15cm}

\begin{figure}[H] 
  \centering 
  \includegraphics[width=0.8\textwidth]{end.png} 
\end{figure}

\newpage

\begin{prettyBox}{Executer Plusieur Instance}{myblue}
    \begin{enumerate}
        \item Apres avoir creer les instance on va ouvrir plusieur instance du terminal et on va utilise la commande 
    \verb|set ORACLE_SID = <nom_instance>| qui va temporairement changer la variable d'environment \\\textbf{ORACLE\_SID} pour
    selectionner l'instance dans notre session terminal.
        \item On lance oracle avec \verb|sqlplus / as sysdba|.
        \item Pour verifier l'instance on va consulte la view \textbf{v\$instance} : \verb|select instance_name from v$instance|
    \end{enumerate}
\end{prettyBox}


\begin{figure}[H] 
  \centering 
  \includegraphics[width=0.8\textwidth]{instance1.png} 
\end{figure}


\begin{figure}[H] 
  \centering 
  \includegraphics[width=0.8\textwidth]{instance2.png} 
\end{figure}


\vspace{0.15cm}

\begin{prettyBox}{Creation Du Lien}{myblue}
    \begin{enumerate}
        \item Pour La Creation de lien on va se utiliser la requete suivant :

            \vspace{-0.25cm}

    \begin{verbatim}   
    CREATE DATABASE LINK NOM_LIEN
    CONNECT TO NOM_UTILISATEUR IDENTIFIED BY MOTDEPASSE_INSTANCE
    USING 'NOM_INSTANCE';
    \end{verbatim}

\vspace{-0.75cm}

        \item Acceder a une table avec le lien : \verb|nom_owner.nom_table@nom_lien|
    \end{enumerate}
\end{prettyBox}

\vspace{0.5cm}

Donc dans notre cas on va creer deux lien :
\begin{itemize}
    \item Lien Vers ORCL (executer depuis ORCL2):

\vspace{-0.25cm}
    \begin{verbatim}   
    CREATE DATABASE LINK link_to_orcl
    CONNECT TO system IDENTIFIED BY Nanno1234
    USING 'ORCL';
    \end{verbatim}

\vspace{-0.75cm}
    \item Lien Vers ORCL2  (executer depuis ORCL) :

\vspace{-0.25cm}
    \begin{verbatim}   
    CREATE DATABASE LINK link_to_orcl2
    CONNECT TO system IDENTIFIED BY Nanno1234
    USING 'ORCL2';
    \end{verbatim}
\end{itemize}

Test Pour Confirmer le lien :

\begin{figure}[H] 
  \centering 
  \includegraphics[width=0.8\textwidth]{test.png} 
\end{figure}
\end{document}
