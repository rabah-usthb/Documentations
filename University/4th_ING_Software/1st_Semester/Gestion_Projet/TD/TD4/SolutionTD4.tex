\documentclass{article}
\usepackage[a4paper, left=1.5cm, right=1.5cm, top=1cm, bottom=2cm]{geometry}

\usepackage{boldline}
\usepackage{tikz,tcolorbox}
\usepackage{amsmath}
\usepackage[table,xcdraw]{xcolor}
\usepackage{listings}
\usepackage{array,multirow} % For customizing tables
\usepackage{booktabs} % For better horizontal lines
\usepackage{makecell}
\setlength{\parindent}{0pt}
\usepackage{siunitx}
\usepackage{tkz-tab}
\usepackage{amssymb}
\usepackage{amsmath}
\usepackage{enumitem}

\usepackage{caption}
\usepackage{float}


\newcommand{\exer}[1]{
  \section*{Exercice #1}
  \vspace{-0.5cm}
  \noindent\rule{\textwidth}{0.5pt}%
}

\newcommand{\tit}[1]{
\begin{center}
    \Large{\textbf{{#1}}}
\end{center}
}

\definecolor{commentgray}{HTML}{676160}
\definecolor{messagegreen}{HTML}{17B867}
\definecolor{myblue}{HTML}{10C2C4}

\tcbuselibrary{skins, breakable, theorems}


\newtcolorbox{prettyBox}[2]{
  enhanced,
  colback=white!90!#2,   % Background color based on the second parameter (color)
  colframe=#2!60!black,  % Frame color based on the second parameter (color)
  coltitle=white,        % Title color (white)
  fonttitle=\bfseries\Large,
  title=#1,              % Title from the first parameter
  boxrule=1mm,
  arc=0.5mm,
  drop shadow=#2!35!gray, % Drop shadow color based on the second parameter (color)
}



\lstdefinestyle{cmd}{
 basicstyle=\ttfamily,
 backgroundcolor=\color{lightgray!20},
 frame=single
}

\lstdefinestyle{pythonstyle}{
    language=python,                    % Language set to Python
    basicstyle=\ttfamily\footnotesize,   % Change basic font size
    keywordstyle=\color{blue}\bfseries, % Different keyword style
    stringstyle=\color{red},         % Different string color
    commentstyle=\color{green!60!black}\itshape, % Adjust comment color
    numbers=left,                       % Line numbers on the left
    numberstyle=\tiny\color{gray},      % Smaller number font and color
    stepnumber=1,                       % Number each line
    frame=single,                       % Single frame around code
    tabsize=4,                          % Adjust tab size
    showstringspaces=false,             % Do not show spaces in strings
    captionpos=b,% Position of caption
    breaklines=true,
    inputencoding=utf8
}









\begin{document}


\tit{TD N\(^{\boldsymbol{\circ}}\)\hspace{0.1cm}4}


\begin{prettyBox}{Diagram De PERT}{myblue}
La méthode de PERT (Program Evaluation and Review Technique) permet de gérer l'ordonnancement des taches 
au sein d’un projet. Cette méthode modélise les tâches et liens entre celles‐ci d’un projet sous forme de réseau
qui permet d’identifier le chemin critique d’un projet et concentrer les
efforts sur les tâches le composant. 
\end{prettyBox}


\vspace{0.25cm}

\begin{prettyBox}{Liaisons}{myblue}
    
\begin{figure}[H] 
  \centering 
  \includegraphics[width=0.9\textwidth]{rel.png} 
\end{figure}

\end{prettyBox}

\vspace{0.25cm}
\begin{prettyBox}{Construction Diagram De PERT}{myblue}
    \begin{enumerate}
        \item Construire le reseau d'activite just les liens entre taches.
        \item PERT au plus tot , on commence depuis la premier tache de gauche a droite on utilise que les predecesseurs et on prend toujours le max.
        \item PERT au plus tard , on commence depuis la derniere tache de droite a gauche on utilise que les successeurs et on prend toujours le min.
    \end{enumerate}
\end{prettyBox}

\vspace{0.25cm}
\begin{prettyBox}{Marge Libre \& Marge Totale}{myblue}
    \begin{itemize}
        \item \textbf{Marge Libre : } le retard maximum que peut prendre une tache sans retarder la date debut plus tot
            de ces successeurs. elle est calculer avec la formule suivante :
         \[\boxed{\text{Marge Libre}(T) = \text{min}(a,b,c-e,d-e) - \text{Date Debut Au Plus Tot De La Tache }T}\]
            \begin{itemize}
                \item $a$ : min( date debut au plus tot des successeurs DD)
                \item $b$ : min( date fin au plus tot des successeurs DF)
                \item $c$ : min( date debut au plus tot des successeurs FD)
                \item $d$ : min( date fin au plus tot des successeurs FF)
                \item $e$ : Duree de la tache $T$
            \end{itemize}

        \item \textbf{Marge Total : } le retard maximum que peut prendre une tache sans retarder la date fin du projet, elle est calculer
            avec la formule suivante :
        \[\boxed{\text{Marge Total}(T)  =  \text{date debut plus tard de }T - \text{date debut plus tot de }T}\]
        \[ \boxed{\text{Marge Total}(T) = \text{date fin plus tard de }T - \text{date fin plus tot de }T }\]

        si Marge Total = 0 $\Rightarrow$ Marge Libre = 0
    \end{itemize}
\end{prettyBox}


\begin{prettyBox}{Liaisons}{myblue}
    
\begin{figure}[H] 
  \centering 
  \includegraphics[width=0.35\textwidth]{tache.png} 
\end{figure}

\end{prettyBox}


\vspace{0.25cm}

\begin{prettyBox}{Tache Critique}{myblue}
   Une tache est dire critique si sa margin total = 0
\end{prettyBox}

\vspace{0.25cm}
\begin{prettyBox}{Chemin Critique}{myblue}
    Chemin le plus long qui ne contient que des taches critiques , la duree de ce chemin represente la duree du projet.
\end{prettyBox}

\vspace{1cm}

{\LARGE Reseau D'activite}

\vspace{1cm}
\begin{figure}[H] 
  \centering 
  \includegraphics[width=0.8\textwidth]{pert.png} 
\end{figure}

\newpage


{\LARGE PERT Plus Tot}

\vspace{0.25cm}

\begin{figure}[H] 
  \centering 
  \includegraphics[width=0.8\textwidth]{pert1.png} 
\end{figure}

\vspace{0.5cm}
{\LARGE PERT Plus Tard}

\vspace{0.25cm}
\begin{figure}[H] 
  \centering 
  \includegraphics[width=0.8\textwidth]{pert2.png} 
\end{figure}

\newpage
{\LARGE Marges \par}
\begin{itemize}
    \item \textbf{A} : MT = 0-0 = 10-10 = 0 $\Rightarrow$ ML = 0.
    \item \textbf{B} : MT = 0-0 = 15-15 = 0 $\Rightarrow$ ML = 0.
    \item \textbf{E} : MT = 15-15 = 25-25 = 0 $\Rightarrow$ ML = 0.
    \item \textbf{G} : MT = 25-25 = 50-50 = 0 $\Rightarrow$ ML = 0.
    \item \textbf{H} : MT = 25-25 = 45-45 = 0 $\Rightarrow$ ML = 0.
    \item \textbf{J} : MT = 45-45 = 65-65 = 0 $\Rightarrow$ ML = 0.
    \item \textbf{I} : MT = 55-55 = 65-65 = 0 $\Rightarrow$ ML = 0.
    \item \textbf{C}: MT= 20-10 = 40-30 = 10  , ML = min(c-e) - Date Debut Plus Tot de C = min(30-10)-20 = 0.
    \item \textbf{D}: MT= 20-10 = 25-15 = 10  , ML = min(c-e) - Date Debut Plus Tot de D = min(25-5,30-5)-10 = 10.
    \item \textbf{D}: MT= 40-30 = 55-45 = 10  , ML = min(c-e) - Date Debut Plus Tot de F = min(55-15)-30 = 10.
\end{itemize}

\vspace{0.15cm}

\begin{figure}[H] 
  \centering 
  \includegraphics[width=0.8\textwidth]{pert3.png} 
\end{figure}

\vspace{0.15cm}

Taches Critique : A,B,E,G,H,J,I\\
Chemin Critique : A B E G H J I\\
Duree du projet : 65jours

\newpage

\begin{figure}[H] 
  \centering 
  \includegraphics[width=0.8\textwidth]{gant.png} 
\end{figure}


\end{document}
