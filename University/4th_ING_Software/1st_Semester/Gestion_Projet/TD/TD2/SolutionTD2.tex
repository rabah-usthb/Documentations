\documentclass{article}
\usepackage[a4paper, left=1.5cm, right=1.5cm, top=1cm, bottom=2cm]{geometry}

\usepackage{boldline}
\usepackage{tikz,tcolorbox}
\usepackage{amsmath}
\usepackage[table,xcdraw]{xcolor}
\usepackage{listings}
\usepackage{array,multirow} % For customizing tables
\usepackage{booktabs} % For better horizontal lines
\usepackage{makecell}
\setlength{\parindent}{0pt}
\usepackage{siunitx}
\usepackage{tkz-tab}
\usepackage{amssymb}
\usepackage{amsmath}
\usepackage{enumitem}

\usepackage{caption}
\usepackage{float}


\newcommand{\exer}[1]{
  \section*{Exercice #1}
  \vspace{-0.5cm}
  \noindent\rule{\textwidth}{0.5pt}%
}

\newcommand{\tit}[1]{
\begin{center}
    \Large{\textbf{{#1}}}
\end{center}
}

\definecolor{commentgray}{HTML}{676160}
\definecolor{messagegreen}{HTML}{17B867}
\definecolor{myblue}{HTML}{10C2C4}

\tcbuselibrary{skins, breakable, theorems}


\newtcolorbox{prettyBox}[2]{
  enhanced,
  colback=white!90!#2,   % Background color based on the second parameter (color)
  colframe=#2!60!black,  % Frame color based on the second parameter (color)
  coltitle=white,        % Title color (white)
  fonttitle=\bfseries\Large,
  title=#1,              % Title from the first parameter
  boxrule=1mm,
  arc=0.5mm,
  drop shadow=#2!35!gray, % Drop shadow color based on the second parameter (color)
}



\lstdefinestyle{cmd}{
 basicstyle=\ttfamily,
 backgroundcolor=\color{lightgray!20},
 frame=single
}

\lstdefinestyle{pythonstyle}{
    language=python,                    % Language set to Python
    basicstyle=\ttfamily\footnotesize,   % Change basic font size
    keywordstyle=\color{blue}\bfseries, % Different keyword style
    stringstyle=\color{red},         % Different string color
    commentstyle=\color{green!60!black}\itshape, % Adjust comment color
    numbers=left,                       % Line numbers on the left
    numberstyle=\tiny\color{gray},      % Smaller number font and color
    stepnumber=1,                       % Number each line
    frame=single,                       % Single frame around code
    tabsize=4,                          % Adjust tab size
    showstringspaces=false,             % Do not show spaces in strings
    captionpos=b,% Position of caption
    breaklines=true,
    inputencoding=utf8
}









\begin{document}


\tit{TD N\(^{\boldsymbol{\circ}}\)\hspace{0.1cm}2}

\vspace{0.15cm}


\begin{prettyBox}{WBS}{red}
    \textbf{WBS} sert à décomposer un projet complexe en tâches récapitulatives (sous-projets, phases, sous-phases, etc.)
    jusqu’à obtenir des feuilles dites tâches subordonnées afin d’avoir une arborescence, un meme projet peut avoir
    plusieur decomposition.
\end{prettyBox}


\vspace{0.15cm}


\begin{prettyBox}{RACI}{red}
    La matrice \textbf{RACI} est une matrice qui attribut les roles et responsabilite des ressources humains aux 
    tâches subordonnées , on a comme ligne les tâches subordonnées et les collones sont les ressources humains
    de chaque serive et a comme valeur :
    \begin{itemize}
        \item \textbf{R} : \textbf{R}éalisateur C'est celui qui effectue le travail. Chaque tâche nécessite une personne responsable, mais plusieurs personnes peuvent partager cette tâche.
        \item \textbf{A} : \textbf{A}pprobateur Il supervise et rend compte du travail effectué. Il doit approuver l'effort produit par le réalisateur. Il veille à ce que la tâche soit exécutée conformément au plan du projet, il valide le livrable (on peut 
            avoir seulment un responsable par tâches subordonnées).
        \item \textbf{C} : \textbf{C}onsulté Il délivre des conseils dans son domaine d'expertise pour accomplir une tâche, évaluer une situation, une production ou un livrable. C'est une partie prenante clé. Il n'a pas le pouvoir de prendre des décisions, mais il doit être consulté avant tout choix.
        \item \textbf{I} : \textbf{I}nformé Il est informé de l'état d'avancement du projet ou de la tâche, des changements et des résultats. Il s'agit généralement des parties prenantes qui ne contribuent pas directement aux projets. 
    \end{itemize}
\end{prettyBox}



\vspace{0.15cm}
\begin{enumerate}
    \item WBS
%
\begin{figure}[H] 
  \centering 
  \includegraphics[width=0.95\textwidth]{WBS.png} 
\end{figure}

%

\item RACI
%
\begin{figure}[H] 
  \centering 
  \includegraphics[width=0.95\textwidth]{RCI.png} 
\end{figure}


\end{enumerate}

\end{document}
