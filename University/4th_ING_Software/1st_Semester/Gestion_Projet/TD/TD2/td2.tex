\documentclass{article}
\usepackage[a4paper, left=1.5cm, right=1.5cm, top=1cm, bottom=2cm]{geometry}

\usepackage{boldline}
\usepackage{tikz,tcolorbox}
\usepackage{amsmath}
\usepackage[table,xcdraw]{xcolor}
\usepackage{listings}
\usepackage{array,multirow} % For customizing tables
\usepackage{booktabs} % For better horizontal lines
\usepackage{makecell}
\setlength{\parindent}{0pt}
\usepackage{siunitx}
\usepackage{tkz-tab}
\usepackage{amssymb}
\usepackage{amsmath}
\usepackage{enumitem}

\usepackage{caption}
\usepackage{float}


\newcommand{\exer}[1]{
  \section*{Exercice #1}
  \vspace{-0.5cm}
  \noindent\rule{\textwidth}{0.5pt}%
}

\newcommand{\tit}[1]{
\begin{center}
    \Large{\textbf{{#1}}}
\end{center}
}

\definecolor{commentgray}{HTML}{676160}
\definecolor{messagegreen}{HTML}{17B867}
\definecolor{myblue}{HTML}{10C2C4}

\tcbuselibrary{skins, breakable, theorems}


\newtcolorbox{prettyBox}[2]{
  enhanced,
  colback=white!90!#2,   % Background color based on the second parameter (color)
  colframe=#2!60!black,  % Frame color based on the second parameter (color)
  coltitle=white,        % Title color (white)
  fonttitle=\bfseries\Large,
  title=#1,              % Title from the first parameter
  boxrule=1mm,
  arc=0.5mm,
  drop shadow=#2!35!gray, % Drop shadow color based on the second parameter (color)
}



\lstdefinestyle{cmd}{
 basicstyle=\ttfamily,
 backgroundcolor=\color{lightgray!20},
 frame=single
}

\lstdefinestyle{pythonstyle}{
    language=python,                    % Language set to Python
    basicstyle=\ttfamily\footnotesize,   % Change basic font size
    keywordstyle=\color{blue}\bfseries, % Different keyword style
    stringstyle=\color{red},         % Different string color
    commentstyle=\color{green!60!black}\itshape, % Adjust comment color
    numbers=left,                       % Line numbers on the left
    numberstyle=\tiny\color{gray},      % Smaller number font and color
    stepnumber=1,                       % Number each line
    frame=single,                       % Single frame around code
    tabsize=4,                          % Adjust tab size
    showstringspaces=false,             % Do not show spaces in strings
    captionpos=b,% Position of caption
    breaklines=true,
    inputencoding=utf8
}









\begin{document}


\tit{TD N\(^{\boldsymbol{\circ}}\)\hspace{0.1cm}2}


\exer{1}

\vspace{0.45cm}

{\Large \textbf{Etude de cas d'un projet industriel}}

\vspace{0.25cm}

L'étude du cas présenté ici a pour objet d'établir le planning initial du projet. Le projet
étudié consiste à réaliser pour un client un prototype d'équipement industriel. Cet
équipement comprend des éléments mécaniques et électroniques. La réalisation du
prototype s'accompagne de la mise en place de I'outillage nécessaire. Le projet se termine
par une première pré-série, qui démontre la capacité à produire le prototype développé.\\[0.15cm]
Le projet fait appel à différents services de l'entreprise :

\begin{itemize} [noitemsep]
    \item marketing
    \item méthodes
    \item planning
    \item achats
    \item logistique
    \item qualité
    \item bureau d'études
\end{itemize}

L'identification des tâches est supposée acquise. Reste à déterminer d'abord l'articulation
des tâches entre elles \\(c'est-à-dire le réseau).\\[0.15cm]
Voici la présentation du projet tâche par tâche:\\[0.25cm]
{\large \textbf{Tâche "prototypage"}}\\[0.15cm]
Marketing : une personne du marketing pendant \textbf{six mois} pour le prototypage (Margot).
Bureau d'études: avec un ingénieur (Bernard).\\[0.25cm]
{\large \textbf{Tâche "planification de projet"}}\\[0.15cm]
Service planning: Dès le début du prototypage, nous démarrerons.\\
Nous travaillerons également \textbf{six mois} pour fournir le planning d'exécution final (Paul).\\
Bureau d'études : Un ingénieur fera partie de l'équipe (Bianca).\\[0.25cm]
{\large \textbf{Tâche "étude des prix et fournisseurs"}}\\[0.15cm]
Méthodes : dès la validation du prototype, nous ferons I'étude des prix et des fournisseurs.\\
Cela devrait prendre \textbf{un mois} (Michel).\\
Achats : deux personnes de chez nous participeront aussi (Albert et Aurélie).\\[0.25cm]
{\large \textbf{Tâche "conception"}}\\[0.15cm]
Bureau d'études : pour un produit de ce genre, la conception devrait prendre \textbf{douze mois}
(Bernard).\\
Le client devra approuver le dossier de conception.\\
La conception pourra démarrer dès que le prototype aura été validé.\\[0.25cm]
{\large \textbf{Tâche "électronique embarquée"}}\\[0.15cm]
Bureau d'études: dés que le planning détaillé sera arrêté, nous pourrons nous lancer dans
le sous-système électronique embarquée.\\
Le développement du sous-système électronique embarquée devrait nous prendre \textbf{six
mois} (Bernard).\\[0.25cm]
{\large \textbf{Tâche "installation de l'outillage"}}\\[0.15cm]
Achats : l'installation de l'outillage prendra \textbf{cinq mois} (Albert).\\
Elle pourra démarrer dès que le client aura approuvé la conception.\\
Méthodes : nous participerons aussi (Michei).\\
Mais il faudra ie sous-système électronique embarquée.\\[0.25cm]
{\large \textbf{Tâche "étude de fiabilité"}}\\[0.15cm]
Qualité : il faudra compter \textbf{trois mois} (Quantin).\\
Mais nous ne démarrerons que lorsque I'installation de l'outillage aura été lancée.\\
Il faudra les résuitats de 1'étude de fiabilité pour lancer le développement du logiciel.\\

\newpage

{\large \textbf{Tâche "développement du logiciel"}}\\[0.15cm]
Bureau d'études: le logiciel n'est même pas défini ... Alors, disons \textbf{six mois} !(Bianca).\\[0.25cm]
{\large \textbf{Tâche "mise en place de la pré-série"}}\\[0.15cm]
Logistique : nous serons chargés de la mise en place de la pré-série.
Avec l'étude des prix et des fournisseurs et le dossier de conception, il nous faudra \textbf{deux semaines} (Laure).\\[0.25cm]
{\large \textbf{Tâche "pré-série"}}\\[0.15cm]
Qualité :Il nous faudra le logiciel pour terminer la pré-série.\\
La pré-série c'est \textbf{une semaine} : nous serons deux (Quantin et Claude).\\
Méthodes: nous, nous serons trois pour suivre la pré-série (après linstallation de
l'outillage).(Michel, Maurice et Maud).

\begin{enumerate}
\item Proposez un WBS pour ce projet jusqu'au niveau des tâches opérationnelles.
\item Proposez une matrice RACI.
\end{enumerate}

\end{document}
