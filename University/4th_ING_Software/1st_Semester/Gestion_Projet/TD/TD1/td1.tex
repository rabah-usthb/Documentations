
\documentclass{article}
\usepackage[a4paper, left=1.5cm, right=1.5cm, top=1cm, bottom=2cm]{geometry}

\usepackage{boldline}
\usepackage{tikz,tcolorbox}
\usepackage{amsmath}
\usepackage[table,xcdraw]{xcolor}
\usepackage{listings}
\usepackage{array,multirow} % For customizing tables
\usepackage{booktabs} % For better horizontal lines
\usepackage{makecell}
\setlength{\parindent}{0pt}
\usepackage{siunitx}
\usepackage{tkz-tab}
\usepackage{amssymb}
\usepackage{amsmath}
\usepackage{enumitem}

\usepackage{caption}
\usepackage{float}


\newcommand{\exer}[1]{
  \section*{Exercice #1}
  \vspace{-0.5cm}
  \noindent\rule{\textwidth}{0.5pt}%
}

\newcommand{\tit}[1]{
\begin{center}
    \Large{\textbf{{#1}}}
\end{center}
}

\definecolor{commentgray}{HTML}{676160}
\definecolor{messagegreen}{HTML}{17B867}
\definecolor{myblue}{HTML}{10C2C4}

\tcbuselibrary{skins, breakable, theorems}


\newtcolorbox{prettyBox}[2]{
  enhanced,
  colback=white!90!#2,   % Background color based on the second parameter (color)
  colframe=#2!60!black,  % Frame color based on the second parameter (color)
  coltitle=white,        % Title color (white)
  fonttitle=\bfseries\Large,
  title=#1,              % Title from the first parameter
  boxrule=1mm,
  arc=0.5mm,
  drop shadow=#2!35!gray, % Drop shadow color based on the second parameter (color)
}



\lstdefinestyle{cmd}{
 basicstyle=\ttfamily,
 backgroundcolor=\color{lightgray!20},
 frame=single
}

\lstdefinestyle{pythonstyle}{
    language=python,                    % Language set to Python
    basicstyle=\ttfamily\footnotesize,   % Change basic font size
    keywordstyle=\color{blue}\bfseries, % Different keyword style
    stringstyle=\color{red},         % Different string color
    commentstyle=\color{green!60!black}\itshape, % Adjust comment color
    numbers=left,                       % Line numbers on the left
    numberstyle=\tiny\color{gray},      % Smaller number font and color
    stepnumber=1,                       % Number each line
    frame=single,                       % Single frame around code
    tabsize=4,                          % Adjust tab size
    showstringspaces=false,             % Do not show spaces in strings
    captionpos=b,% Position of caption
    breaklines=true,
    inputencoding=utf8
}









\begin{document}


\tit{TD N\(^{\boldsymbol{\circ}}\)\hspace{0.1cm}1}


\exer{1}

\vspace{0.25cm}

Une entreprise informatique décide de développer une application interne de gestion
du temps de travail.\\
Le chef de projet est le responsable du service développement, car ce service est le
plus concerné.\\
Pour mener à bien le projet, il doit collaborer avec :

\begin{itemize} [noitemsep]
    \item Le service \textbf{RH} pour définir les règles de pointage,
    \item Le service \textbf{Finance} pour la gestion des heures supplémentaires,
    \item Le service \textbf{Réseau} pour la mise en ligne de l'application.
\end{itemize}
        Chaque service désigne un correspondant chargé de communiquer avec le chef de
        projet.\\[0.35cm]
{\Large \textbf{Question}:}

\begin{enumerate} [noitemsep]
    \item ldentifiez le type d'organisation de projet représenté dans ce cas.
    \item Décrivez le rôle du chef de projet dans cette organisation.    
    \item Expliquez les avantages et les limites de ce type de coordination. 
    \item Proposez une solution pour améliorer la communication entre les services
dans ce contexte.
\end{enumerate}

\vspace{0.75cm}

\exer{2}

\vspace{0.25cm}
Une entrepris souhaite moderniser sa chaîne de production.\\
La \textbf{Direction Générale} nomme un \textbf{chef de projet} chargé de superviser les travaux.\\
Ce chef de projet coordonne plusieurs départements : maintenance, production,
qualité et achats.\\
li ne dépend d'aucun service en particulier, mais directement de la Direction
Générale.\\[0.35cm]
{\Large \textbf{Question}:}
\begin{enumerate}[noitemsep]
    \item ldentifiez le type de coordination utilisée.
    \item Indiquez les avantages de cette organisation pour le suivi du projet.
    \item Quelles difficultés peuvent apparaître si certains services refusent de
collaborer ?
\end{enumerate}
\end{document}
