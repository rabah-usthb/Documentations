
\documentclass{article}
\usepackage[a4paper, left=1.5cm, right=1.5cm, top=1cm, bottom=2cm]{geometry}

\usepackage{boldline}
\usepackage{tikz,tcolorbox}
\usepackage{amsmath}
\usepackage[table,xcdraw]{xcolor}
\usepackage{listings}
\usepackage{array,multirow} % For customizing tables
\usepackage{booktabs} % For better horizontal lines
\usepackage{makecell}
\setlength{\parindent}{0pt}
\usepackage{siunitx}
\usepackage{tkz-tab}
\usepackage{amssymb}
\usepackage{amsmath}
\usepackage{enumitem}

\usepackage{caption}
\usepackage{float}


\newcommand{\exer}[1]{
  \section*{Exercice #1}
  \vspace{-0.5cm}
  \noindent\rule{\textwidth}{0.5pt}%
}

\newcommand{\tit}[1]{
\begin{center}
    \Large{\textbf{{#1}}}
\end{center}
}

\definecolor{commentgray}{HTML}{676160}
\definecolor{messagegreen}{HTML}{17B867}
\definecolor{myblue}{HTML}{10C2C4}

\tcbuselibrary{skins, breakable, theorems}


\newtcolorbox{prettyBox}[2]{
  enhanced,
  colback=white!90!#2,   % Background color based on the second parameter (color)
  colframe=#2!60!black,  % Frame color based on the second parameter (color)
  coltitle=white,        % Title color (white)
  fonttitle=\bfseries\Large,
  title=#1,              % Title from the first parameter
  boxrule=1mm,
  arc=0.5mm,
  drop shadow=#2!35!gray, % Drop shadow color based on the second parameter (color)
}



\lstdefinestyle{cmd}{
 basicstyle=\ttfamily,
 backgroundcolor=\color{lightgray!20},
 frame=single
}

\lstdefinestyle{pythonstyle}{
    language=python,                    % Language set to Python
    basicstyle=\ttfamily\footnotesize,   % Change basic font size
    keywordstyle=\color{blue}\bfseries, % Different keyword style
    stringstyle=\color{red},         % Different string color
    commentstyle=\color{green!60!black}\itshape, % Adjust comment color
    numbers=left,                       % Line numbers on the left
    numberstyle=\tiny\color{gray},      % Smaller number font and color
    stepnumber=1,                       % Number each line
    frame=single,                       % Single frame around code
    tabsize=4,                          % Adjust tab size
    showstringspaces=false,             % Do not show spaces in strings
    captionpos=b,% Position of caption
    breaklines=true,
    inputencoding=utf8
}




\begin{document}
\tit{TD N\(^{\boldsymbol{\circ}}\)\hspace{0.1cm}1}


\begin{prettyBox}{C'est Quoi Un Projet}{red}
Ensemble d'efforts et d'activités fournis dans le but de créer
un produit, un service ou un résultat unique dans un temps limité avec un budget et des ressources contrôlées.
\end{prettyBox}


\vspace{0.25cm}

\exer{1}

\vspace{0.25cm}

\begin{figure}[H] 
  \centering 
  \includegraphics[width=0.75\textwidth]{ex1.png} 
\end{figure}


\vspace{0.25cm}


\begin{enumerate}
    \item Type d'organisation : \textbf{interne} car le chef de projet \underline{appartient à un service existant} (développement).
    \item Rôle du chef :
    \begin{itemize}
        \item Gestion de la partie technique.
        \item Coordonne avec les autres services via des correspondants (pas d'autorité hiérarchique sur eux).
    \end{itemize}
    \item Avantages et limites d'une organisation interne :
        \begin{itemize}
            \item Avantages :
                \begin{itemize}
                    \item Spécialisation, car tout le monde reste à son poste au sein de son service.
                    \item Rapidité et facilité à mettre en œuvre.
                    \item Faible coût (pas besoin d'engager des ressources humaines supplémentaires).
                \end{itemize}
            \item Limites :
                \begin{itemize}
                    \item Lenteur des décisions interservices.
                    \item Dépendance vis-à-vis des autres services, puisque le chef de projet dépend des autres services.
                \end{itemize}
        \end{itemize}
    \item Solutions pour améliorer la communication entre services :
        \begin{itemize}
            \item Réunions régulières interservices.
            \item Plateforme de gestion collaborative pour suivre l'avancement du projet.
        \end{itemize}
\end{enumerate}

\newpage


\exer{2}


\vspace{0.25cm}

\begin{figure}[H] 
  \centering 
  \includegraphics[width=0.75\textwidth]{ex2.png} 
\end{figure}



\vspace{0.25cm}

\begin{enumerate}
    \item Type d'organisation : \textbf{fonctionnelle}, car il \underline{dépend directement de la direction générale}.
    \item Avantages :
        \begin{itemize}
            \item Meilleure autorité du chef de projet (communication directe).
            \item Décisions rapides.
            \item Vision d'ensemble.
        \end{itemize}
    \item Limites :
        \begin{itemize}
            \item Surcharge et pression sur le chef de projet.
            \item Conflit hiérarchique, puisqu'un chef de service peut être en désaccord avec le chef de projet donnant des ordres à son équipe.
        \end{itemize}
\end{enumerate}

\end{document}
