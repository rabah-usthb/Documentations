\documentclass{article}
\usepackage[a4paper, left=1.5cm, right=1.5cm, top=1cm, bottom=2cm]{geometry}

\usepackage{boldline}
\usepackage{tikz,tcolorbox}
\usepackage{amsmath}
\usepackage[table,xcdraw]{xcolor}
\usepackage{listings}
\usepackage{array,multirow} % For customizing tables
\usepackage{booktabs} % For better horizontal lines
\usepackage{makecell}
\setlength{\parindent}{0pt}
\usepackage{siunitx}
\usepackage{tkz-tab}
\usepackage{amssymb}
\usepackage{amsmath}
\usepackage{enumitem}

\usepackage{caption}
\usepackage{float}


\newcommand{\exer}[1]{
  \section*{Exercice #1}
  \vspace{-0.5cm}
  \noindent\rule{\textwidth}{0.5pt}%
}

\newcommand{\tit}[1]{
\begin{center}
    \Large{\textbf{{#1}}}
\end{center}
}

\definecolor{commentgray}{HTML}{676160}
\definecolor{messagegreen}{HTML}{17B867}
\definecolor{myblue}{HTML}{10C2C4}

\tcbuselibrary{skins, breakable, theorems}


\newtcolorbox{prettyBox}[2]{
  enhanced,
  colback=white!90!#2,   % Background color based on the second parameter (color)
  colframe=#2!60!black,  % Frame color based on the second parameter (color)
  coltitle=white,        % Title color (white)
  fonttitle=\bfseries\Large,
  title=#1,              % Title from the first parameter
  boxrule=1mm,
  arc=0.5mm,
  drop shadow=#2!35!gray, % Drop shadow color based on the second parameter (color)
}



\lstdefinestyle{cmd}{
 basicstyle=\ttfamily,
 backgroundcolor=\color{lightgray!20},
 frame=single
}

\lstdefinestyle{pythonstyle}{
    language=python,                    % Language set to Python
    basicstyle=\ttfamily\footnotesize,   % Change basic font size
    keywordstyle=\color{blue}\bfseries, % Different keyword style
    stringstyle=\color{red},         % Different string color
    commentstyle=\color{green!60!black}\itshape, % Adjust comment color
    numbers=left,                       % Line numbers on the left
    numberstyle=\tiny\color{gray},      % Smaller number font and color
    stepnumber=1,                       % Number each line
    frame=single,                       % Single frame around code
    tabsize=4,                          % Adjust tab size
    showstringspaces=false,             % Do not show spaces in strings
    captionpos=b,% Position of caption
    breaklines=true,
    inputencoding=utf8
}









\begin{document}


\tit{TD N\(^{\boldsymbol{\circ}}\)\hspace{0.1cm}3}


\exer{1: WBS}

\vspace{0.45cm}

Contexte: L'universite souhaite organizer une journee portes ouvertes pour presente ses filieres et laboratoires.\\
En tant que chef de projet, vous devez construire la structure de decoupage du projet \textbf{(WBS)} afin d'identifier
toutes les activites necessaires a la reussite de cet evenement.

\begin{itemize}
    \item Proposez un \textbf{WBS} pour ce projet jusqu'au niveau des taches operationnelles.
\end{itemize}

\vspace{0.5cm}

\exer{2: Matric RACI}

\vspace{0.45cm}
Contexte: Toujours dans le cadre de la journee portes ouvertes, le chef de projet doit attribuer les responsabilites pour
chaque tache majeure.\\[0.15cm]

Acteurs identifies: chef de projet\textbf{(CP)}, Responsable communication\textbf{(RC)}, Responsable logistique\textbf{(RL)},\\ Service administratif\textbf{(SA)}, Enseignants participants\textbf{(EP)}.

\begin{itemize}
    \item Completez la matrice \textbf{RACI} ci-dessous.
\end{itemize}

\vspace{0.15cm}

\begin{figure}[H] 
  \centering 
  \includegraphics[width=0.9\textwidth]{empty.png} 
\end{figure}



\end{document}

