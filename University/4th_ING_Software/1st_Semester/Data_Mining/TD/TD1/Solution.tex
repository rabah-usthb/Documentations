
\documentclass{article}
\usepackage[a4paper, left=1.5cm, right=1.5cm, top=1cm, bottom=2cm]{geometry}

\usepackage{boldline}
\usepackage{tikz,tcolorbox}
\usepackage{amsmath}
\usepackage[table,xcdraw]{xcolor}
\usepackage{listings}
\usepackage{array,multirow} % For customizing tables
\usepackage{booktabs} % For better horizontal lines
\usepackage{makecell}
\setlength{\parindent}{0pt}
\usepackage{siunitx}
\usepackage{tkz-tab}
\usepackage{amssymb}
\usepackage{amsmath}
\usepackage{enumitem}

\usepackage{caption}
\usepackage{float}


\newcommand{\exer}[1]{
  \section*{Exercice #1}
  \vspace{-0.5cm}
  \noindent\rule{\textwidth}{0.5pt}%
}

\newcommand{\tit}[1]{
\begin{center}
    \Large{\textbf{{#1}}}
\end{center}
}

\definecolor{commentgray}{HTML}{676160}
\definecolor{messagegreen}{HTML}{17B867}
\definecolor{myblue}{HTML}{10C2C4}

\tcbuselibrary{skins, breakable, theorems}


\newtcolorbox{prettyBox}[2]{
  enhanced,
  colback=white!90!#2,   % Background color based on the second parameter (color)
  colframe=#2!60!black,  % Frame color based on the second parameter (color)
  coltitle=white,        % Title color (white)
  fonttitle=\bfseries\Large,
  title=#1,              % Title from the first parameter
  boxrule=1mm,
  arc=0.5mm,
  drop shadow=#2!35!gray, % Drop shadow color based on the second parameter (color)
}



\lstdefinestyle{cmd}{
 basicstyle=\ttfamily,
 backgroundcolor=\color{lightgray!20},
 frame=single
}

\lstdefinestyle{pythonstyle}{
    language=python,                    % Language set to Python
    basicstyle=\ttfamily\footnotesize,   % Change basic font size
    keywordstyle=\color{blue}\bfseries, % Different keyword style
    stringstyle=\color{red},         % Different string color
    commentstyle=\color{green!60!black}\itshape, % Adjust comment color
    numbers=left,                       % Line numbers on the left
    numberstyle=\tiny\color{gray},      % Smaller number font and color
    stepnumber=1,                       % Number each line
    frame=single,                       % Single frame around code
    tabsize=4,                          % Adjust tab size
    showstringspaces=false,             % Do not show spaces in strings
    captionpos=b,% Position of caption
    breaklines=true,
    inputencoding=utf8
}




\begin{document}
\tit{TD N\(^{\boldsymbol{\circ}}\)\hspace{0.1cm}1}
\exer{1}

\vspace{0.25cm}

\begin{prettyBox}{Les Statistiques}{red}
Les statistiques en data mining nous aident à \textbf{mieux connaître la base de données} et interviennent dans le \textbf{prétraitement}.
\end{prettyBox}

\vspace{0.25cm}
On a comme données sur l’attribut \textit{âge} :\\[0.1cm]
13, 15, 16, 16, 19, 20, 20, 21, 22, 22, 25, 25, 25, 25, 30, 33, 33, 35, 35, 35, 35, 36, 40, 45, 46, 52, 70.


\begin{prettyBox}{Remarque Sur Les Données}{red}
\begin{itemize}
    \item Un seul attribut \textit{âge}.
    \item Les données sont croissantes ce qui facilite les calcules. 
\end{itemize}
\end{prettyBox}

\vspace{0.25cm}


\begin{prettyBox}{Données à plusieurs attributs}{red}
    \begin{itemize}
        \item On a tendance à comparer les attributs d’une même base de données afin de découvrir des relations entre eux (des connaissances).
        \item Pour visualiser une base de données comportant plusieurs attributs (dimensions), on peut utiliser des graphiques comme les \textit{scatter plots} (entre deux attributs) ou \\des cartes thermiques \textit{heatmaps} (plusieurs attributs) pour représenter les relations entre les variables.
    \end{itemize}
\end{prettyBox}

\vspace{0.15cm}

\begin{enumerate}
    \item Formule de la moyenne :

    \[
        \boxed{\overline{X} = \frac{\sum x_i}{N}} \hspace{0.5cm}
    \text{où } N \text{ est l’effectif, et } x_i \text{ la } i^{\text{ème}} \text{ valeur.}
    \]
    
    \begin{align*}
        \overline{x} &= \frac{13+15+16\cdot2+19+20\cdot2+21+22\cdot2+25\cdot4+30+33\cdot2+35\cdot4+36+40+45+46+52+70}{27}\\ 
                     &= \frac{869}{27}\\
                     &= \boxed { 29.96 \textit{ ans}}
    \end{align*}

    \vspace{0.15cm}

    Formule de la mediane 
    
\[
    \text{Med}(X) =
    \begin{cases}
        X\!\left[\dfrac{N+1}{2}\right] & \text{si } N \text{ est impair}, \\[0.5cm]
        \dfrac{X\!\left[\dfrac{N}{2}\right] + X\!\left[\dfrac{N}{2} + 1\right]}{2} & \text{si } N \text{ est pair.}
    \end{cases}
    \]

    \vspace{0.15cm}

    Puisque \(N\) est impair = 27 :
    \begin{align*}
        \text{Med}(X) &= X\!\left[\dfrac{27+1}{2}\right]\\
                      &= X\![14]\\
                      &=\boxed{25\textit{ ans}}
    \end{align*}

    \newpage

\item Le \textit{mode} représente la ou les valeurs dont la fréquence est la plus élevée (le nombre de répétitions).\\[0.1cm]
Le \textit{type de modalité} fait référence au nombre de valeurs distinctes correspondant au mode \\(\textit{bimodale} : 2 valeurs, \textit{trimodale} : 3 valeurs, etc.).

\begin{center}
\begin{tabular}{ |c|c| } 
    \hline
    Valeur & Frequence \\
    \hline
    13 & 1 \\
    \hline
    16 & 2 \\ 
    \hline
    19 & 1 \\
    \hline
    20 & 2 \\ 
    \hline
    21 & 1 \\
    \hline
    22 & 2 \\ 
    \hlineB{3}
    \multicolumn{1}{!{\vrule width 0.5pt}c!{\vrule width 0.5pt}}{\textbf{25}} &
    \multicolumn{1}{!{\vrule width 0.5pt}c!{\vrule width 0.5pt}}{\textbf{4}} \\  
    \hlineB{3}
    30 & 1 \\ 
    \hline
    33 & 2 \\
    \hlineB{3}
    \multicolumn{1}{!{\vrule width 0.5pt}c!{\vrule width 0.5pt}}{\textbf{35}} &
    \multicolumn{1}{!{\vrule width 0.5pt}c!{\vrule width 0.5pt}}{\textbf{4}} \\  
    \hlineB{3}
    36 & 1 \\
    \hline
    40 & 1 \\ 
    \hline
    45 & 1 \\
    \hline
    46 & 1 \\ 
    \hline
    52 & 1 \\
    \hline
    70 & 1 \\ 
    \hline



\end{tabular}
\end{center}


\vspace{0.35cm}
On remarque que les valeurs 25 et 35 ont la plus haute fréquence, répétées 4 fois :\\[-0.15cm]
\[
\boxed{\text{Donc, le \textit{mode} est 25 et 35, et la modalité est dite \textit{bimodale}.}}
\]


\vspace{0.35cm}

\item Les quartiles permettent de diviser nos données en quatre parties égales :

\begin{figure}[H] 
  \centering 
  \includegraphics[width=0.5\textwidth]{Q.png} 
\end{figure}

Les formules :


\[
    Q_1 =
    \begin{cases}
        X\!\left[\dfrac{N+1}{4}\right] & \text{si } N \text{ est impair}, \\[0.5cm]
        X\!\left[\dfrac{N}{4}\right] & \text{si } N \text{ est pair}
    \end{cases}
    \]



\vspace{0.45cm}

\[
    Q_2 = \text{Med}(X)
\]



\vspace{0.45cm}
\[
    Q_3 =
    \begin{cases}
        X\!\left[\dfrac{3\cdot(N+1)}{4}\right] & \text{si } N \text{ est impair}, \\[0.5cm]
        X\!\left[\dfrac{3\cdot N}{4}\right] & \text{si } N \text{ est pair}
    \end{cases}
    \]

\newpage


    Puisque \(N\) est impaire = 27 :
    \begin{align*}
        Q_1 &= X\!\left[\dfrac{27+1}{4}\right]\\
                      &= X\![7]\\
                      &=\boxed{20\textit{ ans}}
    \end{align*}

    \begin{align*}
        Q_3 &= X\!\left[\dfrac{ 3\cdot (27+1)}{4}\right]\\
                      &= X\![21]\\
                      &=\boxed{35\textit{ ans}}
    \end{align*}


\vspace{0.35cm}

\item les cinq nombre de donne sont definie par : min \(Q_1\) \(Q_2\) \(Q_3\) max\\
    \[ \text{min} = 13\hspace{0.1cm}ans \hspace{0.15cm} , \hspace{0.15cm}Q_1 = 20\hspace{0.1cm}ans \hspace{0.15cm}, \hspace{0.15cm}Q_2 = 25\hspace{0.1cm}ans \hspace{0.15cm},\hspace{0.15cm} Q_3 =  35\hspace{0.1cm}ans\]
\vspace{0.5cm}


\item boxPlot 

\begin{prettyBox}{BoxPlot}{red}

\begin{figure}[H] 
  \centering 
  \includegraphics[width=0.5\textwidth]{boxPlot.png} 
\end{figure}

\begin{itemize}
    \item Elle nous permet de repérer les valeurs aberrantes.
    \item La médiane n'est pas forcément située au milieu de la boxplot.
    \item Les formules :
        \[\boxed{\text{lim}_{inf} = Q_1 - (1.5 \times EIQ)} \]
        \[\boxed{\text{lim}_{sup} = Q_3 + (1.5 \times EIQ)}\]
        \vspace{0.1cm}
        \[\boxed{EIQ (\text{Écart interquartile}) = Q_3 - Q_1}\]
    \item Si \(\text{lim}_{inf}\) ou \(\text{lim}_{sup}\) ont des valeurs incohérentes ou trop différentes du 
        min ou max, alors on prendra comme limites min et max respectivement.
\end{itemize}
   
\end{prettyBox}

\newpage

On a :

\[
    \begin{cases}
        X_{min} &= 13 \hspace{0.1cm} ans \\[0.1cm] 
        Q_1 &= 20\hspace{0.1cm}  ans  \\[0.1cm]
        Q_2 &= 25\hspace{0.1cm}  ans  \\[0.1cm]
        Q_3 &= 35\hspace{0.1cm} ans  \\[0.1cm]
        X_{max} &= 70\hspace{0.1cm}  ans  
    \end{cases}
    \]

On calcule les limites :
\begin{align*}
    EIQ &= Q_3 - Q_1 = 35 - 20 = \boxed{15\hspace{0.1cm}ans}\\[0.1cm]
    \text{lim}_{inf} &= Q_1 - (1.5 \times EIQ) = 20 - (1.5 \times 15) = \boxed{-2.5\hspace{0.1cm}ans}\\[0.1cm]
    \text{lim}_{sup} &= Q_3 + (1.5 \times EIQ) = 35 + (1.5 \times 15) = \boxed{57.5\hspace{0.1cm}ans}\\[0.1cm]
\end{align*}


\vspace{0.15cm}

\begin{prettyBox}{Limites}{red}
    \begin{itemize}
        \item On a \(\text{lim}_{inf} = -2.5\) ans, ce qui est une valeur incohérente dans le contexte âge \(\Rightarrow\) on prend comme limite le min = 13 ans.
        \item On a \(\text{lim}_{sup} = 57\) ans, valeur normale et pas loin du max \(\Rightarrow\) on la prend comme limite.
    \end{itemize}
\end{prettyBox}

\vspace{0.5cm}


On prend comme unité :\\[0.1cm]
8 carreaux \( \rightarrow \) 5 ans

\vspace{0.25cm}

 \begin{figure}[H] 
  \centering 
  \includegraphics[width=0.95\textwidth]{box1.png} 
\end{figure}


\vspace{0.25cm}

\begin{prettyBox}{Conclusion}{myblue}
    70 est un outlier
\end{prettyBox}

\end{enumerate}

\newpage

\exer{2}

\vspace{0.35cm}

On dispose des données de l'âge et du taux de graisse de 18 adultes dans un hôpital, sélectionnés au hasard :

\vspace{0.25cm}

\begin{center}
\begin{tabular}{ |c|c|c|c|c|c|c|c|c|c|} 
    \hline
    Âge & 23 & 23 & 27 & 27 & 39 & 41 &47 &49 & 50\\
    \hline
    Fat \% & 9.5 & 26.5 & 7.8 & 17.8 & 31.4 & 25.9 & 27.4 & 27.2 & 31.2 \\
    \hline
    Âge & 52 & 54 & 54 & 56 & 57 & 58 & 58 & 60 & 61\\
    \hline
    Fat \% & 34.6 & 42.5 & 28.8 & 33.4 & 30.2 & 34.1 & 32.9 & 41.2 & 35.7 \\
    \hline

\end{tabular}
\end{center}


\begin{prettyBox}{Remarques sur les données}{red}
\begin{itemize}
    \item Deux attributs : \textit{âge} et \textit{graisse}.
    \item Les données sur l'\textit{âge} sont croissantes, ce qui facilite les calculs.
    \item Les données sur la \textit{graisse} ne sont pas croissantes, donc on doit les ordonner.
\end{itemize}
\end{prettyBox}

\vspace{0.25cm}

\begin{enumerate}

\item Formule de l'écart-type :
    \[
    \sigma =
    \begin{cases}
        \sqrt{\dfrac{\sum (x_i - \overline{x})^2}{N-1}} & \text{si } N \text{ représente un échantillon} \\[0.5cm]
        \sqrt{\dfrac{\sum (x_i - \overline{x})^2}{N}} & \text{si } N \text{ représente la population complète}
    \end{cases}
    \]


Âge : 

\begin{align*}
        \overline{x} &= \frac{23\cdot2+27\cdot2+39+40+49+50+52+54\cdot2+56+57+58+60+61}{18}\\ 
                     &= \frac{836}{18}\\
                     &= \boxed { 46.44 \textit{ ans}}
    \end{align*}

    \vspace{0.5cm}

Puisque \(N\) est pair = 18 :
    \begin{align*}
        \text{Med}(X) &= \dfrac{X\!\left[\dfrac{18}{2}\right] + X\!\left[\dfrac{18}{2} + 1\right]}{2} \\[0.3cm]
                      &= \dfrac{X\!\left[9\right] + X\!\left[10\right]}{2} \\[0.3cm]
                      &= \dfrac{50 + 52}{2} \\[0.3cm]
                      &=\boxed{51\textit{ ans}}
    \end{align*}

\vspace{0.15cm}
\begin{center}
\begin{tabular}{ |c|c|c|c|c|c|c|c|} 
    \hline
    \(x_i\) & 23 & 27 & 39 & 41 &47 &49 & 50\\
    \hline
    \((x_i - \overline{x})^2\) & 549.433 & 377.913 & 55.353 & 29.593 & 0.313 & 6.553 & 12.673  \\
    \hline
    \(x_i\) & 52 & 54 & 56 & 57 & 58 & 60 & 61\\
    \hline
    \((x_i - \overline{x})^2\) & 30.913 & 57.153 & 91.393 & 111.513 & 133.633 & 183.873 & 211.993  \\
    \hline
\end{tabular}
\end{center}

\vspace{0.35cm}

Puisque \(N\) représente un échantillon :

\begin{align*}
        \sigma & =\sqrt{\dfrac{\sum (x_i - \overline{x})^2}{18-1}}\\ 
               & = \sqrt{\dfrac{2970.434}{17}}\\
               & =\boxed{13.218 \textit{ ans}}
\end{align*}


\newpage


Graisse : 


\begin{center}
\begin{tabular}{ |c|c|c|c|c|c|c|c|c|} 
    \hline
    7.8 & 9.5 & 17.8 & 25.9 & 26.5  & 27.2 & 27.4 & 28.8 & 30.2\\
    \hline
    31.2 & 31.4 &  32.9 & 33.4 & 34.1 & 34.6 & 35.7 & 41.2 & 42.5   \\
    \hline

\end{tabular}
\end{center}


\begin{align*}
    \overline{y} &= \frac{
                    \substack {7.8 + 9.5 + 17.8 + 25.9 + 26.5  + 27.2 + 27.4 + 28.8 + 30.2 +31.2 + 31.4 +  32.9 + 33.4+ 34.1 + 34.6 + 35.7 \\
                        + 41.2 + 42.5}
                }{18}\\
                     &= \frac{518.1}{18}\\
                     &= \boxed { 28.78 \textit{ \%}}
    \end{align*}

    \vspace{0.5cm}

Puisque \(N\) est pair = 18 :
    \begin{align*}
        \text{Med}(Y) &= \dfrac{Y\!\left[\dfrac{18}{2}\right] + Y\!\left[\dfrac{18}{2} + 1\right]}{2} \\[0.3cm]
                      &= \dfrac{Y\!\left[9\right] + Y\!\left[10\right]}{2} \\[0.3cm]
                      &= \dfrac{30.2 + 31.2}{2} \\[0.3cm]
                      &=\boxed{30.7\textit{ \%}}
    \end{align*}

\vspace{0.15cm}
\begin{center}
\begin{tabular}{ |c|c|c|c|c|c|c|c|c|c|} 
    \hline
    \(y_i\) & 7.8 & 9.5 & 17.8 & 25.9 & 26.5  & 27.2 & 27.4 & 28.8 & 30.2\\  
    \hline
    \((y_i - \overline{y})^2\) & 440.160 & 371.718 & 120.560 & 8.294 & 5.198 & 2.496  & 1.904 & 0.0001 & 2.016  \\
    \hline
    \(y_i\) & 31.2 & 31.4 &  32.9 & 33.4 & 34.1 & 34.6 & 35.7 & 41.2 & 42.5\\ 
    \hline
    \((y_i - \overline{y})^2\) & 5.856 & 6.864 & 16.974 & 21.344 & 28.302 & 33.872 & 47.886 & 154.256 & 188.238 \\
    \hline
\end{tabular}
\end{center}

\vspace{0.35cm}

Puisque \(N\) représente un échantillon :

\begin{align*}
        \sigma & =\sqrt{\dfrac{\sum (y_i - \overline{y})^2}{18-1}}\\ 
               & = \sqrt{\dfrac{1455.938}{17}}\\
               & =\boxed{9.254 \textit{ \%}}
\end{align*}



\item boxPlots :\\[0.15cm]
    Âge\\[0.1cm]
On a :

\[
    \begin{cases}
        X_{min} &= 23 \hspace{0.1cm} ans \\[0.1cm] 
        Q_1 &= X\left[\frac{N}{4}\right]  = X\left[\frac{18}{4}\right]=X[4.5]=X[5]= 39\hspace{0.1cm}  ans  \\[0.1cm]
        Q_2 &= 51\hspace{0.1cm}  ans  \\[0.1cm]
        Q_3 &= X\left[\frac{3\cdot N}{4}\right]  = X\left[\frac{3\cdot 18}{4}\right]=X[13.5]=X[14]= 57\hspace{0.1cm}  ans  \\[0.1cm]
        X_{max} &= 61\hspace{0.1cm}  ans  
    \end{cases}
    \]

On calcule les limits :
\begin{align*}
    EIQ &= Q_3 - Q_1 = 57 - 39 = \boxed{18\hspace{0.1cm}ans}\\[0.1cm]
    \text{lim}_{inf} &= 39 - (1.5 \times EIQ) = 20 - (1.5 \times 18) = \boxed{12\hspace{0.1cm}ans}\\[0.1cm]
    \text{lim}_{sup} &= 57 + (1.5 \times EIQ) = 35 + (1.5 \times 18) = \boxed{84\hspace{0.1cm}ans}\\[0.1cm]
\end{align*}


\vspace{0.15cm}

\begin{prettyBox}{Limits}{red}

\begin{itemize}
    \item On a \(\text{lim}_{inf} = 12\) ans, valeur normale et pas loin du min \(\Rightarrow\) on la prend comme limite.
    \item On a \(\text{lim}_{sup} = 84\) ans, valeur normale et pas loin du max \(\Rightarrow\) on la prend comme limite.
\end{itemize}
\end{prettyBox}

\vspace{0.25cm}


On prend comme unité :\\[0.1cm]
8 carreaux \( \rightarrow \) 5 ans

\vspace{0.25cm}

 \begin{figure}[H] 
  \centering 
  \includegraphics[width=0.95\textwidth]{box2.png} 
\end{figure}


\vspace{0.25cm}

\begin{prettyBox}{Conclusion}{myblue}
    aucun outlier
\end{prettyBox}

\vspace{1cm}

Graisse\\[0.1cm]
On a :

\[
    \begin{cases}
        Y_{min} &= 7.8 \hspace{0.1cm} \% \\[0.1cm] 
        Q_1 &= Y\left[\frac{N}{4}\right]  = Y\left[\frac{18}{4}\right]=Y[4.5]=Y[5]= 26.5\hspace{0.1cm}  \%  \\[0.1cm]
        Q_2 &= 30.7\hspace{0.1cm}  \% \\[0.1cm]
        Q_3 &= Y\left[\frac{3\cdot N}{4}\right]  = Y\left[\frac{3\cdot 18}{4}\right]=Y[13.5]=X[14]= 34.1 \hspace{0.1cm}  \%  \\[0.1cm]
        Y_{max} &= 42.5\hspace{0.1cm}  \%  
    \end{cases}
    \]

On calcule les limits :
\begin{align*}
    EIQ &= Q_3 - Q_1 = 34.1 - 26.5 = \boxed{7.6\hspace{0.1cm}\%}\\[0.1cm]
    \text{lim}_{inf} &= 26.5 - (1.5 \times EIQ) = 26.5 - (1.5 \times 7.6) = \boxed{15.1\hspace{0.1cm}\%}\\[0.1cm]
    \text{lim}_{sup} &= 34.1 + (1.5 \times EIQ) = 34.1 + (1.5 \times 7.6) = \boxed{\hspace{0.1cm}45.5\%}\\[0.1cm]
\end{align*}


\vspace{0.15cm}

\begin{prettyBox}{Limits}{red}

\begin{itemize}
    \item On a \(\text{lim}_{inf} = 15.1 \) \%, valeur normale et pas loin du min \(\Rightarrow\) on la prend comme limite.
    \item On a \(\text{lim}_{sup} = 45.5\) \%, valeur normale et pas loin du max \(\Rightarrow\) on la prend comme limite.
\end{itemize}
\end{prettyBox}

\newpage

On prend comme unité :\\[0.1cm]
8 carreaux \( \rightarrow \) 5 \%


\vspace{0.25cm}

 \begin{figure}[H] 
  \centering 
  \includegraphics[width=0.95\textwidth]{box3.png} 
\end{figure}


\vspace{0.25cm}

\begin{prettyBox}{Conclusion}{myblue}
    9.5 et 7.8 sont des outliers
\end{prettyBox}

\vspace{1cm}

\item scatter plot

\begin{prettyBox}{Corrélation}{myblue}
    La corrélation dans un scatter plot, c'est la relation entre l'attribut de l'axe \(X\) et l'axe \(Y\) :
    \begin{itemize}
        \item Corrélation positive : quand \(X\) augmente, \(Y\) augmente ; quand \(X\) diminue, \(Y\) diminue aussi.
        \item Corrélation négative : quand \(X\) augmente, \(Y\) diminue ; quand \(X\) diminue, \(Y\) augmente.
        \item Pas de corrélation : aucune relation.
    \end{itemize}
\end{prettyBox}

\vspace{0.25cm}


\item Normalization Z-score:
    \begin{prettyBox}{Z-score}{red}
        \[
            Z = \frac{x_i - \overline{x}}{\sigma}
        \]
Les valeurs sont entre \([0, 1]\). On normalise parce que les algorithmes (modèles) sont plus performants dans cet intervalle.
    \end{prettyBox}


\vspace{0.25cm}

Âge\\[0.15cm]
\begin{center}
\begin{tabular}{ |c|c|c|c|c|c|c|c|} 
    \hline
    \(x_i\) & 23 & 27 & 39 & 41 &47 &49 & 50\\
    \hline
    \(Z\) & -1.77 & -1.47 & -0.56 & -0.41 & 0.04 & 0.19 & 0.26  \\
    \hline
    \(x_i\) & 52 & 54 & 56 & 57 & 58 & 60 & 61\\
    \hline
    \(Z\) & 0.42 & 0.57 & 0.73 & 0.79 & 0.87 & 1.02 & 1.10  \\
    \hline
\end{tabular}
\end{center}

\newpage
\vspace{0.25cm}

Graisse\\[0.15cm]
\begin{center}
\begin{tabular}{ |c|c|c|c|c|c|c|c|c|c|} 
    \hline
    \(y_i\) & 7.8 & 9.5 & 17.8 & 25.9 & 26.5  & 27.2 & 27.4 & 28.8 & 30.2\\  
    \hline
    \(Z\) & -2.26 & -2.08 & -1.18 & -0.31 & -0.24 & -0.17  & -0.14 & 0.002 & 0.15  \\
    \hline
    \(y_i\) & 31.2 & 31.4 &  32.9 & 33.4 & 34.1 & 34.6 & 35.7 & 41.2 & 42.5\\ 
    \hline
    \(Z\) & 0.26 & 0.28 & 0.44 & 0.49 & 0.57 & 0.62 & 0.74 & 1.34 & 1.48 \\
    \hline
\end{tabular}
\end{center}

\vspace{0.5cm}

\item coefficient de corrélation:

\[
    r = 
    \begin{cases}
       \dfrac{\sum (x_i - \overline{x}) \cdot (y_i - \overline{y})}{(N-1) \cdot (\sigma_x \cdot \sigma_y)} & \text{si } N \text{ représente un échantillon} \\[0.75cm]
         \dfrac{\sum (x_i - \overline{x}) \cdot (y_i - \overline{y})}{N \cdot (\sigma_x \cdot \sigma_y)} & \text{si } N \text{ représente la population complète}
    \end{cases}
\]


\vspace{0.25cm}

Interprétation de \( r \in [-1, 1] \) :
\[
    \begin{cases}
        r = -1      &  \text{corrélation négative linéaire parfaite}      \\
        -1 < r < 0  &  \text{corrélation négative}    \\
        r = 0       &  \text{aucune corrélation}   \\
        0 < r < 1   &  \text{corrélation positive} \\
        r = 1       &  \text{corrélation positive linéaire parfaite}
    \end{cases}
\]
\end{enumerate}

\end{document}
